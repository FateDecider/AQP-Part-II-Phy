\documentclass[a4paper]{book}
%% Language and font encodings
\usepackage[english]{babel}
\usepackage[utf8x]{inputenc}
\usepackage[T1]{fontenc}
\usepackage{float}
%% Sets page size and margins
\usepackage[a4paper,top=3cm,bottom=2cm,left=3cm,right=3cm,marginparwidth=1.75cm]{geometry}
\usepackage[caption=false]{subfig}
%% Useful packages
\usepackage{fancyhdr}
\pagestyle{fancy}
\usepackage{amsmath}
\usepackage{amsthm}
\usepackage{enumitem}
\usepackage{eqnarray}
\usepackage{float}
\usepackage{esint}
\usepackage{wrapfig}
\usepackage{gensymb}
\usepackage{lipsum}
\usepackage{amssymb}
\usepackage{array}
\usepackage{tikz}
\usetikzlibrary{arrows,decorations.markings}
\usepackage[colorlinks=true, allcolors=blue]{hyperref}
\usepackage{graphicx}
\usepackage{amsmath}
\usepackage{amssymb}
\usepackage{graphicx}
\usepackage{mathtools}
\usepackage[colorlinks=true, allcolors=blue]{hyperref}
\DeclareMathOperator{\Id}{Id}
\DeclareMathOperator{\lcm}{lcm}
\DeclareMathOperator{\var}{Var}
\DeclareMathOperator{\sech}{sech}
\DeclareMathOperator{\cosech}{cosech}
\DeclareMathOperator{\cov}{Cov}
\DeclareMathOperator{\sgn}{sgn}
\DeclareMathOperator{\Span}{span}
\DeclareMathOperator{\nullity}{nullity}
\DeclareMathOperator{\rank}{rank}
\DeclareMathOperator{\Ker}{Ker}
\DeclareMathOperator{\R}{R}
\DeclareMathOperator{\Tr}{Tr}
\DeclareMathOperator{\sinc}{sinc}
\DeclareMathOperator{\diag}{diag}
\newtheorem{remarks}{Remarks}[section]
\newtheorem{eg}{Example}[section]
\newtheorem{Note}{Note}[section]
\definecolor{darkblue}{RGB}{	0, 0, 139}
\newtheoremstyle{new}% <name>
{2pt}% <Space above>
{2pt}% <Space below>
{\color{darkblue}}% Body font
{}% <Indent amount>
{\bfseries\color{black}}% Theorem head font
{:}% <Punctuation after theorem head>
{.5em}% <Space after theorem headi>
{}% <Theorem head spec (can be left empty, meaning `normal')>
\theoremstyle{new}
\makeatletter\@addtoreset{chapter}{part}\makeatother
\newtheorem{notation}{Notation}[section]
\newtheorem{law}{Law}[section]
\newtheorem{defi}{Definition}[section]
\newtheorem{thm}{Theorem}[section]
\newtheorem{prop}{Proposition}[section]
\newtheorem{lemma}{Lemma}[section]
\newtheorem{cor}{Corollary}[section]
\title{\textbf{AQP (Advanced Quantum Physics) Part II Phy}}
\author{Tai Yingzhe, Tommy (ytt26)}
\date{}
\setlength{\parindent}{0cm}
\begin{document}
\maketitle
\tableofcontents
\subsection*{Acknowledgements:}
Many thanks to my demonstrator Antonios M. Alvertis, and the lecturer Stafford Withington for their guidance. Some parts of this notes are heavily influenced by the previous years' Part II AQP handouts, as well as, the Part II DAMTP course Principles of Quantum Mechanics.
\newpage
\part{Formalism for quantum mechanics}
\chapter{Recap}
\section{Basic ideas}\label{sec:BasicIdeas}
\subsection{Wavefunctions}
\begin{defi}[Wavefunction]
A particle's state is specified by a complex-valued wavefunction $\psi(x)\in\mathbb{C}$.
\end{defi}
\begin{remarks}
If $\psi$ is appropriately normalized, i.e. $\int_{-\infty}^\infty|\psi(x)|^2dx=1$, then when we measure the position of a particle, we get a result $x$ with probability density function $|\psi(x)|^2$, i.e. the probability that the position is found in $[x,x+\delta x]$ for small $\delta x$ is $|\psi(x)|^2\delta x$.
\end{remarks}
\begin{eg}[Gaussian wavefunction]
A common wavefunction which we will encounter is the Gaussian wavefunction 
$$\psi(x)=Ce^{-(x-c)^2/2\alpha}$$
For this to be appropriately normalized, we require the normalization constant to be $C=(\alpha\pi)^{-1/4}$. As a result, small $\alpha$ will give a wavefunction with a sharp peak around the position $x=c$. $\alpha$ determines the width of a wavepacket.
\end{eg}
\begin{remarks}\leavevmode
\begin{enumerate}
    \item Quantum states are equivalence classes of wavefunctions under the equivalence relation $\psi\sim\phi$ if $\phi=\lambda\psi$ for some $\lambda\neq 0$.
    \item While normalized wavefunctions are not always desired, we do require normalizable wavefunctions, i.e.
    $$\int_{-\infty}^\infty|\psi(x)|^2dx<\infty$$
\end{enumerate}
\end{remarks}
\begin{defi}[Complex inner product]
Let $\psi(x)$ and $\phi(x)$ be normalizable wavefunctions, then the complex inner product is defined to be 
$$\langle\phi,\psi\rangle=\int_{-\infty}^\infty \phi(x)^*\psi(x)Dx$$
This complex inner product $\langle\phi,\psi\rangle$ defined on $\mathcal{H}$ must satisfy the following axioms:
$$\langle\phi,\alpha_1\psi_1+\alpha_2\psi_2\rangle=\alpha_1\langle\phi,\psi_1\rangle+\alpha_2\langle\phi,\psi_2\rangle$$
$$\langle\beta_1\phi_1+\beta_2\phi_2,\psi\rangle=\beta_1^*\langle\phi_1,\psi\rangle+\beta_2^*\langle\phi_2,\psi\rangle$$
$$\langle\phi,\psi\rangle=\langle\psi,\phi\rangle^*$$
$$||\psi||^2=\langle\psi,\psi\rangle\geq0$$
$$||\psi||=0\text{ iff }\psi=0$$
\end{defi}
\begin{defi}[Norm]
The norm of a state $\psi$ is
$$||\psi||^2=\langle\psi,\psi\rangle=\int_{-\infty}^\infty|\psi(x)|^2dx$$
\end{defi}
\subsection{Hilbert Space}
\begin{defi}[Hilbert space]
Hilbert space is a vector space $\mathcal{H}$ over $\mathbb{C}$ equipped with an inner product $\mathcal{H}\otimes\mathcal{H}\rightarrow\mathbb{C}$ that is complete. $\mathcal{H}$ is a vector space, where the addition satisfies the axioms $\forall\psi,\phi,\chi\in\mathcal{H}$
\begin{itemize}
    \item commutation: $\psi+\phi=\phi+\psi$
    \item association: $\psi+(\phi+\chi)=(\psi+\phi)+\chi$
    \item identity: $\exists 0\in\mathcal{H}$ such that $\psi+0=\psi$
\end{itemize}
Likewise, we can multiply $\psi\in\mathcal{H}$ by $a,b,c\in\mathbb{C}$, which obeys 
\begin{itemize}
    \item distributive over $\mathcal{H}$: $a(\psi+\phi)=a\psi+a\phi$
    \item distributive over $\mathbb{C}$: $(a+b)\psi=a\psi+b\psi$
\end{itemize}
The inner product obeys
\begin{itemize}
    \item Conjugate symmetry: $(\psi,\phi)=\overline{(\phi,\psi)}$
    \item Linear in the second entry:   $(\phi,a\psi)=a(\phi,\psi)$
    \item Additivity:
    $(\phi,\psi+\chi)=(\phi,\psi)+(\phi,\chi)$
    \item Positive-definite:
    $(\psi,\psi)\geq0,\quad\forall\psi\in\mathcal{H}$
    with equality iff $\psi=0$.
\end{itemize}
\end{defi}
\begin{remarks}\leavevmode
\begin{enumerate}
\item Given the inner product, we automatically have a norm $||\psi||=\sqrt{(\psi,\psi)}$. Further, Cauchy-Schwarz inequalities hold.
\item A set $\{\phi_1,...,\phi_n\}$ of vectors is linearly independent iff the only solution to $a_1\phi_1+...+a_n\phi_n=0$ is $a_i=0$ $\forall i$. The dimension of our vector space is the largest possible number of linearly independent vectors, and is infinite-dimensional if there does not exist such a large number.
\end{enumerate}
\end{remarks}
\begin{eg}
It turns out that all finite-dimensional Hilbert spaces are isomorphic to $\mathbb{C}^n$ for some $n\in\mathbb{N}$, where the norm is $||\mathbf{u}||=\sqrt{\sum_{i=1}^n|u_i|^2}$ and inner product $(\mathbf{u},\mathbf{v})=\sum_i\overline{u}_iv_i$.
\end{eg}
\begin{eg}
The sequence space $\ell^2$ is the space of all infinite sequences $\mathbf{u}=(u_1,u_2,...)$ such that $||\mathbf{u}||=\sqrt{\sum_{i=1}^n|u_i|^2}<\infty$. Heuristically, $\ell^2=\mathbb{C}^\infty$. We take $(\mathbf{u},\mathbf{v})=\sum_{i=1}^\infty\overline{u}_iv_i$ which converges by the Cauchy-Schwarz inequality if $||\mathbf{u}||$, $||\mathbf{v}||<\infty$.
\end{eg}
\begin{eg}
The function space $L^2(\mathbb{R},dx)$ is the space of all functions $\psi:\mathbb{R}\rightarrow\mathbb{C}$ such that $||\psi||^2=\int_{\mathbb{R}}|\psi(x)|^2dx<\infty$, i.e. normalizable. The inner product is $(\phi,\psi)=\int_{\mathbb{R}}\overline{\phi(x)}\psi(x)dx$ and is again finite by the Cauchy-Schwarz inequality.
\end{eg}
\begin{defi}[Dual]
The dual $\mathcal{H}^*$ of $\mathcal{H}$ is the space of all linear maps $\phi:\mathcal{H}\rightarrow\mathbb{C}$. $$\phi:~\psi\mapsto\phi(\psi)\in\mathbb{C},\quad\phi(a\psi+b\chi)=a\phi(\psi)+b\phi(\chi),\quad\forall a,b\in\mathbb{C},~\forall\chi,\psi\in\mathcal{H},~\phi\in\mathcal{H}^*$$
\end{defi}
\begin{remarks}\leavevmode
\begin{enumerate}
    \item One way to construct dual vectors is to use the inner product. Given $\phi\in\mathcal{H}$, one can define $(\phi,\cdot)\in\mathcal{H}^*$ by
$$(\phi,\cdot):~\psi\mapsto(\phi,\psi),\quad\forall\psi\in\mathcal{H}$$ In finite dimension, the inner product provides a natural isomorphism $\mathcal{H}^*\simeq\mathcal{H}$. 
\item Fortunately, the same result also holds in infinite dimensions, but it’s non–trivial to prove and is known as the Riesz Representation Theorem.
\end{enumerate}
\end{remarks}
\begin{defi}[Bra-ket notation]
$\psi\in\mathcal{H}$ denote as $|\psi\rangle$. An element of the dual $\mathcal{H}^*$ denote as $\langle\psi|$. The inner product is written as $\langle\phi|\psi\rangle$.
\end{defi}
\begin{remarks}[Introducing continuum bases]
Given an orthonormal basis $\{|e_a\rangle\}$ of $\mathcal{H}$, we expand $|\psi\rangle=\sum_a\psi_a|e_a\rangle$ where $\psi_a=\langle e_a|\psi\rangle$. If also, $|\chi\rangle=\sum_a\chi_a|e_a\rangle$, then $\langle\chi|\psi\rangle=\sum_{a,b}\overline{\chi}_b\psi_a\langle e_b|e_a\rangle=\sum_a\overline{\chi}_a\psi_a$ where $\langle e_b|e_a\rangle=\delta_{ba}$.\\[5pt]
It is very useful to extend this to function spaces, and we do this by introducing `continuum bases' $\{|a\rangle\}_{a\in\mathbb{R}}$ by writing $|\psi\rangle=\int_\mathbb{R}\psi(a)|a\rangle da$. If we normalize our `basis' by $\langle a'|a\rangle=\delta(a'-a)$ (Dirac Delta function), then
$$\langle\chi|\psi\rangle=\int_\mathbb{R}^2\overline{\chi}(a')\psi(a)dada'\langle a'|a\rangle=\int_\mathbb{R}\overline{\chi}(a)\psi(a)da$$
where $\langle a'|a\rangle=\delta(a'-a)$. This is the usual inner product on $L^2(\mathbb{R},da)$.
\end{remarks}
\begin{eg}[Position and momentum representation]
A common example is to use the `position basis' $\{|x\rangle\}$ or `momentum basis' $\{|p\rangle\}$. We then expand a state $|\psi\rangle=\int_{\mathbb{R}}\psi(x)|x\rangle dx$ in the `position basis' or $|\psi\rangle=\int_{\mathbb{R}}\tilde{\psi}(p)|p\rangle dp$ in `momentum basis'. We have
$\langle x|\psi\rangle=\int_\mathbb{R}\psi(x')\langle x|x'\rangle dx'=\psi(x)$ and similarly, $\langle p|\psi\rangle=\int_\mathbb{R}\tilde{\psi}(p')\langle p|p'\rangle dp'=\tilde{\psi}(p)$. Comparing the two representations, and using $\langle x|p\rangle=\frac{e^{ipx/\hbar}}{\sqrt{2\pi\hbar}}$, we will have $\langle x|\psi\rangle=\int_{\mathbb{R}}\tilde{\psi}(p)\frac{e^{ipx/\hbar}dp}{\sqrt{2\pi\hbar}}$. So $\psi(x)$ and $\tilde{\psi}(p)$ are each other's Fourier transforms.
\end{eg}
\begin{remarks}\leavevmode
\begin{enumerate}
\item The physical information about our system is contained in the abstract state $|\psi\rangle\in\mathcal{H}$. The position/momentum wavefunctions $\psi(x)$/$\tilde{\psi}(p)$ are just coefficients of $|\psi\rangle$ in some basis.
\item If we have the following orthogonality relation $\langle x'|x\rangle=\delta(x'-x)$, then the norm would be $\langle x|x\rangle=|||x\rangle||^2=\delta(0)$. These objects certainly do not lie in the Hilbert space, i.e. $|x\rangle\notin\mathcal{H}$. There is a formulation of QM that makes sense of this, but we will just proceed normally by saying that continuum states such as $|x\rangle$ are allowed as basis elements, but call them non–normalizable states: actual physical particles are never represented by a non-normalizable state.
\end{enumerate}
\end{remarks}
\subsection{Operators}
\begin{defi}[Linear Operators]
A linear operator $\hat{O}$ is a linear map $\hat{O}:~\mathcal{H}\rightarrow\mathcal{H}$, i.e.
$$\hat{O}:c_1|\psi_\rangle+c_2|\psi_2\rangle\mapsto c_1\hat{O}|\psi_1\rangle+c_2\hat{O}|\psi_2\rangle,\quad|\psi_1\rangle,|\psi_2\rangle\in\mathcal{H},~c_1,c_2\in\mathbb{C}$$
\end{defi}
\begin{notation}
We will drop the $\hat{}$ and it will be understood to be an operator, whereever applicable.
\end{notation}
\begin{remarks}
Suppose there are two linear operators $A$ and $B$, we define their sum and compositions to be
$$\alpha A+\beta B:~|\psi\rangle\mapsto\alpha A|\psi\rangle+\beta B|\psi\rangle$$
$$AB:~|\psi\rangle\mapsto A(B|\psi\rangle)$$
for $|\psi\rangle\in\mathcal{H}$, $\alpha,\beta\in\mathbb{C}$. The operator operation is associative, i.e. $(AB)C=A(BC)$, but is in general, not commutative, i.e. $AB$ not necessarily equal to $BA$. The commutator $[A,B]=AB-BA$ obeys
\begin{itemize}
    \item anti-symmetry, i.e. $[A,B]=-[B,A]$
    \item linearity, i.e. $[\alpha A+\beta B,C]=\alpha[A,C]+\beta[B,C]$  $\forall\alpha,\beta\in\mathbb{C}$
    \item Leibniz rule $[A,BC]=[A,B]C+B[A,C]$
    \item Jacobi Identity
    $$[A,[B,C]]+[B,[C,A]]+[C,[A,B]]=0$$
\end{itemize}
\end{remarks}
\begin{defi}[Eigenstates]
$|\psi\rangle\in\mathcal{H}$ is an eigenstate of $A:\mathcal{H}\rightarrow\mathcal{H}$ if $A|\psi\rangle=a_\psi|\psi\rangle$ for some $a_\psi\in\mathbb{C}$. The set of all eigenvalues of an operator is its spectrum, while the number of linearly independent eigenstates with the same eigenvalue is the degeneracy of that eigenvalue. In Dirac notation, we write $Q|q\rangle=q|q\rangle$, where we label the eigenstate by its eigenvalue.
\end{defi}
\begin{defi}[Adjoint of linear operator]
The structure of the inner product allows us to define the adjoint $A^\dag$ of an operator $A$ by
$$\langle\phi|A^\dag|\psi\rangle=\overline{\langle\psi|A|\phi\rangle},\quad\forall|\psi\rangle,|\phi\rangle\in\mathcal{H}$$ Moreover, the adjoint satisfy the properties
\begin{itemize}
    \item $(A+B)^\dag=A^\dag+B^\dag$
    \item $(AB)^\dag=B^\dag A^\dag$
    \item $[A,B]^\dag=[B^\dag,A^\dag]=-[A^\dag,B^\dag]$
\end{itemize}
Also, if $A|a\rangle=a|a\rangle$, then $\langle a|A^\dag=\langle a|\overline{a}$.
\end{defi}
\begin{remarks}[Hermitian operators]
An operator $Q$ is called Hermitian if $Q^\dag=Q$.
\begin{itemize}
    \item Hermitian operators have real spectrum;
    \item Two eigenstates of a Hermitian operator with distinct eigenvalues are orthogonal.
\end{itemize}
This is useful because we can use the complete set of eigenstates of Hermitian operator as a basis. If $\{|n\rangle\}$ is a complete set of eigenstates of $Q$, with $Q|n\rangle=q_n|n\rangle$, then we can write
\begin{equation}
Q=\sum_nq_n|n\rangle\langle n|\label{operatorexpansion}
\end{equation}
such that $Q|\psi\rangle=\sum_nq_n|n\rangle\langle n|\psi\rangle$. 
\end{remarks}
\begin{defi}[Functions of operators]
We can extend our understanding to functions of operators, i.e.
\begin{equation}
f(Q)=\sum_nf(q_n)|n\rangle\langle n|\label{operatorexpansion2}
\end{equation}
whenever $f(q_n)$ makes sense. For example, $Q^{-1}=\sum_nq_n^{-1}|n\rangle\langle n|$ provided $q_n\neq0$ $\forall n$.
\end{defi}
\begin{eg}
Let $\{|p\rangle\}_{p\in\mathbb{R}}$ be a complete set of eigenstates of $\hat{p}$, then on a generic state $|\psi\rangle\in\mathcal{H}$, we have
$$\langle x|p|\psi\rangle=\int\langle x|p|p\rangle\langle p|\psi\rangle dp=\int_\mathbb{R}pe^{ixp/\hbar}\tilde{\psi}(p)\frac{dp}{\sqrt{2\pi\hbar}}=-i\hbar\frac{\partial}{\partial x}\bigg[\int e^{ipx/\hbar}\tilde{\psi}(p)dp\bigg]$$
where $\int|p\rangle\langle p|dp=1$. The action for the operator $p$ on $\psi(x)$ is equal to $-i\hbar\frac{\partial}{\partial x}\psi(x)$ as expected.
\end{eg}
\subsection{Postulates of Quantum Mechanics}
\begin{defi}[First postulate]
Any physical system can be described by some state vector $|\psi\rangle\in\mathcal{H}$. Further, any complete, orthogonal set $\{|\phi_a\rangle,|\phi_b\rangle,...\}$ of states in $\mathcal{H}$ is one-to-one correspondence with all possible outcomes of the measurements of some physical quantity. If a system is in the general state $|\psi\rangle=\sum_nc_n|\phi_n\rangle$, then the probability we obtained as a result corresponding to $|\phi_b\rangle$ is $|c_b|^2$, i.e.
\begin{equation}
\text{prob}(|\psi\rangle\text{ in }|\phi_b\rangle)=\frac{|\langle\phi_b|\psi\rangle|^2}{\langle\phi_b|\phi_b\rangle\langle\psi|\psi\rangle}=|c_b|^2\label{prob_postulate}
\end{equation}
\end{defi}
\begin{remarks}\leavevmode
\begin{enumerate}
    \item Which $\mathcal{H}$ do we use? e.g. $\mathbb{C}^n$, $L^2(\mathbb{R}^3,d^3x)$, etc? Depends on the system we are trying to describe. For example, a single point particle may be determined by using $\mathcal{H}\simeq L^2(\mathbb{R}^3,d^3x)$, whereas to describe a system with some internal structure, we may need a larger $\mathcal{H}$.
    \item QM is inherently probabilistic. This interpretation originally from considering the scattering experiments by Born.
    \item We can clean up the Born rule by working only with normalized states, i.e. $\langle\psi|\psi\rangle=1$ (likewise, using orthonormal basis). Then
    $$\text{prob}(|\psi\rangle\text{ in }|\phi_b\rangle)=|\langle\phi_b|\psi\rangle|^2$$
    i.e. $\sum_k\text{prob}(|\psi\rangle\text{ in }|\phi_k\rangle)=1$. Even when using normalized states, we are still free to rescale $|\psi\rangle\rightarrow e^{i\phi}|\psi\rangle$ for some constant $\phi\in\mathbb{R}$. Combining them with the normalization, physical states really correspond to rays in $\mathcal{H}$, i.e. $\{|\psi\rangle\sim\lambda|\psi\rangle,\lambda\in\mathbb{C}^*\}$. The zero vector $0\in\mathcal{H}$ is unique state with zero norm, so does not represent a physical system.
\end{enumerate}
\end{remarks}
\begin{defi}[Second postulate]
Observable quantities corresponding to Hermitian operators. Let $Q$ be a Hermitian opetrator and $\{|n\rangle\}$ its basis of eigenstates of $Q|n\rangle=q_n|n\rangle$. Then the expectation value of QM state $|\psi\rangle$ is $\langle Q\rangle_\psi=\langle\psi|Q|\psi\rangle=\sum_{n,m}c_n\overline{c}_m\langle m|Q|n\rangle=\sum_n|c_n|^2q_n$, where $|\psi\rangle=\sum_nc_n|n\rangle$.
\end{defi}
\begin{remarks}\leavevmode
\begin{enumerate}
    \item For Hermitian operators,
    $$0\leq||Q|\psi\rangle||^2=\langle\psi|Q^2|\psi\rangle\implies 0\leq||(Q-\langle Q\rangle_\psi)|\psi\rangle||^2=\langle Q^2\rangle_\psi-\langle Q\rangle^2_\psi $$
    where $Q^\dag=Q$ since Hermitian. We define $\sqrt{\langle Q^2\rangle_\psi-\langle Q\rangle_\psi^2}$ to be the uncertainty of $Q$ in state $|\psi\rangle$. So we're only certain of the outcome of an experiment if we already know our system is in an eigenstate of the corresponding operator.
    \item We are not saying anything about how to describe the physics of actually carrying out a measurement. In particular, we do not say this process has anything to do with applying the corresponding operator.
\end{enumerate}
\end{remarks}
\begin{defi}[Third postulate]
Our state evolves in time according to the Schr\"{o}dinger's equation
\begin{equation}
i\hbar\frac{\partial}{\partial t}|\psi\rangle=H(t)|\psi\rangle\label{Schrodinger}
\end{equation}
for some time-dependent Hamiltonian operator, and assuming $|\psi\rangle$ is once-differentiable in time.
\end{defi}
\subsection{Uncertainty principle}
\begin{defi}[Uncertainty]
The uncertainty in a quantity $Q$ (with a corresponding Hermitian observable) is
\begin{equation}
(\sigma_Q)^2_\psi=\langle Q^2\rangle_\psi-\langle Q\rangle_\psi^2\label{uncertainty}
\end{equation}
\end{defi}
\begin{prop}
The uncertainties in position and momentum are real and positive.
\end{prop}
\begin{proof}
Define Hermitian operators $X=x-\alpha$, $P=p-\beta$ for any $\alpha,\beta\in\mathbb{R}$, then
$$\langle x^2\rangle_\Psi-2\alpha\langle x\rangle_\Psi+\alpha^2=\langle\Psi,X^2\Psi\rangle=\langle X\Psi,X\Psi\rangle=||X\Psi||^2\geq0$$
$$\langle p^2\rangle_\Psi-2\alpha\langle p\rangle_\Psi+\alpha^2=(\sigma_p)_\Psi^2=\langle\Psi,P^2\Psi\rangle=\langle P\Psi,P\Psi\rangle=||P\Psi||^2\geq0$$
We can choose $\alpha=\langle x\rangle_\Psi$ and $\beta=\langle p\rangle_\Psi$ which are both real, such that the LHS of the two expressions above give $(\sigma_x)^2_\Psi$ and $(\sigma_p)^2_\Psi$ respectively.
\end{proof}
\begin{thm}[Heisenberg's uncertainty principle]
For normalized $\Psi$, the uncertainties of the position and momentum satisfy the following inequality $(\sigma_x)_\Psi(\sigma_p)_\Psi\geq\frac{\hbar}{2}$.
\end{thm}
\begin{proof}
Again, define $X=x-\alpha$, $P=p-\beta$ but choose $\alpha=\langle x\rangle_\Psi$, $\beta=\langle p\rangle_\Psi$, then 
$$(\sigma_x)_\Psi^2=\langle\Psi,X^2\Psi\rangle=\langle X\Psi,X\Psi\rangle=||X\Psi||^2,\quad(\sigma_p)_\Psi^2=\langle\Psi,P^2\Psi\rangle=\langle P\Psi,P\Psi\rangle=||P\Psi||^2$$
Then, we have $(\sigma_x)_\Psi(\sigma_p)_\Psi$ to be
$$||X\Psi||||P\Psi||\geq|\langle X\Psi,P\Psi\rangle|\geq|\text{Im}\langle X\Psi,P\Psi\rangle|\geq\frac{1}{2i}[\langle X\Psi,P\Psi\rangle-\langle P\Psi,X\Psi\rangle]|=\bigg|\frac{1}{2i}\langle\psi,[X,P]\Psi\rangle\bigg|=\frac{\hbar}{2}$$
where we used Cauchy-Schwarz inequality.
\end{proof}
\begin{eg}
Gaussian wavefunctions saturates the inequality. Consider $\Psi(x)=\frac{1}{\sqrt{\sqrt{\alpha\pi}}}e^{-x^2/\alpha}$. We have $\langle x\rangle_\Psi=\langle p\rangle_\Psi=0$ and $(\sigma_x)^2_\Psi=0.5\alpha$ and $(\sigma_p)^2_\Psi=\frac{\hbar^2}{2\alpha}$. Hence, $(\sigma_x)_\Psi(\sigma_p)_\Psi=\sqrt{\frac{\alpha}{2}\frac{\hbar^2}{2\alpha}}=\frac{\hbar}{2}$.
\end{eg}
\begin{remarks}
The Heisenberg uncertainty relation is a corollary of the generalized uncertainty relation
\begin{equation}
(\sigma_A)_\Psi(\sigma_B)_\Psi\geq\bigg|\frac{1}{2i}\langle[A,B]\rangle_\Psi\bigg|\label{uncertainty}
\end{equation}
\end{remarks}
\begin{defi}[Degeneracy]
For any observable $Q$, the number of linearly independent eigenstates with eigenvalue $\lambda$ is the degeneracy of that eigenvalue.
\end{defi}
If $\lambda$ is degenerate, then there is a large freedom in choosing an orthonormal basis for $V_\lambda$. Physically, we cannot distinguish degenerate states by measuring $Q$ alone.
\begin{prop}
A necessary and sufficient condition for two observables (their operators) to be simultaneously diagonalizable is for $A$ and $B$ to commute.
\end{prop}
\begin{proof}
Consistent with the generalized uncertainty relation. If a state is simultaneously an eigenstate for A and B, the LHS is zero and so the commutator is zero.
\end{proof}
\subsection{Discrete and continuous spectra}
One way of dealing with continous spectra, place the system in a `box' of length $\ell$ with suitable b.c.s and finally take $\ell\rightarrow\infty$.\\[5pt]
For instance, consider $\psi(x)$ with periodic boundary conditions $\psi(x+\ell)=\psi(x)$, $-0.5\ell\leq x\leq0.5\ell$. We have
$$p=-i\hbar\frac{d}{dx},~\chi_n(x)=\frac{1}{\sqrt{\ell}}e^{ik_nx},~k_n=\frac{2\pi n}{\ell},~\lambda_n=\hbar k_n,~\langle\chi_n,\chi_m\rangle=\delta_{mn}$$
Write $\psi(x)=\sum_n\alpha_n\chi_n(x)$ where $\alpha_n=\langle\chi_n,\psi\rangle$. Take $n\rightarrow\infty$ such that the Fourier series becomes a Fourier integral. We could also replace discrete label $n$ with continuous label $\xi$
$$Q\chi_\xi=\lambda_\xi\chi_\xi,\quad\langle\chi_\xi,\chi_\eta\rangle=\delta(\xi-\eta)$$
and $\psi=\int\alpha_\xi\chi_\xi d\xi$ where $\alpha_\xi=\langle\chi_\xi,\psi\rangle$.
\subsection{Gaussian wavepackets}
\begin{defi}[Wavepacket]
A wavefunction localized in space is called a wavepacket, and it is usually Gaussian.
\begin{equation}
\Psi_0(x,t)=\bigg(\frac{\alpha}{\pi}\bigg)^{1/4}\frac{1}{\gamma(t)^{1/2}}e^{-x^2/2\gamma(t)}\label{Gaussian}
\end{equation}
for some $\gamma(t)$. 
\end{defi}
\begin{remarks}
The Gaussian wavepacket is a solution to the time-dependent Schr\"{o}dinger's equation Eqn.~\ref{Schrodinger} (free particle case with $V=0$) such that $\gamma(t)=\alpha+\frac{i\hbar}{m}t$. The probability density is then
$$P_0(x,t)=|\Psi_0(x,t)|^2=\sqrt{\frac{\alpha}{\pi}}\frac{1}{|\gamma(t)|}e^{-\alpha x^2/|\gamma(t)|^2}$$
This is a Gaussian that peaks at $x=0$, and corresponds to a particle at rest at the origin, spreading out with time. A related solution is that of a moving particle,
$$\Psi_u(x,t)=\Psi_0(x-ut)e^{imux/\hbar}e^{-imu^2t/2\hbar}$$
with probability density
$$P_u(x,t)=|\Psi_u(x,t)|^2=P_0(x-ut,t)$$
which indeed corresponds to a particle moving with velocity $u$ where $\langle p\rangle_{\Psi_u}=mu$. In the limit $\alpha\rightarrow\infty$, the uncertainty in position becomes larger but the momentum becomes more definite. Then $\Psi$ has the form
$$\Psi(x,t)=Ce^{ikx}e^{-iEt/\hbar}$$
which is a momentum eigenstate with $\hbar k=mu$ and energy $E=\frac{1}{2}mu^2=\frac{\hbar^2k^2}{2m}$.
\end{remarks}
\subsection{Stationary states}
\begin{defi}[Stationary states]
By using separation of variables, $\Psi(x,t)=\psi(x)T(t)$ for the time-dependent Schr\"{o}dinger's equation, one obtains 
\begin{equation}
\Psi(x,t)=\psi(x)e^{-iEt/\hbar}\label{stationary}
\end{equation}
Such wavefunctions $\Psi(x,t)$ are called stationary states, where $\psi(x)$ is an eigenfunction of the Hamiltonian with eigenvalue $E$.
\end{defi}
\begin{remarks}
Every possible state of a system can be written as a linear combination of stationary states since they form a basis of state space.
\end{remarks}
\begin{cor}
For stationary state, the probability distribution is $|\Psi(x,t)|^2=|\psi(x)|^2$ and is independent of time.
\end{cor}
\begin{proof}
Since the time-dependence part is a complex exponential with $Et/\hbar$ as its phase, then it disappears after taking the modulus of $\Psi(x,t)$.
\end{proof}
\begin{prop}
The probability density $P(x,t)=|\Psi(x,t)|^2$ obeys a conservation equation
\begin{equation}
\frac{\partial P}{\partial t}=-\frac{\partial J}{\partial x}\label{conservation}
\end{equation}
where $J(x,t)$ is the probability current
\begin{equation}
J(x,t)=-\frac{i\hbar}{2m}\bigg(\Psi^*\frac{\partial\Psi}{\partial x}-\frac{\partial\Psi^*}{\partial x}\Psi\bigg)\in\mathbb{R}\label{probcurrent}
\end{equation}
\end{prop}
\begin{proof}
From time-dependent Schr\"{o}dinger Equation Eqn.~\ref{Schrodinger},
$$\frac{\partial P}{\partial t}=\Psi^*\frac{\partial\Psi}{\partial t}+\frac{\partial\Psi^*}{\partial t}\Psi=\Psi^*\frac{i\hbar}{2m}\Psi''-\frac{i\hbar}{2m}\Psi''^*\Psi=-\frac{\partial J}{\partial x}$$
where we assumed $V\in\mathbb{R}$.
\end{proof}
\begin{cor}
The rate of change of probability is the net flux of the probability current.
\begin{equation}
\frac{d}{dt}\int_a^b|\Psi(x,t)|^2dx=\int_a^b-\frac{\partial J}{\partial x}(x,t)dx=J(a,t)-J(b,t)\label{conservation2}
\end{equation}
\end{cor}
\begin{remarks}
For a stationary state, there is no net flux of probability current, follows from Eqn.~\ref{conservation2}.
\end{remarks}
\begin{cor}
Normalized states stay normalized.
\end{cor}
\begin{proof}
Consider a normalizable state with $\Psi,\Psi',J\rightarrow 0$ as $x\rightarrow\pm\infty$ for fixed $t$. Take $a\rightarrow-\infty$, $b\rightarrow+\infty$, then Eqn.~\ref{conservation2} gives $\frac{d}{dt}\int_{-\infty}^\infty|\Psi(x,t)|^2dx=0$.
\end{proof}
There are two types of stationary states which depend on the sign of $E$. 
\begin{itemize}
    \item bound states with $E<0$: If $\psi$ is normalizable, there are two b.c.s for $\psi$:
    $$
\psi(x)=
\left\{
        \begin{array}{ll}
      Ae^{\kappa x} & x\rightarrow-\infty \\
	B e^{-\kappa x}& x\rightarrow+\infty\\
        \end{array}
    \right.
$$
but we only have one possible unknown, so the system is overdetermined. We would typically find several bound states (in particular, the lowest energy bound state is the ground state) or none.
    \item scattering states with $E>0$: 
    $$\psi(x)=
\left\{
        \begin{array}{ll}
      Ie^{ikx}+Re^{-ikx} & x\rightarrow-\infty \\
	Te^{ikx}& x\rightarrow+\infty\\
        \end{array}
    \right.
$$
This is no longer overdetermined since there are more constants now.
\end{itemize} 
\begin{remarks}\leavevmode
\begin{enumerate}
    \item We want to solve $H\psi=-\frac{\hbar^2}{2m}\psi''+V(x)\psi=E\psi$ for normalizable $\psi$. In particular, we can analytically solve problems where $V(x)=U$ is a constant or at least piecewise flat potentials. If $U>E$, the Schr\"{o}dinger's equation Eqn.~\ref{Schrodinger} becomes $\psi''-\kappa^2\psi=0$ where $\kappa=\frac{\sqrt{2m(U-E)}}{\hbar}$ such that without loss of generality, $\kappa>0$. Then we have 
    $$\psi=Ae^{\kappa x}+Be^{-\kappa x}$$
    If $U<E$, the Schr\"{o}dinger's equation Eqn.~\ref{Schrodinger} becomes
    $\psi''+k^2\psi=0$ where $k=\frac{\sqrt{2m(E-U)}}{\hbar}$. Then we have
    $$\psi=Ae^{ikx}+Be^{-ikx}$$
    \item The finite discontinuity in $V$ is allowed as long as $\psi$, $\psi'$ are continuous while $\psi''$ is discontinuous.
\end{enumerate}
\end{remarks}
\subsection{1D Scattering}
Suppose we send a wavepacket towards a barrier, then the probabilities of reflection and transmission are $P_{\text{ref}}$ and $P_{\text{tr}}$ respectively. 
\begin{defi}[Beam]
We allow $\Psi(x,t)$ to represent beams of infinitely many particles with $|\Psi(x,t)|^2$ being the density of number of particles (per unit length) at $x,t$.
\end{defi}
Here, we constantly send in particles and the wavefunctions are called steady states. $\Psi(x,t)$ is bounded but no longer normalizable. 
\begin{defi}[Particle flux]
If $\Psi$ is the particle density instead of probability distribution, then $J$ now represents the flux of particles at $x,t$.
\end{defi}
Consider momentum eigenstates of the time-independent Schr\"{o}dinger's equation (Eqn.~\ref{Schrodinger}) $\Psi(x)=Ce^{ikx}$ with momentum $p=\hbar k$, particle density $|\Psi(x)|^2=|C|^2$ and current (Eqn.~\ref{probcurrent}) $J=\frac{\hbar k}{m}|C|^2$. In 1D scattering problems, the probabilities for transmission and reflection are
\begin{equation}
P_{\text{tr}}=\frac{|J_{\text{tr}}|}{|J_{\text{inc}}|},\quad P_{\text{ref}}=\frac{|J_{\text{ref}}|}{|J_{\text{inc}}|}\label{probtransref}
\end{equation}
\begin{eg}[Potential barrier]
   $$
V(x)=
\left\{
        \begin{array}{ll}
      U & 0<x<a\\
	0& x\leq 0,x\geq a\\
        \end{array}
    \right.
$$
Define $E$ and $U$ the same and consider $0<E<U$. We guess a wavefunction of the form
       $$
\psi(x)=
\left\{
        \begin{array}{ll}
      Ie^{ikx}+Re^{-ikx} & x<0\\
 Te^{ikx}& x>0\\
 Ae^{\kappa x}+Be^{-\kappa x}&0<x<a
        \end{array}
    \right.
$$
Invoke continuity condition of $\psi$ and $\psi'$ at $x=0$ and $x=a$ gives
$$I+R=A+B,\quad ik(I-R)=k(A-B)$$
$$Ae^{\kappa a}+Be^{-\kappa a}=Te^{ika},\quad \kappa (Ae^{\kappa a}-Be^{-\kappa a})=ikTe^{ika}$$
Solve them to get 
$$I+\frac{\kappa\mp ik}{\kappa\pm ik}R=Te^{ika}e^{\mp\kappa a}\implies T=Ie^{-ika}\frac{1}{\cosh\kappa a-i\frac{k^2-\kappa^2}{2k\kappa}\sinh ka}$$
The currents would be 
       $$
J(x)=
\left\{
        \begin{array}{ll}
      J_{\text{inc}}+J_{\text{ref}}=\frac{\hbar k}{m}(|I|^2-|R|^2)& x<0\\
J_{\text{tr}}=|T|^2\frac{\hbar k}{m}& x>0\\
        \end{array}
    \right.
$$
and thus Eqn~.\ref{probtransref} gives
\begin{equation}
P_{\text{tr}}=\frac{|J_{\text{tr}}|}{|J_{\text{inc}}|}=\frac{|T|^2}{|I|^2}=\frac{1}{1+\frac{U^2\sinh^2\kappa a}{4E(U-E)}}\label{tunnelling}
\end{equation}
This is the phenomenon of quantum tunnelling where there is non-zero probability of passing through the barrier even though it classically does not have enough energy. For $\kappa a>>1$, the probability decays as $e^{-2\kappa a}$.
\end{eg}
\newpage
\section{Harmonic Oscillator}\label{sec:HarmonicOscillator}
\subsection{Series solution}
\begin{prop}
The solutions of the quantum harmonic oscillator and their energy eigenvalues are
$$\psi_n(x)=h_n(\sqrt{m\omega/\hbar}x)\exp(-m\omega x^2/2\hbar),\quad E_n=\hbar\omega(n+0.5),\quad n\in\mathbb{Z}^+\cup\{0\}$$
where $h_n(y)$ is a Hermite polynomial of degree $n$, with the property $h_n(-y)=(-1)^nh_n(y)$.
\end{prop}
\begin{proof}
The Schr\"{o}dinger's equation for the one-dimensional harmonic oscillator is
\begin{equation}
H\psi=-\frac{\hbar^2}{2m}\psi''+\frac{1}{2}m\omega^2x^2\psi=E\psi\implies -\frac{d^2\psi}{dy^2}+y^2\psi=\mathcal{E}\psi,\quad y:=\sqrt{\frac{m\omega}{\hbar}}x,~\mathcal{E}=\frac{2E}{\hbar\omega}\label{Schrodinger_HO}
\end{equation}
For large $y,y^2>>\mathcal{E}$, try $\psi\propto e^{-y^2/2}$ so $\psi''$ can offset the large $y^2\psi$. W.l.o.g, set $\psi=f(y)e^{-y^2/2}$, then $f(y)$ satisfies the Hermite's equation
$$\frac{d^2f}{dy^2}-2y\frac{df}{dy}+\mathcal{E}-1=0$$
Try the series solution $f(y)=\sum_{r\geq0}a_ry^r$, then we obtain the recurrence relation:
$$a_{r+2}=\frac{2r+1-\mathcal{E}}{(r+2)(r+1)}a_r,\quad r\geq0$$
Choose $a_0$ and $a_1$ independently to get two linearly independent solutions. What is the behaviour of $f(y)$ when $y$ is large? If the coefficients do not vanish, $\frac{a_p}{a_{p-2}}\sim\frac{1}{p}$. This matches the coefficients of $y^\alpha e^{y^2}$ for some $\alpha$ and this gives $\psi\sim e^{y^2/2}$ which is not normalizable. Thus to ensure normalizability, we require the series $f$ to terminate. For this to do so, $\mathcal{E}=2n+1$ for some $n$. For each $n$, only one of the two independent solutions is normalizable. So for each $\mathcal{E}$, we get exactly one solution, so $f(y)=h_n(y)$ is a polynomial of degree $n$ with the property $h_n(-y)=(-1)^nh_n(y)$. These are the Hermite polynomials. The normalization fixes $a_0$ and $a_1$.
\end{proof}
\subsection{Raising and lowering operators}
Inspired by classical case, we take the Hamiltonian of the harmonic oscillator to be
$$\hat{H}=\frac{1}{2m}\hat{p}^2+\frac{1}{2}m\omega^2\hat{x}^2,\quad \omega\in\mathbb{R},~m\in\mathbb{R}$$
Here, we temporarily reintroduce the operator notation. To analyze this, begin by introducing the dimensionless operators
\begin{defi}[Ladder operators]
\begin{equation}
\hat{a}:=\frac{1}{\sqrt{2m\hbar\omega}}(m\omega\hat{x}+i\hat{p}),\quad\hat{a}^\dag:=\frac{1}{\sqrt{2m\hbar\omega}}(m\omega\hat{x}-i\hat{p})\label{ladderoperator}
\end{equation}
which are called lowering and raising operators respectively. 
\end{defi}
\begin{defi}[Number operator]
$$\hat{n}:=\hat{a}^\dag\hat{a}=\frac{\hat{H}}{\hbar\omega}-0.5\implies\hat{H}=\hbar\omega(\hat{n}+0.5)$$
\end{defi}
\begin{remarks}
Observe that $\hat{n}$ is Hermitian. The raising and lowering operators commute with themselves but 
$$[\hat{a},\hat{a}^\dag]=\frac{1}{2m\hbar\omega}(m^2\omega^2[\hat{x},\hat{x}]+[\hat{p},\hat{p}]-im\omega[\hat{x},\hat{p}]+im\omega[\hat{p},\hat{x}])=1$$
where we used $[\hat{x},\hat{p}]=i\hbar$. Similarly, $[\hat{a}^\dag,\hat{a}]=-1$. We will also need $$[\hat{n},\hat{a}]=[\hat{a}^\dag\hat{a},\hat{a}^\dag]=\hat{a}^\dag[\hat{a},\hat{a}^\dag]+[\hat{a}^\dag,\hat{a}^\dag]\hat{a}=\hat{a}^\dag$$
Similarly, $[\hat{n},\hat{a}]=-\hat{a}$. 
\end{remarks}
\begin{prop}
By representing the excited states as eigenstates of the number operator, the $(n+1)$th excited state is related to the ground state via
\begin{equation}
|n+1\rangle=\frac{(\hat{a}^\dag)^n}{\sqrt{(n+1)!}}|0\rangle\label{excitedstate}
\end{equation}
\end{prop}
\begin{proof}
Let the eigenstates be such that $|n\rangle$ with $\hat{n}|n\rangle=n|n\rangle$ and $\langle n|n\rangle=1$. Then, consider $\hat{a}^\dag|n\rangle$,
$$\hat{n}(\hat{a}^\dag|n\rangle)=([\hat{n},\hat{a}^\dag]+\hat{a}^\dag\hat{n})|n\rangle=(\hat{a}^\dag+\hat{a}^\dag\hat{n})|n\rangle=(n+1)|n+1\rangle$$
Similarly, $\hat{n}(\hat{a}|n\rangle)=n|n-1\rangle$. If $\hat{a}^\dag|n\rangle\neq0$, then $\hat{a}^\dag|n\rangle$ is also an eigenstate of $\hat{n}$ with eigenvalue $n+1$. If $\hat{a}|n\rangle\neq0$, then $\hat{a}|n\rangle$ is an eigenstate of $\hat{n}$ with eigenvalue $n$. Consequently, provided we can find some $|n\rangle$, we will construct a possibly infinite set of eigenstates of energies $E_k=(n+k+0.5)\hbar\omega$ for some $k\in\mathbb{Z}$ by acting repeatedly with $\hat{a}^\dag$ or $\hat{a}$. We can't yet conclude the energies are quantized, because we only know $n\in\mathbb{R}$. To go further, we investigate the norm
$$n=n\langle n|n\rangle=\langle n|\hat{n}|n\rangle=\langle n|\hat{a}^\dag\hat{a}|n\rangle=||\hat{a}|n\rangle||^2\geq0$$
With equality iff $\hat{a}|n\rangle=0$, i.e. $\hat{a}$ annihilates the state. Thus, all eigenvalues of $\hat{n}$ are non-negative, for any eigenstate $n\rangle\in\mathcal{H}$. We have just seen that if $\hat{a}|n\rangle\neq0$, it is an eigenstate of $\hat{n}$ with eigenvalue $n-1$. This lowering process must terminate. This will be the case iff $n\in\mathbb{N}_0$. Consequently, we have a ground state $|0\rangle$ defined by $\hat{a}|0\rangle=0$ and excited states $\sim(\hat{a}^\dag)^n|0\rangle$. The spectrum of $H$ is $E_n=(n+0.5)\hbar\omega$, $n\in\mathbb{N}_0$. We can recover the position space wavefunction of ground state as follow.
$$0=\langle x|\hat{a}|0\rangle=\frac{1}{\sqrt{2m\hbar\omega}}\langle x|m\omega\hat{x}+i\hat{p}|0\rangle=\frac{1}{\sqrt{2m\hbar\omega}}\bigg(m\omega x+\hbar\frac{d}{dx}\bigg)\psi_0(x)$$
where $\langle x|0\rangle=\psi_0(x)$. This is a first order DE for $\psi_0(x)$ such that $\psi_0(x)=Ce^{-m\omega x^2/2\hbar}$. The $x$-space wavefunctions of excited states are likewise obtained from
$$\langle x|\hat{a}^\dag|0\rangle=\frac{1}{\sqrt{2m\hbar\omega}}\langle x|m\omega\hat{x}-i\hat{p}|0\rangle=\frac{C}{\sqrt{2m\hbar\omega}}\bigg(m\omega x-\hbar\frac{d}{dx}\bigg)e^{-m\omega x^2/(2\hbar)}$$
Finally, although $\hat{a}^\dag|n\rangle$ is directly proportional to $|n+1\rangle$. It may not be correctly normalized. Indeed, we have
$$||\hat{a}^\dag|n\rangle||^2=\langle n|\hat{a}\hat{a}^\dag|n\rangle=\langle n|\hat{n}+[\hat{a},\hat{a}^\dag]|n\rangle=n+1$$
where $\langle n|n\rangle=1$. Likewise, $\hat{a}|n\rangle=\sqrt{n}|n-1\rangle$, and in particular, $\hat{a}|0\rangle=0$. Result follows.
\end{proof}
\subsection{Dynamics of oscillator}
\begin{prop}
Like the classical harmonic oscillator, the position expectation of a quantum harmonic oscillator exhibits sinusoidal behaviour.
\end{prop}
\begin{proof}
The motion of our oscillator is governed by the time-dependent Schr\"{o}dinger's Equation Eqn.~\ref{Schrodinger}. The time-dependent state is $$|\psi,t=0\rangle=\sum_{n\in\mathbb{N}_0}c_n|n\rangle\implies|\psi,t\rangle=\sum_{n\in\mathbb{N}_0}e^{-iE_nt/\hbar}c_n|n\rangle=e^{-i0.5\omega t}\sum_{n\in\mathbb{N}_0}c_ne^{-in\omega t}|n\rangle$$
Note that for any harmonic oscillator eigenstate, $\langle n|\hat{x}|n\rangle=0$ $\forall t$. More generally, for generic $|\psi,t\rangle$, we have
$$\langle\psi,t|\hat{x}|\psi,t\rangle=\sum_{n,m}\overline{c}_mc_ne^{-i(E_n-E_m)t/\hbar}\langle m|\hat{x}|n\rangle$$
From Eqn.~\ref{ladderoperator}, we can write $\hat{x}=\sqrt{\frac{\hbar}{2m\omega}}(\hat{a}+\hat{a}^\dag)$, we see
\begin{equation}
\langle m|\hat{x}|n\rangle=\sqrt{\frac{\hbar}{2m\omega}}(\sqrt{n}\langle m|n-1\rangle+\sqrt{n+1}\langle m|n+1\rangle)=\sqrt{\frac{\hbar}{2m\omega}}(\sqrt{n}\delta_{m,n-1}+\sqrt{n+1}\delta_{m,n+1})\label{HOposition}
\end{equation}
Thus, the expected position is oscillatory
$$\langle\psi,t|\hat{x}|\psi,t\rangle=\sqrt{\frac{\hbar}{2m\omega}}\sum_{n=0}^\infty(\overline{c}_{n-1}c_n\sqrt{n}e^{-i\omega t}+\overline{c}_{n+1}c_n\sqrt{n+1}e^{i\omega t})=\sum_nx_n\cos(\omega t+\phi_n)$$
where $x_ne^{i\phi_n}=\sqrt{\frac{2n\hbar}{m\omega}}\overline{c}_nc_m$, where $x_n\geq0$.
\end{proof}
\newpage
\section{Two level systems}\label{sec:Two-LevelSystems}
Two level systems has two basis states, with the state space being a two-dimensional complex vector space.
\begin{defi}[Qubit]
A qubit is a model that may be used to described the two-level system, and could either be in the up or down state.
\end{defi}
\begin{defi}[Pauli Matrices]
The three Pauli matrices are
\begin{equation}
\sigma_x=\begin{bmatrix}0&1\\1&0\\\end{bmatrix},\quad \sigma_y=\begin{bmatrix}0&-i\\i&0\\\end{bmatrix},\quad\sigma_z=\begin{bmatrix}1&0\\0&-1\\\end{bmatrix}\label{pauli}
\end{equation}
\end{defi}
\begin{thm}
The Pauli matrices, together with the identity matrix form a basis of $\mathcal{L}(\mathbb{C}^2)$ such that every Hermitian operator can be written as a linear combination of $\{\sigma_x,\sigma_y,\sigma_z,\text{Id}\}$.
\end{thm}
\begin{proof}
We first verify that the Pauli matrices are Hermitian and unitary (exercise). This implies that the eigenvalues can only be $\pm1$. However, for the identity matrix, there is only one eigenvalue $+1$. Interestingly, $\sigma_k^2=\text{Id}$, where $k=x,y,z$. We can then show the 4 matrices are linearly independent and that every Hermitian operator can be written $H=c_x\sigma_x+c_y\sigma_y+c_z\sigma_z+c_0\text{Id}$, where the coefficients are real.
\end{proof}
\begin{prop}
The Pauli matrices anti-commute. Moreover, their corresponding commutation relations are
\begin{equation}
[\sigma_j,\sigma_k]=2i\varepsilon_{jkl}\sigma_l\label{pauli2}
\end{equation}
\end{prop}
\begin{proof}
One can verify $\sigma_1\sigma_2=-\sigma_2\sigma_1=i\sigma_3$. We can verify the commutation relations as well.
\end{proof}
\begin{prop}
The eigenstates of a Pauli matrices are $\sigma_z|\pm\hat{z}\rangle=\pm|\pm\hat{z}\rangle$, $\sigma_x|\pm\hat{x}\rangle=\pm|\pm\hat{x}\rangle$ and $\sigma_y|\pm\hat{y}\rangle=\pm|\pm\hat{y}\rangle$. Moreover, 
$$|\pm\hat{x}\rangle=\frac{1}{\sqrt{2}}(|+\hat{z}\rangle\pm|-\hat{z}\rangle),\quad|\pm\hat{y}\rangle=\frac{1}{\sqrt{2}}(|+\hat{z}\rangle\pm i|-\hat{z}\rangle)$$
\end{prop}
\begin{thm}
A generic state of a two-level system, referred to as a Bloch vector in the unit sphere $\mathcal{S}_2$ in $\mathbb{R}^3$, reads
\begin{equation}
|+\hat{n}\rangle=\cos0.5\theta |+\hat{z}\rangle+e^{i\phi}\sin(0.5\theta)|-\hat{z}\rangle\label{genericBloch}
\end{equation}
such that the projector is
\begin{equation}
P_{\hat{n}}=|+\hat{n}\rangle\langle+\hat{n}|=\begin{bmatrix}\cos^2(0.5\theta)&e^{-i\phi}\cos(0.5\theta)\sin(0.5\theta)\\e^{i\phi}\cos(0.5\theta)\sin(0.5\theta)&\sin^2(0.5\theta)\\\end{bmatrix}=\frac{1}{2}(\Id+\hat{n}\cdot\sigma)\label{genericBloch2}
\end{equation}
\end{thm}
\begin{proof}
A generic state of a two-level system reads $\alpha|+\hat{z}\rangle+\beta|-\hat{z}\rangle$ with $|\alpha|^2+|\beta|^2=1$. Therefore, up to a global phase, we can parametrize the general state in terms of $\phi$ and $\theta$ which are two important angles in the spherical coordinate system. Expanding out to find the projector, as desired. We can also express it in terms of 
$$|+\hat{n}\rangle\langle+\hat{n}|=0.5(\Id+\sin\theta\cos\phi\sigma_x+\sin\theta\sin\phi\sigma_y+\cos\theta\sigma_z)=0.5\Id+0.5\begin{bmatrix}\cos\theta&e^{-i\phi}\sin\theta\\e^{i\phi}\sin\theta&-\cos\theta\\\end{bmatrix}=0.5(\Id+\hat{n}\cdot\sigma)$$
Essentially, $\hat{n}\cdot\sigma=|+\hat{n}\rangle\langle+\hat{n}|-|-\hat{n}\rangle\langle-\hat{n}|$, which we denote as $\sigma_{\hat{n}}$ because $\sigma_{\hat{n}}|\pm\hat{n}\rangle=\pm\hat{n}\rangle$.
\end{proof}
\begin{cor}
The vector orthogonal to $|+\hat{n}\rangle$ is $e^{i\phi}|-\hat{n}\rangle$.
\end{cor}
\begin{proof}
Clearly, the vector orthogonal to $\alpha|+\hat{z}\rangle+\beta|-\hat{z}\rangle$ is $\beta^*|+\hat{z}\rangle-\alpha^*|-\hat{z}\rangle$, up to a global phase. 
$$e^{i\phi}\sin(0.5\theta)|+\hat{z}\rangle -\cos(0.5\theta)|+\hat{z}\rangle=e^{i\phi}(\cos(0.5(\pi-\theta))|+\hat{z}\rangle+e^{i(\pi+\phi)}\sin(0.5(\pi-\theta))|+\hat{z}\rangle)=e^{i\phi}|-\hat{n}\rangle$$
This is the vector orthogonal to $|+\hat{n}\rangle$ (Eqn.~\ref{genericBloch}).
\end{proof}
\begin{defi}[Projective Measurements]
Projective measurements on a two-level system are labelled by a basis $\{|+\hat{m}\rangle,|-\hat{m}\rangle\}$ with $\hat{m}\in\mathcal{S}_2$. This is what it means to say that one measures the system in the direction $\hat{m}$, or that one measures the observable $\sigma_{\hat{m}}$.
\end{defi}
\begin{thm}
The most general observable whose eigenbasis is $\{|+\hat{m}\rangle,|-\hat{m}\rangle\}$ is
\begin{equation}
\hat{A}=0.5(\lambda_++\lambda_-)\Id+0.5(\lambda_+-\lambda_-)\sigma_{\hat{m}}\label{general2level}
\end{equation}
such that the statistics of the measurement of $\sigma_{\hat{m}}$ on the state $|+\hat{n}\rangle$: 
\begin{equation}
P(\pm\hat{m}|+\hat{n})=0.5(1\pm m\cdot n),\quad \langle+\hat{n}|\sigma_{\hat{m}}|+\hat{n}\rangle=m\cdot n\label{general2level2}
\end{equation}
\end{thm}
\begin{proof}
Choosing any direction will not lose generality of the problem, but greatly simplifies calculations. We could either find the square of the scalar product using Eqn.~\ref{genericBloch}
$$\langle+z|+\hat{n}\rangle=\cos(0.5\theta)\implies|\langle+z|+\hat{n}\rangle|^2=0.5(1+\cos\theta)=0.5(1+\hat{z}\cdot\hat{n})$$
or find the average value of the projector (Eqn.~\ref{genericBloch2})
$$\langle+\hat{z}|P_{\hat{n}}|+\hat{z}\rangle=\cos^2(0.5\theta)=0.5(1+\hat{z}\cdot\hat{n})$$
or find the trace of the product of the projectors
$$\Tr(P_{\hat{n}}P_{\hat{z}})=\begin{bmatrix}\cos^2(0.5\theta)&0\\\cos(0.5\theta)\sin(0.5\theta)e^{i\phi}&0\\\end{bmatrix}=\cos^2(0.5\theta)=0.5(1+\hat{z}\cdot\hat{n})$$
\end{proof}
\begin{eg}[Dynamics of Time-Independent Hamiltonian]
A generic two-level Hamiltonian can be written as $H=E_0\Id+E\sigma_{\hat{m}}$. If $H$ is time-independent, then we straightforwadly diagonalize
$$H=(E_0+E)|+\hat{m}\rangle\langle+\hat{m}|+(E_0-E)|-\hat{m}\rangle\langle-\hat{m}|$$
Write the initial state $|\psi(0)\rangle$ as a superposition of the eigenvectors, then multiply by the suitable phases:
$$|\psi(0)\rangle=c_+|+\hat{m}\rangle+c_-|-\hat{m}\rangle\implies|\psi(t)\rangle=e^{-i(E_0+E)t/\hbar}(c_+|+\hat{m}\rangle+c_-e^{+2iEt/\hbar}|-\hat{m}\rangle)$$
The evolution generated by the time-independent Hamiltonian is a precession around the direction $\hat{m}$ with angular frequency $\omega=2E/\hbar$. The energy gap $2E$ is the difference between the two eigenvalues.
\end{eg}
\begin{eg}[Dynamics of Time-Dependent Hamiltonian]
Let the static Hamiltonian be $H_0=0.5\hbar\omega_0\sigma_z$ and driving term be $H_1(t)=0.5\hbar\omega_1(\sigma_x\cos(\omega t)+\sigma_y\sin(\omega t))$. For a simple mathematical trick, we let $|\psi\rangle=e^{-i\omega t\sigma_z/2}|\psi'\rangle$. Then, solving time-dependent Schr\"{o}dinger Equation (Eqn.~\ref{Schrodinger}),
$$i\hbar\frac{d}{dt}|\psi\rangle=i\hbar(-i0.5\omega\sigma_ze^{-i\omega t\sigma_z/2}|\psi'\rangle+e^{-0.5i\omega t\sigma_z}\frac{d}{dt}|\psi'\rangle)=He^{-i\omega t\sigma_z/2}|\psi'\rangle$$
but $e^{i0.5\omega t\sigma_z}\sigma_xe^{-0.5i\omega t\sigma_z}=\sigma_x\cos(\omega t)-\sigma_y\sin(\omega t)$ and $e^{i0.5\omega t\sigma_z}\sigma_ye^{-i0.5\omega t\sigma_z}=\sigma_y\cos(\omega t)+\sigma_x\sin(\omega t)$. Our trick has led us to a time-independent Hamiltonian from a time-dependent one $H_0+H_1(t)$.
$$i\hbar\frac{d}{dt}|\psi'\rangle=\bigg(\frac{1}{2}\hbar(\omega_0-\omega)\sigma_z+\frac{1}{2}\hbar\omega_1\sigma_x\bigg)|\psi'\rangle$$
The eigenvalues are $E_{\pm}'=\pm\frac{\hbar}{2}\Omega$ with $\Omega=\sqrt{(\omega_0-\omega)^2+\omega_1^2}$. The corresponding eigenvectors are
$$|\phi_{\pm}'\rangle=\sqrt{0.5(1\pm\cos\theta)}|+\hat{z}\rangle\pm\sqrt{0.5(1\mp\cos\theta)}|-\hat{z}\rangle,\quad\cos\theta=\frac{\omega_0-\omega}{\Omega}$$
At $t=0$, we prepared the state $|+\hat{z}\rangle$, and we want to drive a transition to $|-\hat{z}\rangle$. If the Hamiltonian were $H=H_0$, this state is stationary, so no evolution happens. If the Hamiltonian were $H=H_{1x}=0.5\hbar\omega_1\sigma_x$, the Bloch vector would rotate around the $\hat{x}$ axis with an angular velocity $\omega_1$, and we would achieve the rotation in the time $\frac{\pi}{\omega_1}$.\\[5pt]
But $H_0$ is present, otherwise the two states $|+\hat{z}\rangle$ and $|-\hat{z}\rangle$ would not be two distinct energy levels, and usually $\omega_0>>\omega_1$. So, what happens when $H=H_0+H_{1x}$? At the start, the $H_{1x}$ contribution will move the state away from $|+\hat{z}\rangle$. As soon as this happens, the state is no longer stationary: because of $H_0$, the Bloch vector starts precessing around the $\hat{z}$ axis with angular velocity $\omega_0$. \\[5pt]
Suppose now that, instead of $H_{1x}$, we have $H_1(t)$ with $\omega=\omega_0$: the rotation due to $H_1(t)$ is about an axis that rotates at the same frequency at which the Bloch vector precesses due to $H_0$. In its own frame, the Bloch vector always see the same rotation axis (rotating frame). Write the initial state $|+\hat{z}\rangle$ in terms of the eigenvectors $|\phi_\pm'\rangle$: $$|\psi(t=0)\rangle=|\psi'(t=0)\rangle=|+\hat{z}\rangle=\cos(0.5\theta)|\phi_+'\rangle+\sin(0.5\theta)|\phi_-'\rangle$$
In the rotating frame, we have
\begin{eqnarray}
|\psi'(t)\rangle&=&\cos(0.5\theta)e^{-i0.5\Omega t}|\phi_+'\rangle+\sin(0.5\theta)e^{i0.5\Omega t}|\phi_-'\rangle\nonumber\\&=&(\cos(0.5\Omega t)-i\cos\theta\sin(0.5\Omega t))|+\hat{z}\rangle-i\sin\theta\sin(0.5\Omega t)|-\hat{z}\rangle\nonumber
\end{eqnarray}
Back into the non-rotating frame, $|\psi\rangle=e^{-i\omega t\sigma_z/2}|\psi'\rangle$ we have
$$|\psi(t)\rangle=e^{-i0.5\omega t}[\cos(0.5\Omega t)-i\cos\theta\sin(0.5\Omega t)]|+\hat{z}\rangle-ie^{i0.5\omega t}\sin\theta\sin(0.5\Omega t)|-\hat{z}\rangle$$
The probability $P_{z\rightarrow-z}(t)=\sin^2\theta\sin^2(0.5\Omega t)=\sin^2(0.5\Omega t)\omega_1^2\Omega^{-2}$. This oscillation of the populations of the ground and excited state, due to a time-dependent drive, is called Rabi oscillation.
\end{eg}
\section{3D Quantum mechanics}\label{sec:3DQM}
We can extend our concepts discussed so far to three dimensions. The inner product and normalization will instead be a three-dimensional integral. The momentum operator and the canonical commutation relation will be
\begin{equation}
p=-i\hbar\nabla,\quad[x_i,p_j]=-i\hbar\delta_{ij}\label{3Dposmom}
\end{equation}
The Hamiltonian of a structureless particle (observables can be written in terms of position and momentum) 
$$H=-\frac{\hbar^2}{2m}\nabla^2+V(x)$$ 
The probability current $\mathbf{J}$ satisfies the continuity equation
\begin{equation}
\mathbf{J}=-\frac{i\hbar}{2m}(\Psi^*\boldsymbol{\nabla}\Psi-\Psi\boldsymbol{\nabla}\Psi^*),\quad \frac{\partial}{\partial t}|\Psi(x,t)|^2=-\boldsymbol{\nabla}\cdot\mathbf{J}\label{continuity}
\end{equation}
\begin{remarks}
For a separable Hamiltonian of the form
$$H=\sum_iH_i=-\frac{\hbar^2}{2m}\sum_i\frac{\partial^2}{\partial x_i^2}+\sum_iU_i(x_i)$$
We can look for solutions of the form $\psi=\prod_i\chi_i(x_i)$ with $E_i$ as the eigenvalue of $\chi_i$ where $\chi_i$ is the eigenvector of $H_i$.
\end{remarks}
\begin{eg}[2D Harmonic Oscillator]
The energy eigenvalue of a 2D harmonic oscillator is 
$$E=\hbar\omega(n_1+0.5+n_2+0.5)$$
and the eigenfunctions would be
$$\psi(x_1,x_2)=\psi_{n_1}(x_1)\psi_{n_2}(x_2)$$
  \begin{center}
    \begin{tabular}{|c|c|c|}
      \hline
      \textbf{State} & \text{Energy} & \text{Possible states}\\
      \hline
      Ground state & $E = \hbar \omega$ & $\psi = \psi_0(x_1)\psi_0(x_2)$\\
      1st excited state & $E = 2\hbar \omega$ & $\psi = \psi_1(x_1) \psi_0(x_2)$,~$\psi = \psi_0(x_1)\psi_1(x_2)$\\
     \hline
    \end{tabular}
  \end{center}
We see that the degeneracy of the first excited state is 2.
\end{eg}
\subsection{Angular momentum}
\begin{defi}[Orbital angular momentum]
The orbital angular momentum operator is
\begin{equation}
L=-i\hbar x\times\nabla\implies L_i=-i\hbar\varepsilon_{ijk}x_j\frac{\partial}{\partial x_k}\label{angmom}
\end{equation}
Easy to check that $L_i$ is Hermitian $\forall i$ and they form a set of observables. We can also define the total angular momentum
$$L^2=L_iL_i=L_x^2+L_y^2+L_z^2$$
which is also an observable.
\end{defi}
\begin{prop}\leavevmode
\begin{enumerate}
    \item $[L_i,L_j]=i\hbar\varepsilon_{ijk}L_k$, i.e. cannot completely know the angular momentum in all directions.
    \item $[L^2,L_i]=0$
    \item $[L_i,x_j]=i\hbar\varepsilon_{ijk}x_k$, $[L_i,p_j]=i\hbar\epsilon_{ijk}p_k$
\end{enumerate}
\end{prop}
\begin{proof}\leavevmode
  \begin{enumerate}
    \item We have individually
      \begin{align*}
        L_i L_j &= \varepsilon_{iar} x_a p_r \varepsilon_{jbs} x_b p_s\\
        &= \varepsilon_{iar} \varepsilon_{jbs} (x_a p_r x_b p_s)\\
        &= \varepsilon_{iar} \varepsilon_{jbs} (x_a (x_b p_r - [p_r, x_b]) p_s)\\
        &= \varepsilon_{iar} \varepsilon_{jbs} (x_a x_b p_r p_s - i\hbar \delta_{br} x_a p_s)
      \end{align*}
      as well as $L_j L_i = \varepsilon_{iar} \varepsilon_{jbs} (x_b x_a p_s p_r - i\hbar \delta_{as} x_b p_r)$.       Then the commutator is
      \begin{align*}
        L_i L_j - L_j L_i &= -i\hbar \varepsilon_{iar} \varepsilon_{jbs}(\delta_{br} x_a p_s - \delta_{as}x_b p_r)\\
        &= -i\hbar (\varepsilon_{iab} \varepsilon_{jbs} x_a p_s - \varepsilon_{iar} \varepsilon_{jba} x_bp_r)\\
        &= -i\hbar ((\delta_{is} \delta_{ja} - \delta_{ij} \delta_{as})x_a p_s - (\delta_{ib} \delta_{rj} - \delta_{ij}\delta_{rb}) x_b p_r)\\
        &= i\hbar (x_i p_j - x_j p_i)\\
        &= i\hbar \varepsilon_{ijk} L_k.
      \end{align*}
    \item This follows from 1 using the Leibnitz property: $[A, BC] = [A, B] C + B[A, C]$. This property can be proved by directly expanding both sides, and the proof is uninteresting. Using this, we get
$$[L_i, \mathbf{L}^2] = [L_i, L_j L_j]= [L_i, L_j] L_j + L_j [L_i, L_j]= i\hbar \varepsilon_{ijk} (L_k L_j + L_j L_k)= 0$$
      where we get $0$ since we are contracting the antisymmetric tensor $\varepsilon_{ijk}$ with the symmetric tensor $L_k L_j + L_j L_k$.
    \item We will use the Leibnitz property again, but this time with $[AB,C]$ on the LHS.      This follows immediately from the previous version since $[A, B] = -[B, A]$. Hence,
$$[L_i, x_j] = \varepsilon_{iab} [x_a p_b, x_j]  = \varepsilon_{iab} ([x_a, x_j] p_b + x_a [p_b, x_j])= \varepsilon_{iab} x_a (-i\hbar \delta_{bj})= i\hbar \varepsilon_{ija} x_a$$
Similarly, $[L_i,p_j]=i\hbar\varepsilon_{ijb} p_b$.
  \end{enumerate}
\end{proof}
\begin{prop}[Functional expressions of the angular momentum operators]
$$L_3=-i\hbar\frac{\partial}{\partial\phi},\quad L_\pm:=L_1\pm iL_2=\pm\hbar e^{\pm i\phi}\bigg(\frac{\partial}{\partial\theta}\pm i\cot\theta\frac{\partial}{\partial\phi}\bigg)$$
$$L^2=-\hbar^2\bigg(\frac{1}{\sin\theta}\frac{\partial}{\partial\theta}\sin\theta\frac{\partial}{\partial\theta}+\frac{1}{\sin^2\theta}\frac{\partial^2}{\partial\phi^2}\bigg)$$
\end{prop}
\begin{remarks}\leavevmode
\begin{enumerate}
    \item In fact, we can write the $L^2$ operator in terms of $\nabla^2$ in spherical coordinates, specifically
    \begin{equation}
    \nabla^2=\frac{1}{r^2}\frac{\partial^2}{\partial r^2}r-\frac{1}{\hbar^2r^2}L^2\label{LaplacianSpherical}
    \end{equation}
    \item From Proposition 1.4.1.2, we can choose $[L_3,L^2]=0$ without loss of generality. Such a choice is justified due to the simplicity of $L_3$'s expression. Thus, we can find a simultaneous eigenfunctions for both $L_3$ and $L^2$. Let this function be $Y_{\ell m}(\theta,\phi)$ such that
    $$L_3Y_{\ell m}(\theta,\phi)=\hbar mY_{\ell m}(\theta,\phi),\quad L^2Y_{\ell m}(\theta,\phi)=\hbar^2\ell(\ell+1)Y_{\ell m}(\theta,\phi)$$
    Here, $\ell\in\mathbb{Z}^+\cup\{0\}$ and $m\in\mathbb{Z}$ that must satisfy $-\ell\leq m\leq\ell$. We can then work out to show that
    $$Y_{\ell m}(\theta,\phi)\propto e^{im\phi}P_\ell ^m(\cos\theta)$$
    where $P_\ell^m(\cos\theta)$ to be the associated Legendre function. If $m=0$, $P_\ell$ is the Legendre polynomial. The parity of the eigenfunction depends on only $l$, i.e. $Y_{lm}(\pi-\theta,\phi+\pi)=(-1)^lY_{lm}(\theta,\phi)$.
    \item Geometrically, the angular momentum vector $\mathbf{L}$ traces out a cone in 3 dimensional angular momentum space where the length of the vector is $\hbar\sqrt{l(l+1)}$ and the half angle of the cone is $\theta=\cos^{-1}(m/\sqrt{l(l+1)})$.  $m$ is thus the projection of the angular momentum vector along the $z$ axis. As the angular momenta states are quantized, $\theta$ is quantized too and have a discrete set of $2l + 1$ possible values. As $l\rightarrow\infty$ (classical limit), $\theta\rightarrow 0$, this physically means we can measure the angular momentum's direction with great certainty.
\end{enumerate}
\end{remarks}
\begin{eg}[Spherically symmetric potential]
Consider a particle of mass $\mu$ in a spherically symmetric potential $V(r)$, then the Hamiltonian is
\begin{equation}
H=\frac{-\hbar^2}{2\mu}\nabla^2+V(r)=-\frac{\hbar^2}{2\mu}\frac{1}{r^2}\frac{\partial^2}{\partial r^2}r^2+\frac{1}{2\mu r^2}L^2+V(r)\label{sphericalsymmHam}
\end{equation}
We can show $[L_i,H]=0$ since $[L_i,p^2]=0=[L_i,V(r)]$. Trivially, it then follows that $[L^2,H]=0$. Hence, we can use the eigenvalues of $H$, $L^2$ and $L_3$ to label our solutions. They are called good quantum numbers. This joint eigenstate is $\psi(x)=R(r)Y_{\ell m}(\theta,\phi)$. We can then construct an ODE for $R(r)$:
$$-\frac{\hbar^2}{2\mu}\frac{1}{r^2}\frac{d^2}{dr^2}(rR)+\frac{\hbar^2}{2\mu r^2}\ell(\ell+1)R+VR=ER$$
The first two terms are kinetic energy - radial and angular parts respectively.
\end{eg}
\begin{remarks}
 We could have also worked with radial wavefunctions $\chi(r)=rR(r)$, then we will obtain the radial Schr\"{o}dinger's equation
$$-\frac{\hbar^2}{2\mu}\chi''+\frac{\hbar^2}{2\mu r^2}\ell(\ell+1)\chi+V\chi=E\chi$$
We want $\chi$ to obey the same boundary conditions as $R$. We want $R$ to be finite as $r\rightarrow 0$ so we can choose $\chi(r=0)=0$. We can also show that $\chi$ satisfy our desired normalization
$$1=\int|\psi(x)|^2d^3x=\int|R(r)|^2r^2dr\int|Y_{\ell m}(\theta,\phi)|^2\sin\theta d\theta d\phi\iff\int_0^\infty|\chi(r)|^2dr<\infty$$
\end{remarks}
\begin{eg}[3D Well]
Consider the spherically symmetric potential
   $$
V(r)=
\left\{
        \begin{array}{ll}
      -U & r<a\\
	0& r\geq a\\
        \end{array}
    \right.
$$
where $U>0$ is a constant. We look for bound state solutions with $-U<E<0$ and with total angular momentum quantum number $\ell$. The radial Schr\"{o}dinger's equation is
$$\chi''-\frac{\ell(\ell+1)}{r^2}\chi+k^2\chi=0$$
for $r<a$ such that $k$ satisfy $U+E=\frac{\hbar^2k^2}{2\mu}$, and is
$$\chi''-\frac{\ell(\ell+1)}{r^2}\chi-\kappa^2\chi=0$$
for $r\geq a$ such that $\kappa$ satisfy $E=-\frac{\hbar^2\kappa^2}{2\mu}$. Again invoke continuity of $\chi$ and $\chi'$ at the point of potential discontinuity $r=a$. Additional requirement is $\chi(0)=0$, as mentioned before. For $\ell=0$,
   $$
\chi(r)=
\left\{
        \begin{array}{ll}
      A\sin kr & r<a\\
	Be^{-\kappa r}& r> a\\
        \end{array}
    \right.
$$
Match values of $\chi(r)$ at $r=a$, to obtain $k$, $\kappa$ and hence $E$. For $\ell=1$,
   $$
\chi(r)=
\left\{
        \begin{array}{ll}
      A(\cos kr-\frac{1}{k r}\sin kr) & r<a\\
	B(1+\frac{1}{\kappa r})e^{-\kappa r}& r> a\\
        \end{array}
    \right.
$$
The wavefunction will then be
$$\psi(r)=\frac{\chi(r)}{r}Y_{\ell m}(\theta,\phi)$$
where $m$ takes $m=0$, $\pm 1$. The solution for general $\ell$ involves spherical Bessel functions.
\end{eg}
\begin{eg}[Coulomb potential]
Consider specifically $V(r)=-\frac{e^2}{4\pi\epsilon_0r}$ due to a proton stationary ar $r=0$. The radial part of Schr\"{o}dinger's equation is
\begin{equation}
    R''+\frac{2}{r}R'-\frac{\ell(\ell+1)}{r^2}R+\frac{2\lambda}{r}R=\kappa^2R\tag{*}
\end{equation}
where $\lambda=\frac{m_ee^2}{4\pi\epsilon_0\hbar^2}$ and $E=-\frac{\hbar^2\kappa^2}{2m_e}$. We use the same trick as in the 1D harmonic oscillator. For large $r$, (*) gives $R''\sim\kappa^2R\implies R(r)\sim e^{-\kappa r}$ for large $r$. For small $r$, we only know $R$ must be finite. Multiply (*) by $r^2$ and discard $rR$ and $r^2R$ terms to get
$$r^2R''+2rR'-\ell(\ell+1)R\sim 0\implies R\sim r^\ell$$
We try a solution $R(r)=Cr^\ell e^{-\kappa r}$. The $r^{\ell-1}e^{-\kappa r}$ terms match iff
$$2\ell r^{\ell-1}(-\kappa e^{-\kappa r})+2r^{\ell-1}(-\kappa e^{-\kappa r})+2\lambda r^{\ell-1}e^{-\kappa r}=0\iff(\ell+1)\kappa=\lambda$$
for any $n=\ell+1\in\mathbb{Z}^+$. There are bound state energies
\begin{equation}
E_n=-\frac{\hbar^2}{2m_e}\frac{\lambda^2}{n^2}=-\frac{m_e}{2}\bigg(\frac{e^2}{4\pi\epsilon_0\hbar}\bigg)^2\frac{1}{n^2}\label{HatomEnergy}
\end{equation}
which are very similar to the energy levels of the Bohr model. Of course, this particular choice of $R(r)$ is only one solution. Look for $R(r)=e^{-\kappa r}f(r)$. Put in (*) to obtain
$$f''+\frac{2}{r}f'-\frac{\ell(\ell+1)}{r^2}f=2\bigg(\kappa f'+(\kappa-\lambda)\frac{f}{r}\bigg)$$
This is regular singular at $r=0$, so guess a Frobenius series expansion about $r=0$:
$$f(r)=\sum_{p=0}^\infty a_pr^{p+\sigma},~a_0\neq 0$$
We will then obtain
$$\sum_{p\geq0}[(p+\sigma)(p+\sigma-1)-\ell(\ell+1)]a_pr^{p+\sigma-2}=\sum_{p\geq0}2(\kappa(p+\sigma+1)-\lambda)a_pr^{p+\sigma-1}$$
The lowest term gives us the indicial equation
$$\sigma(\sigma+1)-\ell(\ell+1)=(\sigma-\ell)(\sigma+\ell+1)=0\implies\sigma=\ell,~-(\ell+1)$$
We reject the negative solution since this makes $f$ and hence $R$ singular at $r=0$. The desired recurrence relation is
$$a_p=\frac{2(\kappa(p+\ell)-\lambda)}{p(p+2\ell+1)}a_{p-1},\quad p\geq1$$
Unless the series terminates, we must have the asymptotic behaviour $\frac{a_p}{a_{p-1}}\sim\frac{2\kappa}{p}$ as $p\rightarrow\infty$, which matches the behaviour of $r^\alpha e^{2\kappa r}$ for some $\alpha$. So, $R(r)$ is normalizable only if the series terminates, hence the possible values of $\lambda$ are $\kappa n=\lambda$ for some $n\geq\ell+1$. We recover the same expression for $E$. Hence, for any given principle quantum number $n$, the possible angular momentum quantum numbers are $\ell=0,1,2,\dots,n-1$ and $m=0,\pm1,\pm2,\dots,\pm\ell$. The simultaneous eigenstates are then 
$$\psi_{n\ell m}(x)=R_{n\ell}(r)Y_{\ell m}(\theta,\phi)$$
with $R_{n\ell}(r)=r^\ell g_{n\ell}(r)e^{-\lambda r/n}$. We can identify $g_{n\ell}(r)$ with the associated Laguerre polynomials, unique up to a multiplicative constant. The shape of the probability distribution for any electron state depends on $r$, and $\theta,\phi$ mostly through $Y_{\ell,m}$. In particular, for $\ell=0$, the spherically symmetric solutions are
$$\psi_{n00}(r)=g_{n0}(r)e^{-\lambda r/n}$$
The degeneracy of each energy level $E_n$ is
$$\sum_{\ell=0}^{n-1}\sum_{m=-\ell}^\ell1=\sum_{\ell=0}^{n-1}2\ell+1=n^2$$
\end{eg}
\begin{remarks}\leavevmode
\begin{enumerate}
    \item This degeneracy reflects additional symmetries in the Coulomb potential, other than the obvious SO(3) rotation symmetry.
    \item It turns out that the particles (especially in the Hydrogen atom) are not structureless and possess an additional internal degree of freedom, called spin. This is a form of angular momentum which gives an additional degeneracy factor of 2. They may have half-integer angular momentum numbers. Spin is not due to the orbital motion of electrons since the orbital motion has integer values of $\ell$ for well-behaved functions. 
\end{enumerate}
\end{remarks}
\subsection{Angular momentum eigenstates}
\begin{notation}
Angular momentum has two types - orbital angular momentum $L$ and spin $S$. For generic types, we shall renotate angular momentum as $J$. Often, this is a vectorial sum $\mathbf{J}=\mathbf{L}+\mathbf{S}$.
\end{notation}
We saw that $[J_i,J_j]=i\hbar\epsilon_{ijk}J_k$ and $[J_i,J\cdot J]=0$. The commutation relation, together with the norm on $\mathcal{H}$, determine the possible eigenvalues of $J$. We can only find a complete set of simultaneous eigenstates for $J^2$ and any one component of $J$. Without loss of generality, choose $J_z$. 
\begin{defi}[Simultaneous eigenstates for angular momentum]
Let's assume the simultaneous state $\{|\beta,m\rangle\}$ (properly normalized, orthonormal and non-degenerate) obeys
\begin{equation}
J^2|\beta,m\rangle=\beta\hbar^2|\beta,m\rangle,\quad J_z|\beta,m\rangle=m\hbar|\beta,m\rangle\label{simeigen}
\end{equation}
\end{defi}
\begin{defi}[Raising and lowering operators]
We define $J_\pm=J_x\pm iJ_y$ such that $(J_\pm)^\dag=J_\mp$. 
\end{defi}
\begin{prop}
The states $J_\pm|\beta,m\rangle\neq0$ are still eigenstates of both $J^2$ and $J_z$, with the same eigenvalues $\beta\hbar^2$ for $J^2$, but different for $J_z$.
\end{prop}
\begin{proof}
Clearly, $[J_\pm,J^2]=0$ and we have
$$[J_z,J_\pm]=[J_z,J_x]\pm i[J_z,J_y]=i\hbar J_y\pm i(-i\hbar J_x)=\pm\hbar J_\pm$$
\begin{equation}
J^2(J_\pm|\beta,m\rangle)=J_\pm(J^2|\beta,m\rangle)=\beta\hbar^2(J_\pm|\beta,m\rangle),\quad J_zJ_\pm|\beta,m\rangle=([J_z,J_\pm]+J_\pm J_z)|\beta,m\rangle=(\pm1+m)\hbar(J_\pm|\beta,m\rangle)\label{simeigen2}
\end{equation}
So this is still a joint eigenstate with eigenvalues $\beta\hbar^2$ and $(m\pm1)\hbar$. 
\end{proof}
We will also seek the smallest possible Hilbert space $\mathcal{H}$ on which these $J$ can act on. 
\begin{prop}
$\exists$ maximum $m=j$ and minimum $j'$ such that $\beta=j(j+1)=j'(j'-1)\implies j'=-j$ (since $j'\leq j$).
\end{prop}
\begin{proof}
Assume our basis is non-degenerate, then $J_\pm|\beta,m\rangle=C_\pm|\beta,m\pm1\rangle$. Taking the norm $||C_\pm||^2=||J_\pm|\beta,m\rangle||^2$ to find $C_{\pm}$:
$$\langle\beta,m|(J_x\mp iJ_y)(J_x\pm iJ_y)|\beta,m\rangle=\langle\beta,m|J_x^2+J_y^2\pm i[J_x,J_y]|\beta,m\rangle=\langle\beta,m|J^2-J_z^2\pm i(i\hbar)J_z|\beta,m\rangle=\hbar^2(\beta-m^2\mp m)$$
Since each component of $J$ is Hermitian, we have $0\leq||J_i|\psi\rangle||^2=\langle\psi|J_i^2|\psi\rangle$, and hence $\langle\psi|J^2|\psi\rangle\geq\langle\psi|J_z^2|\psi\rangle$ $\forall|\psi\rangle\in\mathcal{H}$. In particular, we must have $\beta\hbar^2\geq m^2\hbar^2$. Hence, we can't keep changing $m\rightarrow m+1$ whilst keeping $\beta$ fixed. The only escape is that when $m=j$ for some $J_+|\beta,j\rangle=0$ which occurs iff $\beta=j(j+1)$. Hence, the action of $J_+$ on $|j,m\rangle$ is
\begin{equation}
J_+|j,m\rangle=\hbar\sqrt{j(j+1)-m(m+1)}|j,m+1\rangle\label{J+}
\end{equation}
Likewise, $\exists$ minimum value $m=j'$ such that $J_-|\beta,j'\rangle=0$ which implies $c_-|_{m=j'}=0$ which implies $\beta=j'(j'-1)$. Equating these, 
$$\beta=j(j+1)=j'(j'-1)\implies j'=-j$$
There is a unique root with $j'\leq j$. Hence, the action of $J_-$ on $|j,m\rangle$ is
\begin{equation}
J_-|j,m\rangle=\hbar\sqrt{j(j+1)-m(m-1)}|j,m-1\rangle\label{J-}
\end{equation}
\end{proof}
Henceforth, we will label our states by $|\beta=j(j+1),m\rangle\rightarrow|j,m\rangle$. 
\begin{prop}
There are $(2j+1)$ states of different $m$ of any given value of $j$.
\end{prop}
\begin{proof}
By applying $J$ repeatedly, we move from $|j,j\rangle\rightarrow|j,j-1\rangle\rightarrow\dots\rightarrow|j,-j-1\rangle\rightarrow|j,-j\rangle\rightarrow0$ and consequently $2j\in\mathbb{N}_0$. Since $(J_-)^{2j}|j,j\rangle$ is directly proportional to $|j,-j\rangle$. This means $j\in\{0,0.5,1,1.5,...\}$, and for any given choice of $j$, the possible values of $m$ are $m\in\{j,j-1,...,-j+1,-j\}$.
\end{proof}
The eigenvalues tells us how much angular momentum our system has in total ($j(j+1)\hbar^2$) and how much is aligned along the $z$-axis $(m\hbar)$. These raising and lowering operators only change the $J_z$ eigenvalue, not the $J^2$ one. They just realign a given system, placing more or less of its angular momentum along the z-axis.
\begin{cor}
There are no states in which the vector $\mathbf{J}$ has a well-defined direction.
\end{cor}
\begin{proof}
Since all three components do not commute with each other, then when $j>0$, we can never determine a common eigenstate for all three. If $j = 0$, every component of $\mathbf{J}$ yields zero, but a null vector in $\mathbb{R}^3$ has no direction.
\end{proof}
\begin{defi}[Highest weight state]
States with $m = j$ are known as highest weight states. They play a key role in the representation theory of any group $G$, because once we know the highest weight state the rest of the multiplet can be constructed by applying lowering operators Physically, a highest weight state is one in which the body’s angular momentum is most nearly aligned along the z-axis.
\end{defi}
\begin{remarks}\leavevmode
\begin{enumerate}
    \item However, even in the highest weight state $|j,j\rangle$, where as much angular momentum as possible lies along $\hat{z}$, we have
$$\frac{\langle j,j|J_x^2+J_y^2|j,j\rangle}{\langle j,j|J_z^2|j,j\rangle}=\frac{\langle j,j|J^2-J_z^2|j,j\rangle}{\langle j,j|J_z^2|j,j\rangle}=\frac{[j(j+1)-j^2]\hbar^2}{j^2\hbar^2}=\frac{1}{j}$$
so it is never possible to perfectly align a system's angular momentum with (any) given axis.
\item Classical (macroscopic) systems will have $j>>1$ so their angular momentum can be accurately aligned with a chosen axis, but when $j=0.5$, there is 2 times as much as in the $x$-$y$ plane, even in state $|0.5,0.5\rangle$. Also, since $J_x=0.5(J_++J_-)$, a state $|j,m\rangle$ with any definite amount $m\hbar$ of angular momentum along $\hat{z}$ is never $(j>0)$ an eigenstate of $J_x$. Likewise, $|j,m\rangle$ is never $(j>0)$ an eigenstate of $J_y$. Thus, we do not know where in the $x$-$y$ plane, this `surplus' total angular momentum is aligned.
\end{enumerate}
\end{remarks}
\subsection{Spin}
\begin{notation}
We temporarily restore the operator notation for clarity.
\end{notation}
\begin{defi}[Spin Operators]
The spin operators for spin-0.5 particles are
\begin{equation}
\hat{S}_x=0.5\hbar\sigma_x,\quad\hat{S}_y=0.5\hbar\sigma_y,\quad\hat{S}_z=0.5\hbar\sigma_z\label{spin}
\end{equation}
We define $\hat{S}_z$ and $\hat{S}^2=\hat{S}_x^2+\hat{S}_y^2+\hat{S}_z^2$ to be
\begin{equation}
\hat{S}^2|s,m_s\rangle=\hbar^2s(s+1)|s,m_s\rangle,\quad\hat{S}_z|s,m_s\rangle=\hbar m_s|s,m_s\rangle\label{spin2}
\end{equation}
We can also define $\hat{S}_\pm=\hat{S}_x\pm i\hat{S}_y$ as the spin ladder operators.
\end{defi}
\begin{prop}
The following commutation relations are true:
\begin{equation}
[\hat{S}_i,\hat{S}_j]=i\hbar\epsilon_{ijk}\hat{S}_k,\quad[\hat{S}^2,\hat{S}_z]=0\label{spin3}
\end{equation}
\end{prop}
\begin{proof}
Can be derived from Eqn.~\ref{pauli2}.
\end{proof}
\begin{prop}
\begin{equation}
e^{-i\theta\hat{n}\cdot\sigma/2}=\cos(\theta/2)\Id-i\sin(\theta/2)\hat{n}\cdot\sigma\label{spin4}
\end{equation}
\end{prop}
\begin{proof} Expanding as a power series of $0.5\theta\hat{n}\cdot\sigma$, we get our desired form.
\end{proof}
\begin{cor}
The spinor has $4\pi$ symmetry instead of $2\pi$.
\end{cor}
\begin{proof}
$e^{-i2\pi\hat{n}\cdot\sigma/2}=-\Id$. So it takes a rotation by $4\pi$ to recover the identity.
\end{proof}
\begin{prop}[Spin as a rotation]
$e^{-i\phi S_z/\hbar}\sigma_xe^{i\phi S_z/\hbar}$ is the same as rotating by an angle $\phi$ around the $z$ axis.
\end{prop}
\begin{proof}
We first show that
$$e^{i\phi(\hat{n}\cdot\sigma)}=\sum_{k=0}^\infty\frac{(-1)^k\phi^{2k}}{(2k)!}\Id+\sum_{k=0}^\infty\frac{(-1)^k\phi^{2k+1}}{(2k+1)!}(i\hat{n}\cdot\sigma)=\cos\alpha \Id+i\sin\alpha\hat{n}\cdot\sigma$$
where $(n\cdot\sigma)^{2k}=\Id$. Then, we have
\begin{eqnarray}
& &e^{-i\phi S_z/\hbar}\sigma_xe^{i\phi S_z/\hbar}\nonumber\\&=&\cos^2(0.5\phi)\sigma_x+i\sin(0.5\phi)\cos(0.5\phi)\sigma_x\sigma_z-i\sin(0.5\phi)\cos(0.5\phi)\sigma_z\sigma_x+\sin^2(0.5\phi)\sigma_z\sigma_x\nonumber\\&=&\cos\phi\sigma_x+i\sin(\phi)0.5[\sigma_x,\sigma_z]\nonumber\\&=&\cos\phi\sigma_x+\sin\phi\sigma_y\nonumber
\end{eqnarray}
\end{proof}
\begin{prop}[Uncertainty for Spin]
\begin{equation}
\langle\hat{S}_x^2\rangle=\langle\hat{S}_y^2\rangle=\frac{\hbar^2}{2}[s(s+1)-m_s^2]\label{uncertainty}
\end{equation}
\end{prop}
\begin{proof}
Since $\hat{S}_{\pm}=\hat{S}_x\pm i\hat{S}_y$ such that
$$\hat{S}_\pm|s,m_s\rangle=\hbar\sqrt{s(s+1)-m_s(m_s\pm1)}|s,m_s\pm 1\rangle$$
and so we can show
$$\langle\hat{S}_x^2\rangle=\langle\hat{S}_y^2\rangle=\frac{1}{2}(\langle\hat{S}^2\rangle-\langle\hat{S}_z\rangle)$$
where we have $\hat{S}^2|s,m_s\rangle=\hbar^2s(s+1)|s,m_s\rangle$ and $\hat{S}_z|s,m_s\rangle=\hbar m_s|s,m_s\rangle$.
\end{proof}
\subsection{Tensor Product of Hilbert Spaces}
We often have to deal with systems with more than one degree of freedom. In QM, we take the Hilbert space of the combined system to be the tensor product of the individual Hilbert spaces.
\begin{defi}[Tensor product]
Suppose $\{|e_a\rangle\}$ is a basis of $\mathcal{H}_1$ with $a=1,...,m$ and $\{|f_\alpha\rangle\}$ is a basis of $\mathcal{H}_2$ with $\alpha=1,...,n$. Then a basis of $\mathcal{H}_1\otimes\mathcal{H}_2$ is given by all pairs $|e_a\rangle\otimes |f_\alpha\rangle$ chosen from the bases. Thus $\dim(\mathcal{H}_1\otimes\mathcal{H}_2)=\dim\mathcal{H}_1\times\dim\mathcal{H}_2$. 
\end{defi}
\begin{remarks}
A general $|\Psi\rangle\in\mathcal{H}_1\otimes\mathcal{H}_2$ can be expanded as 
$$|\Psi\rangle=\sum_{\alpha,a}\Psi_{a,\alpha}|e_a\rangle\otimes|f_\alpha\rangle$$
Note that a general $|\Psi\rangle\in\mathcal{H}_1\otimes\mathcal{H}_2$ cannot be written as $|\phi\rangle\otimes|\chi\rangle$ for any $|\phi\rangle=\sum_a\phi_a|e_a\rangle\in\mathcal{H}_1$, $|\chi\rangle=\sum_\alpha\chi_\alpha|f_\alpha\rangle\in\mathcal{H}_2$. If we can find such a $|\phi\rangle\otimes|\chi\rangle$, then this state is said to be entangled.
\end{remarks}
\begin{defi}[Inner product in this extended space]
We specify that the inner product of two basis elements in $\mathcal{H}_1\otimes\mathcal{H}_2$ is just the usual product of the corresponding inner products on each of $\mathcal{H}_1$, $\mathcal{H}_2$, i.e.
$$(\langle e_b|\otimes\langle f_\beta|)(|e_a\rangle\otimes|f_\alpha\rangle)=\langle e_b|e_a\rangle\langle f_\beta|f_\alpha\rangle$$
and extend to general $|\Psi\rangle|\Phi\rangle\in\mathcal{H}_1\otimes\mathcal{H}_2$ by linearity.
\end{defi}
\begin{eg}
An electron has spin 0.5, so 
$$\mathcal{H}_{\text{electron}}=L^2(\mathbb{R}^3,d^3x)\otimes\mathcal{H}_{s=0.5}$$
Let $\{|\uparrow\rangle,|\downarrow\rangle\}$ be a basis of $\mathcal{H}_{s=0.5}$ and so $\mathcal{H}_{s=0.5}\simeq\mathbb{C}^2$. Then in general
$$|\psi_{\text{electron}}\rangle=|\psi_1\rangle\otimes|\uparrow\rangle+|\psi_2\rangle\otimes|\downarrow\rangle$$
with $\langle\mathbf{x}|\psi_{\text{electron}}\rangle=\psi_1(\mathbf{x})|\uparrow\rangle+\psi_2(\mathbf{x})|\downarrow\rangle$.
\end{eg}
\begin{defi}[Operators in this extended space]
Given linear operators $A:~\mathcal{H}_1\rightarrow\mathcal{H}_1$ and $B:~\mathcal{H}_2\rightarrow\mathcal{H}_2$, we define
$$A\otimes B:~\mathcal{H}_1\otimes\mathcal{H}_2\rightarrow\mathcal{H}_1\otimes\mathcal{H}_2$$
$$|e_a\rangle\otimes|f_\alpha\rangle\mapsto(A|e_a\rangle)\otimes(B|f_\alpha\rangle)$$
and extend by linearity. We sometimes write $A\otimes\Id_{\mathcal{H}_2}$ to mean $A$ acting on $\mathcal{H}_1\otimes\mathcal{H}_2$. 
\end{defi}
Note that
$$[A\otimes\Id_{\mathcal{H}_2},\Id_{\mathcal{H}_1}\otimes B]=0$$
This is true even if $A$ and $B$ do not commute when acting on the same Hilbert space.
\begin{eg}
Consider an example of the Hydrogen atom, which consists of an electron and proton. Its gross structure Hamiltonian is
$$H=\frac{P_e^2}{2m_e}\otimes\Id_P+\Id_e\otimes\frac{P_p^2}{2m_p}-\frac{e^2}{4\pi\epsilon_0|X_e-X_p|}$$
If we instead decompose $\mathcal{H}_{Hatom}=\mathcal{H}_e\otimes\mathcal{H}_p=\mathcal{H}_{COM}\otimes\mathcal{H}_{rel}$, then let
$$X_{COM}=\frac{m_eX_e+m_pX_p}{m_e+m_p},\quad X_{rel}=X_e-X_p$$
$$P_{COM}=P_e+P_p,\quad P_{rel}=\frac{m_pP_e-m_eP_p}{m_e+m_p}$$
In terms of these centre of momentum/relative operators,
$$H=\frac{P_{COM}^2}{2M}\otimes\Id_{rel}+\Id_{COM}\otimes\bigg[\frac{P_{rel}^2}{2\mu}-\frac{e^2}{4\pi\epsilon_0|X_{rel}|}\bigg]$$
where $M=m_e+m_p$, $\mu=\frac{m_em_p}{m_e+m_p}$.
\end{eg}
\newpage
\subsection{Addition of Angular Momentum}
For multi-particle systems, the total angular momentum operator is
$$\mathbf{J}=\sum_{i}\mathbf{S_i}+\sum_{j}\mathbf{L_j}$$
The total angular momentum quantum number $j$ can take on a range of values given by $j=\bigotimes_is_i\bigotimes_j\ell_j$.
\begin{defi}[Direct Sum for Matrices]
$$A\oplus B=\begin{bmatrix}A&0_{n\times m}\\0_{m\times n}&B\\\end{bmatrix}$$
such that the following properties are true: (i) $(A\oplus B)(v\oplus w)=(Av)\oplus(Bw)$; (ii) $(A_1\oplus B_1)(A_2\oplus B_2)=(A_1A_2)\oplus(B_1B_2)$; (iii) $\det(A\oplus B)=\det(A)\det(B)$; (iv) $\Tr(A\oplus B)=\Tr(A)+\Tr(B)$. 
\end{defi}
When we combine states $|j_1,m_1\rangle$ and $|j_2,m_2\rangle$ each of which has definite angular momentum, the `obvious' basis of $\mathcal{H}_{j_1}\otimes \mathcal{H}_{j_2}$ consisting of all pairs $|j_1,m_1\rangle\otimes|j_2,m_2\rangle$ may not be the most convenient. We want to understand how to decompose
$$\mathcal{H}_{j_1}\otimes \mathcal{H}_{j_2}=\bigoplus_j\mathcal{H}_j$$
where each $\mathcal{H}_j$ has definite angular momentum for the combined system.\\[5pt]
To be specific, for the bases  $\{|j_1,m_1\rangle\},~\{|j_2,m_2\rangle\}$, where $-j_i\leq m_i\leq j_i$ for $i=1,2$, the state of the combined system is
$$|\psi\rangle=\sum_{m_1=-j_1}^{j_1}\sum_{m_2=-j_2}^{j_2}C_{m_1,m_2}|j_1,m_1\rangle\otimes|j_2,m_2\rangle$$
for some $C_{m_1,m_2}$. If we were to choose $m_1$ and $m_2$ independently, so there are a total of $(2j_1+1)(2j_2+1)$ number of states, which is also the dimension of $\mathcal{H}_{j_1}\otimes\mathcal{H}_{j_2}$, as expected. We want to understand how $|\psi\rangle$ behaves under rotations. Equivalently, which linear combinations correspond to definite amount of angular momentum for the system as a whole.\\[5pt]
Classically, if we combine angular momentum $\mathbf{J_1}$ and $\mathbf{J_2}$, we would expect $\mathbf{J}=\mathbf{J_1}+\mathbf{J_2}$ to lie anywhere on a sphere of radius $|\mathbf{J_2}|$ centred on $\mathbf{J_1}$. Then the combined angular momentum could have magnitude $J$ with
$$|\mathbf{J_1}|-|\mathbf{J_2}|\leq J\leq|\mathbf{J_1}|+|\mathbf{J_2}|$$
\begin{defi}[Angular momentum operators for this extended space]
Let's define $\mathbf{J}$ to be $\mathbf{J_1}\otimes\Id+\Id\otimes\mathbf{J_2}$, or more casually we write as $J=J_1+J_2$. Then the total angular momentum operator is
$$J^2:=J_1^2\otimes\Id+\Id\otimes J_2^2+2\mathbf{J_1}\otimes\mathbf{J_2}$$
or after an abuse of notation, $J^2=J_1^2+J_2^2+2\mathbf{J_1}\cdot\mathbf{J_2}$.
\end{defi}
\begin{prop}
\begin{equation}
J^2=J_1^2+J_2^2+J_{1+}J_{2-}+J_{1-}J_{2+}+2J_{1z}J_{2z}\label{angmomadd}
\end{equation}
\end{prop}
\begin{proof}
We rewrite $J^2$ using $J_{1\pm}=J_{1x}\pm iJ_{1y}$ and $J_{2\pm}=J_{2x}\pm i J_{2y}$.
\begin{eqnarray}
\mathbf{J_1}\cdot\mathbf{J_2}&=&J_{1x}J_{2x}+J_{1y}J_{2y}+J_{1z}J_{2z}\nonumber\\&=&\frac{J_{1+}+J_{1-}}{2}\frac{J_{2+}+J_{2-}}{2}+\frac{J_{1+}-J_{1-}}{2i}\frac{J_{2+}-J_{2-}}{2i}+J_{1z} J_{2z}\nonumber\\&=&\frac{1}{2}(J_{1+} J_{2-})+\frac{1}{2}(J_{1-} J_{2+})+J_{1z}J_{2z}\nonumber
\end{eqnarray}
This allows us to understand the action of $J^2$ on $|j_1,m_1\rangle|j_2,m_2\rangle$.
\end{proof}
\begin{cor}
For the highest weight states,
\begin{equation}
J_z|j_1,j_1\rangle|j_2,j_2\rangle=\hbar(j_1+j_2)|j_1,j_1\rangle|j_2,j_2\rangle\label{angmomadd2}
\end{equation}
\begin{equation}
J^2|j_1,j_1\rangle|j_2,j_2\rangle=(j_1+j_2)(j_1+j_2+1)\hbar^2|j_1,j_1\rangle|j_2,j_2\rangle\label{angmomadd3}
\end{equation}
From the above, we can thus write $|j_1,j_1\rangle|j_2,j_2\rangle$ as $|j,j\rangle$ for the combined system where $j=j_1+j_2$ is the total angular momentum aligned along $\mathbf{\hat{z}}$.
\end{cor}
\begin{proof}
$$J_z|j_1,j_1\rangle\otimes|j_2,j_2\rangle=(J_{1z}\otimes\Id+\Id\otimes J_{2z})|j_1,j_1\rangle\otimes|j_2,j_2\rangle=\hbar(j_1+j_2)|j_1,j_1\rangle\otimes|j_2,j_2\rangle$$
$$J^2|j_1,j_1\rangle\otimes|j_2,j_2\rangle=\bigg[j_1(j_1+1)+j_2(j_2+1)+2j_1j_2\bigg]\hbar^2|j_1,j_1\rangle\otimes|j_2,j_2\rangle=(j_1+j_2)(j_1+j_2+1)\hbar^2|j_1,j_1\rangle\otimes|j_2,j_2\rangle$$
\end{proof}
\begin{cor}
\begin{equation}
|j,j-1\rangle=\sqrt{j_1/j}|j_1,j_1-1\rangle\otimes|j_2,j_2\rangle+\sqrt{j_2/j}|j_1,j_1\rangle\otimes|j_2,j_2-1\rangle\label{CG1}
\end{equation}
\begin{equation}
|j-1,j-1\rangle=\sqrt{j_2/j}|j_1,j_1-1\rangle\otimes|j_2,j_2\rangle-\sqrt{j_1/j}|j_1,j_1\rangle\otimes|j_2,j_2-1\rangle\label{CG2}
\end{equation}
\end{cor}
\begin{proof}
We have the overall lowering operator $J_-=J_{1-}\otimes\Id+\Id\otimes J_{2-}$ to act on $|j,j\rangle=|j_1,j_1\rangle\otimes|j_2,j_2\rangle$. RHS gives
\begin{eqnarray}
& &(J_{1-}\otimes\Id+\Id\otimes J_{2-})|j_1,j_1\rangle\otimes|j_2,j_2\rangle\nonumber\\&=&\hbar\sqrt{j_1(j_1+1)-j_1(j_1-1)}||j_1,j_1-1\rangle\otimes|j_2,j_2\rangle+\hbar\sqrt{j_2(j_2+1)-j_2(j_2-1)}|j_1,j_1\rangle\otimes|j_2,j_2-1\rangle\nonumber\\&=&\hbar\sqrt{2j_1}|j_1,j_1-1\rangle\otimes|j_2,j_2\rangle+\hbar\sqrt{2j_2}|j_1,j_1\rangle\otimes|j_2,j_2-1\rangle\nonumber
\end{eqnarray}
and LHS gives
$$J_-|j,j\rangle=\hbar\sqrt{2j}|j,j-1\rangle,\quad j=j_1+j_2$$
Comparing would give the first result. This state still has perfect alignment between the two subsystems, but only $(j_1+j_2-1)\hbar$ units of angular momentum along $\mathbf{\hat{z}}$. By repeatedly applying $J_-=J_{1-}\otimes\Id+\Id\otimes J_{2-}$, we will construct $(2j+1)$ states all with subsystems aligned, but with $m\in\{-j,...,+j\}$. All these states have the same eigenvalue $j(j+1)$ of $J^2$ with $j=j_1+j_2$.\\[5pt]
Now let's find another state with $m=j_1+j_2-1$. Any such state is directly proportional to $|j_1,j_1-1\rangle\otimes|j_2,j_2\rangle+\beta|j_1,j_1\rangle\otimes|j_2,j_2-1\rangle$ for some $\beta$. If we want this state to have total angular momentum $j_1+j_2-1$, then it must be orthogonal to the state $|j,j-1\rangle$ (Eqn.~\ref{CG1}), so
$$|j-1,j-1\rangle=\sqrt{j_2/j}|j_1,j_1-1\rangle\otimes|j_2,j_2\rangle-\sqrt{j_1/j}|j_1,j_1\rangle\otimes|j_2,j_2-1\rangle$$
By definition,
$$J^2|j-1,j-1\rangle=j(j-1)\hbar^2|j-1,j-1\rangle,\quad j=j_1+j_2$$
Again, this is a highest weight state, now with total angular momentum $j_1 + j_2 − 1$. Applying $J_-$ to this state produces a multiplot with total angular momentum $j_1+j_2-1$ and $m\in\{-j_1-j_2+1,...,j_1+j_2-1\}$.
\end{proof}
\begin{remarks}
Classically, we would expect $|j_1-j_2|\leq j\leq j_1+j_2$ (assume $j_1\geq j_2$). This is also true in QM: 
$$\dim(H_{j_1}\otimes H_{j_2})=(2j_1+1)(2j_2+1)$$
$$\sum_{j=j_1-j_2}^{j_1+j_2}\dim(H_j)=\sum_{j=j_1-j_2}^{j_1+j_2}(2j+1)=2j_1(2j_2+1)+2j_2+1=(2j_2+1)(2j_1+1)$$
\end{remarks}
\begin{defi}[Clebsch-Gordan coefficients]
In general, we can write 
\begin{equation}
|j,m\rangle=\sum_{m_1,m_2}C_{j,m}(j_1,m_1;j_2,m_2)|j_1,m_1\rangle\otimes|j_2,m_2\rangle\label{CG}
\end{equation}
where $C_{j,m}$ are the Clebsch-Gordan coefficients.
\end{defi}
\begin{eg}[$j\otimes0=j$]
If one subsystem has no angular momentum, then $\mathcal{H}_{j=0}=\mathbb{C}$. We have $\mathcal{H}_{j}\otimes\mathcal{H}_0\simeq\mathcal{H}_{j}$ with basis $\{|j,m\rangle\otimes|0,0\rangle\}$.
\end{eg}
\begin{eg}[$0.5\otimes0.5=1\oplus0$]
Suppose $j_1=j_2=0.5$, we have the highest weight state ($j=j_1+j_2,m=j$) to be $$|1,1\rangle=|0.5,0.5\rangle|0.5,0.5\rangle:=|\uparrow\rangle|\uparrow\rangle$$ 
Here, LHS is $|j,m\rangle$ while RHS is $|j_1,m_1\rangle|j_2m_2\rangle$. Applying $J_-$ once (or just quote Eqn.~\ref{CG1}) to get $$|1,0\rangle=\sqrt{\frac{1}{2}}|0.5,-0.5\rangle|0.5,0.5\rangle+\sqrt{\frac{1}{2}}|0.5,0.5\rangle|0.5,-0.5\rangle:=\frac{1}{\sqrt{2}}(|\downarrow\rangle|\uparrow\rangle+|\uparrow\rangle|\downarrow\rangle)$$
For the remaining $j=1$ state, remember $j_i=0.5\implies m_i=\{0.5,-0.5\}$ and $m=m_1+m_2$, $j=j_1+j_2$.
$$|1,-1\rangle=|0.5,-0.5\rangle|0.5,-0.5\rangle:=|\downarrow\rangle|\downarrow\rangle$$ 
The remaining state is deduced to be (needs to be orthogonal to $|1,1\rangle$, Eqn.~\ref{CG2})
$$|0,0\rangle=\sqrt{\frac{1}{2}}|0.5,-0.5\rangle|0.5,0.5\rangle-\sqrt{\frac{1}{2}}|0.5,0.5\rangle|0.5,-0.5\rangle:=\frac{1}{\sqrt{2}}(|\uparrow\rangle|\uparrow\rangle-|\downarrow\rangle|\uparrow\rangle)$$
For $j=1$, we have a triplet of states, which are symmetric under exchange of two subsystems. For $j=0$, we have a singlet state, anti-symmetric under exchange of two subsystems. Note that $0.5\otimes0.5=1\oplus0$ where we have 4 states (3$+$1 since 3 for $j=1$ and 1 for $j=0$, and 2 $\times$ 2 since 2 for spin-half).
\end{eg}
\begin{eg}[$1\otimes0.5=1.5\oplus0.5$]
We have the highest weight state and the next highest (apply $J_-$ once) to be respectively
$$|1.5,1.5\rangle=|1,1\rangle\otimes|0.5,0.5\rangle,\quad|1.5,0.5\rangle=\sqrt{\frac{2}{3}}|1,0\rangle\otimes|0.5,0.5\rangle+\frac{1}{\sqrt{3}}|1,1\rangle\otimes|0.5,-0.5\rangle$$
For the remaining two, remember $j_1=1\implies m_1=\{1,0,-1\}$, $j_2=0.5\implies m_2=\{0.5,-0.5\}$ and that we are constrained to have $j=j_1+j_2$ and $m=m_1+m_2$.
$$|1.5,-0.5\rangle=|1,-1\rangle\otimes|0.5,0.5\rangle,\quad|1.5,-1.5\rangle=|1,-1\rangle\otimes|0.5,-0.5\rangle$$
All of which have $j=1.5$. The $|0.5,0.5\rangle$ state is obtained from $|1.5,1.5\rangle$ by Eqn.~\ref{CG2}:
$$|0.5,0.5\rangle=\frac{1}{\sqrt{3}}|1,0\rangle\otimes|0.5,0.5\rangle-\sqrt{\frac{2}{3}}|1,1\rangle\otimes|0.5,-0.5\rangle$$
this could also be obtained from the orthogonality relation $\langle1.5,0.5|0.5,0.5\rangle=0$. The last one is obtained by applying $J_-$ to $|0.5,0.5\rangle$.
$$|0.5,-0.5\rangle=\sqrt{\frac{2}{3}}|1,-1\rangle\otimes|0.5,0.5\rangle-\frac{1}{\sqrt{3}}|1,0\rangle\otimes|0.5,-0.5\rangle$$
each of which have $j=0.5$. Altogether this are $4+2=3\times 2$ states. 4 states for $j=1.5$ and 2 for $j=0.5$; 3 states for spin 1 and 2 states from spin 0.5.
\end{eg}
\newpage
\section{Identical Particles}\label{sec:IdenticalParticles}
All detectors in the Universe are `identical' in the sense that they're each described by a copy of the same $\mathcal{H}$ and are indistinguishable by any experiment.
\begin{defi}[Multi-particle state]
Suppose $|\alpha_a\rangle\in\mathcal{H}_a$ is a basis element in the Hilbert space of the `a-th' electron. If we have $N$ electrons, we would describe the system by
$$\bigotimes_{a=1}^N\mathcal{H}_a=|\alpha_1,...,\alpha_N\rangle$$
would correspond to a state where the `a-th' electron is $|\alpha_a\rangle$.
\end{defi}
\begin{prop}
Multi-particle states can only be symmetric or anti-symmetric under exchange of any two particles.
\end{prop}
\begin{proof}
Since all electrons are indistinguishable, this is physically equivalent to the state $|\alpha_2,\alpha_1,\alpha_3,\alpha_4,...,\alpha_N\rangle$ in which all quantum numbers of electrons 1 and 2 are exchanged. Consequently, there can be no physical difference between this state and the state in which these two particles are exchanged. This does not imply that the two states are identical, but does imply that they can only differ by a phase:
$$|\alpha_1,\alpha_2,...,\alpha_N\rangle=e^{i\phi}|\alpha_2,\alpha_1,...,\alpha_N\rangle=e^{2i\phi}|\alpha_1,\alpha_2,...,\alpha_N\rangle\implies e^{i\phi}=\pm1$$
More generally, if all $N$ particles are identical, the state must be either totally symmetric or totally anti-symmetric under exchange of all the quantum numbers of any pair. 
\end{proof}
\begin{defi}[Fermions/Bosons]
Particles for which state is anti-symmetric/symmetric are called fermions/bosons. In QFT, spin-statistics theorem states that all integer spin particles are bosons, while all odd half-integer spin particles are fermions.
\end{defi}
\begin{defi}[Slater's determinant]
The state of $N$ identical fermions can be represented by a determinant:
$$\langle\alpha_1,\alpha_2,\dots,\alpha_N|\psi\rangle=\frac{1}{\sqrt{N!}}\begin{vmatrix}\psi_1(\alpha_1)&\psi_2(\alpha_2)&\dots&\psi_1(\alpha_N)\\\vdots&\vdots&\ddots&\vdots\\\psi_N(\alpha_1)&\psi_N(\alpha_2)&\dots&\psi_N(\alpha_N)\\\end{vmatrix}$$
where $\alpha_i$ denote a complete set of quantum numbers for single particle states.
\end{defi}
\begin{remarks}
The Hilbert space of a single such spin-$s$ fermion is itself a tensor product $\mathcal{H}_{\text{fermion}} =\mathcal{H}_{\text{spatial}}\otimes\mathbb{C}^{2s+1}$, including a factor that describes the fermion’s spatial wavefunction (including details of its likely position, orbital angular momentum etc.) and a separate factor $\mathcal{H}_{\text{spin}}\simeq\mathbb{C}^{2s+1}$, spanned by the possible spin states $\{|s\rangle, |s−1\rangle,\dots , |-s\rangle\}$ of this spin-$s$ particle. We must exchange both the spin and spatial parts of the state in order to find a physically equivalent state.
\end{remarks}
Exchange symmetry has an interesting interplay with orbital angular momentum.
\begin{prop}
Exchange acts trivially on the centre-of-momentum coordinates, but acts on the relative coordinates just like a parity transformation.
\end{prop}
\begin{proof}
Under the exchange of two identical particles, $X_{CoM}\mapsto X_{CoM}$ and $P_{CoM}\mapsto P_{CoM}$, which is trivial. Also, $X_{rel}=X_1-X_2\mapsto X_2-X_1=-X_{rel}$ and $P_{rel}=0.5(P_1-P_2)\mapsto0.5(P_2-P_1)=-P_{rel}$. So exchange acts like a parity transform on the relative coordinates and momenta.
\end{proof}
\begin{eg}
Suppose our system has a relative spatial wavefunction $\sim R(r)Y_\ell^m(x)$, then under exchange, we obtain $Y_\ell^m(-x)=(-1)^lY_\ell^m(x)$. If our two particles are fermions, we need their spins to be in a symmetric state when $l$ is odd, or anti-symmetric if $l$ is even.
\end{eg}
\begin{defi}[Transposition Operator]
Consider two particles with single particle basis $|\alpha\rangle$ and $|\beta\rangle$ respectively. We define the transposition operator
$$P_{12}|\alpha\rangle|\beta\rangle:=|\beta\rangle|\alpha\rangle$$
where $|\alpha\rangle|\beta\rangle=|\alpha\rangle\otimes|\beta\rangle$. Note $P_{12}\in\mathcal{L}(V\otimes V)$.
\end{defi}
\begin{prop}
The transposition operator is Hermitian, i.e. $P_{12}^\dag=P_{12}$.
\end{prop}
\begin{proof}
$\langle u_k\otimes u_l,u_i\otimes u_j\rangle=\delta_{ki}\delta_{lj}$. On one hand, $\langle P_{21}u_k\otimes u_l,u_i\otimes u_j\rangle=\langle u_l\otimes u_k,u_i\otimes u_j\rangle=\delta_{li}\delta_{kj}$. On the other hand, $\langle u_k\otimes u_l,P_{21}u_i\otimes u_j\rangle=\langle u_k\otimes u_l,u_j\otimes u_i\rangle=\delta_{kj}\delta_{li}$. Hence, $P_{12}$ is Hermitian. 
\end{proof}
\begin{prop}
The eigenvalues of $P_{12}$ are $\pm1$, where $+1$ represents symmetric states and $-1$ represents anti-symmetric states.
\end{prop}
\begin{proof}
$P_{12}^2=I$, so the eigenvalues of the identity matrix have to be 1 only.
\end{proof}
\begin{defi}[Identical]
$N$ particles of a system are said to be identical if the various observables of the system are symmetrical when any two particles are interchanged.
\end{defi}
\begin{prop}
If a set of particles are indistinguishable, their Hamiltonian must be unchanged by permutation of particles. However, the converse is not true.
\end{prop}
\begin{defi}[Symmetrizer, Anti-Symmetrizer]
We can define the symmetriser and anti-symmetriser operators as $S_{12}:=\frac{1}{2}(1+P_{12})$ and $A_{12}:=\frac{1}{2}(1-P_{12})$. They are defined such that $S|\psi\rangle\in\text{Sym}(V\otimes V)$ and $A|\psi\rangle\in\text{Anti}(V\otimes V)$. They are thus projectors from $V\otimes V$ to $\text{Sym}(V\otimes V)$ and $\text{Anti}(V\otimes V)$ respectively. They are defined such that
$$P_{21}S|\psi\rangle=S|\psi\rangle,\quad P_{21}A|\psi\rangle=-A|\psi\rangle$$
Also, these operators are orthogonal, i.e. range of $S$ $\perp$ range of $A$, and complementary, i.e. $S+A=\text{Id}$ and $SA=AS=0$. We thus have
$$V\otimes V=\text{Sym}(V\otimes V)\oplus\text{Anti}(V\otimes V)$$
\end{defi}
\begin{defi}[Principle of Indistinguishability]
Dynamical states that differ only by a permutation of
identical particles cannot be distinguished by any observation whatsoever, i.e. $\langle\psi|A|\psi\rangle=\langle\psi|P_{12}^\dag AP_{12}|\psi\rangle\implies P_{12}A=AP_{12}$. The operator $A$ commutes with the transposition operator.
\end{defi}
\begin{cor}
If $[H,P_{ij}]=0$, then the wavefunction will remain symmetric or anti-symmetric for all times.
\end{cor}
\begin{defi}[Symmetrization Postulate]
Particles whose spin is an integer (half odd integer) multiple of $\hbar$ are described by totally symmetrical (anti-symmetric) states. They are bosons (fermions) and satisfy the Bose-Einstein (Fermi-Dirac statistics). Moreover, partially symmetric states do not exist. Essentially, for a system of $N$ identical particles,
$$P_{ij}|N\text{ identical bosons }\rangle=+|N\text{ identical bosons }\rangle$$
$$P_{ij}|N\text{ identical fermions }\rangle=-|N\text{ identical fermions}\rangle$$
where $P_{ij}$ is the permutation operator that interchanges the $i$th and the $j$ th particles, with $i$ and $j$ arbitrary.
\end{defi}
\begin{eg}[System of Two Identical Particles]
Consider two particles with states they can occupy be $|\alpha\rangle$ and $|\beta\rangle$. For bosons, the independent states are $|\alpha\rangle|\alpha\rangle$,$|\beta\rangle|\beta\rangle$,$\frac{1}{\sqrt{2}}(|\alpha\rangle|\beta\rangle+|\beta\rangle|\alpha\rangle)$. For fermions, the only independent state can be written as $\frac{1}{\sqrt{2}}(|\alpha\rangle|\beta\rangle-|\beta\rangle|\alpha\rangle)$. For classical particles (distinguishable), they follow the Maxwell-Boltzmann statistics. the independent states are $|\alpha\rangle|\beta\rangle$, $|\beta\rangle|\alpha\rangle$, $|\alpha\rangle|\alpha\rangle$, $|\beta\rangle|\beta\rangle$.
\end{eg}
\begin{thm}[Pauli Exclusion Principle]
No two fermions can occupy the same state.
\end{thm}
\begin{proof}
For any two fermions, say if $\alpha=\beta$, the state vanishes, i.e. no such state exists.
\end{proof}
\begin{defi}[Generalize to Multi-particle states]
For a system of many identical particles, we can define
$$P_{ij}|k^1\rangle|k^2\rangle...|k^i\rangle|k^{i+1}\rangle...|k^j\rangle...:=|k^1\rangle|k^2\rangle...|k^j\rangle|k^{i+1}\rangle...|k^i\rangle...$$
It is clear that $P_{ij}^2=1$ and the allowed eigenvalues are $\pm 1$ as well. Also, $[P_{ij},P_{kl}]\neq 0$. The normalization is $\sqrt{\prod_{i=1}^nN_i!/N!}$ where $N$ is the total number of particles and $N_i$ is the number of times $|k^{(i)}\rangle$ occurs.
\end{defi}
\begin{defi}[Permutation Operators]
For $N$ particle systems, we can define $N!$ permutation operators, such that all are elements of the symmetric group $\mathcal{S}_N$. We define the permutation $\alpha$ to be a mapping
$$\alpha:~ [1,...,N]\rightarrow[\alpha(1),...,\alpha(N)]$$
The corresponding permutation operator will be $P_\alpha:=P_{\alpha(1),...,\alpha(N)}$. They act via
$$P_{\alpha}|u_1\rangle_{(1)}\otimes...\otimes|u_N\rangle_{(N)}=|u_{\alpha(1)}\rangle_{(1)}\otimes...\otimes|u_{\alpha(N)}\rangle_{(N)}$$
The permutation operator is unitary such that the Hermitian conjugate is its inverse.
\end{defi}
\begin{prop}
While all transpositions are Hermitian, general permutations are not Hermitian since transpositions do not necessarily commute. Any permutation can be written as a product  of transpositions such that the decomposition is not unique, but the number of transpositions is unique modulo 2 (parity).
\end{prop}
\begin{defi}[Complete Symmetrizer, Complete Asymmetrizer]
Consider $N$ particles each with state space $V$ so that the collection of particles live on $V^{\otimes N}$. Let $P_\alpha$ denote arbitrary permutation in $\mathcal{S}_N$.\\[5pt]
We have $S$ to be the complete symmetrizer such that $S|\psi\rangle$ to be the symmetric states and hence $P_\alpha S|\psi\rangle=S|\psi\rangle$ $\forall\alpha$. These symmetric states are eigenstates of all permutation operators with eigenvalue $+1$. Of course, these symmetric states cannot form a basis for $V^{\otimes N}$, and can only form a subspace $\text{Sym}^NV$. We have $S=\frac{1}{N!}\sum_\alpha P_\alpha$.\\[5pt]
We have $A$ to be the complete antisymmetrizer such that $A|\psi\rangle$ to be the anti-symmetric states and hence $P_\alpha A|\psi\rangle=\epsilon_\alpha A|\psi\rangle$. These anti-symmetric states are eigenstates of all permutation operators with eigenvalue, depending on the parity of the permutation, and they form a subspace $\text{Anti}^NV$. We have $A=\frac{1}{N!}\sum_\alpha\epsilon_\alpha P_\alpha$. Essentially, $\text{Sym}^NV\oplus\text{Anti}^NV\subset V\otimes V$.
\end{defi}
\begin{defi}[Spatial versus Spin States]
The total wavefunction may be written as a product of spatial $\psi(\mathbf{r_i})$ and spin wavefunction $\chi(\mathbf{m_{s_i}})$. We denote it as $|\Psi\rangle\in V\otimes W$, where $V$ and $W$ are spaces for spatial and spin states respectively. 
\end{defi}
\begin{eg}[Para and Ortho H$_2$]
Fermions have an overall anti-symmetric wavefunction, so either anti-symmetric spatial with symmetric spin (ortho-), or symmetric spatial with anti-symmetric spin (para-). Ortho and para have $S_{total}$ 1 (triplet) and 0 (singlet) respectively. Since $PY_{l,m}=Y_{l,m}(\pi-\theta,\pi+\phi)=(-1)^lY_{l,m}(\theta,\phi)$, we thus have even $l$ (symmetric spatial so anti-symmetric spin) to be para and odd $l$ (anti-symmetric spatial so symmetric spin) to be ortho.
\end{eg}
\begin{prop}
Identical particles are closer together, or further apart, on average than distinguishable particles occupying the same spatial states.
\end{prop}
\begin{proof}
Consider two non-interacting particles moving one dimension, with each particle occupying distinct orthonormal spatial states $|a\rangle$ and $|b\rangle$. For distinguishable particles, each particle unambiguously occupies a particular spin state, giving two possible two-particle states $|1,a\rangle|2,b\rangle$ or $|1,b\rangle|2,a\rangle$. For identical particles, the two-particle spatial state must be symmetric or anti-symmetric under particle exchange, i.e. $\frac{1}{\sqrt{2}}(|1,a\rangle|2,b\rangle\pm|1,b\rangle|2,a\rangle)$. The mean-squared separation of the two particles is given by
$$\langle x_{12}^2\rangle=\int\int(x_1-x_2)^2|\psi(x_1,x_2)|^2dx_1dx_2=\langle x_1^2\rangle+\langle x_2^2\rangle-2\langle x_1x_2\rangle$$
For distinguishable particles, $\langle x_{12}^2\rangle=\langle x^2\rangle_a+\langle x^2\rangle_b-2\langle x\rangle_a\langle x\rangle_b$. For indistinguishable particles, we however have $\langle x_1^2\rangle=\frac{1}{2}\langle x^2\rangle_a+\frac{1}{2}\langle x^2\rangle_b$ (noting that the spatial states are orthonormal). Similarly, interchange $a$ and $b$ to get $\langle x_2^2\rangle$. Finally, $\langle x_1 x_2\rangle=\frac{1}{2}\langle x\rangle_a\langle x\rangle_b+\frac{1}{2}\langle x\rangle_a\langle x\rangle_b\pm\frac{1}{2}x_{ab}x_{ba}\pm\frac{1}{2}x_{ab}x_{ba}$. Thus, giving $\langle x_{12}^2\rangle=\langle x^2\rangle_a+\langle x^2\rangle_b-2\langle x\rangle_a\langle x\rangle_b\mp2|x_{ab}|^2$ for symmetric/antisymmetric case.
\end{proof}
\begin{remarks}
Indistinguishable particles tend to be mutually attracted or repelled (depending on their statistics), giving rise to `exchange forces'. The energy associated is the `exchange interactions' while their correlated behaviour is called `exchange correlations'.
\end{remarks}
\chapter{Transformations}
\begin{notation}
From here onwards, we will drop the hat notation and no longer distinguish operators. More often than not, the context is clear.
\end{notation}
\section{Continuous transformations}\label{sec:cont_trans}
When we move our physical system through $\mathbb{R}^3$ in some way, it should be described by different states $|\psi\rangle,|\psi'\rangle\in\mathcal{H}$ before and after transformation. This suggests that transformations (e.g. translations, rotations, etc) of $\mathbb{R}^3$ should correspond to some operator $U$ such that $|\psi\rangle\mapsto U|\psi\rangle=|\psi'\rangle$. If initially the state is properly normalized, i.e. $\langle\psi|\psi\rangle=1$, then we should still have $\langle\psi'|\psi'\rangle=1$ after transforming, since all we've done is translate/rotate it. 
\begin{prop}
$\forall|\psi\rangle\in\mathcal{H}$ to stay normalized after a transformation, then this transformation must be represented by a unitary operator.
\end{prop}
\begin{proof}
Let $|\psi\rangle=|\phi\rangle+\lambda|\chi\rangle,\lambda\in\mathbb{C}$. Then the condition says
$$1=\langle\psi|\psi\rangle=\langle\phi|\phi\rangle+|\lambda|^2\langle\chi|\chi\rangle+\lambda\langle\phi|\chi\rangle+\overline{\lambda}\langle\chi|\phi\rangle$$
but
$$1=\langle\psi'|\psi'\rangle=\langle\psi|U^\dag U|\psi\rangle=||U|\phi\rangle||^2+|\lambda|^2||U|\chi\rangle||^2+\lambda\langle\phi|U^\dag U|\chi\rangle+\overline{\lambda}\langle\chi|U^\dag U|\phi\rangle$$
We thus have 
$$\lambda(\langle\phi|\chi\rangle-\langle\phi|U^\dag U|\chi\rangle)=\overline{\lambda}(\langle\chi|U^\dag U|\phi\rangle-\langle\chi|\phi\rangle)$$
This is true $\forall\lambda\in\mathbb{C}$, so each side must vanish separately, i.e. $\langle\phi|\chi\rangle=\langle\phi|U^\dag U|\chi\rangle$ $\forall|\phi\rangle,|\chi\rangle\in\mathcal{H}$. This is possible for arbitrary $|\phi\rangle$ and $|\chi\rangle$ iff $U^\dag U=\Id$, or $U^\dag=U^{-1}$, so transformations are represented by unitary operators $U:~\mathcal{H}\rightarrow\mathcal{H}$. 
\end{proof}
\begin{remarks}\leavevmode
\begin{enumerate}
\item Translations, rotations, etc form groups, e.g. the composition of two rotations around $\mathbf{O}\in\mathbb{R}^3$ is again a rotation). We require our operators $U$ to reflect this group structure, i.e.
$$U(g_1)\circ U(g_2)=U(g_1g_2)$$
for $g_1,g_2\in G$. Also, $U(e)=\Id_{\mathcal{H}}$ where $e$ is the identity in $G$. Note:  $(U_1U_2)^\dag=U_2^\dag U_1^\dag=U_2^{-1}U_1^{-1}=(U_1U_2)^{-1}$ preserves unitarity. 
\item We can also show how transformations act on operators. If $|\psi\rangle\mapsto U|\psi\rangle$, then $\langle\psi|A|\psi\rangle\mapsto\langle\psi|U^\dag AU|\psi\rangle$. Thus, we can equally work with the original untransformed state $|\psi\rangle$, but instead change operators $A\mapsto U^\dag AU$, i.e. conjugation or similarity transform. They satisfy the following commutation relation:
$$[U^\dag AU,U^\dag BU]=U^\dag[A,B]U$$
\item Transformations can often be continuous. Suppose our transformation depends on some continuous parameter $\theta$ such that $\theta=0$ corresponds to the identity. For small $\theta$, we have
\begin{equation}
U(\delta\theta)=\Id_{\mathcal{H}}-i\delta\theta~T+O(\delta\theta^2)\label{infinitesimal}
\end{equation}
for some operator $T$. Since $U$ is unitary, $U^\dag(\delta\theta)=\Id_{\mathcal{H}}+i\delta\theta~ T^\dag+O(\delta\theta^2)=\Id_H+i\delta\theta~T+O(\delta\theta^2)=U^{-1}(\delta\theta)$. Comparing at $O(\delta\theta)$, $T^\dag=T$, and so $T$ is a Hermitian operator. These first order terms are called \emph{generators} of the transformation and are thus good candidates for observables. In particular, if $|\psi'\rangle=U(\delta\theta)|\psi\rangle$ is our transformed state, then 
\begin{equation}
|\psi'\rangle-|\psi\rangle=(U(\delta\theta)-1)|\psi\rangle=-i\delta\theta T|\psi\rangle\implies\frac{\partial|\psi\rangle}{\partial\theta}=\lim_{\delta\theta\rightarrow 0}\frac{|\psi'\rangle-|\psi\rangle}{\delta\theta}=-iT|\psi\rangle\label{generator}
\end{equation}
The generator $T$ tells us the rate of change of states. Similarly, since
\begin{equation}
A'=U^{-1}(\delta\theta)AU(\delta\theta)=[1+i\delta\theta T]A[1-i\delta\theta T]+O(\delta\theta^2)=A+i\delta\theta[T,A]+O(\delta\theta^2)\label{commutator_generator}
\end{equation}
Hence, $i[T,A]$ tells us how operators change under infinitesimal transformation. In another words, the rate of change of states themselves is given by the action of the generator, whereas the rate of change of operators under a transformation is determined by the commutator of the generator with the operator. We can build up a finite transformation by repeatedly performing the same infinitesimal one. Let $\delta\theta=\theta/N$, then
\begin{equation}
U(\theta)=\lim_{N\rightarrow\infty}\bigg(1-\frac{i\delta\theta}{N}\bigg)^N=e^{-i\theta T}\label{finite}
\end{equation}
\end{enumerate}
\end{remarks}
\subsection{Translations}
\begin{eg}
Consider the example of translation. If we translate our system through some fixed vector $\mathbf{a}$, then $$\langle\psi|\mathbf{X}|\psi\rangle\mapsto\langle\psi'|\mathbf{X}|\psi'\rangle=\langle\psi|U^{-1}(\mathbf{a})\mathbf{X}U(\mathbf{a})|\psi\rangle=\langle\psi|\mathbf{X}|\psi\rangle+\mathbf{a}$$
Since this defines our translation. Thus must hold $\forall|\psi\rangle\in\mathcal{H}$, so
$$U^{-1}(\mathbf{a})\mathbf{X}U(\mathbf{a})=\mathbf{X}+\mathbf{a}\Id_{\mathcal{H}}$$
\end{eg}
\begin{remarks}\leavevmode
\begin{enumerate}
    \item For an infinitesimal translation (Eqn.~\ref{infinitesimal}), we write
$$U(\delta\mathbf{a})=1-\frac{i}{\hbar}\delta\mathbf{a}\cdot\mathbf{P}+O(\delta a^2)$$
$$\implies\mathbf{X}+\delta\mathbf{a}=U^{-1}(\delta\mathbf{a})\mathbf{X}U(\delta\mathbf{a})=\bigg(1+\frac{i}{\hbar}\delta\mathbf{a}\cdot\mathbf{P}\bigg)\mathbf{X}\bigg(1-\frac{i}{\hbar}\delta\mathbf{a}\cdot\mathbf{P}\bigg)+...=\mathbf{X}-\frac{i}{\hbar}[\mathbf{X},\delta\mathbf{a}\cdot\mathbf{P}]$$
which is consistent with Eqn.~\ref{commutator_generator}. Equivalently, since this is true $\forall\delta\mathbf{a}$, $i\hbar\delta_{ij}=[\mathbf{X_i},\mathbf{P_j}]$, hence re-obtaining the standard canonical commutation relations. 
\item Translations form an Abelian group, i.e. $U(\mathbf{a})U(\mathbf{b})=U(\mathbf{a}+\mathbf{b})$ which should be equal to $U(\mathbf{b}+\mathbf{a})=U(\mathbf{b})U(\mathbf{a})$. 
\item For infinitesimal translations to be abelian, we must have
\begin{align}
    \bigg(1-\frac{i}{\hbar}\delta\mathbf{a}\cdot\mathbf{P}\bigg)\bigg(1-\frac{i}{\hbar}\delta\mathbf{b}\cdot\mathbf{P}\bigg)&=\bigg(1-\frac{i}{\hbar}\delta\mathbf{b}\cdot\mathbf{P}\bigg)\bigg(1-\frac{i}{\hbar}\delta\mathbf{a}\cdot\mathbf{P}\bigg)\nonumber\\\implies-\frac{1}{\hbar^2}\delta a_iP_i\delta b_jP_j&=-\frac{1}{\hbar^2}\delta b_jP_j\delta a_iP_i\nonumber\\\implies[P_i,P_j]&=0\nonumber
    \end{align}
\item For finite translations, we have $U(\mathbf{a})=e^{-i\mathbf{a}\cdot\mathbf{P}/\hbar}$. What is the resultant state of $U(a)|x\rangle$? Act the position operator:
$$X(U(\mathbf{a})|\mathbf{x}\rangle)=([X,U(\mathbf{a})]+U(\mathbf{a})\mathbf{X})|x\rangle=(x+a)U(\mathbf{a})|\mathbf{x}\rangle$$
i.e. $U(\mathbf{a})|\mathbf{x}\rangle$ is at position $x+a$. We must have $U(\mathbf{a})|\mathbf{x}\rangle=c|\mathbf{x}+\mathbf{a}\rangle$ for some constant $c\in\mathbb{C}$. 
$$\delta^3(\mathbf{x}-\mathbf{y})=\langle\mathbf{y}|\mathbf{x}\rangle=\langle\mathbf{y}|U^\dag(\mathbf{a})U(\mathbf{a})|\mathbf{x}\rangle=(U(\mathbf{a})|\mathbf{y}\rangle)^\dag U(\mathbf{a})|\mathbf{x}\rangle=\overline{c}c\langle\mathbf{y}+\mathbf{a}|\mathbf{x}+\mathbf{a}\rangle=\delta^3(\mathbf{x}-\mathbf{y})|c|^2$$
We thus must have $|c|=1$. By convention, we choose $c=1$ and hence $U(\mathbf{a})|\mathbf{x}\rangle=|\mathbf{x}+\mathbf{a}\rangle$. For a generic state if $|\psi\rangle\mapsto|\psi_{\mathbf{a}}\rangle=U(\mathbf{a})|\psi\rangle$, then
$$\langle x|\psi_{\mathbf{a}}\rangle=\langle x|U(\mathbf{a})|\psi\rangle=(U^{-1}(\mathbf{a})|\mathbf{x}\rangle)^\dag|\psi\rangle=\langle\mathbf{x}-\mathbf{a}|\psi\rangle$$
In another words, the new wavefunction in the position representation behaves like how we expected, i.e. $\psi_{\mathbf{a}}(\mathbf{x})=\psi(\mathbf{x}-\mathbf{a})$.
\end{enumerate}
\end{remarks}
\begin{eg}
As a special case, if we translate an eigenstate of $\mathbf{P}$, then
$$U(\mathbf{a})|\mathbf{p}\rangle=e^{-i\mathbf{a}\cdot\mathbf{P}/\hbar}|\mathbf{p}\rangle=e^{-i\mathbf{a}\cdot\mathbf{p}/\hbar}|\mathbf{p}\rangle$$
Hence $\langle\mathbf{x}|U(\mathbf{a})|\mathbf{p}\rangle=e^{-i\mathbf{a}\cdot\mathbf{p}/\hbar}\langle\mathbf{x}|\mathbf{p}\rangle$ or $\psi_p(\mathbf{x}-\mathbf{a})=e^{-iap/\hbar}\psi_p(x)$. We thus have $\psi_p(\mathbf{x})=Ce^{i\mathbf{p}\cdot\mathbf{x}/\hbar}$ and it is conventional to choose $C=\frac{1}{(2\pi\hbar)^{3/2}}$. We have
$$\langle\mathbf{p'}|\mathbf{p}\rangle=\int\langle\mathbf{p'}|\mathbf{x}\rangle\langle\mathbf{x}|\mathbf{p}\rangle d^3x=\frac{1}{(2\pi\hbar)^3}\int e^{i\mathbf{x}\cdot(\mathbf{p}-\mathbf{p'})/\hbar}d^3x=\delta^3(\mathbf{p}-\mathbf{p'})$$
We also have $\langle\mathbf{x}|\mathbf{P}|\psi\rangle=-i\hbar\boldsymbol{\nabla}(\langle x|\psi\rangle)$.
\end{eg}
\begin{remarks}
To discuss position and momentum of a dynamical quantum system described by a Hamiltonain $H[\mathbf{X},\mathbf{P}]$, we need to be in the world of function spaces. For non-trivial finite-dimensional Hilbert space $\mathcal{H}$, the standard commutation relation $[X_i,P_j]=i\hbar\delta_{ij}\Id_{\mathcal{H}}$ gives us
$$\dim\mathcal{H}\delta_{ij}=-\frac{i}{\hbar}\Tr_{\mathcal{H}}[X_iP_j-P_jX_i]=0$$
But in an infinitesimal dimensional Hilbert space, neither $\Tr_{\mathcal{H}}\Id_{\mathcal{H}}$ nor $\Tr_{\mathcal{H}}X_iP_j$ are defined. So finite-dimensional matrices cannot represent the actions of the translation group on the Hilbert space.
\end{remarks}
\subsection{Rotations}
Next we consider the example of rotation. For any vector $\mathbf{v}\in\mathbb{R}^3$, rotating anti-clockwise through $|\boldsymbol{\alpha}|$ around the $\boldsymbol{\hat{\alpha}}$-axis, we obtain $\mathbf{v}\mapsto\mathbf{v'}=R(\boldsymbol{\alpha})\mathbf{v}$ where $\mathbf{v'}\cdot\mathbf{v'}=\mathbf{v}\cdot\mathbf{v}$ and $\det(R(\boldsymbol{\alpha}))=+1$. Infinitesimally, we have $\mathbf{v'}=\mathbf{v}+\delta\mathbf{v}=\mathbf{v}+\delta\boldsymbol{\alpha}\times\mathbf{v}$. However, rotations do not generically commute, i.e. $R(\boldsymbol{\alpha})R(\boldsymbol{\beta})\neq R(\boldsymbol{\beta})R(\boldsymbol{\alpha})$ such that
$$R(\delta\boldsymbol{\beta})R(\delta\boldsymbol{\alpha})\mathbf{v}=R(\delta\boldsymbol{\beta})(\mathbf{v}+\delta\boldsymbol{\alpha}\times\mathbf{v})=\mathbf{v}+\delta\alpha\times \mathbf{v}+\delta\boldsymbol{\beta}\times(\mathbf{v}+\delta\alpha\times \mathbf{v})=\mathbf{v}+\delta\boldsymbol{\alpha}\times \mathbf{v}+\delta\boldsymbol{\beta}\times\mathbf{v}+\delta\boldsymbol{\beta}\times(\delta\boldsymbol{\alpha}\times\mathbf{v})$$
It then follows that the commutator is
\begin{equation}
[R(\delta\boldsymbol{\beta}),R(\delta\boldsymbol{\alpha})]\mathbf{v}=\delta\boldsymbol{\beta}\times(\delta\boldsymbol{\alpha}\times\mathbf{v})-\delta\boldsymbol{\alpha}\times(\delta\boldsymbol{\beta}\times\mathbf{v})=(\delta\boldsymbol{\beta}\times\delta\boldsymbol{\alpha})\times\mathbf{v}=R(\delta\boldsymbol{\beta}\times\delta\boldsymbol{\alpha})\mathbf{v}-\mathbf{v}\label{commutator_rotation}
\end{equation}
There exists a unitary operator $U(\boldsymbol{\alpha})$ corresponding to rotation $R(\boldsymbol{\alpha})$.
$$\mathbf{X}\mapsto\mathbf{X'}=U^{-1}(\boldsymbol{\alpha})\mathbf{X}U(\boldsymbol{\alpha})=R(\boldsymbol{\alpha})\mathbf{X}=\begin{pmatrix}\cos\alpha&\sin\alpha&0\\-\sin\alpha&\cos\alpha&0\\0&0&1\\\end{pmatrix}\begin{pmatrix}x\\y\\z\\\end{pmatrix}$$
\newpage
\begin{prop}
For an infinitesimal rotation, we write $U(\delta\alpha)=1-\frac{i}{\hbar}\delta\boldsymbol{\alpha}\cdot\mathbf{J}+O(\delta\alpha^3)$ where $\mathbf{J}$ has units of angular momentum, such that $\mathbf{J}/\hbar$ are the generators of rotations. Then, the commutation relations will be
\begin{enumerate}
    \item $[J_i,X_j]=i\hbar\epsilon_{ijk}X_k$
    \item $[J_i,J_j]=i\hbar\epsilon_{ijk}J_k$
    \item $[J_i,P_j]=i\hbar\epsilon_{ijk}P_k$
\end{enumerate}
\end{prop}
\begin{proof}\leavevmode
\begin{enumerate}
\item From Eqn.~\ref{commutator_generator}, we have
$$\mathbf{X'}=U^\dag \mathbf{X} U=\bigg(1+\frac{i}{\hbar}\delta\boldsymbol{\alpha}\cdot\mathbf{J}\bigg)\mathbf{X}\bigg(1-\frac{i}{\hbar}\delta\boldsymbol{\alpha}\cdot\mathbf{J}\bigg)=R(\delta\boldsymbol{\alpha})\mathbf{X}=\mathbf{X}+\delta\boldsymbol{\alpha}\times\mathbf{X}$$
This leads to $\frac{i}{\hbar}[\delta\boldsymbol{\alpha}\cdot\mathbf{J},\mathbf{X}]=\delta\boldsymbol{\alpha}\times\mathbf{X}\implies\delta\alpha_i[J_i,X_j]=-i\hbar\epsilon_{ijk}\delta\alpha_iX_k$ and thus $[J_i,X_j]=i\hbar\epsilon_{ijk}X_k$ since it is true $\forall\alpha_i$. 
\item Apply our homomorphism $U$ (from SO(3) group of rotations to the space of unitary operators on the Hilbert space) to the identity (Eqn.~\ref{commutator_rotation}) to find $[U(\delta\boldsymbol{\beta}),U(\delta\boldsymbol{\alpha})]=U(\delta\boldsymbol{\beta}\times\delta\boldsymbol{\alpha})-\Id$. This implies
$$-\frac{1}{\hbar^2}[\delta\boldsymbol{\beta}\cdot\mathbf{J},\delta\boldsymbol{\alpha}\cdot\mathbf{J}]=-\frac{i}{\hbar}(\delta\boldsymbol{\beta}\times\delta\boldsymbol{\alpha})\cdot\mathbf{J}\implies[J_i,J_j]=i\hbar\epsilon_{ijk}J_k$$
since this is true for arbitrary $\delta\boldsymbol{\alpha}$ and $\delta\boldsymbol{\beta}$. 
\item We can more generally consider both rotations and translations. Clearly, they do not commute. Consider the commutator of an infinitesimal rotation and translation 
$$R(\delta\boldsymbol{\alpha})(\mathbf{v}+\delta\boldsymbol{\alpha})-(R(\delta\boldsymbol{\alpha})\mathbf{v}+\delta\mathbf{a})=\delta\boldsymbol{\alpha}\times\delta\mathbf{a}$$
which is independent of $\mathbf{v}$. Again, apply our homomorphism
$$[U_R(\delta\boldsymbol{\alpha}),U_T(\delta\mathbf{a})]=U_T(\delta\boldsymbol{\alpha}\times\delta\mathbf{a})-I$$
which equivalently gives $[J_i,P_j]=i\hbar\epsilon_{ijk}P_k$.
\end{enumerate}
\end{proof}
\begin{remarks}
More generally, any 3-component linear operator $V:~\mathcal{H}\rightarrow\mathcal{H}$ is said to transform as a vector under rotations if it obeys $$U^{-1}(\boldsymbol{\alpha})\mathbf{V}U(\boldsymbol{\alpha})=R(\boldsymbol{\alpha})\mathbf{V}$$ for any $\boldsymbol{\alpha}$. But if an operator $S$ transform like
$$U^{-1}(\boldsymbol{\alpha})SU(\boldsymbol{\alpha})=S$$
for any $\boldsymbol{\alpha}$, then $S$ is a scalar operator. We will return to this in the Wigner-Eckart theorem later.
\end{remarks}
\begin{thm}
We can show for a generic vector $V$ and a scalar $S$ that
\begin{equation}
[J_i,V_j]=i\hbar\epsilon_{ijk}V_k,\quad [J_i,S]=0\label{vector_scalar_commutator}
\end{equation}
\end{thm}
\begin{eg}
If we have two vector operators $\mathbf{V}$, $\mathbf{W}$, then
$$U^{-1}(\boldsymbol{\alpha})\mathbf{V}\cdot\mathbf{W}U(\boldsymbol{\alpha})=(U^{-1}(\boldsymbol{\alpha})\mathbf{V}U(\boldsymbol{\alpha}))\cdot(U^{-1}(\boldsymbol{\alpha})\mathbf{W}U(\boldsymbol{\alpha})=(R(\boldsymbol{\alpha})\mathbf{V})\cdot (R(\boldsymbol{\alpha})\mathbf{W})=\mathbf{V'}\cdot\mathbf{W'}=\mathbf{V}\cdot\mathbf{W}$$
Hence $\mathbf{V}\cdot\mathbf{W}$ is a scalar operator (see spin orbit coupling later). For example, let $\mathbf{V}$, $\mathbf{W}$ be both $\mathbf{J}$, then $\mathbf{J}^2$ is a scalar operator, i.e. $[J_i,J^2]=0$. Together with Proposition 2.1.2.2, we see that we cannot form a complete set of simultaneous eigenstates of $\mathbf{J}^2$ and any one component of $\mathbf{J}$.
\end{eg}
\subsubsection{Orbital angular momentum}
Suppose we translate our system around an $N$-sided regular polygon. As $N\rightarrow\infty$, this will be a circular translation.
\begin{prop}
$\mathbf{L}$ is a Hermitian operator which is a generator of circular translations.
\end{prop}
\begin{proof}
If our system is initially located at some $\mathbf{x}\in\mathbb{R}^3$, then we translate it through $\delta\mathbf{a}=\delta\alpha\mathbf{n}\times\mathbf{x}$, where $\mathbf{n}$ is the unit normal to the plane of the polygon. From Eqn.~\ref{commutator_generator}, our translation is
$$U(\delta\mathbf{a})^{-1}\mathbf{X}U(\delta\mathbf{a})=\mathbf{X}+\delta\alpha(\mathbf{n}\times\mathbf{X})$$
The associated unitary translation operator is thus (discarding second order terms in $\delta\alpha^2$)
$$U(\delta\mathbf{a})=1-\frac{i}{\hbar}\delta\mathbf{a}\cdot\mathbf{P}=1-\frac{i}{\hbar}\delta\alpha(\mathbf{n}\times\mathbf{X})\cdot\mathbf{P}=1-\frac{i}{\hbar}\delta\boldsymbol{\alpha}\cdot(\mathbf{X}\times\mathbf{P})$$
where we identify $\mathbf{L}=\mathbf{X}\times\mathbf{P}$ which is a Hermitian operator by inspection.
\end{proof}
\begin{remarks}
By inspection, $\mathbf{L}$ clearly commutes with itself. We can thus define finite translations around a circle by exponentiating it to find $e^{-i\alpha\mathbf{n}\cdot\mathbf{L}/\hbar}$.
\end{remarks}
\subsubsection{Spin}
In terms of its action on operators built just from $\mathbf{X}$ and $\mathbf{P}$, the action of $\mathbf{J}$ and $\mathbf{L}$ are identical. This may not always be the case if $\mathcal{H}$ is not isomorphic to $L^2(\mathbb{R}^3,d^3x)$. We define the spin operator to be
$$\mathbf{S}=\mathbf{J}-\mathbf{L}$$
We expect this operator to generate a rotation of the body around its own centre of mass.
\begin{prop}\leavevmode
\begin{enumerate}
    \item $[S_i,X_j]=0$
    \item $[S_i,P_j]=0$
    \item $[S_i,S_j]=i\hbar\epsilon_{ijk}S_k$
    \item $[S_i,L_j]=0$
\end{enumerate}
\end{prop}
\begin{proof}\leavevmode
\begin{enumerate}
    \item 
    $$[S_i,X_j]=[J_i-L_i,X_j]=[J_i,X_j]-[L_i,X_j]=i\hbar\epsilon_{ijk}X_k-i\hbar\epsilon_{ijk}X_k=0$$
    \item 
    $$[S_i,P_j]=[J_i-L_i,P_j]=i\hbar\epsilon_{ijk}P_k-i\hbar\epsilon_{ijk}P_k=0$$
    \item
    $$[S_i,S_j]=[J_i,J_j]-[J_i,L_j]-[L_i,J_j]+[L_i,L_j]=i\hbar\epsilon_{ijk}J_k-i\hbar\epsilon_{ijk}L_k+i\hbar\epsilon_{jik}L_k+i\hbar\epsilon_{ijk}L_k=i\hbar\epsilon_{ijk}S_k$$
    \item Since $\mathbf{S}$ commutes with both $\mathbf{X}$ and $\mathbf{P}$, it also commutes with $\mathbf{L}$.
\end{enumerate}
\end{proof}
\begin{remarks}
These commutation relations confirm that the spin operator has nothing to do with an object’s location in or motion through space, but is purely to do with rotating its intrinsic orientation. The final commutation confirms that we can think of a quantum rotation as consisting of a translation of a body’s centre of mass along an arc centred on the origin together with a simultaneous rotation of the body around it’s own centre of mass by the same amount. The order in which we perform these two operations makes no difference.
$$U(\boldsymbol{\alpha})=e^{-i\boldsymbol{\alpha}\cdot\mathbf{J}/\hbar}=e^{-i\boldsymbol{\alpha}\cdot(\mathbf{L}+\mathbf{S})/\hbar}=e^{-i\boldsymbol{\alpha}\cdot\mathbf{S}/\hbar}e^{-i\boldsymbol{\alpha}\cdot\mathbf{L}/\hbar}$$
\end{remarks}
\subsection{Time Translations}
Time translations also form an Abelian group, so will be represented on $\mathcal{H}$ by some unitary $U(t)$ obeying $U(t)U(t')=U(t+t')$ and $U(0)=\Id_\mathcal{H}$.
\begin{prop}
By treating time evolution as a unitary operator, we can obtain the time-dependent Schr\"{o}dinger's equation.
\end{prop}
\begin{proof}
We write $U(t)=e^{-iHt/\hbar}$ and we have $|\psi(t)\rangle=U(t)|\psi(0)\rangle$. From Eqn.~\ref{generator},
$$|\psi(t+\delta t)\rangle-|\psi(t)\rangle=[U(t+\delta t)-U(t)]|\psi(0)\rangle=(U(\delta t)-1)U(t)|\psi(0)\rangle=\bigg(1-i\frac{\delta t}{\hbar}H-1\bigg)|\psi(t)\rangle$$
or equivalently, $i\hbar\frac{\partial}{\partial t}|\psi(t)\rangle=H(t)|\psi(t)\rangle$, which is the time-dependent Schr\"{o}dinger's equation.
\end{proof}
\section{Discrete transformation}\label{sec:discrete_transform}
The most prominent example of a discrete transformation is the parity transformation $\mathcal{P}$, acting on $\mathbb{R}^3$ as $\mathcal{P}:~\mathbf{x}\mapsto-\mathbf{x}$. Since $det(\mathcal{P}) = 1$, this is different from a rotation, so may have consequences that cannot be deduced by considering only rotations. Parity transformations form the $\mathbb{Z}_2$ group since carrying out a non-trivial parity transformation twice just brings us back to where we were.
\begin{defi}[Parity transform]
The parity operator $P:~\mathbb{R}^3\rightarrow\mathbb{R}^3$ is represented by $-\Id_{3\times 3}$. On $\mathcal{H}$, we have a unitary operator $\Pi$ obeying $\Pi^2=\Id_{\mathcal{H}}$. Parity transformations are discrete, so there is no associated generator (cannot take an infinitesimally small parameter for its transformation). The spectrum is $\{+1,-1\}$. We must have
\begin{equation}
\Pi^{-1}X\Pi=PX=-X\label{parity}
\end{equation}
which is true iff the anticommutator vanishes, i.e. $\{\Pi,X\}=0$. 
\end{defi}
\begin{eg}
Likewise, $\Pi^{-1}U(\mathbf{a})\Pi=U(-\mathbf{a})$ where $U$ is a translation generator. And so $\Pi^{-1}\mathbf{P}\Pi=-\mathbf{P}$ for momentum operator. More generally, if $\mathbf{V}$ is any collection of 3 operators such that
$$U^{-1}(\boldsymbol{\alpha})VU(\boldsymbol{\alpha})=R(\boldsymbol{\alpha})\mathbf{V}$$
and $\Pi^{-1}\mathbf{V}\Pi=-\mathbf{V}$, then we call $\mathbf{V}$ a vector operator. Pseudo-vector operators instead have $\Pi^{-1}\tilde{V}\Pi=+\tilde{V}$. $\mathbf{J}$ and $\mathbf{L}$ are pseudovectors. 
$$\Pi^{-1}\mathbf{L}\Pi=\Pi^{-1}(\mathbf{X}\times\mathbf{P})\Pi=(\Pi^{-1}\mathbf{X}\Pi)\times(\Pi^{-1}\mathbf{P}\Pi)=(-\mathbf{X})\times(-\mathbf{P})=+\mathbf{L}$$
For $\mathbf{J}$, observe that since $P=-\Id_{\mathbb{R}^3}$, it commutes with any rotation
$$P^{-1}R(\boldsymbol{\alpha})P=+R(\boldsymbol{\alpha})\implies\Pi^{-1}U(\boldsymbol{\alpha})\Pi=U(\boldsymbol{\alpha})$$
in Hilbert space, and hence $\Pi^{-1}J\Pi=+J$.
\end{eg}
\begin{eg}
Let $|x\rangle$ be an eigenstate of $X$ and $|x'\rangle=\Pi|x\rangle$. We have
$$X|x'\rangle=X\Pi|x\rangle=-\Pi X|x\rangle=-x\Pi|x\rangle=-x|x'\rangle$$
so we must have $|x'\rangle=c|-x\rangle$. Applying $\Pi$ again,
$$\Pi^2|x\rangle=c\Pi|-x\rangle=c^2|x\rangle=|x\rangle$$
which means $c=\pm1$. We fix our convention for $\Pi$ by saying $\Pi|x\rangle=+|x\rangle$ and $\Pi|x\rangle=|-x\rangle$. For a general state $|\psi\rangle\in L^2(\mathbb{R}^3,d^3x)$, we have
$$\langle x|\Pi|\psi\rangle=(\Pi^\dag|x\rangle)^\dag|\psi\rangle=(\Pi|x\rangle)^\dag|\psi\rangle=\langle -x|\psi\rangle$$
So the position wavefunction of $\Pi|\psi\rangle$ is $\psi_\Pi(x)=\psi(-x)$.
\end{eg}
\begin{eg}
Suppose some system is completely described by the wavefunction $\psi(x)=R(r)Y_m^l(x)$, where $Y_m^l(x)$ are the spherical harmonics, then $\psi_\Pi(x)=\psi(-x)=R(r)Y_m^l(-\hat{x})=(-1)^lR(r)Y_m^l(x)=(-1)^l\psi(x)$.
\end{eg}
\begin{remarks}[Intrinsic parity]
More complicated systems with internal structure are described by other (bigger) Hilbert spaces on which $\Pi$ can act differently. We still have $\Pi^2=\Id_\mathcal{H}$, so its spectrum can only ever be $\pm1$, but there can be intrinsic parities associated to the internal structure.
\end{remarks}



\newpage
\chapter{Three Pictures}
\section{Heisenberg's picture}\label{sec:Heisenberg}
\begin{defi}[Heisenberg's picture versus Schr\"{o}dinger's picture]
Instead of working with time-dependent states, we can notice
\begin{equation}
\langle\psi(t)|\mathbf{Q_S}|\psi(t)\rangle=\langle\psi(0)|U^{-1}(t)\mathbf{Q_S}U(t)|\psi(0)\rangle\label{Heisenberg}
\end{equation}
and hence work with $|\psi(0)\rangle$ using time-dependent operators $\mathbf{Q_H}(t)=U^{-1}(t)\mathbf{Q_S}U(t)$ via conjugation. This is known as the Heisenberg's picture, where states are time-independent but operators vary with time. This is in contrast with the Schr\"{o}dinger's picture (states change by time-dependent Schr\"{o}dinger's equation but operators are time-independent).
\end{defi}
\begin{remarks}
Classically, there are also two ways of thinking about time evolution. On the one hand, we can think of a particle moving in some way through the phase space manifold $\mathcal{M}$. If we know it’s location $(\mathbf{x}(t), \mathbf{p}(t))\in\mathcal{M}$ for every time $t$ we can compute any quantity we wish, represented by some function $f:~\mathcal{M}\rightarrow\mathbb{R}$ by evaluating $f$ at the location of our particle, obtaining the value $f(\mathbf{x}(t), \mathbf{p}(t))$.\\[5pt]
Unlike quantum mechanics, Newton’s Laws are deterministic, so, given a force, the entire trajectory is determined by the initial conditions $(\mathbf{x_0}, \mathbf{p_0})$. This suggests a perspective in which the ‘state’ of our particle is simply a choice of initial conditions. These initial conditions do not themselves evolve, rather, it is the quantities we measure that vary in time. Thus, instead of thinking of a physical quantity $f$ as a map from phase space, we treat it just as a map from time, so $f:~[t_0,\infty)\rightarrow\mathbb{R}$.
\end{remarks}
\begin{thm}\leavevmode
\begin{itemize}
    \item $A_H(t=0)=A_S$
    \item $C_S=A_SB_S\implies C_H(t)=A_H(t)B_H(t)$
    \item $[A_S,B_S]=C_S\implies[A_H(t),B_H(t)]=C_H(t)$
    \item $\langle A_H(t)\rangle=\langle A_S\rangle$
\end{itemize}
\end{thm}
\begin{thm}[Heisenberg's equation of motion]
The rate of change of a time-dependent Heisenberg operator $Q_H(t)$ is
\begin{equation}
\frac{d}{dt}Q_H(t)=\frac{i}{\hbar}[H,Q_H(t)]+U^{-1}\frac{\partial Q_S}{\partial t}U(t)\label{EoM}
\end{equation}
where $Q_S$ is the corresponding Schr\"{o}dinger's operator.
\end{thm}
\begin{proof}
Differentiating $Q_H(t)$ (Eqn.~\ref{Heisenberg}) with respect to time
$$\frac{d}{dt}Q_H(t)=\frac{d}{dt}(U^{-1}(t)Q_SU(t))=\frac{i}{\hbar}U^{-1}(t)[H,Q_S]U(t)+U^{-1}\frac{\partial Q_S}{\partial t}U(t)=\frac{i}{\hbar}[H,Q_S(t)]+U^{-1}\frac{\partial Q_S}{\partial t}U(t)$$
where we used the time-dependent Schr\"{o}dinger's equation (Eqn.~\ref{Schrodinger}) and that since $U(t)$ depend only on $H$ $\implies[U(t),H]=0$.
\end{proof}
\begin{eg}
The creation and annihilation operators satisfy the Heisenberg's equation of motion:
$$\frac{d}{dt}a_H(t)=-\frac{i}{\hbar}[a_H(t),H]=U^{-1}[a_S,H]U=-i\omega a_H(t)$$
which has the solution $a_H(t)=e^{-i\omega t}a_S$ since $a_S=a_H(t=0)$. If the energy of the system is reduced by a single excitation at $t$, the phase accrued during forward propagation is greater than that during backward propagation, regardless of the original state.
\end{eg}
\section{Symmetries and Conserved Quantities}\label{sec:symmetries_conserved}
\begin{defi}[Conserved operators]
Suppose an operator $Q$ does not depend on $t$ even in the Heisenberg picture, i.e. $Q(t)=U^{-1}(t)QU(t)=Q$ $\forall t$ true iff $[Q,H]=0$. Such operators are said to be conserved. If $Q$ is Hermitian and conserved, then it is useful to choose its eigenstates to be a basis of $H$ since system will remain in the eigenstate of $Q$ at all subsequent times.
\end{defi}
\begin{remarks}
For this reason, it’s usually sensible to expand our states in a basis of eigenstates of a maximal set of conserved operators, rather than a maximal commuting set of any old operators.
\end{remarks}
\begin{defi}[Symmetry transformation]
Now suppose the Hamiltonian is invariant under some transformation represented by $U(\theta):~\mathcal{H}\rightarrow\mathcal{H}$ with generator $T$. Then $H'=U(\theta)^{-1}HU(\theta)=H$. The Hamiltonian is invariant under such transformations. Equivalently, for an infinitesimal symmetry transformation $T$, we must have $[H,T]=0$. This is the same equation as for conserved quantities, so symmetries of a given $H(X,P,...)$ correspond to conserved quantities.
\end{defi}
\begin{eg}
Consider the example of the free particle again, where $H=\frac{P^2}{2m}$ is invariant under translations and rotations, i.e. $[H,P]=0$ and $[H,J]=0$. So momentum $P$ and angular momentum $J$ will be conserved. Since $[J_i,P_j]=i\hbar\epsilon_{ijk}P_k$, we cannot find a simultaneous eigenbasis for $J$ and $P$. We usually label states of rotational invariant $H$ by $|n,l,m\rangle$, where
$$H|n,l,m\rangle=E(n)|n,l,m\rangle,\quad J^2|n,l,m\rangle=\hbar^2l(l+1)|n,l,m\rangle,\quad J|n,l,m\rangle=m\hbar|n,l,m\rangle$$
\end{eg}
\begin{remarks}\leavevmode
\begin{enumerate}
\item Symmetry transformations are either active (move the system, keeping the axes fixed) or passive (move the axes, keeping the system fixed).
\item Most symmetry transformations are unitary, with the exception of time-reversal symmetry. The weak interactions were found not to be invariant under time-reversal symmetry.
\item Symmetries are associated with conservation laws, and leads to degeneracies. Often, symmetries determine which processes are allowed or forbidden.
\end{enumerate}
\end{remarks}
\section{Interaction Picture}\label{sec:interaction}
If $H(t)=H_0+\Delta(t)$ where $H_0$ is independent of time (and can be solved) and $\Delta$ is time-dependent even in the Schr\"{o}dinger's picture. 
\begin{defi}[Interaction picture state]
Define the interaction picture state
\begin{equation}
|\Psi_I(t)\rangle=U_0^{-1}(t)|\Psi_S(t)\rangle,\quad U_0(t)=e^{-iH_0t/\hbar}\label{interactionpicture}
\end{equation}
is the time evolution operator just for $H_0$, $|\Psi_S(t)\rangle$ is the Schr\"{o}dinger picture state for the full TDSE using $H(t)=H_0+\Delta(t)$.
\end{defi}
\begin{prop}
The Schr\"{o}dinger's equation in the interaction picture is
\begin{equation}
i\hbar\frac{\partial}{\partial t}|\Psi_I(t)\rangle=U_0^{-1}(t)\Delta(t)U_0(t)|\Psi_I(t)\rangle\label{interactionpicture2}
\end{equation}
\end{prop}
\begin{proof}
In the absence of $\Delta(t)$, the Schr\"{o}dinger state evolves via Eqn.~\ref{Schrodinger}, i.e. $|\Psi_S(t)\rangle=U_0(t)|\Psi_S(0)\rangle$ such that by Eqn.~\ref{interactionpicture}, $|\Psi_I(t)\rangle_{\Delta=0}=|\Psi_S(0)\rangle$ is time-independent. Thus $|\Psi_I(t)\rangle$ evolves only due to the perturbation. 
\begin{align}
i\hbar\frac{\partial}{\partial t}|\Psi_I(t)\rangle&=-H_0e^{iH_0t/\hbar}|\Psi_S(t)\rangle+e^{iH_0t/\hbar}(H_0+\Delta(t))|\Psi_S(t)\rangle\nonumber\\&=\Delta(t)e^{iH_0t/\hbar}|\Psi_S(t)\rangle=U_0^{-1}(t)\Delta(t)U_0(t)|\Psi_I(t)\rangle\nonumber
\end{align}
where we took the time derivative.
\end{proof}
\begin{prop}
The interaction picture of an arbitrary operator is
\begin{equation}
A_I(t)=U_0^{-1}(t)A_SU_0(t)\label{interactionpicture3}
\end{equation}
while the corresponding Heisenberg's equation of motion gives
\begin{equation}
\frac{d}{dt}A_I(t)=\frac{i}{\hbar}[H_0,A_I(t)]+U_0^{-1}(t)\frac{\partial A_S(t)}{\partial t}U_0(t)\label{interactionpicture4}
\end{equation}
\end{prop}
\begin{proof}
Since the expectation value is independent of the picture, we have
$$\langle\Psi_S(t)|A|\Psi_S(t)\rangle=\langle\Psi_I(t)|U_0^{-1}(t)AU_0(t)|\Psi_I(t)\rangle\implies A_I(t)=U_0^{-1}(t)A_SU_0(t)$$
Take the time derivative to find the equation of motion.
\end{proof}
\begin{remarks}
Schr\"{o}dinger states evolve according to the Hamiltonian. Schr\"{o}dinger operators do not evolve unless they are explicitly time-dependent. Heisenberg states do not evolve but Heisenberg operators evolve like $A_H(t)=U^{-1}(t)A_SU(t)$. Interaction states evolve only through the time-dependent perturbation $\Delta(t):=H(t)-H_0$, i.e. $\Delta_I(t)=U_0^{-1}(t)\Delta(t)U_0(t)$. Interaction operators evolve only through $U_0(t)$ (indirectly via $H_0$) and may have explicit time-dependence.
\end{remarks}
\begin{prop}
The time evolution operator may be written as a Dyson series.
\begin{equation}
U_I(t)=\sum_{n=0}^\infty(-i/\hbar)^n\int_0^t\int_0^{t_1}\int_0^{t_{n-1}}U_0^{-1}(t_1)\Delta(t)U_0(t_1-t_2)\Delta(t_2)...U_0(t_{n-1}-t_n)\Delta(t_n)U_0(t_n)dt_n...dt_1\label{Dyson}
\end{equation}
\end{prop}
\begin{proof}
If we define $\Delta_I(t):=U_0^{-1}(t)\Delta(t)U_0(t)$ as the perturbation in the interaction picture, we can say that $|\Psi_I(t)\rangle=U_I(t)|\Psi(0)\rangle$ for some unitary evolution operator that depends only on $\Delta(t)$. However, we should be careful: it is not usually true that $U_I(t)=\exp(-\frac{i}{\hbar}\int_0^t\Delta(t')dt')$. This is not true since typically $[\Delta(t'),\Delta(t'')]\neq0$. The Schr\"{o}dinger's equation in the interaction picture (Eqn.~\ref{interactionpicture2}) may be rewritten as
$$i\hbar\frac{\partial}{\partial t}|\Psi_I(t)\rangle=\Delta_I(t)|\Psi_I(t)\rangle=\Delta_I(t)U_I(t)|\Psi(0)\rangle$$
which is true $\forall|\Psi(0)\rangle\in\mathcal{H}$. We thus obtain an integral equation
$$U_I(t)=1-\frac{i}{\hbar}\int_0^t\Delta_I(t')U_I(t')dt'$$
This equation is exact but not immediately useful. To make progress, we approximate by repeatedly substituting the LHS of the integral equation into the integral on the RHS, hence obtaining a time-nested integral.
\begin{eqnarray}
U_I(t)&=&1-\frac{i}{\hbar}\int_0^t\Delta_I(t_1)dt_1+(-i/\hbar)^2\int_0^t\Delta_I(t_1)\int_0^{t_1}\Delta_I(t_2)U_I(t_2)dt_2dt_1\nonumber\\&=&\sum_{n=0}^\infty(-i/\hbar)^n\int_0^t\int_0^{t_1}\int_0^{t_{n-1}}U_0^{-1}(t_1)\Delta(t)U_0(t_1-t_2)\Delta(t_2)...U_0(t_{n-1}-t_n)\Delta(t_n)U_0(t_n)dt_n...dt_1\nonumber
\end{eqnarray}
Thus, at $n$th order in the expansion, the state evolves according to $H_0$ except for $n$ `strikes' of the perturbation.
\end{proof}
\begin{remarks}
We sometimes write
$$U_I(t)=\mathcal{T}[e^{-(i/\hbar)\int_0^t\Delta_I(t')dt'}]$$
where $\mathcal{T}$ is a time-ordered operator (more later), and this series is known as the Dyson series. We sometimes call $U_I(t)$ to be the scattering matrix.
\end{remarks}
\section{Time Ordering}\label{sec:timeordering}
Recall that time evolution is achieved by acting the state using a unitary operator in Hilbert space.
\begin{defi}[Time evolution operator]
The state vector at $t$ is related to that at some earlier time $t_0$ through
$$|\psi(t)\rangle=U(t,t_0)|\psi(t_0)\rangle$$
\end{defi}
For a time-independent Hamiltonian, we may separate the time dependence from the state, i.e. temporal behaviour is described by a superposition of `frequency' components. As time proceeds, the frequency components changes in phase relative to each other, giving rise to an evolving state vector. If only one frequency is present, the energy is certain and the state is a stationary state. However, in general, the Hamiltonian is time dependent.
\begin{thm}
The time evolution operator is
\begin{itemize}
    \item If $H$ is time-independent, $U(t,t_0)=e^{-iH(t-t_0)/\hbar}$.
    \item If $[H(t_1),H(t_2)]=0$ $\forall t_1\neq t_2$, $U(t,t_0)=e^{-i\hbar^{-1}\int_{t_0}^tH(t')dt'}$.
    \item If $[H(t_1),H(t_2)]\neq 0$, then $U(t,t_0)$ is a time-ordered exponential, i.e.
    \begin{equation}
    U(t,t_0)=\mathcal{T}\bigg[\exp\bigg\{\bigg(\frac{-i}{\hbar}\bigg)\int_{t_0}^t H(t')dt'\bigg\}\bigg]\label{time-ordered}
    \end{equation}
\end{itemize}
\end{thm}
\begin{proof}
If $H$ is time-independent, we have from Schr\"{o}dinger equation $\frac{dU}{dt}=-i\hbar^{-1}HU$ and can be solved easily. If $H$ is time-dependent, we verify the given claim. Let $R(t):=-\frac{i}{\hbar}\int_{t_0}^tH(t')dt'$, then by Fundamental Theorem of Calculus, $R'=-\frac{i}{\hbar}H(t)$ and we can show $[R'(t),R(t)]=0$ as long as $[H(t),H(t')]=0$. Then, we try $\frac{d}{dt}e^R$, which gives us $R'e^R$ and hence recover Schr\"{o}dinger equation, as desired. Now, if Hamiltonians at different times don't commute, we have
\begin{align}
U(t,t_0)&=1-\frac{i}{\hbar}\int_{t_0}^tH(t')U(t',t_0)dt'\nonumber\\&=1+(-i\hbar^{-1})\int_{t_0}^tH(t_1)dt_1+(-i\hbar^{-1})^2\int_{t_0}^t\int_{t_0}^{t_1}H(t_1)H(t_2)dt_2dt_1+\dots\nonumber
\end{align}
where without loss of generality, let $t\geq t_0$. The term time-ordered refers to the fact that in the $n$th term of the series, we have a product of $H(t_1)H(t_2)...H(t_n)$ of non-commuting operators with integration ranges that force ordered times $t_1\geq t_2\geq...\geq t_n$. Consider for instance:
\begin{align}
\int_{t_0}^t\int_{t_0}^{t_1}H(t_1)H(t_2)dt_1dt_2&=\int_{t_0}^t\int_{t_0}^tH(t_1)H(t_2)\Theta(t_1-t_2)dt_1dt_2\nonumber\\&=\int_{t_0}^{t}\int_{t_0}^tH(t_2)H(t_1)\Theta(t_2-t_1)dt_1dt_2\nonumber\\&=\frac{1}{2}\int_{t_0}^t\int_{t_0}^t\mathcal{T}[H(t_2)H(t_1)]dt_1dt_2\nonumber
\end{align}
where $\Theta(t_1-t_2)$ is a step function that only gives 1 if $t_1\geq t_2$, which allows the limit on the second integral to be extended to $t$. The third line follows by exchanging of variables. The last line includes the time ordering operator which gives $H(t_2)H(t_1)$ if $t_2>t_1$ and $H(t_1)H(t_2)$ if $t_2<t_1$. By induction, we can show that
$$\int_{t_0}^t\int_{t_0}^{t_1}\int_{t_0}^{t_2}\dots\int_{t_0}^{t_{n-1}}H(t_1)H(t_2)\dots H(t_n)dt_1\dots dt_2=\mathcal{T}\bigg[\frac{1}{n!}\int_{t_0}^t\dots\int_{t_0}^t H(t_1)H(t_2)\dots H(t_n)dt_1\dots dt_n\bigg]$$
Hence, $U(t,t_0)$ is a series, i.e.
$$U(t,t_0)=\mathcal{T}\bigg[\sum_{n=0}^\infty\bigg(-\frac{i}{\hbar}\bigg)^n\frac{1}{n!}\int_{t_0}^t\dots\int_{t_0}^t H(t_1)H(t_2)\dots H(t_n)dt_1dt_2\dots dt_n\bigg]$$
where we note that for the $O(1/\hbar^n)$ term, there are a total of $n!$ permutations we have to account for. Finally, this is basically the power series of the exponential of an operator.
\end{proof}
\begin{remarks}
Now suppose $t<t_0$ instead, then $\mathcal{T}$ is the anti-time ordering operator which gives $H(t_1)H(t_2)$ if $t_2>t_1$ and vice-versa.
\end{remarks}
\newpage
\chapter{Mixed States}
\section{Density Matrices and Entropy}
\begin{defi}[Density operator]
We define the density operator $\rho:~\mathcal{H}\rightarrow\mathcal{H}$ to be
\begin{equation}
\rho=\sum_ip_i|\Psi_i\rangle\langle\Psi_i|\label{density_operator}
\end{equation}
where $p_i$ refers to some classical uncertainty in our knowledge of the system.
\end{defi}
\begin{prop}\leavevmode
\begin{enumerate}
    \item $\rho^\dag=\rho$
    \item $\Tr_{\mathcal{H}}\rho=1$
    \item $\langle\chi|\rho|\chi\rangle\geq0$ $\forall|\chi\rangle\in\mathcal{H}$
\end{enumerate}
Essentially, we summarize the above as $\rho\geq0$ (positive semi-definite operator).
\end{prop}
\begin{proof}\leavevmode
\begin{enumerate}
    \item The density operator Eqn.~\ref{density_operator} obeys $\rho^\dag=\rho$ since probabilities are real, i.e. $p_i\in\mathbb{R}$ $\forall i$.
    \item $\Tr_{\mathcal{H}}\rho=1$ since probabilities sum to 1, i.e. $\sum_ip_i=1$.
    \item $\langle\chi|p|\chi\rangle\geq0$ $\forall|\chi\rangle\in\mathcal{H}$ since probabilities are non-negative. 
\end{enumerate}
\end{proof}
\begin{prop}
The expectation of an operator $Q$ written in terms of the density matrix is
\begin{equation}
\langle Q\rangle=\Tr_{\mathcal{H}}(Q\rho)=\sum_\alpha p_\alpha\langle\psi_\alpha|Q|\psi_\alpha\rangle\label{expectation}
\end{equation}
\end{prop}
\begin{proof}
Use the definition of trace, $\Tr_{\mathcal{H}}(Q\rho)=\sum_n\langle q_n|\rho Q|q_n\rangle=\sum_np_\alpha\langle q_n|\psi_\alpha\rangle\langle\psi_\alpha|Q|q_n\rangle$, where we used $|\psi_\alpha\rangle\langle\psi_\alpha|=\Id_{\mathcal{H}}$. The result follows from the cyclic property of trace.
\end{proof}
\begin{defi}[Pure state, mixed state]
If we have perfect knowledge of the system state $|\psi\rangle$ (with probability 1), then $\rho=|\psi\rangle\langle\psi|$. This is a pure state. However, if more than one $p_\alpha>0$ for $\rho=\sum_\alpha p_\alpha|\psi_\alpha\rangle\langle\psi_\alpha|$, then the system is in a mixed state.
\end{defi}
\begin{remarks}
Quantum states are added coherently by adding state vectors, whereas quantum states are added incoherently by adding density operators. The resultant vector depends on the relative quantum phase difference between the two states which is measurable in experiments, i.e.
$$|\psi\rangle=e^{i\phi_1}|\psi_1\rangle+e^{i\phi_2}|\psi_2\rangle=e^{i\phi_1}[|\psi_1\rangle+e^{i(\phi_2-\phi_1)}|\psi_2\rangle]\implies \rho=c_1^*c_1|\psi_1\rangle\langle\psi|+c_2^*c_2|\psi_2\rangle\langle\psi_2|$$
where the resultant density operator does not depend on the relative phase difference between the states, i.e. no experiments can detect any quantum correlations.
\end{remarks}
\begin{eg}[Bloch sphere]
Suppose we have a two-state system (qubit) $\{|\uparrow\rangle,|\downarrow\rangle\}$ with $\mathcal{H}\simeq\mathbb{C}^2$. If $\rho=|\uparrow\rangle\langle\uparrow|$, system is definitely in the state $|\uparrow\rangle$, hence a pure state. If $\rho=\frac{1}{2}|\uparrow\rangle\langle\uparrow|+\frac{1}{2}|\downarrow\rangle\langle\downarrow|=\frac{1}{2}\Id$, hence equally likely to be in $|\uparrow\rangle$ or $|\downarrow\rangle$. We are not saying $|\Psi\rangle=\frac{1}{\sqrt{2}}(|\uparrow\rangle+|\downarrow\rangle)$! This is an impure or a mixed state.\\[5pt]
We could also have $\rho=\frac{1}{2}|\uparrow\rangle\langle\uparrow|+\frac{1}{2}|\uparrow_x\rangle\langle\uparrow_x|$, where $S_x|\uparrow_x\rangle=\frac{\hbar}{2}|\uparrow_x\rangle$. Note that $\langle\uparrow_x|\uparrow\rangle\neq0$. In this case, since $|\uparrow_x\rangle=\frac{1}{\sqrt{2}}(|\uparrow\rangle+|\downarrow\rangle)$, we can write this as
$$\rho=\frac{1}{4}\Id_{\mathcal{H}}+\frac{1}{2}|\uparrow\rangle\langle\uparrow|+\frac{1}{4}|\uparrow\rangle\langle\downarrow|+\frac{1}{4}|\downarrow\rangle\langle\uparrow|\implies\rho=\begin{pmatrix}3/4&1/4\\1/4&1/4\\\end{pmatrix}$$
in the $\{|\uparrow\rangle,|\downarrow\rangle\}$ basis. With this $\rho$, we have $\langle S_x\rangle=0.5\hbar=\langle S_z\rangle$ but $\langle S_y\rangle=0$ since we do not know anything of its spin in the $y$ direction.
\end{eg}
\begin{remarks}
This terminology of purity of a state refers to our incomplete knowledge of the system. The state is a precise state but we are not certain which state it is.
\end{remarks}
\begin{prop}
For a generic two-level system, we have the projector Eqn.~\ref{genericBloch2} to give $\rho=\frac{1}{2}(\Id+\mathbf{b}\cdot\boldsymbol{\sigma})$.
\begin{enumerate}
    \item At least one eigenvalue of $\rho$ is positive. Moreover, both eigenvalues of $\rho$ are non-negative if $\det(\rho)=0.25(1-|\mathbf{b}|^2)\geq0$.
    \item If $|\mathbf{b}|=1$, then the state is pure. Otherwise if $|\mathbf{b}|<1$, then the state is impure.
\end{enumerate}
\end{prop}
\begin{proof}
First one follows from $\Tr_\mathcal{H}\rho=1$ and the given condition $\det\rho\geq0$. For the second, if our $\rho$ has $|\mathbf{b}|=1$, then the eigenvalues are 1 and 0, so the state must be pure, i.e. $\exists|\uparrow_\mathbf{b}\rangle\in\mathcal{H}$ such that $\rho=\frac{1}{2}(\Id+\mathbf{\hat{b}}\cdot\boldsymbol{\sigma})=|\uparrow_{\mathbf{b}}\rangle\langle\uparrow_{\mathbf{b}}|$. In fact, this state indeed has spin $+0.5\hbar$ along the direction $\mathbf{\hat{b}}$. If $|\mathbf{b}|<1$, then both eigenvalues of $\rho$ are positive, so state is necessary impure since we cannot cast $\rho$ in the form $|\uparrow_\mathbf{b}\rangle\langle\uparrow_\mathbf{b}|$ for any direction $\mathbf{\hat{b}}$. 
\end{proof}
\begin{remarks}
For both mixed and pure states, the direction of $\mathbf{b}$ define the polarization of the state (preferential measurement direction).  $\Tr_\mathcal{H}(\sigma)=0$ and $\Tr_{\mathcal{H}}(\Id)=2$, fixing the overall factor.
\end{remarks}
\begin{defi}[von Neumann entropy]
To quantify how pure/mixed our system (how imperfect our knowledge of the system) is, we define the von Neumann entropy 
\begin{equation}
S=-\Tr_{\mathcal{H}}(\rho\ln\rho)\label{vonNeumann}
\end{equation}
\end{defi}
\begin{prop}$S=-\sum_r\rho_r\ln\rho_r$
\end{prop}
\begin{proof}
Let $|\phi_r\rangle$ be the orthonormal eigenstates of $\rho=\rho^\dag$ with eigenvalue $\rho_r$. Then, the von Neumann entropy is
$$-\Tr_{\mathcal{H}}(\rho\ln\rho)=-\sum_n\langle n|\bigg(\sum_r\rho_r|\phi_r\rangle\langle\phi_r|\bigg)\bigg(\sum_{r'}\ln\rho_{r'}|\phi_{r'}\rangle\langle\phi_{r'}|\bigg)|n\rangle=-\sum_r\rho_r\ln\rho_r|\langle n|\phi_r\rangle|^2=-\sum_r\rho_r\ln\rho_r$$
since $\sum_R|\langle n|\phi_r\rangle|^2=1$, if bases $\{|n\rangle\}$ and $\{|\phi_r\rangle\}$ both properly normalized. 
\end{proof}
\begin{remarks}\leavevmode
\begin{enumerate}
\item We can see that since each $0\leq\rho_r\leq 1$, $S(\rho)\geq0$ with equality iff the system is pure. 
\item $S(\rho)$ is defined in a basis-independnet way.
\end{enumerate}
\end{remarks}
\begin{prop}
The maximum entropy is $S(\rho_{max})=\ln\dim(\mathcal{H})$, where $\mathcal{H}$ is the Hilbert space.
\end{prop}
\begin{proof}
To maximize the entropy, extremize $S(\rho)$ over $\rho$, subject to $\Tr(\rho)=1$,
$$0=\delta(-\Tr(\rho\ln\rho)+\lambda(\Tr\rho-1))=-\Tr(\delta\rho\ln\rho+\rho\rho^{-1}\delta\rho-\lambda\delta\rho)+\delta\lambda(\Tr(\rho)-1)$$
At an extremum, need both $\delta\rho$ and $\delta\lambda$ terms to vanish, which leads to $\rho=\Id e^{\lambda-1}$ and $\Tr(\rho)=1$ respectively. So the proportionality factor is fixed and $\rho_{max}=\frac{1}{\dim(\mathcal{H})}\Id_{\mathcal{H}}$. This is the density operators when all the states are equally likely. We have
$$S(\rho_{max})=\frac{1}{\dim(\mathcal{H})}\bigg[-\Tr_{\mathcal{H}}(\Id)\ln\frac{1}{\dim \mathcal{H}}\bigg]=\ln\dim(\mathcal{H})$$
where the trace of identity is the dimension of the Hilbert space.
\end{proof}
Density operators are very important in statistical physics, because we can't know the exact quantum state of $10^{23}$ atoms.
\begin{prop}[Gibbs distribution]
Suppose we know our system has fixed average energy $U=\Tr_{\mathcal{H}}(\rho H)$ where $H$ is the Hamiltonian, then the $\rho$ with maximum entropy is 
\begin{equation}
\rho_{Gibbs}=\frac{e^{-\beta H}}{\Tr_{\mathcal{H}}e^{-\beta H}}=\frac{1}{Z(\beta)}\sum_ne^{-\beta E_n}|E_n\rangle\langle E_n|\label{Gibbs}
\end{equation}
where $Z(\beta):=\Tr_{\mathcal{H}}(e^{-\beta H})$ is the partition function of our system.
\end{prop}
\begin{proof}
Consider a first-order variation again
$$0=\delta[-\Tr_{\mathcal{H}}\rho\ln\rho-\lambda(\Tr_{\mathcal{H}}\rho-1)-\beta(\Tr_{\mathcal{H}}(\rho H)-U)]$$
which gives
$$0=-\Tr_{\mathcal{H}}[\delta\rho(\ln\rho+1+\beta H+\lambda)],\quad 0=\delta\lambda(\Tr_{\mathcal{H}}(\rho)-1),\quad 0=\delta\beta(\Tr_{\mathcal{H}}(\rho H)-U)$$
we obtain respectively $\rho=e^{-\beta H}e^{-\lambda-1}$, $\Tr_{\mathcal{H}}\rho=1$ and $\Tr_{\mathcal{H}}(\rho H)=U$. We have $\rho_{Gibbs}=\frac{e^{-\beta H}}{\Tr_{\mathcal{H}}[e^{-\beta H}]}$. So states of high energy are exponentially suppressed. Here, $\frac{1}{\beta}=k_BT$ plays the role of temperature.
\end{proof}
\begin{eg}
Consider $H$ to be the Hamiltonian of the simple harmonic oscillator. Evaluate the partition function $\frac{1}{Z}=2\sinh(\hbar\omega/2k_BT)$, and so $\rho=2\sinh(\hbar\omega/2k_BT)\sum_{n=0}^\infty e^{-E_n/k_BT}|\phi_n\rangle\langle\phi_n|$. The energy expectation is then
$$\langle E\rangle=\Tr_{\mathcal{H}}[\rho H]=\hbar\omega\bigg[\frac{1}{2}+\frac{1}{e^{\hbar\omega/k_BT}-1}\bigg]$$
\end{eg}
Suppose $\mathcal{H}_{AB}=\mathcal{H}_A\otimes\mathcal{H}_B$ and we only keep track of $A$, treating $B$ as the environment. Recall that $|\psi\rangle\in\mathcal{H}_A\otimes\mathcal{H}_B$ is called entangled if it cannot be written as a simple product $|\psi\rangle=|\phi\rangle|\chi\rangle$ for $|\phi\rangle\in\mathcal{H}_A$ and $|\chi\rangle\in\mathcal{H}_B$. Reduced density operators enables us to obtain expectation values of a subsystem's observables.
\begin{defi}[Reduced density operator]
If we were to measure some property $Q$ of $A$, represented by $Q\otimes\Id_B$ on $\mathcal{H}_{AB}$. We expect to obtain
\begin{equation}
\overline{Q}=\Tr_{\mathcal{H}_{AB}}(\rho_{AB}(Q\otimes\Id_B))=\Tr_{\mathcal{H}_A}(\Tr_{\mathcal{H}_B}(\rho_{AB}(Q\otimes\Id_B)))=\Tr_{\mathcal{H}_A}(\rho_AQ)\label{reduced}
\end{equation}
where $\rho_A=\Tr_{\mathcal{H}_B}(\rho_{AB})$ is the reduced density operator for subsystem $A$. Note that the traces were performed independently.
\end{defi}
\begin{eg}
Consider two 2-state systems and the state $|\Psi\rangle=\frac{1}{\sqrt{2}}(|\uparrow\downarrow\rangle-|\downarrow\uparrow\rangle)$, then the density operator is
$$\rho_{AB}=\frac{1}{2}\bigg(|\uparrow\downarrow\rangle-|\downarrow\uparrow\rangle\bigg)\bigg(\langle\uparrow\downarrow|-\langle\downarrow\uparrow|\bigg)=\frac{1}{2}\bigg(|\uparrow\downarrow\rangle\langle\uparrow\downarrow|+|\downarrow\uparrow\rangle\langle\downarrow\uparrow|-|\downarrow\uparrow\rangle\langle\uparrow\downarrow|-|\uparrow\downarrow\rangle\langle\downarrow\uparrow|\bigg)$$
Tracing over directions of the second spins, we have
$$\rho_A=\Tr_{\mathcal{H}_B}(\rho_{AB})=\frac{1}{2}\bigg(|\downarrow\rangle_A\langle\downarrow|_A-|\uparrow\rangle_A\langle\uparrow|_A\bigg)$$
which means $\rho_A=\frac{1}{2}\Id_A$, which is a mixed state and indeed is the state of maximum entropy. Thus, if the total state is correlated/entangled, the reduced density matrix of a subsystem will be mixed. 
\end{eg}
\begin{remarks}
In fact, the entropy follows a subadditivity property, i.e. $S(\rho_{AB})\leq S(\rho_A)+S(\rho_B)$. The equality is saturated iff the two subsystems are unentangled so that $\rho_{AB}=\rho_A\otimes\rho_B$.
\end{remarks}
\section{Time Evolution of Density Operators}
\begin{prop}
The equation of motion for the density operator is
\begin{equation}
i\hbar\frac{d\rho}{dt}=[H,\rho(t)]\label{EOMdensity}
\end{equation}
\end{prop}
\begin{proof}
In the Schr\"{o}dinger's picture, the density operator evolves as
$$\rho(t)=U(t)|\Psi_0\rangle\langle\Psi_0|U^{-1}(t)=U(t)\rho_0U^{-1}(t),\quad U(t)=e^{-iHt/\hbar}$$
$$\implies\frac{\partial\rho}{\partial t}=\frac{\partial U}{\partial t}\rho_0U^{-1}+U\rho_0\frac{\partial U^{-1}}{\partial t}=-i\frac{H}{\hbar}U\rho _0U^{-1}+U\rho_0\frac{i}{\hbar}HU^{-1}$$
But $U$ and $H$ commute, so multiplying by $i\hbar$, we then have
$$i\hbar\frac{\partial\rho}{\partial t}=HU\rho_0U^{-1}+U\rho_0HU^{-1}=U(H\rho_0-\rho_0H)U^{-1}=U[H,\rho_0]U^{-1}=[H,\rho(t)]$$
where we used the definition for Heisenberg operator.
\end{proof}
\begin{remarks}
The equation of motion for the density operator is the quantum analogue of the Liouville's equation $\frac{d\rho}{dt}=\{H,\rho\}$ in classical dynamics for the probability density $\rho$ on phase space.
\end{remarks}
\begin{prop}
Even when we have imperfect knowledge of a system, the expected rate of change of an operator $Q$ is the appropriately weighted average of the rates of change of $Q$ for each of the possible states of the system.
\begin{equation}
i\hbar\frac{d}{dt}\Tr_{\mathcal{H}}(\rho Q)=\Tr_{\mathcal{H}}(\rho[Q,H])\label{EOM_trace}
\end{equation}
\end{prop}
\begin{proof}
Using Eqn.~\ref{EOMdensity}, $i\hbar\frac{d}{dt}\Tr_{\mathcal{H}}(\rho Q)=\Tr_{\mathcal{H}}([H,\rho]Q)=\Tr_{\mathcal{H}}(\rho[Q,H])$. Last equality follows from the cyclic property of trace.
\end{proof}
\begin{prop}
The evolution of a reduced density matrix of a composite system with initial density operator 
$$\rho_{AB}(t=0)=|\phi\rangle|\chi\rangle\langle\chi|\langle\phi|:=|\Psi_0\rangle\langle\Psi_0|$$
where subsystems A and B are unentangled, is
$$\rho_A(t)=\sum_bM_b(t)\rho_A(0)M_b^\dag(t),\quad M_b(t):~\mathcal{H}_A\rightarrow\mathcal{H}_A,~M_b(t):=\Tr_{\mathcal{H}_B}[U_{AB}(t)|\chi\rangle\langle b|]$$
where $\{|b\rangle\}$ is the orthonormal basis of $\mathcal{H}_B$.
\end{prop}
\begin{proof}
We have the reduced density operator for A to be
$$\rho_{A}(t)=\Tr_{\mathcal{H}_B}[U_{AB}(t)|\Psi_0\rangle\langle\Psi_0|U_{AB}^{-1}(t)]=\sum_b\langle b|U_{AB}(t)|\Psi_0\rangle\langle\Psi_0|U_{AB}^{-1}(t)|b\rangle$$
where we perform the trace over $\{|b\rangle\}$, an orthonormal basis of $\mathcal{H}_B$. Since $U_{AB}(t)$ is unitary, we can show that the newly defined operator $M_B(t)$ obeys a completeness relation.
$$\sum_bM_b^\dag(t)M_b(t)=\sum_b\langle\chi|U_{AB}^\dag(t)|b\rangle\langle b|U_{AB}(t)|\chi\rangle=\Id_A$$
it then follows that $\rho_A(t)=\sum_bM_b(t)\rho_A(0)M_b^\dag(t)$.
\end{proof}
\begin{eg}
If $H_{AB}=H_A\otimes\Id_B+\Id_A\otimes H_B$ where our subsystems don't interact with one another. Then, the unitary is $U_{AB}(t)=U_A(t)\otimes U_B(t)$ and we have $M_b(t)=\langle b|U_B(t)|m\rangle U_A(t)$.
By completeness, we must have $\rho_A(t)=U_A(t)\rho_A(0)U_A^{-1}(t)$. So, if $\rho_0$ is initially both pure and unentangled (with no interaction with the environment), it will remain pure and unentangled $\forall t$.
\end{eg}
\begin{defi}[Decoherence]
To make a measurement, we must have some interaction between A (system) and B (apparatus). Thus during measurement, A and B do become entangled. This is known as decoherence-interaction between our system and environment/apparatus, cause A to be described by a mixed $\rho_A$.
\end{defi}
%\begin{eg}
%Let's suppose our system $A$ consists of a single qubit, either $|\uparrow\rangle$ or $|\downarrow\rangle$. Imagine the environment has only three possible states, $|0\rangle$, $|1\rangle$ and $|2\rangle$. An ideal measurement will change the state of the measuring apparatus without affecting the system A. Let's suppose the measurement process is described by a unitary operator $U$, that acts like
%$$U|\uparrow\rangle\otimes|0\rangle=|\uparrow\rangle(\sqrt{1-p}|0\rangle+\sqrt{p}|1\rangle),\quad U|\downarrow\rangle\otimes|0\rangle=|\downarrow\rangle(\sqrt{1-p}|0\rangle+\sqrt{p}|2\rangle)$$
%Suppose the system $A$ is initially described by some density matrix
%$$\rho_A=\begin{pmatrix}\rho_{\uparrow\uparrow}&\rho_{\uparrow\downarrow}\\\rho_{\downarrow\uparrow}&\rho_{\downarrow\downarrow}\\\end{pmatrix}$$
%For this evolution, we have
%$$M_0=\langle 0|U|0\rangle=\sqrt{1-p}\Id_A,\quad M_1=\langle 1|U|0\rangle=\sqrt{p}|\uparrow\rangle\langle\downarrow|,\quad M_2=\langle 2|U|0\rangle=\sqrt{p}|\downarrow\rangle\langle\downarrow|$$
%where they obey $\sum_iM_i^\dag M_i=\Id_A$. Contact with our measuring apparatus thus causes $\rho_A$ to evolve as
%$$U:~\begin{pmatrix}\rho_{\uparrow\uparrow}&\rho_{\uparrow\downarrow}\\\rho_{\downarrow\uparrow}&\rho_{\downarrow\downarrow}\\\end{pmatrix}\mapsto \begin{pmatrix}\rho_{\uparrow\uparrow}&(1-p)\rho_{\uparrow\downarrow}\\(1-p)\rho_{\downarrow\uparrow}&\rho_{\downarrow\downarrow}\\\end{pmatrix}$$
%i.e. suppressing the off-diagonal components. For successive evolution, we discretize an evolution of time $t$ into $N$ discrete steps $\delta t$, then the off-diagonal terms will thus be suppressed by $(1-p)^n\approx e^{-pt/\delta t}$ for large $N:=t/\delta t$. In particular, if we initially prepare A to be in the superposition
%$$|\psi\rangle=a|\uparrow\rangle+b|\downarrow\rangle,\quad |a|^2+|b|^2=1$$
%then eventually, $A$'s density matrix will become
%$$\lim_{t\rightarrow\infty}\rho_A(t)=\begin{pmatrix}|a|^2&0\\0&|b|^2\\\end{pmatrix}$$
%The matrix now only have real entries and this is called phase damping.
%\end{eg}
%\begin{remarks}
%The measurement unitary $U$ was defined with respect to a preferred basis. This corresponds to local interactions between the apparatus and the system. 
%\end{remarks}

\part{Approximation Methods}
\chapter{Variational Principle}
The variational method is used to place an upper bound on the ground state energy of a quantum system. Consider a Hamiltonian $H$ with a discrete spectrum $H|n\rangle=E_n|n\rangle$, $n=0,1,...$ with $E_{n+1}>E_n$.
\section{Upper Bound on Ground State}
\begin{prop}
For any state $|\psi\rangle$, the expectation is definitely greater than the ground state energy
\begin{equation}
\langle E\rangle=\langle\psi|H|\psi\rangle\geq E_0\label{upperbound}
\end{equation}
\end{prop}
\begin{proof}
Let $|\psi\rangle=\sum_{m=0}^\infty a_m|m\rangle$ with $\sum_m|a_m|^2=1$. Then,
$$\langle E\rangle=\sum_{m,m'}a_ma_{m'}^*\langle m'|H|m\rangle=\sum_m|a_m|^2E_m=\sum_n|a_n|^2E_0+\sum_n(E_n-E_0)|a_n|^2\geq E_0$$
\end{proof}
\begin{defi}[Rayleigh-Ritz quotient]
Now consider a family of states $|\psi(\alpha)\rangle$ depending on some parameters $\alpha_n$. If $|\psi(\alpha)\rangle$ are not normalized, define
\begin{equation}
E(\alpha):=\frac{\langle\psi(\alpha)|H|\psi(\alpha)\rangle}{\langle\psi(\alpha)|\psi(\alpha)\rangle}\label{RayleighRitz}
\end{equation}
This is the Rayleigh-Ritz quotient. We have $E(\alpha)\geq E_0$ $\forall\alpha$. 
\end{defi}
\begin{remarks}\leavevmode
\begin{enumerate}
\item The most stringent bound occurs at the minimum $\alpha=\alpha^*$, i.e.
$$\frac{\partial E(\alpha)}{\partial\alpha}\bigg|_{\alpha=\alpha^*}=0,\quad\text{and}\quad E(\alpha^*)\geq E_0$$
\item We choose an ansatz $\psi_{\text{trial}}$ (which is a function of variational parameters $\{\alpha_i\}$) based on the relevant symmetry properties and boundary conditions it is required to satisfy. The closer $\psi$ is to the true (but unknown) ground state, the closer the minimum will be to the true ground state energy. 
\end{enumerate}
\end{remarks}
\begin{eg}[Hydrogen atom]
For the Hydrogen atom orbital, we guess $\psi(r)=e^{-\beta r}$ with $\psi(r)\rightarrow 0$ as $r\rightarrow\infty$. The Rayleigh-Ritz quotient (Eqn.~\ref{RayleighRitz}) is
$$E(\beta)=\frac{1}{\langle\psi|\psi\rangle}\bigg(-\frac{\hbar^2}{2m_e}\langle\psi|\nabla^2|\psi\rangle-\frac{e^2}{4\pi\varepsilon_0r}\langle\psi|\psi\rangle\bigg)=\frac{1}{\pi/\beta^3}\bigg(-2\beta\frac{\pi}{\beta^2}+\beta^2\frac{\pi}{\beta^3}-\frac{e^2}{4\pi\varepsilon_0r}\frac{\pi}{\beta^2}\bigg)$$
which is minimized at $\beta_0=\frac{e^2}{4\pi\varepsilon_0}\frac{m_e}{\hbar^2}$, i.e. $E(\beta_0)=-(\frac{e^2}{4\pi\varepsilon_0\hbar c})^2\frac{1}{2}m_ec^2\approx -13.6$ eV (actually exact).
\end{eg}
\begin{eg}[Quartic Potential]
We have $H=-\frac{d^2}{dx^2}+x^4$. What is $E_0$? We guess $\psi(x,\alpha)=(\frac{\alpha}{\pi})^{1/4}e^{-\alpha x^2/2}$ which is also the ground state of the harmonic oscillator. Note that this is not the ground state of the quartic potential! Consider the following illustration: quartic potential (blue) and ground state ansatz (orange).
\begin{center}
\begin{tikzpicture}
      \draw[->] (-3,0) -- (3,0) node[right] {$x$};
      \draw[->] (0,0) -- (0,4);
      \draw[domain=-2.5:2.5,smooth,variable=\x,blue] plot ({\x},{0.1*(\x)^4)});
      \draw[domain=-3:3,smooth,variable=\x,orange] plot ({\x},{exp(-(\x)^2)});
      \draw (0,0) node[below]{0};
\end{tikzpicture}
\end{center}
The Rayleigh-Ritz quotient (Eqn.~\ref{RayleighRitz}) is
$$E(\alpha)=\sqrt{\frac{\alpha}{\pi}}\int_{-\infty}^\infty(\alpha-\alpha^2x^2+x^4)e^{-\alpha x^2}dx=\frac{\alpha}{2}+\frac{3}{4\alpha^2}$$
The minimum is $(\alpha^*)^3=3$. The true ground state is $E_0=1.06$ but $E(\alpha^*)=1.08$, which is a reasonable estimate to the true ground state.
\end{eg}
\begin{remarks}
In many cases, the variational method can be very accurate. For instance, guessing the ground state of Helium. However, we have no way knowing how accurate it is unless someone tells us $E_0$. For complicated systems, it can be difficult to guess an ansatz for the ground state $\psi(\alpha)$.
\end{remarks}
\begin{cor}
For the ansatz $\psi=\sum_j\alpha_j\psi_j$, the minimum energy and $\alpha_j$'s can be found from
$$\det(\langle\psi_j|H|\psi_k\rangle-E_{\text{min}}\langle\psi_j|\psi_k\rangle)=0$$
\end{cor}
\begin{proof}
The Rayleigh-Ritz quotient (Eqn.~\ref{RayleighRitz}) gives
$$\langle E\rangle=\frac{\sum_{j,k}\alpha_j\alpha_kH_{jk}}{\sum_{jk}\alpha_j\alpha_kS_{jk}},\quad H_{jk}=\langle\psi_j|H|\psi_k\rangle,~S_{jk}=\langle\psi_j|\psi_k\rangle$$
To minimize this,
$$0=\frac{\partial\langle E\rangle}{\partial\alpha_i}=\frac{\sum_k\alpha_kH_{ik}+\sum_j\alpha_jH_{ji}}{\sum_{j,k}\alpha_j\alpha_kS_{jk}}-\frac{\sum_k\alpha_k\alpha_jH_{jk}(\sum_k\alpha_kS_{ik}+\sum_j\alpha_jS_{ji})}{(\sum_{j,k}\alpha_j\alpha_kS_{jk})^2}$$
which gives $\sum_k\alpha_kH_{ik}(\sum_{j,k}\alpha_j\alpha_kS_{jk})=\sum_k\alpha_kS_{ik}(\sum_{j,k}\alpha_j\alpha_kH_{jk})$. Compare with $E_{\text{min}}\sum_{j,k}\alpha_j\alpha_kS_{jk}=\sum_{j,k}\alpha_j\alpha_kH_{jk}$. This gives
\begin{equation}
\sum_k\alpha_k(H_{ik}-E_{\text{min}}S_{ik})=0\label{linearsystem}
\end{equation}
where the non-trivial solutions are obtained from setting the determinant $\det$ to be zero.
\end{proof}
\begin{eg}[Hydrogen atom]
Suppose the proton mass $m_p$ cannot be neglected. We replace $m_e$ with the reduced mass $\mu=\frac{m_pm_e}{m_p+m_e}$ in the Hamiltonian. The ground state is $E_0=-\frac{\mu}{m_e}R_\infty\approx -0.999456 R_\infty$. Try an ansatz $\psi_{\text{trial}}=\alpha_1|100\rangle+\alpha_2|200\rangle$ where the basis states are in terms of $m_e=\frac{4\pi\varepsilon_0\hbar^2}{e^2a_0}$, and not $\mu$. The terms $H_{jk}$ are
$$H_{11}=4H_{22}=\bigg(\frac{m_e}{m_p}-1\bigg)R_\infty,\quad H_{21}=H_{12}=\frac{16}{27\sqrt{2}}\frac{m_e}{m_p}R_\infty$$
Solving the linear system gives 
$$E_{\text{min}}=-\bigg(1-\frac{m_e}{m_p}\bigg)\bigg[\frac{5}{8}\pm\frac{3}{8}\sqrt{1+2\bigg(\frac{64}{81}\frac{m_e}{m_p-m_e}\bigg)^2}\bigg]R_\infty\approx -0.999455 R_\infty$$
\end{eg}
\newpage
\section{Bound States}
This method is useful to determine whether bound states exist for a given potential. Consider potentials $V(x)$ such that $V(x)\rightarrow 0$ as $x\rightarrow\infty$. A bound state has $E<0$. If we can find $E(\alpha)<0$ for some $\alpha$, then the bound state must exist.
\begin{prop}
Consider $V(x)$ such that $V(x)=0$ for $|x|>L$ for some $L$. A bound state exists whenever $\int_{-\infty}^\infty V(x)dx<0$.
\end{prop}
\begin{proof}
Use ansatz $\psi(X)=(\frac{\alpha}{\pi})^{1/4}e^{-\alpha X^2/2}$ for generic Hamiltonian $H=-\frac{\hbar^2}{2m}\frac{\partial^2}{\partial x^2}+V(x)$, then
$$E(\alpha)=\langle H\rangle=\frac{\hbar^2\alpha}{4m}+\sqrt{\frac{\alpha}{\pi}}\int_{-\infty}^\infty V(x)e^{-\alpha x^2}dx$$
Consider the function $\frac{E(\alpha)}{\sqrt{\alpha}}$. This is a continuous function of $\alpha$. In the limit $\alpha\rightarrow0^+$,
$$\frac{E(\alpha)}{\sqrt{\alpha}}\rightarrow\frac{1}{\sqrt{\pi}}\int_{-\infty}^\infty V(x)dx$$
If $\int V(x)dx<0$, then $\lim_{\alpha\rightarrow 0}\frac{E(\alpha)}{\sqrt{\alpha}}<0$ and by continuity, $\exists\alpha>0$ for which $E(\alpha)<0$.
\end{proof}
The converse, however, is not true. There are potentials with $\int Vdx>0$ that admits bound states. This theorem is also true in 2D but not in 3D. This theorem can be powerful. For isntance, it can be shown that nuclei bound using Yukawa force. 
\section{Upper Bound on Excited States}
It is harder to bound excited states. Consider an ansatz $|\psi(\alpha)\rangle$ such that $\langle\psi(\alpha)|0\rangle=0$ $\forall\alpha$ where $|0\rangle$ is the ground state. Then by the previous argument, we can show
$$E(\alpha)=\langle\psi(\alpha)|H|\psi(\alpha)\rangle\geq E_1$$
but such $\psi(\alpha)$ is hard to find unless there is a symmetry.
\begin{eg}
Consider the quartic potential again. The Hamiltonian is invariant under parity $P:X\rightarrow -X$. Then, the ground state must be even under parity. The ansatz for the first excited states must thus be odd (since orthogonal by construction). For instance, try $\psi(\alpha)=(\frac{4\alpha^3}{\pi})^{1/4}xe^{-\alpha x^2/2}$ such that $E(\alpha^*)=3.85$. This is close to the exact value $E_1=3.80$.
\end{eg}
\newpage
\section{Molecules}
Unlike the Hydrogen atom, molecules cannot be solved.
\begin{defi}[Born-Oppenheimer approximation]
The nuclei have a much greater mass than the electrons, and we can treat the electrons and nuclei separately - nuclei in fixed positions and study the eigenstates of the electron system.
\end{defi}
\begin{remarks}
To study nuclear motion, such as molecular rotations and vibrations, assume that the electrons adjust instantly to changes in nuclear positions.
\end{remarks}
As the molecular conformation is varied, by changing the nuclear positions $\mathbf{R_a},\mathbf{R_b},\dots$, the ground state energy $E_0$ can be minimized to find the equilibrium conformation for these positions.
\begin{eg}[H$^+_2$ ion]
The Hamiltonian of the H$_2^+$ ion is
$$H=-\frac{\hbar^2}{2m_e}\nabla_r^2+\frac{e^2}{4\pi\varepsilon_0}\bigg(\frac{1}{R}-\frac{1}{r_a}-\frac{1}{r_b}\bigg),\quad R=|\mathbf{R_a}-\mathbf{R_b}|,~r_a=|\mathbf{R_a}-\mathbf{r}|,~r_b=|\mathbf{R_b}-\mathbf{r}|$$
Use the Rayleigh-Ritz method, with the trial wavefunction being a linear combination of two Hydrogen-like ground state atomic wavefunctions. 
$$\psi_{\text{trial}}(\mathbf{r};R,Z')=\alpha_a\psi_a(\mathbf{r})+\alpha_b\psi_b(\mathbf{r}),\quad\psi_{a,b}(\mathbf{r})=\sqrt{\frac{\beta^3}{\pi}}e^{-\beta|\mathbf{R_{a,b}}-\mathbf{r}|},~\beta=\frac{Z'}{a_0}$$
where $\mathbf{R_a}=(0,0,-R/2)$, $\mathbf{R_b}=(0,0,+R/2)$ and $r_a^2=x^2+y^2+(z+0.5R)^2$, $r_b^2=x^2+y^2+(z-0.5R)^2$. There are 4 variational parameters: coefficients $\alpha_a,\alpha_b$, the proton-proton separation $R$ and the effective nuclear charge $Z'$ (due to screening). For any given $R$ and $Z'$, use Rayleigh-Ritz to find the coefficients of $\alpha_a,\alpha_b$. Eqn.~\ref{linearsystem} gives 
$$0=\det(H-E_{\text{min}}S),\quad S=\begin{pmatrix}1&S_{\text{ab}}\\S_{\text{ba}}&1\\\end{pmatrix},~H=\begin{pmatrix}H_{aa}&H_{ab}\\H_{ba}&H_{bb}\\\end{pmatrix}$$
with matrix elements $S_{aa}=S_{bb}=1$, $S_{ab}=S_{ba}=\langle\psi_a|\psi_b\rangle$ and $H_{aa}=H_{bb}$, $H_{ab}=H_{ba}$. The resulting eigenstates are $\alpha_a=\pm\alpha_b$, giving 
$$\psi_g=\frac{\psi_a+\psi_b}{\sqrt{2(1+S_{\text{ab}})}},~E_g=\frac{H_{aa}+H_{ab}}{1+S_{ab}},\quad\psi_u=\frac{\psi_a-\psi_b}{\sqrt{2(1-S_{ab})}},~E_u=\frac{H_{aa}-H_{ab}}{1-S_{ab}}$$
The eigenstate $\psi_g$ ($\psi_u$) is even (odd) under the interchange $a\leftrightarrow b$. For each of these two cases, vary $R$ and $Z'$ and look (numerically) for the overall minima of the energies $E_{g,u}$. 
\end{eg}
\begin{remarks}\leavevmode
\begin{enumerate}
\item For $|\psi_g\rangle$, the two atomic wavefunctions interfere constructively, giving enhanced electron density between the two protons, thereby screening the proton charge and reducing the p-p repulsion, giving the bonding molecular orbital. For $|\psi_u\rangle$, the two atomic wavefunctions interfere destructively, giving no screening, hence not a bound state. This is the anti-bonding molecular orbital.
\item $H_2$ is not a trivial extension of $H_2^+$ and cannot be solved exactly. Again, we may use the variational method (variable effective nuclear charge $Z'$) to estimate the ground state energy. But yet, $\psi_g(\mathbf{r_1})\psi_g(\mathbf{r_2})$ is not a good representation of the true ground state since there is an equal contribution of covalent bonding (electrons shared between the two protons) and ionic bonding (both electrons assigned to the same proton). To allow varying covalent/ionic composition, we introduce an extra variational parameter $\lambda$ for the ionic part. To further improve the trial wavefunctions, we may add more variational parameters.
\end{enumerate}
\end{remarks}
\newpage
\chapter{Time Independent Perturbation}
\begin{defi}[Time-independent perturbation]
Consider a time-independent perturbation $\Delta$, which is a Hermitian operator. We can interpolate from the unperturbed Hamiltonian $H^{(0)}$ to the Hamiltonian of interest $H^{(0)}+\Delta$, via $H(\lambda):=H^{(0)}+\lambda\Delta$, which is a family of Hamiltonians, $\lambda\in[0,1]$. Here, $\lambda\Delta<<H^{(0)}$. 
\end{defi}
\begin{remarks}
We assume the eigenstates and eigenvalues of $H^{(0)}$ are fully understood. The goal would be to find the eigenstates and eigenvalues of $H(\lambda)$. In general, $[H(\lambda),\Delta]\neq 0$, so they do not share eigenstates. The key assumption in time-independent perturbation theory would be the eigenvalues and eigenvectors of $H(\lambda)$ can be found as a series expansion (assuming it converges for sufficiently small $\lambda$) in positive powers of $\lambda$.
\end{remarks}
To analyze the evolution of states and energies as a function of $\lambda$, we treat separately for (1) a non-degenerate state and (2) a collection of degenerate states.
\section{Non-degenerate}\label{sec:NDTIP}
\begin{notation}
Start form a discrete spectrum with an orthonormal basis $|k^{(0)}\rangle$, $k\in\mathbb{Z}$ for original unperturbed Hamiltonian $H^{(0)}$, i.e.
$$H^{(0)}|k^{(0)}\rangle=E_k^{(0)}|k^{(0)}\rangle,\quad\langle k^{(0)}|l^{(0)}\rangle=\delta_{kl}$$
where $E_0^{(0)}\leq E_1^{(0)}\leq E_2^{(0)}\leq\dots$. Consider a non-degenerate state $|n^{(0)}\rangle$ with fixed $n$ such that $\dots\leq E_{n-1}^{(0)}<E_n^{(0)}<E_{n+1}^{(0)}\leq\dots$.
\end{notation}
\begin{prop}[Non-degenerate time-independent perturbation theory]
In the presence of the perturbation $\Delta$, the first and second order corrections to the energies are respectively
\begin{equation}
E_n^{(1)}=\langle n^{(0)}|\Delta|n^{(0)}\rangle,\quad E_n^{(2)}=-\sum_{k\neq n}\frac{\langle k^{(0)}|\Delta|n^{(0)}\rangle}{E_k^{(0)}-E_n^{(0)}}|k^{(0)}\rangle\label{TINDP_E}
\end{equation}
and the first order correction to the state is
\begin{equation}
|n^{(1)}\rangle=-\sum_{k\neq n}\frac{\langle k^{(0)}|\Delta|n^{(0)}\rangle}{E_k^{(0)}-E_n^{(0)}}|k^{(0)}\rangle\label{TINDP_S}
\end{equation}
\end{prop}
\begin{proof}
The Hamiltonian $H(\lambda)$ satisfies
$$H(\lambda)|n\rangle_\lambda=E_n(\lambda)|n\rangle_\lambda,\quad|n\rangle_{\lambda=0}=|n^{(0)}\rangle,\quad E_n(\lambda=0)=E_n^{(0)}$$
Assume a Taylor series expansion for the perturbed states and energies:
\begin{equation}
    |n\rangle_\lambda=\sum_{i=0}^\infty\lambda^i|n^{(i)}\rangle,\quad E_n(\lambda)=\sum_{i=0}^\infty \lambda^i E_n^{(i)}\label{TINDP_taylor}
\end{equation}
All these states and energies are, by definition, $\lambda$ independent. We will not impose normalization requirement for $|n\rangle_\lambda$, just require normalizable, which will be for sufficiently small perturbations. Substitute Eqn.~\ref{TINDP_taylor} to the Schr\"{o}dinger's equation, we have
\begin{equation}
O(1):\quad (H^{(0)}-E_n^{(0)})|n^{(0)}\rangle=0\label{TINDP_zeroorder}
\end{equation}
\begin{equation}
O(\lambda):\quad (H^{(0)}-E_n^{(0)})|n^{(1)}\rangle=(E_n^{(1)}-\Delta)|n^{(0)}\rangle\label{TINDP_firstorder}
\end{equation}
\begin{equation}
O(\lambda^2):\quad (H^{(0)}-E_n^{(0)})|n^{(2)}\rangle=(E_n^{(1)}-\Delta)|n^{(1)}\rangle+E_n^{(2)}|n^{(0)}\rangle\label{TINDP_secondorder}
\end{equation}
Without loss of generality, we claim 
\begin{equation}
    0=\langle n^{(0)}|n^{(1)}\rangle=\dots=\langle n^{(0)}|n^{(k)}\rangle\label{claim}
\end{equation}
which we will prove later. From Eqn.~\ref{TINDP_zeroorder}, Eqn.~\ref{TINDP_firstorder} and Eqn.~\ref{TINDP_secondorder} must give respectively
$$0=\langle n^{(0)}|E_n^{(1)}-\Delta|n^{(0)}\rangle\implies E_n^{(1)}=\Delta_{00}:=\langle n^{(0)}|\Delta|n^{(0)}\rangle$$
$$0=\langle n^{(0)}|\bigg((E_n^{(1)}-\Delta)|n^{(1)}\rangle+E_n^{(2)}|n^{(0)}\rangle\bigg)\implies E_n^{(2)}=\Delta_{01}$$
where $\langle n^{(0)}|n^{(1)}\rangle=0$. To find $|n^{(1)}\rangle$, start from Eqn.~\ref{TINDP_firstorder}, and take
$$\langle k^{(0)}|(H^{(0)}-E_n^{(0)}|n^{(1)}\rangle=\langle k^{(0)}|E_n^{(1)}-\Delta|n^{(0)}\rangle\implies\langle k^{(0)}|n^{(1)}\rangle=-\frac{\langle k^{(0)}|\Delta|n^{(0)}\rangle}{E_n^{(0)}-E_n^{(0)}},\quad k\neq n$$
By completeness, $|n^{(1)}\rangle=\sum_{k\neq n}|k^{(0)}\rangle\langle k^{(0)}|n^{(1)}\rangle$, and we thus obtain Eqn.~\ref{TINDP_S} and hence $E_n^{(2)}$.
\end{proof}
\begin{remarks}\leavevmode
\begin{enumerate}
    \item $|n^{(1)}\rangle$ can have components along all basis states, except the non-degenerate unperturbed state $|n^{(0)}\rangle$. If $|n^{(0)}\rangle$ is degenerate, i.e. $\exists w\neq n$ such that $E_n^{(0)}=E_w^{(0)}$, then Eqn.~\ref{TINDP_S} will diverge, hence incorrect.
    \item $O(\lambda)$ corrections overstates the exact ground state energy, which can be easily seen by considering
    $$E_0^{(0)}+\lambda E_0^{(1)}=\langle 0^{(0)}|H(\lambda)|0^{(0)}\rangle\geq E_0(\lambda)$$
    The inequality is true, from variational principle. In contrast, the $O(\lambda^2)$ correction always understates, since $E_k^{(0)}-E_0^{(0)}<0$ $\forall k$.
    \item $O(\lambda^2)$ correction to $|n^{(0)}\rangle$ exhibits level repulsion: levels with $k>n$ push state down, while levels with $k<n$ push state up.
    $$-\lambda^2\sum_{k\neq n}\frac{|\Delta_{kn}|^2}{E_k^{(0)}-E_n^{(0)}}=-\lambda^2\sum_{k>n}\frac{|\Delta_{kn}|^2}{E_k^{(0)}-E_n^{(0)}}+\lambda^2\sum_{k<n}\frac{|\Delta _{kn}|^2}{E_n^{(0)}-E_k^{(0)}},\quad\Delta_{kn}=\langle k^{(0)}|\Delta |n^{(0)}\rangle$$
\end{enumerate}
\end{remarks}
\begin{lemma}
Without loss of generality, $|n^{(k)}\rangle$, $k\geq1$ contains no vector along the unperturbed state $|n^{(0)}\rangle$. 
\end{lemma}
\begin{proof}
Suppose this is not true, then we can write $|n^{(k)}\rangle=|n^{(k)}\rangle'-a_k|n^{(0)}\rangle$, $k\geq1$, where $\langle n^{(0)}|n^{(k)}\rangle'=0$. Then
$$|n\rangle_\lambda=|n^{(0)}\rangle+\lambda(|n^{(1)}\rangle'-a_1|n^{(0)}\rangle)+\lambda^2(|n^{(2)}\rangle'-a_2|n^{(0)}\rangle)+\dots=(1-a_1\lambda-a_2\lambda^2-\dots)|n^{(0)}\rangle+\lambda|n^{(1)}\rangle'+\lambda^2|n^{(2)}\rangle'+\dots$$
Divide by $1-a_1\lambda-a_2\lambda^2-\dots$, then
$$|n\rangle_\lambda'\approx|n^{(0}\rangle+(1+a_1\lambda)(\lambda|n^{(1)}\rangle'+\lambda^2|n^{(2)}\rangle'+\dots)=|n^{(0)}\rangle+\lambda|n^{(1)}\rangle'+\lambda^2(|n^{(2)}\rangle'+a_1|n^{(1)}\rangle')+\dots$$
Here, $|n\rangle_\lambda$ is physically identical to $|n_\lambda\rangle'$. Now, the state corrections are all orthogonal to $|n^{(0)}\rangle$, as desired.
\end{proof}
\begin{prop}[Validity of theory]
For the series expansion to converge, the perturbation $\Delta$ must be small compared with the energy differences in $H^{(0)}$, i.e. the off-diagonal terms $V$ satisfy
\begin{equation}
    \frac{|\lambda||V|}{0.5|E_1^{(0)}-E_2^{(0)}|}<1\label{validity}
\end{equation}
\end{prop}
\begin{proof}
Let the Hamiltonian be $H(\lambda)=\begin{pmatrix}E_1^{(0)}&\lambda V\\\lambda V^*&E_2^{(0)}\\\end{pmatrix}$, then its eigenvalues are
$$E_\pm(\lambda)=\frac{1}{2}(E_1^{(0)}+E_2^{(0)})\pm\frac{1}{2}(E_1^{(0)}-E_2^{(0)}\sqrt{1+\bigg(\frac{\lambda|V|}{0.5(E_1^{(0)}-E_2^{(0)})}\bigg)^2}$$
The function $f(z)=\sqrt{1+z^2}$ has branch cut at $z=\pm i$ and has a series expansion $f(z)=1+z^2/2-z^4/8+z^6/18+O(z^8)$, that is only valid if $|z|<1$, i.e. the radius of convergence is 1. Hence, $ \frac{|\lambda||V|}{0.5|E_1^{(0)}-E_2^{(0)}|}<1$.
\end{proof}
\begin{eg}[Infinite square well]\leavevmode
\begin{enumerate}
    \item Consider an infinite square well with a central bump. The perturbation is $\Delta=\varepsilon$ for $|x|<b$ and zero otherwise. The unperturbed energies and states are
    $$E_n^{(0)}=\frac{\hbar^2\pi^2n^2}{2m(2a)^2},\quad|n^{(0)}\rangle=\frac{1}{\sqrt{a}}\sin\frac{n\pi}{2a}(x+a)$$
    The first energy correction to the energy (Eqn.~\ref{TINDP_E}) is
    $$O(\lambda):\quad E_n^{(1)}=\frac{\varepsilon}{a}\int_{-b}^b\sin^2\frac{n\pi}{2a}(x+a)dx=\varepsilon\bigg[\frac{b}{a}-\frac{(-1)^n}{n\pi}\sin\frac{n\pi b}{a}\bigg]$$
    Here, the first energy corrections are tangent to the exact energy of the perturbed system. For $b<<a$, $E_n^{(1)}\approx\frac{\varepsilon b}{a}(1-(-1)^n)$. it is non-zero for $n=1$ since $|1^{(0)}\rangle$ is maximal at $x=0$, where the bump is. The central bump is symmetric, so the symmetry of unperturbed wavefunctions is preserved, i.e.
    $$\langle m^{(0)}|\Delta|n^{(0)}\rangle=\frac{2\varepsilon}{\pi}\bigg[\frac{(-1)^{(m-n)/2}}{m-n}\sin\frac{(m-n)\pi b}{2a}-\frac{(-1)^{(m+n)/2}}{m+n}\sin\frac{(m+n)\pi b}{2a}\bigg]$$
    if the parity matches. Otherwise, it vanishes.
    \item Consider the infinite square well under an electric field $\mathcal{E}$. The perturbation is $\Delta=-q\mathcal{E}x$. The first energy correction (Eqn.~\ref{TINDP_E}) vanishes - an example of a parity selection rule.
    $$E_n^{(1)}=q\mathcal{E}\int_{-a}^ax\sin^2\frac{n\pi}{2a}(x+a)dx=0$$
    The first energy correction to the state (Eqn.~\ref{TINDP_S}) is
    $$|n^{(1)}\rangle=-\frac{q\mathcal{E}}{E_0}\sum_{m\neq n}|m\rangle\frac{\langle m^{(0)}|x|n^{(0)}\rangle}{n^2-m^2}=-\frac{q\mathcal{E}a}{E_0}\frac{16n}{\pi^2\sqrt{a}}\sum_{m+n\text{ odd}}\frac{m}{(m^2-n^2)^3}\sin\frac{m\pi}{2a}(x+a)$$
    where $E_i^{(0)}=E_0i^2$ and $\langle m^{(0)}|x|n^{(0)}\rangle=-\frac{16a}{\pi^2}\frac{mn}{(m^2-n^2)^2}$ if the parities do not match (if the parity matches, this term vanishes). The second energy correction (Eqn.~\ref{TIDP_E}) is then
    $$E_n^{(2)}=-\bigg(\frac{q\mathcal{E}a}{E_0}\bigg)^2\bigg(\frac{16}{\pi^2}\bigg)^2E_n^{(0)}\sum_{m+n\text{ odd}}\frac{m^2}{(m^2-n^2)^3}$$
    For say, $n=1$, only anti-symmetric states contribute, and this sum dominated by $m=2$.
\end{enumerate}
\end{eg}
\newpage
\begin{eg}[Harmonic oscillator]\leavevmode
\begin{enumerate}
    \item Consider a linear perturbation $\Delta=\varepsilon x$ (for example, external electric field) to the harmonic oscillator. The first energy correction vanishes but the second energy correction does not. From Eqn.~\ref{TINDP_E}:
    $$E_n^{(1)}=\varepsilon\langle n|x|n\rangle=0,\quad E_n^{(2)}=\varepsilon^2\sum_{m\neq n}\frac{|\langle m|x|n\rangle|^2}{E_n^{(0)}-E_m^{(0)}}$$
    but $\langle n-1|x|n\rangle=\sqrt{\hbar n/2m\omega}$ and $\langle n+1|x|n\rangle=\sqrt{\hbar(n+1)/2m\omega}$. Hence, we have
    $$E_n=E_n^{(0)}+\frac{\hbar\varepsilon^2}{2m\omega}\bigg(\frac{n}{E_n^{(0)}-E_{n-1}^{(0)}}+\frac{n+1}{E_n^{(0)}-E_{n+1}^{(0)}}\bigg)=(n+0.5)\hbar\omega-\frac{\varepsilon^2}{2m\omega^2}$$
    All energy levels are lowered by a common amount. This is an exact solution, and can be obtained by completing the square, i.e. akin to translating the harmonic oscillator via $x\rightarrow x'=x+\frac{\varepsilon}{m\omega^2}$.
\item Consider an anharmonic oscillator Hamiltonian, which consists of the quadratic harmonic oscillator and a perturbation 
$$\Delta=\hbar\omega x^4/d^4=\frac{m^2\omega^3}{\hbar}x^4=0.25\hbar\omega(a+a^\dag)^4$$
where $d^2=\frac{\hbar}{m\omega}$ is the length scale of the unperturbed harmonic oscillator. The first energy correction is given by Eqn.~\ref{TINDP_firstorder}
$$E_0^{(1)}=\langle 0|0.25\hbar\omega(a+a^\dag)^4|0\rangle=0.75\hbar\omega$$
Since $(a+a^\dag)^4|0\rangle=3|0\rangle+6\sqrt{2}|2\rangle+\sqrt{4!}|4\rangle$, then $\Delta_{00}=0.75\hbar\omega$, $\Delta_{20}=(3\sqrt{2}/2)\hbar\omega$, $\Delta_{40}=\sqrt{6}\hbar\omega/2$, then
$$E_0^{(2)}=-\sum_{k\neq 0}\frac{|\Delta_{k0}|^2}{E_k^{(0)}-E_0^{(0)}}=-\frac{|\Delta_{20}|^2}{2\hbar\omega}-\frac{|\Delta_{40}|^2}{4\hbar\omega}=-\frac{21}{8}\hbar\omega$$
Including more higher order corrections will give
$$E_0(\lambda)=\frac{1}{2}\hbar\omega\bigg(1+\frac{3}{2}\lambda-\frac{21}{4}\lambda^2+\frac{333}{8}\lambda^3-\frac{30885}{64}\lambda^4+\frac{916731}{128}\lambda^5\bigg)+O(\lambda^6)$$
The radius of convergence is actually zero. $E_0(\lambda)$ is, in fact, non-analytic at $\lambda=0$, but the error is small for small $\lambda$, i.e. asymptotic expansion. The magnitude for successive higher order terms generally decrease until at some point, they start growing again. The first order correction to the ground state is
$$|0^{(1)}\rangle=-\frac{\Delta_{20}}{2\hbar\omega}|2\rangle-\frac{\Delta_{40}}{4\hbar\omega}|4\rangle=-\frac{3}{4}a^\dag a^\dag|0\rangle-\frac{1}{16}a^\dag a^\dag a^\dag a^\dag|0\rangle$$
\end{enumerate}
\end{eg}
\begin{eg}[van der Waals interaction]
Consider two Hydrogen atom interacting with each other via
$$H=H_0+V,~H_0=-\frac{\hbar^2}{2m_e}(\nabla_1^2+\nabla_2^2)-\frac{e^2}{4\pi\varepsilon_0}\bigg(\frac{1}{r_1}+\frac{1}{r_2}\bigg),\quad V=\frac{e^2}{4\pi\varepsilon_0}\bigg(\frac{1}{r}+\frac{1}{|r+r_2-r_1|}-\frac{1}{|r_1+r_2|}-\frac{1}{|r-r_1|}\bigg)$$
If $r$ is large enough ($r>>a_0$), we can treat $V\approx\frac{e^2}{4\pi\varepsilon_0r^3}(x_1x_2+y_1y_2-2z_1z_2)$ as a perturbation $\Delta$. The unperturbed ground state is $|0^{(0)}\rangle=|100(\mathbf{r_1})\rangle|100(\mathbf{r_2})\rangle$, where $H_0|0^{(0)}\rangle=-2R_\infty|0^{(0)}\rangle$. The first energy correction (Eqn.~\ref{TINDP_E}) vanishes, i.e.
$$\langle 100|x_{1,2}|100\rangle=\langle100|y_{1,2}|100\rangle=\langle100|z_{1,2}|100\rangle=0\implies E_0^{(1)}=0$$
while the second energy correction (Eqn.~\ref{TINDP_E}) is
$$E^{(2)}_0(r)=\bigg(\frac{e^2}{4\pi\varepsilon_0r^3}\bigg)^2\sum_{k\neq 0}\frac{|\langle k^{(0)}|(x_1x_2+y_1y_2-2z_1z_2)|0^{(0)}\rangle|^2}{E_0^{(0)}-E_k^{(0)}}<0$$
\end{eg}
\newpage
\section{Degenerate}\label{sec:DTIP}
When there is degeneracy, the eigenvectors of $H^{(0)}$ are `ambiguous' since any combination of eigenvectors is equally valid. The perturbation selects a particular combination of these eigenvectors.
\begin{notation}
We have the full Hamiltonian to be
$$H(\lambda)=H^{(0)}+\lambda\Delta,\quad H^{(0)}=\diag(E_1^{(0)}, E_2^{(0)},\dots, E_n^{(0)},\dots, E_n^{(0)},\dots),\quad N>1$$
where there are $N$ numbers of $E_n^{(0)}$. We focus on the degenerate subspace of eigenvectors of dimension $N>1$. In this degenerate subspace, we choose a collection of $N$ orthonormal eigenstates $\{|n^{(0)};1\rangle,\dots,|n^{(0)};N\rangle\}$, where $\langle n^{(0)};p|n^{(0)};\ell\rangle=\delta_{p,\ell}$ and $H^{(0)}|n^{(0)};k\rangle=E_n^{(0)}|n^{(0)};k\rangle$. They span a degenerate subspace $\mathbb{V}_N$ such that the Hilbert space can be decomposed into a direct sum of two orthogonal subspaces, i.e.
$$\mathcal{H}=\mathbb{V}_N\oplus\hat{V}$$
where $\hat{V}$ is spanned by eigenstates of $H^{(0)}$ not in $\mathbb{V}_N$, namely $\{|p^{(0)}\}\rangle$, $p\in\mathbb{Z}$, which satisfy $\langle p^{(0)}|q^{(0)}\rangle=\delta_{pq}$ and $\langle p^{(0)}|n^{(0)};k\rangle=0$..
\end{notation}
\begin{remarks}
In degenerate perturbation theory, the energy correction is different for each value of $k$ (there are $N$ possible values for $k$).
\end{remarks}
From a similar procedure as time-independent non-degenerate perturbation, we assume a series expansion ansatz to the perturbed states and energies:
\begin{equation}
    |n;k\rangle_\lambda=\sum_{i=0}^\infty\lambda^i|n^{(i)};k\rangle,\quad E_{n,k}(\lambda)=\sum_{i=0}^\infty\lambda^iE^{(i)}_{n,k}\label{TIDP_taylor}
\end{equation}
Again, we demand $|n^{(p)};k\rangle$ ($O(\lambda^p)$ correction to state) for $p\geq1$ has no component along $|n^{(0)};k\rangle$, i.e. $\langle n^{(0)};k|n^{(p)};k\rangle=0$, but $|n^{(p)};k\rangle$ may have components along $|n^{(0)};\ell\rangle$ (some other degenerate state) with $\ell\neq k$, i.e. $|n^{(p)};k\rangle$ may have component in $\mathbb{V}_N$. Plug Eqn.~\ref{TIDP_taylor} into the Schr\"{o}dinger's equation $H(\lambda)|n;k\rangle_\lambda=E_{n,k}(\lambda)|n;k\rangle_\lambda$, we have the following:
\begin{equation}
    O(\lambda^0):\quad(H^0-E_n^{(0)})|n^{(0)};k\rangle=0\label{TIDP_zerothorder}
\end{equation}
\begin{equation}
    O(\lambda^1):\quad(H^0-E_n^{(0)})|n^{(1)};k\rangle=(E_{n,k}^{(1)}-\Delta)|n^{(0)};k\rangle\label{TIDP_firstorder}
\end{equation}
\begin{equation}
    O(\lambda^2):\quad(H^0-E_n^{(0)})|n^{(2)};k\rangle=(E_{n,k}^{(1)}-\Delta)|n^{(1)};k\rangle+E_{n,k}^{(2)}|n^{(0)};k\rangle\label{TIDP_secondorder}
\end{equation}
The solution to time-independent degenerate perturbation theory will differ depending to which order of the perturbation, is this degeneracy for $|n^{(0)};k\rangle$ broken at. 
\subsection{Lifted at first order}
We first discuss a solution to first order, for the case, in which degeneracy in $\mathbb{V}_N$ is completely broken to first order in perturbation theory, i.e. first order corrections to energies split $N$ states completely.
\begin{defi}[Degeneracy lifted]
The degeneracy for $|n^{(0)};k\rangle$ is said to be lifted if $E_{n,k}^{(1)}\neq E_{n,\ell}^{(1)}$ whenever $k\neq\ell$.
\end{defi}
\begin{prop}
When the degeneracy is lifted at $O(\lambda)$, the perturbed state and corrected energy is
\begin{equation}
    |n;k\rangle_\lambda=|n^{(0)};k\rangle-\lambda\bigg(\sum_p\frac{\Delta_{p,nk}}{E_p^{(0)}-E_n^{(0)}}|p^{(0)}\rangle+\sum_{\ell\neq k}\frac{|n^{(0)};\ell\rangle}{E_{n,k}^{(1)}-E_{n,\ell}^{(1)}}\sum_p\frac{\Delta_{m\ell,p}\Delta_{p,nk}}{E_{n,k}^{(1)}-E_{n,\ell}^{(1)}}\bigg)+O(\lambda^2)\label{TIDP_S}
\end{equation}
\begin{equation}
    E_{n,k}(\lambda)=E_n^{(0)}+\lambda\Delta_{nk,nk}-\lambda^2\sum_p\frac{\Delta_{nk,p}\Delta_{p,nk}}{E_p^{(0)}-E_n^{(0)}}+O(\lambda^3)\label{TIDP_E}
\end{equation}
where $\Delta_{p,q}:=\langle n^{(0)};p|\Delta|n^{(0)};q\rangle$. Here, $\Delta$ is diagonal in the chosen basis for $\mathbb{V}_N$.
\end{prop}
\begin{proof}
Take $\langle n^{(0)};\ell|\times$ Eqn.~\ref{TIDP_firstorder},
$$0=E^{(1)}_{n,k}\langle n^{(0)};\ell|n^{(0)};k\rangle-\langle n^{(0)};\ell|\Delta|n^{(0)};k\rangle\implies E^{(1)}_{nk}=\Delta_{nk,nk}$$
where $\langle n^{(0)};\ell|n^{(0)};k\rangle=\delta_{\ell,k}$ and $\langle n^{(0)};\ell|(H^0-E_n^{(0)})=0$. To obtain $|n^{(1)};k\rangle$ along $\hat{V}$, take $\langle p^{(0)}|\times$ Eqn.~\ref{TIDP_firstorder}, we have
$$\langle p^{(0)}|(H^{(0)}-E_n^{(0)})|n^{(1)};k\rangle=\langle p^{(0)}|(E_{n,k}^{(1)}-\Delta)|n^{(0)};k\rangle\implies(E_p^{(0)}-E-n^{(0)})\langle p^{(0)}|n^{(1)};k\rangle=-\Delta_{p,nk}$$
where $\langle p^{(0)}|n^{(0)};k\rangle=0$ since $|p^{(0)}\rangle\in\hat{V}$ and $|n^{(0)};k\rangle\in\mathbb{V}_N$. Hence, we can determine $|n^{(1)};k\rangle$, up to some undetermined component in $\mathbb{V}_N$, $|n^{(1)};k\rangle|_{\mathbb{V}_N}$, 
\begin{equation}
|n^{(1)};k\rangle=-\sum_p\frac{\Delta_{p,nk}}{E_p^{(0)}-E_n^{(0)}}|p^{(0)}\rangle+|n^{(1)};k\rangle|_{\mathbb{V}_N}\label{intermediate4}
\end{equation}
which disappears when we dot $|n^{(1)};k\rangle$ with $|p^{(0)}\rangle$.\\[5pt]
To obtain this component, we take $\langle n^{(0)};\ell|\times$ Eqn.~\ref{TIDP_secondorder}.
\begin{equation}
0=-\langle n^{(0)};\ell|(E^{(1)}_{n,k}-\Delta)\sum_p|p^{(0)}\rangle\frac{\Delta_{p,nk}}{E_p^{(0)}-E_n^{(0)}}+\langle n^{(0)};\ell|(E^{(1)}_{n,k}-\Delta)|n^{(1)};k\rangle|_{\mathbb{V}_N}+E_{n,k}^{(2)}\delta_{k,\ell}\label{intermediate3}
\end{equation}
The term $\langle n^{(0)};\ell|E_{n,k}^{(1)}|p^{(0)}\rangle$ vanishes by orthonormality. In reference to an equation (Eqn.~\ref{intermediate_step}) we will explain later, we have
\begin{equation}
\langle n^{(0)};\ell|\Delta=E_{n,k}^{(1)}\langle n^{(0)};\ell|+\sum_p\langle n^{(0)};\ell|\Delta|p^{(0)}\rangle\langle p^{(0)}|\label{intermediate2}
\end{equation}
The piece in $\hat{V}$ drops out for our case of interest, when we take $|n^{(1)};k\rangle|_{\mathbb{V}_N}\times$ to Eqn.~\ref{intermediate2}. This, together with Eqn.~\ref{intermediate3} gives
$$\sum_p\frac{\Delta_{n\ell,p}\Delta_{p,nk}}{E_p^{(0)}-E_n^{(0)}}+(E_{n,k}^{(1)}-E_{n,\ell}^{(1)})\langle n^{(0)};\ell|n^{(1)};k\rangle|_{\mathbb{V}_N}+E_{n,k}^{(2)}\delta_{k,\ell}=0$$
which gives $E_{n,k}^{(2)}$ when we set $k=\ell$ and otherwise when $k\neq\ell$, we have
$$\sum_p\frac{\Delta_{n\ell,p}\Delta_{p,nk}}{E_p^{(0)}-E_n^{(0)}}+(E_{n,k}^{(1)}-E_{n,\ell}^{(1)})\langle n^{(0)};\ell|n^{(1)};k\rangle=0\implies\langle n^{(0)};\ell|n^{(1)};k\rangle=\frac{-1}{E_{n,k}^{(1)}-E_{n,\ell}^{(1)}}\sum_p\frac{\Delta_{n\ell,p}\Delta_{p,nk}}{E_p^{(0)}-E_n^{(0)}}$$
as long as $E_{n,k}^{(1)}\neq E_{n,\ell}^{(1)}$, i.e. degeneracy broken to first order. Finally, the desired component in $\mathbb{V}_N$ can be obtained by bringing $\langle n^{(0)};\ell|$ to the right and sum over $\ell\neq k$, i.e.
\begin{equation}
|n^{(1)};k\rangle|_{\mathbb{V}_N}=-\sum_{\ell\neq k}|n^{(0)};\ell\rangle\frac{1}{E_{n,k}^{(1)}-E_{n,\ell}^{(1)}}\sum_p\frac{\Delta_{n\ell,p}\Delta_{p,nk}}{E_p^{(0)}-E_n^{(0)}}\label{intermediate5}
\end{equation}
Eqns.~\ref{intermediate4} and ~\ref{intermediate5} give the first order correction to $|n;k\rangle$.
\end{proof}
\begin{remarks}\leavevmode
\begin{enumerate}
\item As we will see, $E_{n,k}^{(1)}=\Delta_{nk,nk}$ is still true even if the degeneracy is not lifted in the first order. If the degeneracy is lifted in $O(\lambda)$, $|n^{(0)};k\rangle$ basis states, that make $\Delta$ diagonal in $\mathbb{V}_N$, are `good states'. Otherwise, the determination of good basis has to be attempted to $O(\lambda^2)$, which we will see later.
\item $\Delta$ is diagonalized in $\mathbb{V}_N$ and not diagonal on the full Hilbert space $\mathcal{H}=\mathbb{V}_N\oplus\hat{V}$. To see this, consider
\begin{align}
    \Delta|n^{(0)};\ell\rangle&=\sum_q|n^{(0)};q\rangle\langle n^{(0)};q|\Delta|n^{(0)};\ell\rangle+\sum_p|p^{(0)}\rangle\langle p^{(0)}|\Delta|n^{(0)};\ell\rangle\nonumber\\&=\sum_qE^{(1)}_{n,\ell}\delta_{\ell,q}|n^{(0)};q\rangle+\sum_p|p^{(0)}\rangle\langle p^{(0)}|\Delta|n^{(0)};\ell\rangle\nonumber\\&=E_{n,\ell}^{(1)}|n^{(0)};\ell\rangle+\sum_p|p^{(0)}\rangle\langle p^{(0)}|\Delta|n^{(0)};\ell\rangle\label{intermediate_step}
\end{align}
There is an `extra state' along $\hat{V}$.
\item A rule of thumb: $\Delta$ is diagonal for a choice in basis in $\mathbb{V}_N$, if for any two different basis vectors, there is a Hermitian $K$ that $[K,\Delta]$ for which these two basis vectors are $K$ eigenstates. To see this, consider two different basis states in $\mathbb{V}_N$: $|n^{(0)};p\rangle$ and $|n^{(0)};q\rangle$ with $p\neq q$. Assume $K|n^{(0)};p\rangle=\lambda_p|n^{(0)};p\rangle$ and $K|n^{(0)};q\rangle=\lambda_q|n^{(0)};q\rangle$. Since $[\Delta,K]=0$, then
$$0=\langle n^{(0)};p|[\Delta,K]|n^{(0)};q\rangle=(\lambda_q-\lambda_p)\langle n^{(0)};p|\Delta|n^{(0)};q\rangle$$
Since $\lambda_q\neq\lambda_p$, $\Delta$ must be diagonal in this choice of basis, of course, if such a $K$ exists.
\end{enumerate}
\end{remarks}
\begin{eg}[2D infinite square well]
Consider the perturbation $\Delta=\varepsilon xy$ to the 2D infinite square well. The unperturbed energies and states are
$$E_{n_x,n_y}^{(0)}=(n_x^2+n_y^2)E_0,\quad E_0=\frac{\hbar^2\pi^2}{8ma^2},\quad|n_x,n_y\rangle^{(0)}=\frac{1}{a}\sin\frac{n_x\pi}{2a}(x+a)\sin\frac{n_y\pi}{2a}(y+a)$$
All states with $n_x\neq n_y$ are degenerate with degeneracy $g=2$. Consider the degenerate states $|1,2\rangle$, $|2,1\rangle$, then we need to find a suitable linear combination that diagonalizes $\Delta$. In the basis of $\{|1,2\rangle,|2,1\rangle\}$, $\Delta$ is an off-diagonal matrix, i.e.
$$\Delta=\begin{pmatrix}0&A\\A&0\\\end{pmatrix},\quad A:=\langle 1,2|\Delta|2,1\rangle=\varepsilon a^2\bigg(\frac{32}{9\pi^2}\bigg)^2$$
Diagonalizing it gives eigenvalues $\pm A$ and eigenvectors $\frac{1}{\sqrt{2}}(|1,2\rangle\pm|2,1\rangle)$. The first order energy correction is thus $E^{(1)}=\pm A$. The degeneracy is not lifted to first order of the perturbation.
\end{eg}
\begin{eg}[Linear Stark effect]
Level splitting occurs when a static electric field is applied to a Hydrogen atom - the linear Stark effect. The state of a Hydrogen atom is specified by $|n,\ell,m\rangle$. There are four degenerate $n=2$ states, with associated spatial wavefunctions
$$|2,0,0\rangle=2-\frac{r}{a_0},\quad|2,1,0\rangle=\frac{r}{a_0}\cos\theta,\quad|2,1,\pm1\rangle=\frac{r}{a_0}\sin\theta e^{\pm i\phi}$$
all multiplied by a factor $e^{-r/2a_0}/\sqrt{32\pi a_0^3}$. Consider a perturbing electric field in the $z$ direction: $V(r,\theta,\phi)=e\mathcal{E}r\cos\theta$. The diagonal matrix elements of the perturbations requires $\langle2,\ell,m|\Delta|2,\ell,m\rangle$ to be evaluated. The diagonal elements become
$$e\mathcal{E}\int_0^\infty\int_0^\pi\int_0^{2\pi}d\phi~|\psi_{2,\ell,m}(r,\theta,\phi)|^2r^3\cos\theta\sin\theta d\theta dr =0$$
The only matrix elements that survive are $\langle2,0,0|\Delta|2,1,0\rangle$ and $\langle 2,1,0|\Delta|2,0,0\rangle$, which gives
$$\frac{e\mathcal{E}}{32\pi a_0^4}\int_0^\infty r^4e^{-r/a_0}(2-r/a_0)dr\int_0^\pi\cos^2\theta\sin\theta d\theta\int_0^{2\pi}d\phi=\frac{2e\mathcal{E}}{3(16)a_0^4}\int_0^\infty r^4e^{-r/a_0}(2-r/a_0)dr=-3e\mathcal{E}$$
The appropriate two-dimensional subspace of the four-dimensional degenerate subspace will be off-diagonal and can be diagonalized to show that two of the four degenerate states split into energy states at $\mp3e\mathcal{E}$.
\end{eg}
\begin{eg}[Bandgaps]
Consider the situation where a particle moves through a region having a weakly periodic potential $V(x)=2V\cos(2\pi x/a)$. The matrix elements of the perturbation in terms of the unperturbed energy eigenstates are
$$\frac{2V}{L}\int_0^Le^{i(k'-k)x}\cos\frac{2\pi x}{a}dx=\frac{V}{L}\int_0^Le^{i(k'-k+2\pi/a)x}+e^{i(k'-k-2\pi/a)x}=V\delta_{k-k',\pm2\pi/a}$$
The only off-diagonal terms that survive (diagonal terms vanish) are those for which $k'-k=\pm\frac{2\pi}{a}$. In the region of degenercies, we need to use degenerate perturbation theory which reveals bandgaps.
\end{eg}
\newpage
\subsection{Lifted at second order}
Suppose degeneracy is not broken to first order in $\Delta$, i.e.
$$\langle n^{(0)};\ell|\Delta|n^{(0)};k\rangle=E_n^{(1)}\delta_{\ell,k}$$
but suppose it is, to second order. 
\begin{notation}
We want to find exactly $N$ number of good states to span the degenerate subspace. Try the ansatz
$$|\psi_I^{(0)}\rangle=\sum_{k=1}^N|n^{(0)};k\rangle a_{Ik}^{(0)},\quad k=1,\dots,N,~I=1,\dots,N$$
where this set of `good states' satisfy
$$\langle\psi_J^{(0)}|\psi_I^{(0)}\rangle=\delta_{IJ}\implies\sum_k(a^{(0)}_{Jk})^*a^{(0)}_{Ik}=\delta_{IJ}$$
\end{notation}
Again, assume a series expansion for $|\psi_I\rangle_\lambda$ and $E_{nI}(\lambda)$, 
\begin{equation}
    |\psi_I\rangle_\lambda=\sum_{i=0}^\infty\lambda^i|\psi_I^{(i)}\rangle,\quad E_{nI}(\lambda)=E_n^{(0)}+\lambda E_n^{(1)}+\sum_{i=2}^\infty E_{nI}^{(i)}\label{TIDP_Taylor2}
\end{equation}
and plug it to Schr\"{o}dinger's equation $H(\lambda)|\psi_I\rangle_\lambda=E_{nI}(\lambda)|\psi_I\rangle_\lambda$ to get
\begin{equation}
    O(1):\quad(H^{(0)}-E_n^{(0)})|\psi_I^{(0)}\rangle=0\label{TIDP_zerothorder2}
\end{equation}
\begin{equation}
    O(\lambda):\quad(H^{(0)}-E_n^{(0)})|\psi_I^{(1)}\rangle=(E_n^{(1)}-\Delta)|\psi_I^{(0)}\rangle\label{TIDP_firstorder2}
\end{equation}
\begin{equation}
    O(\lambda^2):\quad(H^{(0)}-E_n^{(0)})|\psi_I^{(2)}\rangle=(E_n^{(1)}-\Delta)|\psi_I^{(1)}\rangle+E_{nI}^{(2)}|\psi_I^{(0)}\label{TIDP_secondorder2}
\end{equation}
\begin{prop}
When the degeneracy is lifted only at the second order, the second order correction $E^{(2)}_{nI}$ is the eigenvalues of the matrix
\begin{equation}
    M^{(2)}_{\ell,k}=\frac{-\sum_p\Delta_{n\ell,p}\Delta_{p,nk}}{E_p^{(0)}-E_n^{(0)}}\label{M}
\end{equation}
\end{prop}
\begin{proof}
Take $\langle p^{(0)}|\times$ Eqn.~\ref{TIDP_firstorder2},
$$(E_p^{(0)}-E_n^{(0)})\langle p^{(0)}|\psi_I^{(1)}\rangle=-\langle p^{(0)}|\Delta|\psi_I^{(0)}\rangle=-\sum_{k=1}^N\Delta_{p,nk}a_{Ik}^{(0)}\implies\langle p^{(0)}|\psi_I^{(1)}\rangle=\frac{-\sum_{k=1}^N\Delta_{p,nk}a_{Ik}^{(0)}}{E_p^{(0)}-E_n^{(0)}}$$
Use the above result after taking $\langle n^{(0)};\ell|\times$ Eqn.~\ref{TIDP_secondorder2}:
$$0=\langle n^{(0)};\ell|E_n^{(1)}-\Delta|\psi_I^{(1)}\rangle|_{\hat{V}}+\langle n^{(0)};\ell|E_n^{(1)}-\Delta|\psi_I^{(1)}\rangle|_{\mathbb{V}_N}+E_{nI}^{(2)}a_{I\ell}^{(0)}$$
where $\mathbb{V}_N\perp\hat{V}\implies\langle n^{(0)};\ell|E_n^{(1)}|\psi_I^{(1)}\rangle|_{\hat{V}}=0$ and the second term is zero since $\Delta$ is diagonal in $\mathbb{V}_N$. This gives
$$0=-\sum_p\langle n^{(0)};\ell|\Delta|p^{(0)}\rangle\langle p^{(0)}|\psi_I^{(1)}\rangle+E_{nI}^{(2)}a_{I\ell}^{(0)}=-\sum_p\frac{\Delta_{n\ell,p}\sum_{k=1}^N\Delta_{p,nk}a_{Ik}^{(0)}}{E_p^{(0)}-E_n^{(0)}}+\sum_{k=1}^NE_{nI}^{(2)}\delta_{\ell k}a_{Ik}^{(0)}$$
which we get an eigenvalue-eigenvector equation, where $a_{Ik}^{(0)}$ are components of an orthonormal basis of the good zeroth order states.
\end{proof}
\begin{remarks}
The label `I' in Eqn.~\ref{TIDP_Taylor2} is only for $i>2$. For $E_n^{(1)}$, the expression is the same as in Eqn.~\ref{TIDP_E}, i.e. $\lambda\Delta_{nk,nk}$.
\end{remarks}
\newpage
\chapter{Time Dependent Perturbation}
In time-dependent perturbation theory, we consider a time-independent Hamiltonian $H^{(0)}$ that is modified by a time-dependent perturbation $\Delta (t)$ (non-zero over a time interval $t_0<t<t_f$). We will need the interaction picture (discussed earlier).
\begin{remarks}\leavevmode
\begin{enumerate}
\item By describing our formalism in the interaction picture, the time dependence generated by $H^{(0)}$ is folded into operators, while the time dependence generated by $\Delta$ is realized in the states. This is because $H(t)$ do not have energy eigenstates (which we could have solved for in $H\psi(\mathbf{x})=E\psi(\mathbf{x})$, if $H$ was time-independent). 
\item Both initial and final states can be described in terms of eigenstates of $H^{(0)}$ (since $\Delta=0$ in the initial and final state). Even during the time interval ($t_0<t<t_f$) when $\Delta$ is switched on, we can use the eigenstates of $H^{(0)}$ to describe and analyze the system, since any time dependent state can be written as a superposition of the complete eigenbasis with time-dependent coefficients.
\end{enumerate}
\end{remarks}
\begin{prop}
In the interaction picture, the Schr\"{o}dinger's equation is
\begin{equation}
i\hbar\frac{d}{dt}|\tilde{\psi}(t)\rangle=\tilde{\Delta}(t)|\tilde{\psi}(t)\rangle,\quad|\tilde{\psi}(t)\rangle=e^{iH^{(0)}t/\hbar}|\psi(t)\rangle,\quad\tilde{\Delta}(t)=e^{iH^{(0)}t/\hbar}\Delta(t)e^{-iH^{(0)}t/\hbar}\label{interaction1}
\end{equation}
\end{prop}
\begin{proof}
As before, we define the quantum state in the interaction picture as $|\tilde{\psi}(t)\rangle=e^{iH^{(0)}t/\hbar}|\psi9t)\rangle$, where $|\tilde{\psi}(0)\rangle=|\psi(0)\rangle$. We have
\begin{align}
i\hbar\frac{d}{dt}|\tilde{\psi}(t)\rangle&=-H^{(0)}|\tilde{\psi}(t)\rangle+e^{iH^{(0)}t/\hbar}i\hbar\frac{d}{dt}|\psi(t)\rangle\nonumber\\&=\bigg[-H^{(0)}+e^{iH^{(0)}t/\hbar}(H^{(0)}+\Delta(t))e^{-iH^{(0)}t/\hbar}\bigg]|\tilde{\psi}(t)\rangle\nonumber\\&=e^{iH^{(0)}t/\hbar}\Delta(t)e^{-iH^{(0)}t/\hbar}|\tilde{\psi}(t)\rangle\nonumber
\end{align}
Result follows from the operator definition, in the interaction picture.
\end{proof}
\begin{cor}
For $|\psi(t)\rangle=\sum_nc_n(t)e^{-iE_nt/\hbar}|n\rangle$, the Schr\"{o}dinger's equation is
\begin{equation}
i\hbar\dot{c}_m(t)=\sum_n\tilde{\Delta}_{mn}(t)c_n(t)\label{interaction2}
\end{equation}
\end{cor}
\begin{proof}
We have $|\tilde{\psi}(t)\rangle=\sum_nc_n(t)|n\rangle$, so by Eqn.~\ref{interaction1}, we have
$$i\hbar\frac{d}{dt}\sum_mc_m(t)|m\rangle=\tilde{\Delta}(t)\sum_nc_n(t)|n\rangle=\sum_m|m\rangle\langle m|\tilde{\Delta}(t)\sum_nc_n(t)|n\rangle=\sum_{m,n}\tilde{\Delta}_{mn}(t)c_n(t)|m\rangle$$
\end{proof}
\section{Fermi's Golden rule}\label{sec:Fermi}
\begin{prop}
To first order, the probability of transitioning from $|n\rangle$ at $t=0$ to $|m\rangle$ with $m\neq n$ at time $t$ is
\begin{equation}
    P_{n\rightarrow m}^{(1)}(t)=\bigg|\int_0^te^{i\omega_{mn}t'}\frac{\Delta_{mn}(t')}{i\hbar}dt'\bigg|^2\label{TDP_firstorder}
\end{equation}
\end{prop}
\begin{proof}
We interpolate the full Hamiltonian by $\lambda\in[0,1]$, i.e. $H(t)=H^{(0)}+\lambda\Delta(t)$. Assume a series expansion for our perturbed state, i.e. $|\tilde{\psi}(t)\rangle=\sum_{i=0}^\infty|\tilde{\psi}^{(0)}(t)\rangle$, in the interaction picture. Plug into the Schr\"{o}dinger's equation (Eqn.~\ref{interaction1})
$$i\hbar\partial_t|\tilde{\psi}^{(0)}(t)\rangle+i\hbar\lambda\partial_t|\tilde{\psi}^{(1)}(t)\rangle+\dots=\lambda\tilde{\Delta}|\tilde{\psi}^{(0)}(t)\rangle+\lambda^2\tilde{\Delta}|\tilde{\psi}^{(1)}(t)\rangle+\dots$$
$$\implies i\hbar\partial_t|\tilde{\psi}^{(n+1)}(t)\rangle=\tilde{\Delta}|\tilde{\psi}^{(n)}(t)\rangle,~i\hbar\partial_t|\tilde{\psi}^{(0)}(t)\rangle=0$$
This means $|\tilde{\psi}^{(0)}(t)\rangle=|\tilde{\psi}^{(0)}(0)\rangle=|\psi(0)\rangle$ where the initial state is $|\psi(0)\rangle$. To first order, we have
\begin{equation}
|\tilde{\psi}^{(1)}(t)\rangle=\int_0^t\frac{1}{i\hbar}\tilde{\Delta}(t')|\psi(0)\rangle dt'\label{interaction3}
\end{equation}
and we have a nested integral for $|\tilde{\psi}^{(n)}(t)\rangle$ for $n>1$. The initial condition also gives us $|\tilde{\psi}^{(n)}(0)\rangle=0$ for $n\geq1$. We want to calculate probability $P_{n\rightarrow m}(t)$, to first order:
$$P_{n\rightarrow m}(t)=|\langle m|\psi(t)\rangle|^2=|\langle m|e^{-iH^{(0)}t/\hbar}|\tilde{\psi}(t)\rangle|^2\approx|\langle m|\tilde{\psi}^{(1)}(t)\rangle|^2$$
where $\langle m|\psi(0)\rangle=0$.
\end{proof}
\begin{cor}
To first order, $P^{(1)}_{m\rightarrow n}(t)=P^{(1)}_{n\rightarrow m}(t)$.
\end{cor}
\begin{cor}
For $|\psi(t)\rangle=\sum_nc_n(t)e^{-iE_nt/\hbar}|n\rangle$, we have
\begin{equation}
c_m^{(1)}(t)=\sum_n\int_0^te^{i\omega_{mn}t'}\frac{\Delta_{mn}(t')}{i\hbar}c_n(0)dt'\label{TDP_firstorder2}
\end{equation}
\end{cor}
\begin{proof}
We have $|\psi(0)\rangle=|\tilde{\psi}(0)\rangle=\sum_nc_n(0)|n\rangle=|\tilde{\psi}^{(0)}(0)\rangle$. Let $|\tilde{\psi}^{(k)}(t)\rangle=\sum_nc_n^{(k)}(t)|n\rangle$, then $c_n(t)=\sum_{i=0}^\infty\lambda^ic_n^{(i)}(t)$, then 
$$|\tilde{\psi}^{(0)}(t)\rangle=\sum_nc_n^{(0)}(t)|n\rangle=|\psi(0)\rangle=\sum_nc_n(0)|n\rangle\implies c_n^{(0)}(t)=c_n^{(0)}(0)=c_n(0)$$
We have
$$c_m^{(1)}=\langle m|\tilde{\psi}^{(1)}(t)\rangle=\sum_nc_n^{(1)}(t)\langle m|n\rangle=\langle m|\int_0^t\tilde{\Delta}(t')dt'\sum_nc_n(0)|n\rangle$$
giving our result.
\end{proof}
\begin{eg}[Nuclear magnetic resonance]\leavevmode
\begin{enumerate}
    \item Consider a time-independent perturbation $\Delta=\Omega S_x$, where $H^{(0)}=\omega_0S_z$. Assume $\Omega<<\omega_0$, i.e. perturbation is small. $H$ has exact energy eigenstates $|\mathbf{n};\pm\rangle$ (spin states that point to $\mathbf{n}=\omega/|\boldsymbol{\omega}|$, with energies $\pm0.5\hbar|\boldsymbol{\omega}|$). In the interaction picture, the perturbation is
    $$\tilde{\Delta}(t)=\exp(i\omega_0t\sigma_z/2)\Omega S_x\exp(-i\omega_0t\sigma_z/2)=\Omega(S_x\cos(\omega_0t)-S_y\sin(\omega_0t))$$
    The first order perturbed state (Eqn.~\ref{interaction3}) is
    $$|\tilde{\psi}^{(1)}(t)\rangle=\frac{\Omega}{i\hbar}\int_0^t(S_x\cos(\omega_0t')-S_y\sin(\omega_0t'))|\psi(0)\rangle dt'=\frac{\Omega}{i\hbar\omega_0}(S_x\sin(\omega_0t)+(\cos(\omega_0t)-1)S_y)|\psi(0)\rangle$$
    The full state is
    $$|\psi(t)\rangle=e^{-i\omega_0t\sigma_z/2}\bigg(1+\frac{\Omega}{i\hbar\omega_0}(S_x\sin(\omega_0t)+(\cos(\omega_0t)-1)S_y)\bigg)|\psi(0)\rangle+O((\Omega/\omega_0)^2)$$
    \item Consider a time-dependent perturbation $\Delta(t)=\Omega(S_x\cos(\omega_0t)+S_y\sin(\omega_0t)$, then in the interaction picture, 
    $$\tilde{\Delta}(t)=e^{i\omega_0t\sigma_z/2}\Omega(S_x\cos(\omega_0t)+S_y\sin(\omega_0t))e^{-i\omega_0t\sigma_z/2}=\Omega S_x$$
    which turns out to be time-independent, i.e. $\tilde{\Delta}(t)=\tilde{\Delta}(0)=\Omega S_x$. The state is then
    $$|\tilde{\psi}^{(0)}(t)\rangle=e^{-i\tilde{\Delta}t/\hbar}|\tilde{\psi}(0)\rangle=e^{-i\Omega t\sigma_x/2}|\psi(0)\rangle\implies|\psi^{(0)}(t)\rangle=e^{-i\omega_0t\sigma_z/2}e^{-i\Omega t\sigma_x/2}|\psi(0)\rangle$$
    i.e. the spin aligned along $\hat{z}$ at $t=0$ will move towards the $x$-$y$ plane, with angular velocity $\Omega$ while rotating around the $z$ axis with $\omega_0$. By first order perturbation theory (Eqn.~\ref{interaction3}),
    $$|\tilde{\psi}^{(1)}(t)\rangle=-i\Omega t\frac{\sigma_x}{2}|\psi(0)\rangle\implies|\psi(t)\rangle\approx e^{-i\omega_0t\sigma_x/2}\bigg(e^{-i\Omega t\sigma_x/2}-i\Omega t\frac{\sigma_x}{2}\bigg)|\psi(0)\rangle$$
    which is accurate only when $\Omega t<<1$.
\end{enumerate}
\end{eg}
We are also interested in transition between a discrete state and a state in a continuum. 
\begin{lemma}[Density of states]
The density of states is
\begin{equation}
\rho(E):=\frac{dN}{dE}=\bigg(\frac{L}{2\pi}\bigg)^3k\frac{m}{\hbar^2}d\Omega\label{DoS}
\end{equation}
\end{lemma}
\begin{proof}
To count states in the continuum, we replace the infinite space by a box of length $L$, with periodic boundary conditions (PBC) imposed. If the potential is short range, momentum eigenstates $\psi=L^{-3/2}e^{i\mathbf{k}\cdot\mathbf{r}}$ of large energy are a good representation of the continuum. PBC implies $k_iL=2\pi n_i$ for $i=x,y,z$. The total number of states in small volume $d^3k$ is $(L/2\pi)^3d^3k$. The energy dispersion is $E=\hbar^2k^2/2m$, and so
$$dE=\frac{\hbar^2k}{m}dk\implies d^3k=k^2dkd\Omega=k\frac{m}{\hbar^2}dEd\Omega\implies\rho(E)=\frac{dN}{dE}=\bigg(\frac{L}{2\pi}\bigg)^3d^3k=\bigg(\frac{L}{2\pi}\bigg)^3k\frac{m}{\hbar^2}d\Omega$$
\end{proof}
\begin{remarks}
The box construction conveniently replaces the continuous spectrum by a discrete spectrum, where the separation between the states is infinitesimal and can be made arbitrarily small by making $L$ sufficiently large. As $L\rightarrow\infty$, the discrete sum becomes a continuous integral in energy space.
\end{remarks}
\begin{thm}[Fermi's Golden rule]
If the final state is part of a continuum, the transition probability is linear in time.
\end{thm}
\begin{proof}
There are two cases to consider:
\begin{enumerate}
    \item Constant perturbation case: take $c_n(0)=\delta_{n,i}$, $|f\rangle\neq|i\rangle$ at $t_0$, then from Eqn.~\ref{TDP_firstorder2}:
    $$c_f^{(1)}(t_0)=\frac{1}{i\hbar}\int_0^{t_0}V_{fi}e^{i\omega_{fi}t'}dt'=\frac{V_{fi}e^{i\omega_{fi}t_0/2}}{E_f-E_i}(-2i)\sin\frac{\omega_{fi}t_0}{2}$$
    The probability is then
    $$P_{i\rightarrow f}(t_0)=|c_f^{(1)}(t_0)|^2=\frac{|V_{fi}|^2}{\hbar^2}\frac{\sin^2(\omega_{fi}t_0/2)}{((E_f-E_)/2\hbar)^2}$$
    This is accurate at $t_0$ if $P_{i\rightarrow f}(t_0)<<1$. For transition from a discrete state to the continuum, we want the probability that transition to any state in the continuum.
    $$\sum_fP_{i\rightarrow f}(t_0)=\int P_{i\rightarrow f}(t_0)\rho(E_f)dE_f=\frac{|V_{fi}|^2}{\hbar^2}\rho(E_f)\frac{\sin^2(\omega_{fi}t_0/2)}{((E_f-E_i)/2\hbar)^2}dE_f$$
    The function $\sinc^2(\omega_{fi}t_0/2)$ is large only over a range $4\pi/t$ (distance between two zeros closest to the origin), around $\omega_{fi}=0$. The bulk of the contribution to the integral will occur for a narrow range of energies $E_f$ near $E_i$. Assume $|V_{fi}|^2\rho$ is a slowly varying function of $E_f$, then this is approximately constant over the narrow energy interval.
    $$\sum_fP_{i\rightarrow f}(t_0)\approx\frac{|V_{fi}|^2}{\hbar^2}\rho(E_f=E_i)\hbar\int_{-\infty}^\infty\frac{\sin^2(\omega_{fi}t_0/2)}{(\omega_{fi}/2)^2}d\omega_{fi}$$
    Restricting the integral to the main lobe (which gives 90 percent of the contribution), the integral evaluates to $\pi$. The desired result is
    $$\sum_fP_{i\rightarrow f}(t)\approx\frac{|V_{fi}|^2}{\hbar^2}\rho(E_f)2\pi\hbar t$$
    \item Harmonic perturbation case: take $\Delta(t)=2H'\cos\omega t$ for $t>0$ and zero for $t\leq 0$. Here, $H'$ is time-independent and $\omega>0$. Again, from  Eqn.~\ref{TDP_firstorder2}:
    $$c_f^{(1)}(t_0)=\frac{H_{fi}'}{i\hbar}\int_0^{t_0}(e^{i(\omega_{fi}+\omega)t'}+e^{i(\omega_{fi}-\omega)t'})dt'=-\frac{H'_{fi}}{\hbar}\bigg[\frac{e^{i(\omega_{fi}+\omega)t_0}-1}{\omega_{fi}+\omega}+\frac{e^{i(\omega_{fi}-\omega)t_0}-1}{\omega_{fi}-\omega}\bigg]$$
    The first and second terms represent stimulated emission and absorption respectively, and are relevant for $\omega_{fi}\approx -\omega$ ($E_i>E_f$) and $\omega_{fi}\approx\omega$ ($E_f<E_i$) respectively. We consider the absorption case (the other case is similar), $\omega_{fi}\approx\omega\implies|\omega-\omega_{fi}|<<|\omega_{fi}|$. Then, the transition probability is
    \begin{equation}
    P_{i\rightarrow f}(\omega;t_0)=|c_f^{(1)}(t_0)|^2=\frac{|H'_{fi}|^2}{\hbar^2}\frac{\sin^2((\omega_{fi}-\omega)t_0/2)}{((\omega_{fi}-\omega)/2)^2}\label{harmonic}
    \end{equation}
    which is the same as for the constant perturbation case, with $V\rightarrow H'$ and $\omega_{fi}\rightarrow\omega_{fi}-\omega$. Summing this result over $f$ gives the same desired result, with $E_f=E_i+\hbar\omega$ for absorption.
\end{enumerate}
\end{proof}
\begin{defi}[Transition rate]
We define the transition rate to be $w=\sum_fP_{i\rightarrow f}(t)/t$, which by Fermi's Golden rule, it is a constant.
\end{defi}
\begin{eg}
Example for constant perturbation case and harmonic perturbation case respectively are auto-ionization and interaction of EM fields with atoms.
\end{eg}
\begin{remarks}\leavevmode
\begin{enumerate}
\item In comparison, the probability of transition between two discrete states is periodic in time.
\item If the transition does not conserve energy, i.e. $E_f\neq E_i$, then the initial state is an eigenstate of $H^{(0)}$ and after some time evolution, the state will become one that it is also a superposition of the $H^{(0)}$ eigenstates.
\item For the constant perturbation case, if $\frac{4|V_{fi}|^2}{(E_f-E_i)^2}<<1$, then $P_{i\rightarrow f}(t_0)$ is accurate $\forall t_0$. The amplitude is suppressed as $|E_f-E_i|$ grows. But if $E_f=E_i$, then take $\omega_{fi}\rightarrow 0$, and so $P_{i\rightarrow f}(t_0)|_{E_f=E_i}=\frac{|V_{fi}|^2}{\hbar^2}t_0^2$, which can only be trusted for small enough $t_0$ such that $P_{i\rightarrow f}(t_0)<<1$. We define the energy scale $\Delta_E(|V|^2\rho)$ over which the change in $|V_{fi}|^2\rho$ to be comparable to itself. We need the width of the main lobe to be much smaller than this scale:
$$\frac{4\pi\hbar}{t_0}<<\Delta_E(|V|^2\rho)\implies t_0>>\frac{\hbar}{\Delta_E(|V|^2\rho)}$$
Since $\sum_fP_{i\rightarrow f}(t_0)=wt_0$ and that $t_0$cannot be arbitrarily large, we require $t_0<<1/w$ at the first order. In another words,
$$\frac{\hbar}{\Delta_E(|V|^2\rho}<<t_0<<\frac{1}{w}\implies w=\frac{2\pi}{\hbar}|V_{fi}|^2\rho(E_f=E_i)<<\frac{1}{\hbar}\Delta_E(|V|^2\rho)$$
Let $|V_{fi}|^2\rho$ be $\tilde{E}$ which is a function of $E$, and evaluate it at $E_i$, then the inequality is
$$\tilde{E}(E_i)<<\Delta_E(\tilde{E})\implies\Delta_E(\tilde{E})\bigg|\frac{d\tilde{E}}{dE}\bigg|\approx\tilde{E}(E_i)\implies1>>\bigg|\frac{d\tilde{E}}{dE}\bigg|\propto\bigg|\frac{d\hbar w}{dE}\bigg|_{E_f=E_i}$$
\item For the harmonic perturbation case, we need the width of the main lobe to be small compared to the distance $2|\omega_{fi}|$ between the peaks of the probability distributions:
$$\frac{4\pi}{t_0}<<2|\omega_{fi}|\approx 2\omega\implies t_0>>\frac{1}{\omega}$$
i.e. $t_0$ must include a number of periods of the wave in order to identify the perturbation to be oscillatory. We also do not want $P_{i\rightarrow f}(\omega;t_0)$ to be too large. Consider the case of resonance $\omega=\omega_{fi}$, 
$$1>>P_{i\rightarrow f}(\omega_{fi};t_0)=\frac{|H'_{fi}|^2}{\hbar^2}t_0^2\implies t_0<<\frac{\hbar}{|H'_{fi}|}$$
This gives $|H_{fi}'|<<\hbar|\omega_{fi}|$, i.e. the matrix element of the perturbation is an energy that must be much smaller than the $H^{(0)}$ energy separating the two levels. A similar discussion would then give
\begin{equation}
    \bigg|\frac{d\hbar w}{dE}\bigg|_{E_f}<<1\label{validity}
\end{equation}
This is the regime where the theory is valid in, and it is true for either type of time-dependent perturbation.
\end{enumerate}
\end{remarks}
\begin{eg}[Ionization of Hydrogen]
We want to find the ionization rate for Hydrogen when hit by the harmonically varying electric field of an electromagnetic wave. Assume the Hydrogen atom has its electron on the ground state, then during this ionization, a photon ejects the bound electron which becomes free. If the wavelength of the photon is much bigger than the Bohr radius, we can neglect spatial dependence.
$$1<<\frac{\lambda}{a_0}=\frac{2\pi\hbar c}{\hbar\omega a_0}=\frac{4\pi}{\alpha}\frac{R_y}{\hbar\omega}$$
where $R_y\approx 13.6$eV is the Rydberg constant (magnitude of energy of the ground state) For the ejected electron to be free, require it to not `feel' the Coulomb field, i.e. $\hbar\omega>>R_y$. The perturbation is
$$\Delta(t)=-e\Phi(t)=e2E_0\cos\omega t~r\cos\theta\implies H'=eE_0r\cos\theta$$
The initial and final states are respectively $|i\rangle=\frac{1}{\sqrt{\pi a_0^3}}e^{-r/a_0}$ and $|f\rangle=L^{-3/2}e^{i\mathbf{k}\cdot\mathbf{r}}$ respectively. We expect the electron to be ejected maximally in the direction of $\mathbf{E}$, so align the electron momentum to be along the $z$ axis. $\theta$ is defined to be the angle between the electron momentum and the electric field, while $\theta''$ is that between the field and $\mathbf{r}$. The matrix element is
$$\langle f|H'|i\rangle=\frac{eE_0}{\sqrt{\pi a_0^3L^3}}\int r^2\sin\theta'e^{-ikr\cos\theta'}r\cos\theta''e^{-r/a_0}drd\theta'd\phi'$$
$\cos\theta''$ is the dot product between $\mathbf{E}/E$ and $\mathbf{r}/r$, i.e.
$$\cos\theta''=(\sin\theta\cos\phi)(\sin\theta'\cos\phi')+(\sin\theta\sin\phi)(\sin\theta'\sin\phi')+\cos\theta\cos\theta'=\sin\theta\sin\theta'\cos(\phi-\phi')+\cos\theta\cos\theta'$$
But $\int\cos(\phi-\phi')d\phi'=0$, so
$$\langle f|H'|i\rangle=\frac{eE_0}{\sqrt{\pi a_0^3L^3}}\cos\theta 2\pi\int_0^\infty r^3e^{-r/a_0}\int_{-1}^1\sin\theta'e^{-ikr\cos\theta'}\cos\theta'd\theta'dr=-i32\sqrt{\pi}eE_0a_0\frac{ka_0^4}{\sqrt{a_0^3L^3}}\frac{\cos\theta}{(1+k^2a_0^2)^3}$$
Fermi's Golden rule would then give the probability of ionization per unit time and per unit solid angle
$$dw=\frac{2\pi}{\hbar}|H_{fi}'|^2\rho(E_e)=\frac{2\pi}{\hbar}1024\pi(eE_0a_0)^2\frac{k^2a_0^5}{L^3}\frac{\cos^2\theta}{(1+k^2a_0^2)^6}\frac{L^3}{8\pi^3}\frac{m}{\hbar^2}kd\Omega$$
The total ionization probability per unit time is then $w=\int\frac{dw}{d\Omega}d\Omega$, where $\int\cos^2\theta d\Omega=4\pi/3$.
\end{eg}

\newpage
\section{Light and atoms}\label{sec:lightsnatoms}
We will discuss this topic in greater detail later.
\subsection{Emission and absorption}
Consider two discrete levels in an atom, then let $\omega_{ba}=\frac{1}{\hbar}(E_b-E_a)>0$. Let $N_a$ and $N_b$ be the population of atoms in the levels $|a\rangle$ and $|b\rangle$ respectively. The atoms are assumed to be in thermal equilibrium with a bath of photons, where the whole system is at temperature $T$. Let $U(\omega)d\omega$ to be the energy per unit volume (of thermal blackbody radiation of temperature $T$) in the frequency range $d\omega$.
\begin{defi}[Absorption]
An absorption process occurs when $|a\rangle\rightarrow|b\rangle$ at the resonant frequency $\omega_{ba}$. The absorption rate is defined to be proportional to $N_aU(\omega_{ba})$.
\end{defi}
\begin{defi}[Stimulated emission]
A stimulated emission process occurs when an incoming photon stimulates the process $|b\rangle\rightarrow|a\rangle$, with the release of additional photons, which are all coherent with the photon that does the stimulation. The stimulated emission rate is defined to be proportional to $N_bU(\omega_{ba})$.
\end{defi}
\begin{defi}[Spontaneous emission]
A spontaneous emission process is an emission process that occurs without the need of photons. The spontaneous emission rate is defined to be proportional to $N_b$.
\end{defi}
\begin{prop}
We cannot achieve equilibrium with just absorption and stimulated emission.
\end{prop}
\begin{proof}
Equilibrium is attained when $\dot{N}_b=0$. Let the proportionality constants of absorption and stimulated emission be $N_{ab}$ and $N_{ba}$ respectively
$$0=\dot{N}_b=B_{ab}U(\omega_{ba})N_a-B_{ba}U(\omega_{ba})N_b$$
But $U(\omega_{ba})\neq 0$ and $\frac{N_a}{N_b}=\frac{e^{-\beta E_a}}{e^{-\beta E_b}}=e^{-\beta\hbar\omega_{ab}}$, so $B_{ab}-B_{ba}e^{-\beta\hbar\omega_{ba}}=0$.
\end{proof}
\begin{prop}[Einstein's argument]
\begin{equation}
B_{ab}=B_{ba},\quad A=\frac{\hbar\omega_{ba}^3}{\pi^2c^3}\label{rate1}
\end{equation}
\end{prop}
\begin{proof}
Let the proportionality constant of spontaneous emission be $A$. With all three processes, we can attain equilibrium
$$0=\dot{N}_b=B_{ab}U(\omega_{ba})N_a-B_{ba}U(\omega_{ba})N_b-AN_b\implies U(\omega_{ba})=\frac{A}{B_{ab}}\frac{1}{e^{\beta\hbar\omega_{ba}}-\frac{B_{ba}}{B_{ab}}}$$
Comparing to the expression for blackbody radiation, we have $U(\omega_{ba})=\frac{\hbar\omega_{ba}^3}{\pi^2c^3}\frac{1}{e^{\beta\hbar\omega_{ba}}-1}$, then we get our desired results.
\end{proof}
\begin{prop}
Using time-dependent perturbation theory, we can work out the rate proportionality constants:
\begin{equation}
A=\frac{4}{3}\frac{\omega_{ba}^3}{\hbar c^3}|\mathbf{d}_{ab}|^2,\quad B_{ba}=\frac{4\pi^2}{3\hbar^2}|\mathbf{d_{ab}}|^2\label{rate2}
\end{equation}
where $\mathbf{d_{ab}}$ is the matrix element $q\langle a|\mathbf{r}|b\rangle$.
\end{prop}
\begin{proof}
Assume bound state of electron is not relativistic and optical frequencies $\lambda>>a_0$ so that we can ignore the spatial dependence of the electric field. The electric field is $\mathbf{E}(t)=2E_0\cos\omega t\mathbf{\hat{n}}$, where $\mathbf{\hat{n}}$ specifies polarization. The time-dependent part acts like a perturbation.
$$\Delta=-q\mathbf{r}\cdot\mathbf{E}(t)=2(-\mathbf{d}\cdot\mathbf{n} E_0)\cos\omega t\implies H'=-\mathbf{d}\cdot\mathbf{n}E_0$$
where we identify this as a harmonic perturbation. The electric field is an incoherent superposition of many waves with frequency $\omega_k$, amplitude $2E_0(\omega_k)$ and polarization $n_{\omega_k}$. Using Eqn.~\ref{harmonic}, we have
$$P_{b\rightarrow a}^k(t)=\frac{E_0^2(\omega_k)}{\hbar^2}|\mathbf{d}_{ab}\cdot\mathbf{n_{\omega_k}}|^2\frac{\sin^2((\omega_{ba}-\omega_k)t/2)}{((\omega_{ba}-\omega_k)/2)^2}$$
The energy density $u_E$ in the electric field $\mathbf{E}(t)=2E_0\cos(\omega t)\mathbf{n}$ is
$$u_E=\frac{|E(t)|^2}{8\pi}=\frac{E_0^2}{2\pi}\cos^2\omega t$$
The time-averaged value is $E_0^2/4\pi$. In a wave, the electric and magnetic energy are the same, and so the total energy density is $u=2\times\frac{E_0^2}{4\pi}\implies E_0^2=2\pi u$.\\[5pt]
Moreover, since the superposition of light is incoherent, we will add probabilities of transition due to each component of light. Even when we fix a frequency and a field amplitude, we have modes with all polarization directions. Averaging over all directions of $\mathbf{n}$,
$$\langle[\mathbf{d_{ab}}\cdot\mathbf{n}]\rangle=\bigg\langle|\sum_id^i_{ab}n_i|^2\bigg\rangle=\sum_{i,j}(d^i_{ab})^*d^j_{ab}\langle n_in_j\rangle$$
The average $\langle n_in_j\rangle$ is computed by integrating $\mathbf{n}$ over solid angle and dividing by $4\pi$, and is easily computed from symmetry considerations. 
\begin{itemize}
    \item $\langle n_in_j\rangle=0$ for $i\neq j$ since a reflection across a plane orthogonal to $x_i$ changes the sign of $n_i$ but not that of $n_j$, changing the sign of the integral
    \item $\langle n_xn_x\rangle=\langle n_yn_y\rangle=\langle n_zn_z\rangle$ due to rotational symmetry. Since $\sum_i\langle n_in_i\rangle=1$, we have $\langle n_in_j\rangle=\frac{1}{3}\delta_{ij}$.
\end{itemize}
We then have $\langle|\mathbf{d_{ab}}\cdot\mathbf{n}|^2\rangle=\frac{1}{3}|\mathbf{d_{ab}}|^2$.
Finally, we arrive at the contribution of each mode to be
$$P^k_{b\rightarrow a}(t)=\frac{2\pi}{3\hbar^2}|\mathbf{d_{ab}}|^2u(\omega_k)\frac{\sin^2((\omega_{ba}-\omega_k)t/2)}{((\omega_{ba}-\omega_k)/2)^2}$$
Consider a little interval $d\omega$ containing a large number of modes $\omega_k$ which we denote as $k\in d\omega$, we then have 
$$\sum_{k\in d\omega}u(\omega_k)\approx U(\omega)d\omega\implies\sum_k u(\omega_k)f(\omega_k)\approx\int U(\omega)f(\omega)d\omega$$
where $U(\omega)d\omega$ is the thermal radiation energy per unit volume in the range $d\omega$. Finally, we sum over the modes to obtain $P_{b\rightarrow a}(t)$, noting that we may pull $U(\omega)$ out of the integral for sufficiently large $t$.
$$P_{b\rightarrow a}(t)=\sum_{k}P^{k}_{b\rightarrow a}(t)\approx\frac{2\pi}{3\hbar^2}|\mathbf{d_{ab}}|^2U(\omega_{ba})\int\frac{\sin^2((\omega_{ba}-\omega_k)t/2)}{((\omega_{ba}-\omega_k)/2)^2}d\omega=\frac{4\pi^2}{3\hbar^2}|\mathbf{d_{ab}}|^2U(\omega_{ba})t$$
The rate associated with the transition is $P_{b\rightarrow a}(t)/t$. But this rate is $B_{ba}U(\omega_{ba})$, hence we obtained the desired results from Einstein's argument (Eqn.~\ref{rate1}).
\end{proof}
\begin{remarks}\leavevmode
\begin{enumerate}
\item If a particle can decay in more than one way, there is a decay rate $A_i$ for each decay mode. With decay rates $A_i$, $i=1,\dots, k$, the rates add up to give the total lifetime to be $\tau=1/\sum_{i=1}^kA_i$. 
\item If a state $|a\rangle$ can spontaneously decay into a state $|b\rangle$ and the state $|b\rangle$ can spontaneously decay into a state $|c\rangle$. The lifetime of $|a\rangle$ is determined solely by the decay rate from $|a\rangle$ to $|b\rangle$.
\end{enumerate}
\end{remarks}
\newpage
\subsection{Selection rules}
The result Eqn.~\ref{rate2} suggests that the spontaneous decay rate $A$ applied to an atomic transition, only requires an explicit calculation of the dipole matrix element $\mathbf{d_{ab}}$ connecting the initial and final states. 
\begin{defi}[Dipole operator]
$$\mathbf{d}:=q\mathbf{r}$$
with corresponding matrix element $\mathbf{d_{12}}=q\langle 1|\mathbf{r}|2\rangle$.
\end{defi}
\begin{remarks}
If $\mathbf{d_{12}}$ vanishes, the decay rate vanishes, then the decay does not happen in the electric dipole approximation.
\end{remarks}
\begin{prop}
To tell when the matrix element $\mathbf{d_{12}}$ vanishes, we require the parity operator $P:~\mathbf{r}\mapsto-\mathbf{r}$, i.e. $P\mathbf{r}P=-\mathbf{r}$. If the states have the same parity then $\mathbf{d_{12}}=0$. Otherwise, it may not be zero (another symmetry might kill it instead).
\end{prop}
\begin{proof}
Assume the states in the expectation value $\mathbf{d_{12}}$ are parity eigenstates such that
$$P|1\rangle=\varepsilon_1|1\rangle,\quad P|2\rangle=\varepsilon_2|2\rangle,\quad\varepsilon_1,\varepsilon_2=\pm1$$
Since $PP=1$, then 
$$\mathbf{d_{12}}=q\langle 1|PP\mathbf{r}PP|2\rangle=q\langle 1|P^\dag(P\mathbf{r}P)P|2\rangle=-\varepsilon_1\varepsilon_2\mathbf{d_{12}}$$
Hence, unless $\varepsilon_1\varepsilon_2=-1$, $\mathbf{d_{12}}=\boldsymbol{0}$. 
\end{proof}
\begin{prop}[Selection Rules]
For a central potential, the transition between states are constrained by the selection rules
\begin{equation}
    \ell'-\ell=\pm1,\quad m'-m=0,~\pm1\label{selectionrules}
\end{equation}
where $\ell$ and $m$ are the azimuthal and magnetic quantum numbers respectively.
\end{prop}
\begin{proof}
Consider the nested commutator
$$\frac{1}{(2i\hbar)^2}[L^2,[L^2,\mathbf{r}]]=(\mathbf{r}\cdot\mathbf{L})\mathbf{L}-\frac{1}{2}(L^2\mathbf{r}+\mathbf{r}L^2)$$
But $\mathbf{r}\cdot\mathbf{L}=0$, we then have $[L^2,[L^2,\mathbf{r}]]=2\hbar^2(L^2\mathbf{r}+\mathbf{r}L^2)$. Hit this with $\langle n',\ell',m'|$ from the left and $|n,\ell,m\rangle$ from the right. Then, we have
\begin{align}
0&=\bigg[(\ell'(\ell'+1)-\ell(\ell+1))^2-2\ell(\ell+1)-2\ell'(\ell'+1)\bigg]\langle n',\ell',m'|\mathbf{r}|n,\ell,m\rangle\nonumber\\&=\bigg[(\ell'(\ell'+1)-\ell(\ell+1))^2-2\ell(\ell+1)-2\ell'(\ell'+1)\bigg]\langle n',\ell',m'|\mathbf{r}|n,\ell,m\rangle\nonumber
\end{align}
This means $\ell+\ell'+1=\pm1$ or $\ell'-\ell=\pm1$. But $\ell,\ell'\geq0$, then we can only have $\ell+\ell'+1=1$ or $\ell=\ell'=0$, which obviously vanish by parity. The second possibility gives Eqn.~\ref{selectionrules}. For the second selection rule, we have
$$[L_z,z]=0\implies\langle n',\ell',m'|z|n,\ell,m\rangle=0$$
unless $m'=m$. Similarly,
$$[L_z,x]=i\hbar y\implies\hbar(m'-m)\langle n',\ell',m'|x|n,\ell,m\rangle=i\hbar\langle n',\ell',m'|y|n,\ell,m\rangle$$
$$[L_z,y]=-i\hbar x\implies\hbar(m'-m)\langle n',\ell',m'|y|n,\ell,m\rangle=-i\hbar\langle n',\ell',m'|x|n,\ell,m\rangle$$
Combine both results to get
$$[(m'-m)^2-1]\langle n',\ell',m'|y|n,\ell,m\rangle=0$$
Similar for $x$. The matrix elements must vanish when $m'-m=\pm1$.
\end{proof}
\begin{eg}
Any of the 2P states of the Hydrogen atom can decay to the 1S state. The 2S states of Hydrogen cannot dipole decay via spontaneous emission of a single photon. The only possible decay is to 1S and this is forbidden by the selection rules.
\end{eg}
\newpage
\section{Scattering}\label{sec:Scattering}
In a typical scattering experiment, we count the number of particles per second scattered into a detector. The incoming beam may contain a spread of particle energies, with the incoming particle flux varying with the transverse position across the beam. The target may be inhomogeneous and not regularly shaped. The detector will cover a finite spread of scattering angles $(\theta,\phi)$.
\begin{defi}[Incident flux]
We define the incident flux as the number of incoming particles per unit time per unit transverse area.
\end{defi}
\begin{defi}[Differential scattering cross-section]
The differential scattering cross-section is defined (per target particle) as the number of particles scattered per unit time into an infinitesimal element of solid angle $d\Omega$ in the direction $(\theta,\phi)$ per incident flux per infinitesimal solid angle. This depends on the beam energy, scattering angles $\theta,\phi$, and the nature of the incoming and target particles, but independent of the details of the experimental setup (the target density and geometry, the beam intensity and profile). The scattering rate into a given detector can be found by integrating the differential cross-section over all these details.
\end{defi}
\begin{defi}[Scattering amplitude]
By Born's rule, the differential scattering cross-section is the modulus squared of the scattering amplitude $f(\theta,\phi)$.
\end{defi}
\begin{defi}[Total cross-section]
The total scattering cross-section $\sigma$ characterises the scattering rate integrated over all angles, and is defined per target particle as the total number of particles scattered in any direction per unit time per incident flux, i.e. $\sigma=\int_0^{2\pi}\int_{-1}^{+1}\frac{d\sigma}{d\Omega}d\cos\theta~d\phi$.
\end{defi}
\begin{eg}[Measuring the cross-section]
Assume the target is thin enough that a given beam particle undergoes at most one scatter as it traverses the target. Under these conditions, the total cross-section $\sigma$ is
$$\sigma=\frac{N_{\text{scatt}}/T}{N_T\times(N_{\text{in}}/T/A)}=f_{\text{scatt}}\frac{A}{N_T}$$
where $A$ is the transverse area presented by the target to the incoming beam, $N_T$ is the total number of particles in the target. $N_{\text{in}}$ and $N_{\text{scatt}}$ are the total number of beam particles incident on the target and total number of particles scattered in any direction respectively. $f_{\text{scatt}}=N_{\text{scatt}}/N_{\text{in}}$ is the fraction of scattered particles.\\[5pt]
For a uniform foil of density $\rho$ consisting of particles of mass $m$, the final factor $A/N_T$ is $m/\rho$, i.e. the measured cross-section depends only on the nature of the target and not its geometry.\\[5pt]
Suppose that scattered particles are measured by a detector located in the direction $(\theta,\phi)$ and covering only a small region $\Delta\Omega$ of solid angle. Then the differential cross-section for that direction is measured as
$$\frac{d\sigma}{d\Omega}=\frac{N^{\Delta\Omega}_{\text{scatt}}/T}{N_T\times(N_{\text{in}}/T/A)\times\Delta\Omega}=\frac{N^{\Delta\Omega}_{\text{scatt}}}{N_{\text{in}}}\frac{A}{N_T}\frac{1}{\Delta\Omega}$$
\end{eg}
\begin{remarks}[Interpretation of cross-section]
The total cross-section has dimensions of area. It can be thought of as the effective transverse area $\sigma_{\text{eff}}$ presented by each scattering centre to the incoming beam. For a target containing $N_T$ particles, each having an effective transverse area $\sigma_{\text{eff}}$, the total effective area presented to the beam is $A_{\text{eff}}=N_T\sigma_{\text{eff}}$. If the physical transverse area presented by the target is $A$, then $f_{\text{scatt}}=A_{\text{eff}}/A$ and hence
$$\sigma=\frac{A_{\text{eff}}}{A}\frac{A}{N_T}=\frac{A_{\text{eff}}}{N_T}=\sigma_{\text{eff}}$$
\end{remarks}
\begin{remarks}[Scattering in classical dynamics]
Consider classical scattering in a central potential $V(r)$: each incoming particle follows a well-defined trajectory. There is a one-to-one correspondence between the scattering angle $\theta$ and the impact parameter $b=b(\theta)$. Incoming particles between $b$ and $b+db$ scatter into outgoing angles between $\theta$ and $\theta+d\theta$, hence
$$\frac{d\sigma}{d\Omega}=\frac{j~2\pi bdb}{j~d\Omega}=b(\theta)\bigg|\frac{db(\theta)}{d\cos\theta}\bigg|$$
where $d\Omega=2\pi d\cos\theta$. For a hard sphere elastic scattering, $b(\theta)=R\sin\alpha=R\cos(\theta/2)$ where $\alpha=0.5(\pi-\theta)$. The total cross-section is
$$\sigma=\int\frac{d\sigma}{d\Omega}d\Omega=\int b(\theta)\bigg|\frac{db(\theta)}{d\cos\theta}\bigg|d\Omega=\frac{1}{4}R^2\times 4\pi$$
hence equal to the projected transverse area of the sphere.
\end{remarks}
\begin{eg}
For classical Coulomb scattering, the incoming particle follows a hyperbolic trajectory. The Coulomb potential is $V(r)=\frac{Z_1Z_2e^2}{4\pi\varepsilon_0r}$. The connection between impact parameter and scattering angle is the same for the repulsive and attractive cases: $b(\theta)=\frac{|C|}{2E}\cot(\theta/2)$, which leads to the Rutherford formula $\frac{d\sigma}{d\Omega}=\frac{C^2}{16E^2}\frac{1}{\sin^4(\theta/2)}$.
\end{eg}
In the quantum description of scattering, we no longer have a well-defined correspondence $b(\theta)$. The scattering is inherently probabilistic in nature. 
\begin{defi}[Born's approximation]
We limit to consider the high energy (but still non-relativistic) scattering from a fixed `small' potential $V(\mathbf{r})$.
\end{defi}
The Born cross-section is obtained using the Fermi's Golden Rule in the zero frequency limit $\omega=0$.
\begin{prop}
$$\frac{d\sigma}{d\Omega}=\bigg(\frac{m}{2\pi\hbar^2}\bigg)^2\bigg|\int V(\mathbf{r})e^{i\mathbf{q}\cdot\mathbf{r}}d^3\mathbf{r}\bigg|^2$$
\end{prop}
\begin{proof}
Approximating the incoming and outgoing particles by plane waves, the initial $\psi_i(\mathbf{r},t)=\frac{1}{L^{3/2}}e^{i\mathbf{k}\cdot\mathbf{r}-iEt/\hbar}$ and final state $\psi_f(\mathbf{r},t)=\frac{1}{L^{3/2}}e^{i\mathbf{k'}\cdot\mathbf{r}-iEt/\hbar}$ has 3-momenta $\mathbf{p}=\hbar\mathbf{k}$ and $\mathbf{p'}=\hbar\mathbf{k'}$. Fermi's Golden rule gives the transition rate as
$$\Gamma(i\rightarrow f)=\frac{2\pi}{\hbar}|\langle\psi_f|V(\mathbf{r})|\psi_i\rangle|^2g(E),\quad\langle\psi_f|V(\mathbf{r})|\psi_i\rangle=\frac{1}{L^3}\int V(\mathbf{r})e^{i\mathbf{q}\cdot\mathbf{r}}d^3\mathbf{r}$$
where for an elastic scattering, $|\mathbf{k}|=|\mathbf{k'}|$ and $\mathbf{q}=\mathbf{k}-\mathbf{k'}$ is the momentum transfer. Restricting to a solid angle $d\Omega$, the density of states $g(E)$ is available to the final state scattered particle is
$$g(E)=\frac{L^3mk}{2\pi^2\hbar^2}\frac{d\Omega}{4\pi}=\frac{L^3mk}{8\pi^3\hbar^2}d\Omega$$
For an incoming plane wave, $\psi(\mathbf{r})=L^{-3/2}e^{i\mathbf{k}\cdot\mathbf{r}}$, the flux is $\mathbf{j}=\frac{\hbar\mathbf{k}}{mL^3}$. The differential cross-section is
$$\frac{d\sigma}{d\Omega}=\frac{\Gamma(i\rightarrow f)}{(\hbar k/mL^3)d\Omega}=\bigg(\frac{m}{2\pi\hbar^2}\bigg)^2\bigg|\int V(\mathbf{r})e^{i\mathbf{q}\cdot\mathbf{r}}d^3\mathbf{r}\bigg|^2$$
where the arbitrary normalisation volume $L^3$ is now cancelled. The scattering angle is $\sqrt{d\sigma/d\Omega}$.
\end{proof}
\begin{cor}
For a central potential $V(\mathbf{r})=V(r)$, 
$$\frac{d\sigma}{d\Omega}=\bigg(\frac{m}{2\pi\hbar^2}\bigg)^2\bigg|\frac{4\pi}{q}\int_0^\infty V(r)r\sin(qr)dr\bigg|^2$$
\end{cor}
\begin{proof}
For a given incoming momentum $p$ and scattering angle $\theta$, 
$$\mathbf{q}=\mathbf{k}-\mathbf{k'}=k(0-\sin\theta,0,1-\cos\theta)\implies q=2k\sin(\theta/2)$$
To compute $V(q)$, set up spherical polar coordinates defined relative to $\mathbf{q}$. $\mathbf{q}$ is a constant independent of $\mathbf{r}$ in the FT integration. Define $\theta'$ to be the relative angle, i.e. $\mathbf{q}\cdot\mathbf{r}=qr\cos\theta'$. For an isotropic potential, the FT integration is
$$\int_0^\infty\int_{-1}^1V(r)e^{iqr\cos\theta'}2\pi r^2dr~d\cos\theta'=\frac{2\pi2}{q}\int_0^\infty V(r)\frac{r^2}{r}\sin(qr)dr$$
\end{proof}
\begin{eg}[Yukawa potential]
Yukawa potential is an isotropic screened Coulomb potential $V(r)=\frac{Z_1Z_2e^2}{4\pi\varepsilon_0r}e^{-\lambda r}$. The FT is
$$\int V(\mathbf{r})e^{i\mathbf{q}\cdot\mathbf{r}}d^3\mathbf{r}=\frac{4\pi}{q}\frac{Z_1Z-2e^2}{4\pi\varepsilon_0}\int_0^\infty e^{-\lambda r}\sin(qr)dr=\frac{Z_1Z-2e^2}{q\varepsilon_0}\frac{q}{\lambda^2+q^2}$$
but $q=|\mathbf{k}-\mathbf{k'}|=2k\sin(\theta/2)$. By Born approximation, the differential cross-section is
$$\frac{d\sigma}{d\Omega}=\bigg(\frac{m}{2\pi\hbar^2}\bigg)^2\bigg|\frac{Z_1Z_2e^2}{\varepsilon_0}\frac{1}{4k^2\sin^2(\theta/2)+\lambda^2}\bigg|^2$$
Take $\lambda\rightarrow 0$, we obtain the Rutherford cross-section, similar to the classical case.
\end{eg}
\part{Applications}
\chapter{Quantum Optics}
\section{Coherent states}
\begin{defi}[Coherent states]
Coherent states are quantum states with classical behaviour. 
\end{defi}
\begin{eg}[Simple example]
Consider the unitary translation operator $T_{x_0}=e^{-ipx_0/\hbar}$ where $p$ is the momentum operator. Since it is unitary, its action on the position operator $x$ (Eqn.~\ref{commutator_generator}) is $T^\dag_{x_0}xT_{x_0}=x+x_0$, where $\langle x\rangle_{T_{x_0}\psi}=\langle x\rangle_{\psi}+x_0$. For an arbitrary momentum state $|p\rangle$, we have
$$\langle p|T_{x_0}|x_1\rangle=\langle p|e^{-ipx_0/\hbar}|x_1\rangle=e^{-ipx_0/\hbar}\frac{e^{-ipx_1/\hbar}}{\sqrt{2\pi\hbar}}=\langle p|x_1+x_0\rangle\implies T_{x_0}|x_1\rangle=|x_1+x_0\rangle$$
The translation operator's action is consistent with expectation. A coherent state is constructed by translating the ground state of the harmonic oscillator by a distance $x_0$, i.e. $|\tilde{x}_0\rangle=T_{x_0}|0\rangle$, such that $\langle\tilde{x_0}|\tilde{x}_0\rangle=1$ and $\psi_{\tilde{x}_0}=\langle x|\tilde{x}_0\rangle=\langle x-x_0|0\rangle=\psi_0(x-x_0)$. Further, it satisfies
\begin{enumerate}
    \item $\langle\tilde{x}_0|x|\tilde{x}_0\rangle=\langle 0|T^\dag_{x_0}xT_{x_0}|0\rangle=\langle 0|x+x_0|0\rangle=x_0$ where $\langle 0|x|0\rangle=0$
    \item $\langle\tilde{x}_0|p|\tilde{x}_0\rangle=\langle 0|T^\dag_{x_0}pT_{x_0}|0\rangle=\langle 0|p|0\rangle=0$
    \item The action of $T_{x_0}$ on the harmonic oscillator Hamiltonian is $H+m\omega^2x_0x+\frac{1}{2}m\omega^2x_0^2$, so
    $$\langle\tilde{x}_0|H|\tilde{x}_0\rangle=\langle 0|T_{x_0}^\dag HT_{x_0}|0\rangle=\frac{1}{2}\hbar\omega+\frac{1}{2}m\omega^2x_0^2$$
\end{enumerate}
The time evolution is easily evaluated in the Heisenberg's picture. Using the earlier results,
$$\langle x\rangle_{\tilde{x}_0}(t)=\langle\tilde{x}_0|x_H(t)|\tilde{x}_0\rangle=x_0\cos\omega t,\quad \langle p\rangle_{\tilde{x}_0}(t)=\langle\tilde{x}_0|p_H(t)|\tilde{x}_0\rangle=-m\omega x_0\sin\omega t=m\frac{d}{dt}\langle x\rangle(t)$$
Next, we find the uncertainities. At a fixed time, we have
$$\langle\tilde{x}_0|x^2|\tilde{x}_0\rangle=\langle 0|T^\dag_{x_0}x^2T_{x_0}|0\rangle=\langle 0|(x+x_0)^2|0\rangle=\frac{\hbar}{2m\omega}+x_0^2,\quad \langle\tilde{x}_0|p^2|\tilde{x}_0\rangle=\langle 0|p^2|0\rangle=\frac{1}{2}m\hbar\omega$$
The coherent state thus saturates the Heisenberg uncertainty relation:
$$(\langle x^2\rangle-(\langle x\rangle)^2)(\langle p^2\rangle-(\langle p\rangle)^2)=\frac{\hbar}{2m\omega}\frac{m\hbar\omega}{2}=\frac{\hbar^2}{4}\implies\Delta x\Delta p=\frac{\hbar}{2}$$
We can extend our computation to generic times
$$(\Delta x)^2(t)=\langle\tilde{x}_0|x^2_H(t)|\tilde{x}_0\rangle-x_0^2\cos^2\omega t=\langle\tilde{x}_0|x^2|\tilde{x}_0\rangle\cos^2\omega t+\langle\tilde{x}_0|p^2|\tilde{x}_0\rangle\frac{\sin^2\omega t}{m^2\omega^2}+\frac{\cos\omega t\sin\omega t}{m\omega}\langle\tilde{x}_0|xp+px|\tilde{x}_0\rangle$$
The last term can be shown to be zero:
$$\langle\tilde{x}_0|xp+px|\tilde{x}_0\rangle=\langle 0|(x+x_0)p+p(x+x_0)|0\rangle=\langle 0|xp+px|0\rangle=\frac{1}{2}i\hbar\langle 0|aa^\dag+(-a)(a^\dag)|0\rangle=0$$
using earlier results, we can show $\Delta x(t)\Delta p(t)=\hbar/2$ $\forall t$. In another words, the state remains Gaussian $\forall t$. Since $\Delta x$ is time-independent, the Gaussian does not change shape, i.e. the state moves `coherently' without changing its shape. Finally, we define our length scale to be $d=\sqrt{\frac{\hbar}{m\omega}}$, then the uncertainities are $\Delta x(t)=d/\sqrt{2}$ and $\Delta p(t)=m\omega d/\sqrt{2}$. The ratio $\Delta x(t)/\sqrt{\overline{\langle x^2\rangle(t)}}$ is
$$\frac{\Delta x(t)}{\sqrt{\overline{\langle x^2\rangle(t)}}}=\frac{d/\sqrt{2}}{\sqrt{0.5d^2+0.5x_0^2}}=\frac{1}{\sqrt{1+(x_0/d)^2}}$$
We get the same result for the momentum ratio.
\end{eg}
\begin{defi}[Generalized coherent states and displacement operator]
A coherent state $|\alpha\rangle$ is defined as a state which is unchanged when operated on by an annihilation operator
\begin{equation}
    a|\alpha\rangle=\alpha|\alpha\rangle\label{coherent}
\end{equation}
$a$ is not Hermitian and thus $\alpha$ is in general complex. It is constructed by displacing the ground state of a harmonic oscillator by the displacement operator, i.e. $D(\alpha)|0\rangle$ where
\begin{equation}
    D(\alpha)=e^{\alpha a^\dag-\alpha^*a}\label{displacement}
\end{equation}
\end{defi}
\begin{prop}
\begin{equation}
    |\alpha\rangle=e^{-|\alpha|^2/2}\sum_{n=0}^\infty\frac{\alpha^n}{\sqrt{n!}}|n\rangle\label{coherent2}
\end{equation}
\end{prop}
\begin{proof}
Expand the state $|\alpha\rangle$ in terms of the complete set of number states $|n\rangle$: $|\alpha\rangle=\sum_{n=0}^\infty c_n|n\rangle$, $c_n=\langle n|\alpha\rangle$. Recall $a^\dag|n\rangle=\sqrt{n+1}|n+1\rangle$, we have $\langle n|a|\alpha\rangle=\sqrt{n+1}\langle n+1|\alpha\rangle$. But, $\langle n|a|\alpha\rangle$ is also $\alpha\langle n|\alpha\rangle$. We thus obtain a recursive relation between successive coefficients, i.e. a series expansion:
$$\sqrt{n+1}c_{n+1}=\alpha c_n\implies c_n=\frac{\alpha^n}{\sqrt{n!}}c_0=\frac{\alpha^n}{\sqrt{n!}}\langle 0|\alpha\rangle\implies|\alpha\rangle=\langle 0|\alpha\rangle\sum_{n=0}^\infty\frac{\alpha^n}{\sqrt{n!}}|n\rangle$$
The normalization factor $\langle 0|\alpha\rangle$ is determined by $1=\langle\alpha|\alpha\rangle=|\langle 0|\alpha\rangle|^2\sum_{n=0}^\infty\frac{|\alpha|^{2n}}{n!}\langle n|n\rangle=|\langle 0|\alpha\rangle|^2e^{|\alpha|^2}$.
\end{proof}
\begin{cor}
The probability of having $n$ quanta in the state $|\alpha\rangle$ follows a Poisson distribution with mean value $|\alpha|^2$.
\end{cor}
\begin{proof}
The probability of having $n$ quanta in the state $|\alpha\rangle$ is
$$P(n)=|\langle n|\alpha\rangle|^2=\bigg|e^{-|\alpha|^2/2}\sum_{j=0}^\infty\frac{\alpha^j}{\sqrt{j!}}\langle n|j\rangle\bigg|^2=\frac{\alpha^{2n}e^{-|\alpha|^2}}{n!}=\frac{\langle n\rangle^ne^{-\langle n\rangle}}{n!}$$
with $\langle n\rangle=\langle\alpha|a^\dag a|\alpha\rangle=|\alpha|^2$.
\end{proof}
\begin{cor}
The coherent state saturates the Heisenberg uncertainty.
\end{cor}
\begin{proof}
The harmonic oscillator (temporarily restore operator notation)'s expectation values are
$$\hat{x}=\sqrt{\frac{\hbar}{2m\omega}}(\hat{a}+\hat{a}^\dag)\implies\langle\hat{x}\rangle=\sqrt{\frac{\hbar}{2m\omega}}(\alpha+\alpha^*),\quad\hat{p}=-i\sqrt{\frac{\hbar m\omega}{2}}(\hat{a}-\hat{a}^\dag)\implies\langle\hat{p}\rangle=-i\sqrt{\frac{\hbar m\omega}{2}}(\alpha-\alpha^*)$$
$$\langle\hat{x}^2\rangle=\frac{\hbar}{2m\omega}\langle\alpha|\hat{a}^2+\hat{a}\hat{a}^\dag+\hat{a}^\dag\hat{a}+(\hat{a}^\dag)^2|\alpha\rangle=\frac{\hbar}{2m\omega}(1+(\alpha+\alpha^*)^2)$$
$$\langle\hat{p}^2\rangle=-\frac{\hbar m\omega}{2}\langle\alpha|\hat{a}^2-\hat{a}\hat{a}^\dag-\hat{a}^\dag\hat{a}+(\hat{a}^\dag)^2|\alpha\rangle=\frac{\hbar m\omega}{2}(1-(\alpha-\alpha^*)^2)$$
The uncertainties are
$$(\Delta x)^2=\langle x^2\rangle-\langle x\rangle^2=\frac{\hbar}{2m\omega}(1+(\alpha+\alpha^*)^2)-\frac{\hbar}{2m\omega}(\alpha+\alpha^*)^2=\frac{\hbar}{2m\omega}$$
$$(\Delta p)^2=\langle p^2\rangle-\langle p\rangle^2=\frac{\hbar m\omega}{2}(1-(\alpha+\alpha^*)^2)-\frac{\hbar m\omega}{2}(\alpha-\alpha^*)^2=\frac{\hbar m\omega}{2}$$
This gives the saturated Heisenberg uncertainty relation $(\Delta p)^2(\Delta x)^2=\hbar^2/4$.
\end{proof}
\begin{cor}
The position representation of a coherent state is a Gaussian wavefunction.
\end{cor}
\begin{proof}
Let the position representation of the coherent state $|\alpha\rangle$ be represented by $\psi(x)$. The eigenvalue equation gives
$$a|\alpha\rangle=\alpha|\alpha\rangle\implies\sqrt{\frac{m\omega}{2\hbar}}\bigg(x+\frac{\hbar}{m\omega}\frac{\partial}{\partial x}\bigg)\psi(x)=\alpha\psi(x)$$
One can then verify $\psi(x)\propto\exp[-\frac{(x-x_0)^2}{4(\Delta x)^2}+i\frac{p_0x}{\hbar}]$ is a solution.
\end{proof}
\begin{cor}
Consider the time evolution of a coherent state in the case where the Hamiltonian is time-independent. Using the time evolution operator $\hat{U}(t)$, show that a coherent state at $t = 0$ always evolves into another coherent state at some subsequent time.
\end{cor}
\begin{proof}
Using the displacement operator (Eqn.~\ref{displacement}) and $D(\alpha)=e^{\alpha a^\dag}e^{-\alpha^* a}e^{-|\alpha|^2/2}$ but $e^{-\alpha^*a}|0\rangle=\frac{(\alpha^*a)^0}{\sqrt{0!}}|0\rangle=|0\rangle$. Since the number basis is complete, we may write the coherent state $|\alpha\rangle$ in terms of the number basis. Introducing time-dependence
$$|\alpha\rangle=e^{-|\alpha|^2/2}\sum_{n=0}^\infty\frac{(\alpha)^n}{n!}\sqrt{n!}|n\rangle\implies|\alpha(t)\rangle=e^{-i\omega t/2}e^{-|\alpha|^2/2}\sum_{n=0}^\infty\frac{\alpha^n}{\sqrt{n!}}e^{-in\omega t}|n\rangle=e^{-i\omega t/2}|e^{-i\omega t}\alpha\rangle$$
where $|n(t)\rangle=e^{-iE_nt/\hbar}|n(0)\rangle$, $E_n=\hbar\omega(n+0.5)$. Up to a phase, this is a coherent state.
\end{proof}
\begin{remarks}
The coherent state preserves the coherent state form as it evolves, where $|\alpha(t)\rangle$ has eigenvalue $\alpha e^{-i\omega t}$ (for the eigenoperator $a$), then the expectations may replace $\alpha\rightarrow\alpha e^{-i\omega t}$:
$$\langle x\rangle=\sqrt{\frac{\hbar}{2m\omega}}2\text{Re}[\alpha e^{-i\omega t}],\quad \langle p\rangle=-i\sqrt{\frac{m\hbar\omega}{2}}2i\text{Im}[\alpha e^{-i\omega t}]$$
Write $\alpha=|\alpha|e^{i\phi}$, then the position and momentum of centre of the coherent state follow that of the classical state, i.e. $x_0(t)=\sqrt{\frac{2\hbar}{m\omega}}|\alpha|\cos(\phi-\omega t)$, $p_0(t)=\sqrt{2m\hbar\omega}|\alpha|\sin(\phi-\omega t)$. The width of the coherent state wavepacket $\Delta x$ remains time independent. The classical character is lost once the energy or photon number is measured.
\end{remarks}
\begin{cor}
Two different coherent states are not orthogonal.
\end{cor}
\begin{proof}
$$\langle\alpha_2|\alpha_1\rangle=e^{-0.5(|\alpha_1|^2+|\alpha_2|^2)}\sum_n\frac{(\alpha_2^*\alpha_1)^n}{n!}=e^{-0.5(|\alpha_1|^2+|\alpha_2|^2-2\alpha_2^*\alpha_1)}=e^{-0.5|\alpha_1-\alpha_2|^2}e^{-0.5(\alpha_2^*\alpha_1-\alpha_2\alpha_1^*)}$$
which means $|\langle\alpha_2|\alpha_1\rangle|^2=e^{-|\alpha_1-\alpha_2|^2}$, i.e. the degree of overlap diminishes rapidly as the two eigenvalues are moved apart in the complex plane. Coherent states are said to be overcomplete.
\end{proof}
\section{Polarsation states}
\begin{defi}[Plane wave vector potential]
The solution to the classical equation (under Coulomb gauge) $\nabla^2\mathbf{A}=\frac{1}{c^2}\frac{\partial^2\mathbf{A}}{\partial t^2}$ is a plane-wave solution of the form $\mathbf{A}(\mathbf{r},t)=A_ke^{i(\mathbf{k}\cdot\mathbf{r}-\omega t)}\mathbf{e}(\mathbf{k})~ \forall\mathbf{k}$, where $\mathbf{e}(\mathbf{k})$ is a unit vector. 
\end{defi}
\begin{remarks}
Since $\boldsymbol{\nabla}\cdot(e^{i\mathbf{k}\cdot\mathbf{r}}\mathbf{e}(\mathbf{k}))=i(\mathbf{k}\cdot\mathbf{e})e^{i\mathbf{k}\cdot\mathbf{r}}$, the Coulomb gauge $\boldsymbol{\nabla}\cdot\mathbf{A}=0$ gives the constraint $\mathbf{k}\cdot\mathbf{e}(\mathbf{k})=0$. For each $\mathbf{k}$, we obtain two linearly independent plane wave solutions.
$$\mathbf{k}\cdot\mathbf{e_1}(\mathbf{k})=0=\mathbf{k}\cdot\mathbf{e_2}(\mathbf{k})$$
The 3-wavevector $\mathbf{k}$ defines the direction of propagation of the EM wave, with $\mathbf{e_1}$ and $\mathbf{e_2}$ being perpendicular to this direction.
\end{remarks}
\begin{eg}
For $\mathbf{k}$ directed along the $z$ axis, $\mathbf{k}=(0,0,\mathbf{k})$, a possible choice of the vectors is $\mathbf{e_1}(\mathbf{k})=(1,0,0)$, $\mathbf{e_2}(\mathbf{k})=(0,1,0)$. The electric fields corresponding to this choice are the plane polarisation states.  We could have also used
$$\begin{pmatrix}\mathbf{e_L}(\mathbf{k})\\\mathbf{e_R}(\mathbf{k})\\\end{pmatrix}=\frac{1}{\sqrt{2}}\begin{pmatrix}1&i\\1&-i\\\end{pmatrix}\begin{pmatrix}\mathbf{e_1}(\mathbf{k})\\\mathbf{e_2}(\mathbf{k})\\\end{pmatrix}$$
which gives the left- and right-circular polarisation. 
\end{eg}
\begin{remarks}
Each wavevector $\mathbf{k}$ thus gives rise to two independent modes $(\mathbf{k},\lambda)$ of the EM field
$$\mathbf{A}(\mathbf{r},t)=\sum_{\lambda}A_{\mathbf{k},\lambda}e^{i(\mathbf{k}\cdot\mathbf{r}-\omega t)}\mathbf{e_\lambda}(\mathbf{k})$$
\end{remarks}
\begin{prop}
For a general direction $\mathbf{k}$ (not just along the $z$ axis) in 3D spherical coordinates, the corresponding plane polarisation vectors $\{\mathbf{e_1},\mathbf{e_2}\}$ and circular polarization vectors $\{\mathbf{e_L},\mathbf{e_R}\}$ are
$$\mathbf{e_1}=\begin{pmatrix}\cos\theta\cos\phi\\\cos\theta\sin\phi\\-\sin\theta\\\end{pmatrix},~\mathbf{e_2}=\begin{pmatrix}-\sin\phi\\\cos\phi\\0\\\end{pmatrix},\quad\mathbf{e_L}=-\frac{1}{\sqrt{2}}\begin{pmatrix}\cos\theta\cos\phi-i\sin\phi\\\cos\theta\sin\phi+i\cos\phi\\-\sin\theta\\\end{pmatrix},~\mathbf{e_R}=\frac{1}{\sqrt{2}}\begin{pmatrix}\cos\theta\cos\phi+i\sin\phi\\\cos\theta\sin\phi-i\cos\phi\\-\sin\theta\\\end{pmatrix}$$
\end{prop}
\begin{proof}
To take $\mathbf{k}=(0,0,1)$ to $(\sin\theta\cos\phi,\sin\theta\sin\phi,\cos\theta)$, we act using:
$$R_\phi R_\theta=\begin{pmatrix}\cos\phi&-\sin\phi&0\\\sin\phi&\cos\phi&0\\0&0&1\\\end{pmatrix}\begin{pmatrix}\cos\theta&0&\sin\theta\\0&1&0\\-\sin\theta&0&\cos\theta\\\end{pmatrix}$$
i.e. rotate through $\theta$ about the $y$ axis first, followed by rotation through $\phi$ about the $z$ axis.
\end{proof}
\section{Mode expansion}
\begin{defi}[Mode expansion]
The most general real solution $\mathbf{A}(\mathbf{r},t)$ to the wave equation is obtained by integrating over all possible modes $(\mathbf{k},\lambda)$ of the EM field:
$$\mathbf{A}(\mathbf{r},t)=\sum_{\lambda=1}^2\int\bigg[A_{\mathbf{k},\lambda}e^{i(\mathbf{k}\cdot\mathbf{r}-\omega t)}\mathbf{e_\lambda}(\mathbf{k})+A_{\mathbf{k},\lambda}^*e^{-i(\mathbf{k}\cdot\mathbf{r}-\omega t)}\mathbf{e_\lambda}^*(\mathbf{k})\bigg]d^3\mathbf{k}$$
To integrate over $\mathbf{k}$, we employ the standard trick of imposing periodic boundary conditions on the surface of an arbitrary volume $V=L^3$. The allowed $\mathbf{k}$ vectors occupy a regular 3D lattice of points with spacing $2\pi/L$ along $k_x$, $k_y$, $k_z$, i.e. the continuous integral over $\mathbf{k}$ space then becomes a summation over an infinite cubic lattice of discrete $\mathbf{k}$ points. The vector potential thus has the mode (plane wave, Fourier) expansion
\begin{equation}
    \mathbf{A}(\mathbf{r},t)=\sum_{\mathbf{k},\lambda}\bigg[A_{\mathbf{k},\lambda}e^{i(\mathbf{k}\cdot\mathbf{r}-\omega t)}\mathbf{e_{\lambda}}(\mathbf{k})+A^*_{\mathbf{k},\lambda}e^{-i(\mathbf{k}\cdot\mathbf{r}-\omega t)}\mathbf{e^*_\lambda}(\mathbf{k})\bigg]\label{mode}
\end{equation}
\end{defi}
\begin{cor}
The electromagnetic fields are
\begin{equation}
\mathbf{E}(\mathbf{r},t)=\sum_{\mathbf{k},\lambda}i\omega(\mathbf{k})\bigg[A_{\mathbf{k},\lambda}e^{i(\mathbf{k}\cdot\mathbf{r}-\omega t)}\mathbf{e_{\lambda}}(\mathbf{k})-A^*_{\mathbf{k},\lambda}e^{-i(\mathbf{k}\cdot\mathbf{r}-\omega t)}\mathbf{e_\lambda^*}(\mathbf{k})\bigg]\label{Efield}
\end{equation}
\begin{equation}
\mathbf{B}(\mathbf{r},t)=\sum_{\mathbf{k},\lambda}i\mathbf{k}\times\bigg[A_{\mathbf{k},\lambda}e^{i(\mathbf{k}\cdot\mathbf{r}-\omega t)}\mathbf{e_{\lambda}}(\mathbf{k})-A^*_{\mathbf{k},\lambda}e^{-i(\mathbf{k}\cdot\mathbf{r}-\omega t)}\mathbf{e_\lambda^*}(\mathbf{k})\bigg]\label{Bfield}
\end{equation}
hence the total electromagnetic energy $U$ contained within the normalisation volume $V$ is
$$U=V\sum_{\mathbf{k},\lambda}\varepsilon_0\omega(\mathbf{k})^2\bigg[A_{\mathbf{k},\lambda}A_{\mathbf{k},\lambda}^*+A_{\mathbf{k},\lambda}^*A_{\mathbf{k},\lambda}\bigg]$$
\end{cor}
\begin{proof}
Follows from $\mathbf{E}=-\frac{\partial\mathbf{A}}{\partial t}$ and $\mathbf{B}=\boldsymbol{\nabla}\times\mathbf{B}$. We use the vector identity:
$$\boldsymbol{\nabla}\times(A_{\mathbf{k},\lambda}e^{i\mathbf{k}\cdot\mathbf{r}}\mathbf{e_\lambda}(\mathbf{k}))=iA_{\mathbf{k},\lambda}e^{i\mathbf{k}\cdot\mathbf{r}}(\mathbf{k}\times\mathbf{e_\lambda}(\mathbf{k}))$$
where $\boldsymbol{\nabla}e^{i\mathbf{k}\cdot\mathbf{r}}=i\mathbf{k}e^{i\mathbf{k}\cdot\mathbf{r}}$. Finally, the energy is $U=\frac{1}{2}\int_V(\varepsilon_0|\mathbf{E}|^2+\frac{1}{\mu_0}|\mathbf{B}|^2)dV$. Expand out and use:
$$\int_Ve^{\pm i(\mathbf{k}-\mathbf{k'})\cdot\mathbf{r}}dV=\delta^{(3)}(\mathbf{k}-\mathbf{k'})V$$
The $\delta$ function imposes $\mathbf{k}=\mathbf{k'}\implies\omega'=\omega$. After carrying out the summation over $\mathbf{k'}$, the time-dependent factors cancel. Similarly, 
$$\int_Ve^{\pm i(\mathbf{k}+\mathbf{k'})\cdot\mathbf{r}}dV=\delta^{(3)}(\mathbf{k}+\mathbf{k'})V$$
The $\delta$ function imposes $\mathbf{k}=-\mathbf{k'}\implies\omega'=\omega$. After carrying out the summation over $\mathbf{k'}$, here the time-dependent factors do not cancel. The time-dependent terms will eventually cancel. Result follows after some tedious algebra.
\end{proof}
\begin{remarks}\leavevmode
\begin{enumerate}
    \item Since $\mathbf{e_\lambda}(\mathbf{k})\perp\mathbf{k}$, $\mathbf{E}\perp\mathbf{k}$. 
    \item The total EM field energy is in fact independent of time and position, and only dependent upon the coefficients. We may further simplify to 
    $$U=2V\sum_{\mathbf{k},\lambda}\varepsilon_0\omega(\mathbf{k})^2|A_{\mathbf{k},\lambda}|^2$$
\end{enumerate}
\end{remarks}
\begin{prop}[Quantizing the EM field]
\begin{equation}
    U=\sum_{\mathbf{k},\lambda}\frac{1}{2}\hbar\omega(\mathbf{k})\bigg[a_{\mathbf{k},\lambda}a^*_{\mathbf{k},\lambda}+a^*_{\mathbf{k},\lambda}a_{\mathbf{k},\lambda}\bigg]\label{QED}
\end{equation}
\end{prop}
\begin{proof}
Again, introducing dimensionless coefficients $a_{\mathbf{k},\lambda}$ by writing
$$A_{\mathbf{k},\lambda}=\sqrt{\frac{\hbar}{2\varepsilon_0\omega(\mathbf{k})V}}a_{\mathbf{k},\lambda},\quad A_{\mathbf{k},\lambda}^*=\sqrt{\frac{\hbar}{2\varepsilon_0\omega(\mathbf{k})V}}a_{\mathbf{k},\lambda}^*$$
To quantize the field, we convert the coefficients for each mode into harmonic oscillator ladder operators for that mode (See Section~\ref{sec:HarmonicOscillator}). Using the commutation relation, we could also write the Hamiltonian as $\hat{H}=\sum_{\mathbf{k},\lambda}\hbar\omega(\mathbf{k})(n_{\mathbf{k},\lambda}+0.5)$.
\end{proof}
\begin{remarks}
The EM field consists of an infinite number of independent quantum harmonic oscillators - two oscillators for each possible $\mathbf{k}$, corresponding to the two available polarisation states. The ladder operators for a given mode $(\mathbf{k},\lambda)$ change the mode occupancy by $\pm1$ (creating/annihilating photons), leaving all other mode occupanices unchanged. The ladder operators for distinct modes commute since photons are bosons, i.e. symmetric under particle exchange.
\end{remarks}
\begin{defi}[Vacuum state]
The vacuum state $|0\rangle$ contains no photons in any mode.
\end{defi}
\begin{defi}[Fock state]
The $n$-photon states (Fock states) contain a well-defined number of photons, with no fluctuations in photon number $\langle(n_{\mathbf{k},\lambda})^2\rangle=(n_{\mathbf{k},\lambda})^2=\langle n_{\mathbf{k},\lambda}\rangle^2$. The Fock states $|n_{\mathbf{k},\lambda}\rangle$ form a complete, orthonormal set of eigenstates. The $n$-photon state $|n_{\mathbf{k},\lambda}\rangle$ is an eigenstate of the number operator $n_{\mathbf{k},\lambda}:=a^\dag_{\mathbf{k},\lambda}a_{\mathbf{k},\lambda}$. 
\end{defi}
\begin{remarks}
Since by definition, the vacuum state contains no photons in any mode, the vacuum energy is $U_0=\sum_{\mathbf{k},\lambda}\frac{1}{2}\hbar\omega(\mathbf{k})=\infty$. Nevertheless, only relative energies are observable and the infinity is not an issue for leading-order perturbation theory. We thus define the vacuum energy to be the zero-point energy, and rewrite the Hamiltonian to be 
$$H=\sum_{\mathbf{k},\lambda}\hbar\omega(\mathbf{k})a^\dag_{\mathbf{k},\lambda} a_{\mathbf{k},\lambda}\implies\langle 0|H|0\rangle=0$$
\end{remarks}
\begin{eg}
To create a photon,
\begin{align}
    H|k,\lambda\rangle&=\bigg(\sum_{\mathbf{k'},\lambda'}\hbar\omega(\mathbf{k'})a^\dag_{\mathbf{k'},\lambda'}a_{\mathbf{k'},\lambda'}\bigg)a^\dag_{\mathbf{k},\lambda}|0\rangle\nonumber\\&=\sum_{\mathbf{k'},\lambda'}\hbar\omega(\mathbf{k'})a^\dag_{\mathbf{k'},\lambda'}\bigg(\delta_{\mathbf{k'},\mathbf{k}}\delta_{\lambda',\lambda}+a^\dag_{\mathbf{k,\lambda}}a_{\mathbf{k}',\lambda'}\bigg)|0\rangle\nonumber\\&=\sum_{\mathbf{k'},\lambda'}\hbar\omega(\mathbf{k'})a^\dag_{\mathbf{k'},\lambda'}\delta_{\mathbf{k'},\mathbf{k}}\delta_{\lambda'\lambda}|0\rangle=\hbar\omega(\mathbf{k})a^\dag_{\mathbf{k},\lambda}|0\rangle\nonumber
\end{align}
where we use the commutation and $a_{\mathbf{k},\lambda}|0\rangle=0$ for all modes. (Temporarily restore operator notation for clarity.) By promoting the classical vector potential (Eqn.~\ref{mode}) to a quantum field operator:
$$\mathbf{\hat{A}}(\mathbf{r},t)=\sum_{\mathbf{k},\lambda}N(\mathbf{k})\bigg[\hat{a}_{\mathbf{k},\lambda}e^{i(\mathbf{k}\cdot\mathbf{r}-\omega t)}\mathbf{e_\lambda}(\mathbf{k})+\hat{a}^\dag_{\mathbf{k},\lambda}e^{-i(\mathbf{k}\cdot\mathbf{r}-\omega t)}\mathbf{e^*_\lambda}(\mathbf{k})\bigg],\quad N(\mathbf{k})=\sqrt{\frac{\hbar}{2\varepsilon_0\omega(\mathbf{k})V}}$$
From the classical linear momentum (Poynting vector), we have
$$\mathbf{\hat{P}}=\frac{1}{c}\int(\mathbf{E}\times\mathbf{H})d^3\mathbf{r}=-\varepsilon_0\int\frac{\partial\mathbf{\hat{A}}(\mathbf{r},t)}{\partial t}\times(\boldsymbol{\nabla}\times\mathbf{\hat{A}}(\mathbf{r},t))d^3\mathbf{r}=\sum_{\mathbf{k},\lambda}\hbar\mathbf{k}\hat{a}^\dag_{\mathbf{k},\lambda}\hat{a}_{\mathbf{k},\lambda}$$
By a similar argument, a single photon in a mode $|\mathbf{k},\lambda\rangle$ has linear momentum $\hbar\mathbf{k}$. The total angular momentum of the EM field about a given point $\mathbf{r_0}$ (the component independent of $\mathbf{r_0}$, i.e. intrinsic angular momentum/spin associated with the photon) is
$$\mathbf{J_s}=\int\mathbf{E}\times\mathbf{A}(\mathbf{r},t)d^3\mathbf{r}\implies\mathbf{\hat{J}_s}=\hbar\sum_{\mathbf{k}}\frac{\mathbf{k}}{|\mathbf{k}|}\bigg[\hat{a}^\dag_{\mathbf{k},L}\hat{a}_{\mathbf{k},L}-\hat{a}^\dag_{\mathbf{k},R}\hat{a}_{\mathbf{k},R}\bigg]$$
Operating (using the projection of the photon spin along the direction of motion of the photon $\mathbf{k}\cdot\mathbf{\hat{J}_s}/|\mathbf{k}|$) on a circularly polarised single photon states gives the eigenvalue $-\hbar$ and $+\hbar$ for R and L polarized states respectively, i.e. anti-parallel and parallel to the photon direction respectively.
\end{eg}
\begin{remarks}
Photons are spin-one particles, but possess only two spin degrees of freedom. The longitudinal spin (polarisation) state $m_s=0$ is missing. Photons are always transversely polarized.
\end{remarks}
\begin{eg}[Coherent states]
Any electromagnetic field generated by a classical current is in a coherent state. As the current is varied, the complex eigenvalue moves around in the complex plane.
For a mode $(\mathbf{k},\lambda)$ of the EM field, the coherent states are eigenstates of the photon annihilation operator for that mode (each mode) is independent
$$\hat{a}_{\mathbf{k'},\lambda'}|\alpha_{\mathbf{k},\lambda}\rangle=\alpha_{\mathbf{k},\lambda}|\alpha_{\mathbf{k},\lambda}\rangle\delta_{\mathbf{k'},\mathbf{k}}\delta_{\lambda',\lambda}$$
It follows that the average number of photons in a coherent state is $\langle n_{\mathbf{k},\lambda}\rangle=|\alpha_{\mathbf{k},\lambda}|^2$. By writing $\alpha_{\mathbf{k},\lambda}=|\alpha_{\mathbf{k},\lambda}|e^{i\theta_{\mathbf{k},\lambda}}$, the average value of the electric field (Eqn.~\ref{Efield}) is
$$\langle\mathbf{\hat{E}}\rangle=\langle\alpha_{\mathbf{k},\lambda}|\mathbf{\hat{E}}(\mathbf{r},t)|\alpha_{\mathbf{k},\lambda}\rangle=-\sqrt{\frac{2\langle n_{\mathbf{k},\lambda}\rangle\hbar\omega}{\varepsilon_0V}}\sin(\mathbf{k}\cdot\mathbf{r}-\omega t+\theta)\mathbf{e_\lambda}(\mathbf{k})$$
in complete contrast to the $n$-photon Fock state where $\langle\mathbf{\hat{E}}\rangle=0$, i.e. the electric field in a coherent state is a `classical'-like harmonic travelling wave. One can further show the uncertainty is a constant $\frac{\hbar\omega(\mathbf{k})}{2\varepsilon_0V}$ and it is the same as for vacuum.
\end{eg}
\newpage
\chapter{Particles in Magnetic Fields}
\section{Schr\"{o}dinger's equation}
\begin{defi}[Gauge field]
The electric field $\mathbf{E}=-\boldsymbol{\nabla}\phi-\frac{\partial\mathbf{A}}{\partial t}$ and magnetic field $\mathbf{B}=\boldsymbol{\nabla}\times\mathbf{A}$ are expressed by the gauge fields $\phi$ and $\mathbf{A}$ - scalar potential field and vector potential field respectively. One can obtain a different set of gauge fields $\phi'$ and $\mathbf{A}'$ via a gauge transformation:
\begin{equation}
\mathbf{A}'=\mathbf{A}+\boldsymbol{\nabla}\Lambda,\quad\phi'=\phi-\frac{1}{c}\frac{\partial\Lambda}{\partial t}\label{gaugetransform}
\end{equation}
where $\Lambda(\mathbf{x},t)$ is called a gauge.
\end{defi}
\begin{remarks}\leavevmode
\begin{enumerate}
    \item If we cannot find appropriate gauge fields for a proposed electromagnetic field, then the electromagnetic field is not valid/physical.
    \item Potentials related via a gauge transformation give rise to the same physical fields, i.e. $\mathbf{E}(\mathbf{A'},\phi')=\mathbf{E}(\mathbf{A},\phi),\quad\mathbf{B}(\mathbf{A'},\phi')=\mathbf{B}(\mathbf{A},\phi)$     is an equivalent representation of the electromagnetic field. An equivalence class of gauge fields are thus all related via a gauge transformation.
    \item Quantum mechanics can distinguish two physically identical electromagnetic fields with gauge fields not related to each other via a gauge transformation.
\end{enumerate}
\end{remarks}
\begin{defi}[Canonical momentum]
\begin{equation}
\mathbf{p}=m\mathbf{\dot{x}}+q\mathbf{A}\label{canonicalmomentum}
\end{equation}
\end{defi}
\begin{defi}[Minimal coupling]
The Hamiltonian is
\begin{equation}
H=\frac{1}{2m}(\mathbf{p}-q\mathbf{A})^2+q\phi=\frac{p^2}{2m}-\frac{q}{2m}(\mathbf{p}\cdot\mathbf{A}+\mathbf{A}\cdot\mathbf{p})+\frac{q^2A^2}{2m}+q\phi\label{minimalcoupling}
\end{equation}
This is called minimal coupling (shift of kinetic energy to include $\mathbf{A}$). 
\end{defi}
\begin{remarks}\leavevmode
\begin{enumerate}
\item In the quantum theory, we have the canonical commutation relations $[x_i,p_j]=\delta_{ij}$ and $[x_i,x_j]=0=[p_i,p_j]$. $p$ is the canonical momentum and not $m\mathbf{\dot{x}}$. 
\item Eqn.~\ref{canonicalmomentum} arises from classical mechanics via $\frac{\partial\mathcal{L}}{\partial\mathbf{\dot{x}}}$, where the Lagrangian of a charged particle of mass $m$ and charge $q$ is $\mathcal{L}=\frac{1}{2}m|\mathbf{\dot{x}}|^2+q\mathbf{\dot{x}}\cdot\mathbf{A}-q\phi$ (which gives rise to the equation of motion $m\mathbf{\ddot{x}}=q(\mathbf{E}+\mathbf{\dot{x}}\times\mathbf{B})$. Similarly, Eqn.~\ref{minimalcoupling} is given by $\mathbf{\dot{x}}\cdot\mathbf{p}-\mathcal{L}$.
\item The canonical momentum $\mathbf{p}$ is not gauge invariant since it transforms as $\mathbf{p}\rightarrow\mathbf{p}+q\boldsymbol{\nabla}\alpha$, and hence not physical.
\item We can further simplify Eqn.~\ref{minimalcoupling} by choosing the Coulomb gauge. Specifically, we have $\frac{q}{2m}\mathbf{p}\cdot\mathbf{A}=\frac{q}{2m}\frac{\hbar}{i}\boldsymbol{\nabla}\cdot\mathbf{A}=0$.
\end{enumerate}
\end{remarks}
We upgrade the classical operators to quantum operators, i.e. $\mathbf{p}\rightarrow -i\hbar\boldsymbol{\nabla}$. Then the time-dependent Schr\"{o}dinger's equation is
\begin{equation}
i\hbar\frac{\partial}{\partial t}\psi=H\psi=\frac{1}{2m}(-i\hbar\boldsymbol{\nabla}-q\mathbf{A})^2\psi+q\phi\psi\label{minimalcoupling2}
\end{equation}
\begin{prop}
We require Eqn.~\ref{minimalcoupling2} to be gauge invariant. This requires the wavefunction to change under a gauge transformation. Specifically, $\psi\rightarrow\psi'=e^{iq\Lambda/\hbar}\psi$ for the gauge $\Lambda$.
\end{prop}
\begin{proof}
To show this, we first demonstrate the following identity:
$$\bigg(\frac{\hbar}{i}\boldsymbol{\nabla}-q\mathbf{A'}\bigg)e^{iq\Lambda/\hbar}\psi=e^{iq\Lambda/\hbar}\bigg(\frac{\hbar}{i}\boldsymbol{\nabla}-q\mathbf{A}\bigg)\psi$$
Specifically, by commuting $(\mathbf{p}-q\mathbf{A})$ with $e^{iq\Lambda/\hbar}$, we `perform a gauge transformation'. The LHS gives
$$\frac{\hbar}{i}i\frac{q}{\hbar}(\boldsymbol{\nabla}\Lambda)\psi+e^{iq\Lambda/\hbar}\frac{\hbar}{i}\boldsymbol{\nabla}\psi-e^{iq\Lambda/\hbar}q(\mathbf{A}+\boldsymbol{\nabla}\Lambda)\psi=e^{iq\Lambda/\hbar}\frac{\hbar}{i}\boldsymbol{\nabla}\psi-e^{iq\Lambda/\hbar}qA\psi$$
which is the RHS. Now, we check $e^{iq\Lambda/\hbar}\psi$ is indeed a solution to the Schr\"{o}dinger's equation (Eqn.\ref{minimalcoupling2}) if $\psi$ is a solution. LHS and the RHS (earlier identity) respectively gives
$$-e^{iq\Lambda/\hbar}q(\partial_t\Lambda)\psi+e^{iq\Lambda/\hbar}i\hbar\frac{\partial\psi}{\partial t},\quad \frac{e^{iq\Lambda/\hbar}}{2m}\bigg(\frac{\hbar}{i}\boldsymbol{\nabla}-q\mathbf{A}\bigg)^2\psi+e^{iq\Lambda/\hbar}q\bigg(\phi-\frac{1}{c}\partial_t\Lambda\bigg)\psi$$
which gives Eqn.~\ref{minimalcoupling2} multiplied by $e^{iq\Lambda/\hbar}$.
\end{proof}
\begin{defi}[Gauge covariant operator]
Gauge covariant operators yield gauge-invariant measurements. For the operator $O(\phi,\mathbf{A})$ to be gauge covariant, we require $O'=O(\phi',\mathbf{A}')$. Specifically, 
\begin{equation}
    O'=O(\phi',\mathbf{A'})=e^{iq\Lambda/\hbar}O(\phi,\mathbf{A})e^{-iq\Lambda/\hbar}\label{gaugecovariant}
\end{equation}
\end{defi}
\begin{remarks}
Physical observables are, by construction, Hermitian operators that are gauge covariant. The probability density $|\psi'|^2=|\psi|^2$ is unchanged, as expected.
\end{remarks}
We need to use the gauge fields $\mathbf{A}$ and $\phi$ to formulate the theory at the quantum level. Under a gauge transformation, the wavefunction should change as $\psi\mapsto e^{iq\alpha/\hbar}\psi$, so to ensure the physical probabilities $|\psi|^2$ invariant. With this choice, the Schr\"{o}dinger's equation is covariant. 
\begin{prop}[Covariant derivatives]
By defining the covariant derivatives as 
\begin{equation}
D_t\psi:=\frac{\partial\psi}{\partial t}+i\frac{q\phi}{\hbar}\psi,\quad D_i\psi:=\frac{\partial\psi}{\partial x^i}-i\frac{qA_i}{\hbar}\psi\label{cov_deriv}
\end{equation}
then the Schr\"{o}dinger's equation (Eqn.~\ref{minimalcoupling2}) has a covariant form $i\hbar D_t\psi=-\frac{\hbar^2}{2m}D_iD_i\psi$.
\end{prop}
\begin{proof}
Under a gauge transformation, $D_t\psi\mapsto e^{iq\alpha/\hbar}D_t\psi$ and $D_i\psi\mapsto e^{iq\alpha/\hbar}D_i\psi$. The Schr\"{o}dinger Equation (Eqn.~\ref{minimalcoupling2}) becomes $i\hbar D_t\psi=-\frac{\hbar^2}{2m}D_iD_i\psi$ and this transforms covariantly.
\end{proof}
\begin{prop}
In the Heisenberg picture, the equation of motion for a charged particle is
\begin{equation}
m\frac{d\mathbf{v_H}}{dt}=q\mathbf{E_H}+\frac{1}{2}q(\mathbf{v_H}\times\mathbf{B_H}-\mathbf{B_H}\times\mathbf{v_H})\label{EOMcharged}
\end{equation}
\end{prop}
\begin{proof}
The velocity $v_{H,i}=\frac{dx_{H,i}}{dt}$, in the Heisenberg picture, is obtained from the Heisenberg's equation of motion with the Hamiltonian (Eqn.~\ref{minimalcoupling2}):
$$\frac{i}{\hbar}\bigg[\frac{1}{2m}(p_H-qA_H)^2+q\phi_H,x_{H,i}\bigg]=\frac{i}{\hbar m}\sum_j(p_H-qA_H)_j[(p_H-qA_H)_j,x_{H,i}]=\frac{i}{\hbar m}\sum_j(p_H-qA_H)_j[p_{H,j},x_{H,i}]$$
which gives $v_{H,i}=\frac{1}{m}(p_H-qA_H)_i$. Invoking the equation of motion (taking into account $\mathbf{A}$ is time-dependent) again, we have $m\frac{dv_{H,i}}{dt}=\frac{i}{\hbar}[0.5mv_H^2+q\phi_H,(p_H-qA_H)_i]-q(\partial A_i/\partial t)_H$:
$$\frac{i}{\hbar}[0.5 mv_H^2,mv_{H,i}]-q(\boldsymbol{\nabla}\phi_H)_i-q\frac{\partial A_{H,i}}{\partial t}=\frac{i}{2\hbar}i\hbar q(-\mathbf{v_H}\times\mathbf{B_H}+\mathbf{B_H}\times\mathbf{v_H})_i+qE_{H,i}$$
where $[mv_{H,i},mv_{H,j}]=i\hbar q(\partial_iA_{H,j}-\partial_jA_{H,i})=i\hbar q\varepsilon_{ijk}B_{H,k}$ since $mv_i=p_i-qA_i$ (Eqn.~\ref{canonicalmomentum}).
\end{proof}
\section{Examples}
\subsection{Magnetic field on a torus}
\begin{defi}[Torus]
We may identify the torus $\mathcal{T}^2$ as a rectangle with periodic boundary conditions $(x,y)\sim(x+L_x,y)$ and $(x,y)\sim(x,y+L_y)$.
\end{defi}
\begin{prop}
Consider a constant $\mathbf{B}=B_0\mathbf{\hat{z}}$, then then the magnetic flux through the torus $\mathcal{T}^2$ is a multiple of the flux quantum $\Phi_0=h/q$.
\end{prop}
\begin{proof}
The $B$ field gives $B_0=\partial_xA_y-\partial_yA_x$. Without loss of generality, $A_x(x,y)=0$ and $A_y(x,y)=B_0x$ is a solution. For configuration of $A$ to be well-defined, it needs to be periodic (since subject rectangle with PBC), i.e. $A_y(x,y)=A_y(x+L_x,y)$ and $A_y(x,y)=A_y(x,y+L_y)$. The periodicity is sufficient for well-defined gauge but not strictly necessary since we have freedom from the gauge transform. $A_y$ on the left and right vertical lines on the rectangle to differ by a gauge transform:
$$A_y^R(y)=A_y(x=L_x,y)=B_0L_x,\quad A_y^L(y)=A_y(x=0,y)=0,\quad A_y^R(y)=A_y^L(y)+\partial_y\Lambda\implies\Lambda=B_0L_xy$$
We also require $\Lambda$ to be well-defined on $\mathcal{T}^2$. $\Lambda$ has no $x$-dependence, so it is sufficient for it to be well-defined on the circle $y\sim y+L_y$. But, $\Lambda$ is not periodic on this circle. We can resolve this issue by requiring $e^{iqB_0L_xy/\hbar}$ to be periodic. This is obviously periodic in $x$. Assert $y$ to be periodic, then
$$\frac{q}{\hbar}B_0L_xL_y=2\pi n,\quad n\in\mathbb{Z}\implies B_0L_xL_y=\frac{2\pi\hbar}{q}n$$
where we define the flux quantum to be $\Phi_0:=h/q$.
\end{proof}
\subsection{Landau Levels}
\begin{defi}[Landau gauge]
We could have also chosen a gauge that breaks translational symmetry in the $x$ direction, as well as, rotational symmetry, i.e. $\mathbf{A}=xB\mathbf{\hat{y}}$
\end{defi}
\begin{prop}
For unbound motion in the $(x,y)$ plane and with $\mathbf{B}=B_0\mathbf{\hat{z}}$, the corresponding solution to Schr\"{o}dinger's equation looks like strips, extended in the $y$ direction but localized around $x=k_y\ell_B^2$ in the $x$-direction.
\end{prop}
\begin{proof}
With our chosen Landau gauge, the Hamiltonian (Eqn.~\ref{minimalcoupling2}) becomes $H=\frac{1}{2m}[p_x^2+(p_y-qBx)^2]$. We have translational invariance in the $y$ direction, i.e. $[p_y,H]=0$, so the good quantum number is $k_y$. Look for solutions of the form  $\psi(\mathbf{x})=e^{ik_yy}\chi(x)$. We have $p_{y}\psi=\hbar k_{y}\psi$ and ${H}=\frac{1}{2m}p_x^2+\frac{1}{2}m\omega_B^2(x-k_yl_B^2)^2$ where $\omega_B=\frac{qB}{m}$ and the characteristic length scale $l_B=\sqrt{\frac{\hbar}{qB}}$, This is the Hamiltonian of a harmonic oscillator. The momentum $k_y$ has turned into the position in the $y$ direction. Hence, the energy eigenvalues and the wavefunctions are respectively
$$E=\hbar\omega_B(n+0.5),~n\in\mathbb{Z}^+\cup\{0\},\quad\psi_{n,k_y}=e^{ik_yy}H_m(x-k_yl_B^2)e^{-\frac{(x-k_yl_B^2)^2}{2l_B^2}}$$
excluding the normalization factor, where $H_n$ is the usual Hermite polynomial wavefunctions of the harmonic oscillator. The energy eigenstate is a harmonic oscillator eigenstate centered at some $x_0=k_yl_B^2>0$ with an approximate width $l_B$ and fully delocalized in the $x$-direction (infinite degeneracy).
\end{proof}
\begin{remarks}We can still construct energy eigenstates that are localized in $x$, by superposing the momentum eigenstates in the $x$ direction. 
\end{remarks}
\begin{eg}
Solutions in different gauges can look very different. For the symmetric gauge $\mathbf{A}=-0.5B(-y,x,0)$, angular momentum is a good quantum number to label states. We will get a distinctly different wavefunctions - form concentric rings around the origin with radii depending on the angular momentum number. The lowest Landau level take the form
$$\psi_0(w,w^*)=f(w)e^{-|w|^2/4\ell_B^2},\quad H\psi_0(w,w^*)=0.5\hbar\omega_B\psi_0(w,w^*)$$
for any holomorphic function $f(w=x+iy)$. These wavefunctions are not gauge invariant and hence the fact that these wavefunctions are different from the previous attempt have no physical meaning. However, the degeneracy will still be $\frac{qBA}{2\pi\hbar}$. This is the springboard to lots of interesting physics such as the Quantum Hall Effect.
\end{eg}
\begin{prop}
For bound motion instead, the degeneracy of any given Landau level is finite.
\end{prop}
\begin{proof}
Finite size in $y$ implies the quantization of $k_y$ via the required periodicity of $e^{ik_yy}$ when $y\rightarrow y+L_y$. This gives $k_yL_y=2\pi n_y$, $n_y\in\mathbb{Z}$. Since $x_0=k_y^2l_B$ in order for the states to be inside the simple, we have $n_y>0$. Let the lowest allowed value of $n_y$ be $D\in\mathbb{Z}^+$ such that
$$L_x=k_yl_B^2=\frac{2\pi D}{L_y}l_B^2\implies D=\frac{L_xL_y}{l_B^2}\frac{1}{2\pi}=\frac{A}{\hbar/qB}\frac{1}{2\pi}=\frac{\Phi}{h/q}=\frac{\Phi_B}{\Phi_0}$$
$D$ is the degeneracy of the Landau level (number of possible $n_y$ allowed by the geometry). 
\end{proof}
\begin{eg}
For a finite square with $B=0.1$ T and $A=1$ cm$^2$, we have $10^{10}$ Landau levels. 
\end{eg}
\begin{cor}
The area occupied by each degenerate state in a maximally filled Landau level $A_0=2\pi l_B^2$.
\end{cor}
\begin{proof}
We have $x_0=k_yl_B^2=\frac{2\pi n_y}{L_y}l_B^2$. We see that a change in $n_y$ by one unit, will imply a change $\Delta x=2\pi l_B^2/L_x$ in $x_0$. Since the state extends from $y=0$ to $y=L_y$, the area $A_0$ is $L_y\Delta x=2\pi l_B^2$.
\end{proof}
\begin{remarks}
The area $A_0$ is independent of the Landau level, and does not correlate with the expected area of the orbit.
\end{remarks}
\begin{eg}
The electrons fill up the $\frac{eBA}{2\pi\hbar}$ states in the lowest Landau level $n=0$, before filling successive Landau levels. When we increase $B$ field, the number of states housed in each Landau level will increase, leading to a depletion of the higher Landau levels. This successive depletion gives rise to de Haas van Alphen oscillations in magnetic susceptibility, as well as, plateaus in the Hall resistivity $\rho_{xy}=\frac{2\pi\hbar}{e^2\nu}$, $\nu\in\mathbb{N}$ with a corresponding vanishing longitudinal resistivity, i.e. Integer Quantum Hall Effect - plateaus occur when precisely $\nu\in\mathbb{Z}^+$ Landau levels are filled.
\end{eg}
\subsection{Aharonov-Bohm Effect}
We have shown that under a gauge transform (Eqn.~\ref{gaugetransform}) $A^\mu\rightarrow A^\mu+\partial^\mu\Lambda$, $\psi\rightarrow\psi e^{-iq\Lambda/\hbar}$, the Schr\"{o}dinger's equation (Eqn.~\ref{minimalcoupling2}) is gauge invariant.

\begin{prop}
Consider a region with $\mathbf{A}\neq\boldsymbol{0}$ ($\mathbf{B}$ may be zero), the quantum particle will pick up a phase, due to the gauge field, which depends on the path taken. 
\end{prop}
\begin{proof}
For a field-free region $\mathbf{B}=\boldsymbol{0}$, can either take $\mathbf{A}=\boldsymbol{0}$ everywhere, or choose a continuum of gauges with $\mathbf{A}\neq\boldsymbol{0}$, i.e. 
$$\mathbf{A}(\mathbf{r})=-\boldsymbol{\nabla}\chi(\mathbf{r})\iff\chi(\mathbf{r_0})-\chi(\mathbf{r})=\int_{\mathbf{r_0}}^{\mathbf{r}}\mathbf{A}(\mathbf{r'})\cdot d\mathbf{r'}$$
The connection between wavefunctions in different gauges for a field-free region can be taken to be 
$$\psi_\chi(\mathbf{r})=\psi_0(\mathbf{r})\exp\bigg(i\frac{q}{\hbar}\int_{\mathbf{r_0}}^{\mathbf{r}}\mathbf{A}(\mathbf{r'})\cdot d\mathbf{r'}\bigg)$$
i.e. we can gauge away $\mathbf{A}$ by an exponential phase factor $\exp((iq/\hbar)\int^x\mathbf{A}(\mathbf{x'})\cdot d\mathbf{x'})$.
\end{proof}
\begin{eg}[Flux tube]
Consider a double-slit arrangement, initially with zero field everywhere. A fringe pattern will be produced on the screen as a result of interference between contributions associated with the two possible paths A and B, with intensity $I\propto|\psi_A+\psi_B|^2$. For a gauge with $\mathbf{A}=\boldsymbol{0}$ everywhere, the wavefunction on the screen is $\psi_{A,0}+\psi_{B,0}$. Choose $\mathbf{r_0}$ to be the source point S, the wavefunction on the screen for a general gauge $\mathbf{A}=-\boldsymbol{\chi}$ is
$$\psi_{A,0}\exp\bigg(i\frac{q}{\hbar}\int_A\mathbf{A}(\mathbf{r'})\cdot d\mathbf{r'}\bigg)+\psi_{B,0}\exp\bigg(i\frac{q}{\hbar}\int_B\mathbf{A}(\mathbf{r'})\cdot d\mathbf{r'}\bigg)$$
The line integrals above are the same for paths A and B. This is seen by applying Stokes' theorem to a clockwise loop. Hence, the two phase factors are equal, and the interference pattern is independent on the choice of gauge.
$$I\propto|\psi_{A,\chi}+\psi_{B,\chi}|^2=|\psi_{A,0}+\psi_{B,0}|^2$$
Now place a solenoid carrying flux $\Phi$, none of which can escape, directly behind the two slits. The solenoid is shielded such that electrons cannot physically penetrate the solenoid region. If $\Phi$ is contained entirely within the solenoid, then we still have $\mathbf{B}=\boldsymbol{0}$ everywhere outside the solenoid, especially at all points along the two trajectories A and B. But, $\mathbf{A}$ outside the solenoid must change and cannot have $\mathbf{A}=\boldsymbol{0}$ at all points along both trajectories. This seen in Stokes' theorem again:
$$\oint\mathbf{A}(\mathbf{r'})\cdot d\mathbf{r'}=\int_S\mathbf{B}(\mathbf{r})\cdot d\mathbf{S}=\Phi\neq 0$$
In cylindrical polar coordinates, $\mathbf{A}(r>a)$ can be taken to be $\frac{\Phi}{2\pi r}\boldsymbol{\hat{\phi}}\implies\chi(\mathbf{r})=-\frac{\Phi\phi}{2\pi}$, a multiple-valued gauge. By extracting away the common phase factor, the intensity on the screen is
\begin{equation}
I\propto|\psi_{A,0}e^{iq\Phi/\hbar}+\psi_{B,0}|^2\label{AharonovBohm}
\end{equation}
still independent of the choice of gauge, but now dependent on the flux $\Phi$ contained within the solenoid. The phase difference picked up from a closed path is the Aharonov-Bohm phase. The Hamiltonian (Eqn.~\ref{minimalcoupling2}) becomes
$$H=\frac{1}{2m}(p_\phi-qA_\phi)^2=\frac{1}{2mr^2}\bigg(-i\hbar\frac{\partial}{\partial\phi}-q\frac{\Phi}{2\pi}\bigg)^2\implies\psi=\frac{1}{\sqrt{2\pi r}}e^{in\phi},~n\in\mathbb{Z},\quad E=\frac{\hbar^2}{2mr^2}\bigg(n-\frac{\Phi}{\Phi_0}\bigg)^2$$
The spectrum thus remains unchanged when $\Phi=m\Phi_0$, for $m\in\mathbb{Z}$. But it depends on $\Phi$ otherwise. We can always gauge this away via $\psi\rightarrow e^{-iq\Lambda/\hbar}\psi=e^{-i\phi(\Phi/\Phi_0)}\psi$. But, the wavefunction should be single-valued, hence $\Phi/\Phi_0\in\mathbb{Z}$.
\end{eg}
\begin{eg}
Consider a GaAs/AlGaAs semiconductor ring structure at $T\sim 25$ mK. Large oscillations in resistance were observed as a function of the magnetic field strength $B$ applied through the centre of the ring. The peak at $h/2e$ is due to paths which encircle the ring twice.
\end{eg}
\subsection{Spin in a Magnetic Field}
\begin{defi}[Orbital and spin magnetic moment]
$\hat{\boldsymbol{\mu}}_L=\gamma_L\hat{\boldsymbol{L}}$ is the orbital magnetic moment operator where $\gamma_L=\frac{q}{2m}$ being the gyromagnetic ratio. $\boldsymbol{\hat{\mu}}_S=\gamma_S\boldsymbol{\hat{S}}$ with $\gamma_S=g_e\mu_B/\hbar$ where $\mu_B=\frac{e\hbar}{2m_e}\approx 5.79\times10^{-5}$ eV/T being the Bohr magneton.
\end{defi}
\begin{prop}
For a stationary uniform $\mathbf{B}$, we may use the symmetric gauge $\mathbf{A}(\mathbf{r})=-\frac{1}{2}\mathbf{r}\times\mathbf{B}$ subjected to Coulomb gauge $\boldsymbol{\nabla}\cdot\mathbf{A}=0$, and further neglecting the quadratic $B$ term in the Hamiltonian, we obtain a linear magnetic field contribution (magnetic dipole moment) $\boldsymbol{\hat{\mu}_e}\cdot\mathbf{B}$.
\end{prop}
\begin{proof}
Let $\mathbf{B}$ be directed along $z$, then the term in Eqn.~\ref{minimalcoupling} is
$$\mathbf{A}\cdot\boldsymbol{\nabla}=\frac{1}{2}B_z\bigg(-y\frac{\partial}{\partial x}+x\frac{\partial}{\partial y}\bigg)=\frac{i}{2\hbar}\mathbf{B}\cdot\mathbf{\hat{L}}$$
where $\hat{L}_z=-i\hbar(x\partial_y-y\partial_x)$. In the Coulomb gauge, $\mathbf{p}\cdot\mathbf{A}$ in Eqn.~\ref{minimalcoupling} is 0. Further,
$A^2=\frac{1}{4}(\mathbf{r}\times\mathbf{B})^2=\frac{1}{4}(r^2B^2-(\mathbf{r}\cdot\mathbf{B})^2)$. Neglect the quadratic $B^2$ terms.
\end{proof}
\begin{remarks}
For an electron confined within in an atom, estimate $\langle\hat{L}_z\rangle\sim\hbar$, $\langle r\rangle\sim a_0\approx 0.5\times10^{-10}$ m. Subject it to external magnetic field $\mathbf{B}$, the ratio of quadratic and linear $B$-field terms is very small, i.e. $\frac{e^2B^2a_0^2/8m_e}{eB\hbar/2m_e}=\frac{eBa_0^2}{4\hbar}=1.1\times10^{-6}\times(B/T)$.
\end{remarks}

\begin{defi}[Scalar magnetic moment]
For a spin 1/2 particle, the scalar magnetic moment $\mu$ is defined via the spin-up state: $\mu=\langle\uparrow|\hat{\mu}_z|\uparrow\rangle=\gamma_S\frac{\hbar}{2}$.
\end{defi}
\begin{prop}[Spin precession]
In an external magnetic field, the spin precesses:
\begin{equation}
\frac{d}{dt}\langle\mathbf{\hat{S}}\rangle=\gamma_S\langle\mathbf{\hat{S}}\rangle\times\mathbf{B}\label{spinprecession}
\end{equation}
\end{prop}
\begin{proof}
Consider a particle at rest in a magnetic field $\mathbf{B}$, the spin interaction Hamiltonian is $\hat{H}=-\gamma_S\mathbf{\hat{S}}\cdot\mathbf{B}$. By Ehrenfest's theorem, we have $\frac{d}{dt}\langle\mathbf{\hat{S}}\rangle=\frac{i}{\hbar}\langle[\hat{H},\mathbf{\hat{S}}]\rangle$, but the commutator is
$$[\mathbf{B}\cdot\mathbf{\hat{S}},\hat{S}_x] =B_y[\hat{S}_y,\hat{S}_x]+B_z[\hat{S}_z,\hat{S}_x]=i\hbar(-B_y\hat{S}_z+B_z\hat{S}_y)=i\hbar(\mathbf{\hat{S}}\times\mathbf{B})_x$$
similarly for $y$ and $z$. our desired result follows.
\end{proof}
\begin{defi}[Larmor frequency]
Thus, the spin precesses with a Larmor frequency $\omega_S=\gamma_SB$. For $\mathbf{B}=B\mathbf{\hat{z}}$, we have $\langle\hat{S}_z\rangle$ to be constant, and $\langle\hat{S}_x\rangle,\langle\hat{S}_y\rangle$ to vary sinusoidally. 
\end{defi}
\begin{remarks}
Naively, since $\gamma_S=q/m$, we expect the neutron to hae a zero internal magnetic moment. However, this is incorrect. We have
$$\boldsymbol{\hat{\mu}_p}=g_p\frac{\mu_N}{\hbar}\boldsymbol{\hat{S}_p},\quad \boldsymbol{\hat{\mu}_n}=g_n\frac{\mu_N}{\hbar}\boldsymbol{\hat{S}_n},\quad \mu_N=\frac{e\hbar}{2m_p}\approx 3.15\times10^{-8}\text{ eV T}^{-1}$$
where $\mu_N$ is the nuclear magneton. We find that $g_p\approx +5.586$ and $g_n\approx -3.826$. This suggests that protons and neutrons are not fundamental particles.
\end{remarks}
\begin{defi}[Stern-Gerlach experiment]
A magnetic dipole moment $\boldsymbol{\mu}$ in an external magnetic field $\mathbf{B}$ is subjected to a torque $\boldsymbol{\mu}\times\mathbf{B}$. If $\mathbf{B}$ is non-uniform, then $\boldsymbol{\mu}$ is subjected to a force $\mathbf{F}=\boldsymbol{\nabla}(\boldsymbol{\mu}\cdot\mathbf{B})$. Classically, for a sample of randomly oriented dipoles, $\boldsymbol{\mu}\cdot\mathbf{B}$ takes on a continuous range of values between $-\mu B$ and $+\mu B$, leading to a continuum of possible trajectories through the magnetic field. Quantum mechanically, particles possess an internal magnetic dipole moment $\mu$ follow only a finite number of distinct spatial trajectories.
\end{defi}
\begin{eg}
Consider a beam of neutral particles sent into a slowly varying $B$ field of the form $\mathbf{B}(\mathbf{r})=\mathbf{B_0}+\mathbf{B_1}(\mathbf{r})$, where $\mathbf{B_0}$ is constant, uniform and `large', while $\mathbf{B_1}$ is `small' and satisfy $\boldsymbol{\nabla}\cdot\mathbf{B_1}=0$ and $\boldsymbol{\nabla}\times\mathbf{B_1}=0$. Without loss of generality, orient the $z$ axis along $\mathbf{B_0}=(0,0,B_0)$. Let the particles have a magnetic dipole moment $\hat{\mu}_S=\gamma_S\hat{S}$. By Eqn.~\ref{minimalcoupling}, the Hamiltonian is
$$\hat{H}=\frac{\hat{p}^2}{2m}-\gamma_S(\hat{S}_zB_0+\mathbf{\hat{S}}\cdot\mathbf{B_1}(\mathbf{r}))$$
The spin operator $\hat{S}$ commutes with $\hat{r}$ and $\hat{p}$. Hence, from Ehrenfest's theorem, we necessarily have $\frac{d}{dt}\langle\hat{r}\rangle=\langle\hat{p}\rangle/m$. Similarly, the second equation of motion is 
$$[\hat{H},\mathbf{\hat{p}}]=-i\hbar\gamma_S(\sum_{i=x,y,z}\hat{S}_i\boldsymbol{\nabla}B_{1,i}(\mathbf{r}))\implies\frac{d}{dt}\langle\mathbf{\hat{p}}\rangle=\gamma_S\sum_{i=x,y,z}\langle\hat{S}_i\rangle\langle\boldsymbol{\nabla}B_{1,i}(\mathbf{r})\rangle$$
\end{eg}
\begin{eg}
For an arbitrary initial spin state $|\psi(0)\rangle=\sum_{m_s}c_{m_s}|sm_s\rangle$. The state evolution and and spin expectation gives:
    $$|\psi(t)\rangle=\sum_{m_s}c_{m_s}e^{im_s\gamma_SB_0t}|sm_s\rangle,\quad\langle\psi(t)|\mathbf{\hat{S}}|\psi(t)\rangle=\sum_{m_s',m_s}c^*_{m_s'}c_{m_s}e^{i(m_s'-m_s)\gamma_SB_0t}\langle sm_s'|\mathbf{\hat{S}}|sm_s\rangle$$
    The $x$ and $y$ components involve matrix elements of the form $\langle sm_s'|\hat{S}_{x,y}|sm_s\rangle$ which are non-zero only for $m_s'-m_s=\pm1$, giving the matrix elements for $x$ and $y$ to have a time dependence of the form $\exp(\pm i\gamma_SB_0t)$. For $B_0$ large, this is a rapidly oscillating phase factor which vanishes when averaged over short time intervals, hence effectively the $x$ and $y$ elements vanish. For $z$, the matrix elements are non-zero only for $m_s'=m_s$ and are hence time-independent, i.e.
    $$\langle sm_s'|\hat{S}_z|sm_s\rangle=c^*_{m_s'}c_{m_s}\delta_{m_s',m_s}m_s\hbar\implies\langle\mathbf{\hat{S}}\rangle=\sum_{m_s}|c_{m_s}|^2m_s\hbar$$
    It follows that the equation of motion is $m\frac{d^2\langle\mathbf{\hat{r}}\rangle}{dt^2}=\gamma_S\hbar\sum_{m_s}|c_{m_s}|^2m_s\langle\boldsymbol{\nabla}B_{1,z}(\mathbf{r})\rangle$. The incoming beam separates again into $2s+1$ outgoing beams, each with different $m_s$, and with the intensity in each beam proportional to $|c_{m_s}|^2$.
\end{eg}
\newpage
\chapter{Real atoms}
\section{Structure of Hydrogen atom}
The first treatment of the Hydrogen atom has been non-relativistic and has neglected the effects of electron and proton spin. To improve the accuracy of our description, we do the following:
\begin{enumerate}
    \item `switch on' electron spin `by hand', remaining non-relativistic
    \item add the leading-order relativistic effects predicted by the Dirac equation - relativistic corrections to kinetic energy, spin-orbit coupling, the Darwin term
    \item `switch on' nuclear spin.
\end{enumerate}
\subsection{Electron spin only}
Since the zeroth-order Hamiltonian $H_0$ of the Hydrogen atom does not contain the electron spin operator, the energy eigenstates may be written as $|n,\ell,m_\ell\rangle|s,m_s\rangle$. The degeneracy of each level is $g=2n^2$.\\[5pt]
An alternative set of basis states is provided by the eigenstates $|n,j,m_j,\ell,s\rangle$ (coupled basis) of the total angular momentum operator $\mathbf{J}=\mathbf{L}+\mathbf{S}$ with $j=\ell\otimes s=\ell\pm\frac{1}{2}$. These alternative bases are linear combinations of each other (they are all good quantum numbers), and we can use whichever is most convenient.
\begin{eg}[$n=2$ level]\leavevmode
\begin{center}
    \begin{tabular}{cc|c|c}
        $\ell$ & $m_\ell$ & $m_s$ & $g$ \\ 0 & 0 & $\pm1/2$ & $1\times 2=2$\\
        $1$ & $0,\pm1$ & $\pm1/2$ & $3\times 2=6$\\ 
    \end{tabular}
    \begin{tabular}{cc|c|c}
        $\ell$ & $j$ & $m_j$ & $g$\\  0 & $1/2$ & $\pm 1/2$ & $2$\\
        $1$ & $1/2$ & $\pm1/2$ & $2$\\
        1 & $3/2$ & $\pm 3/2,\pm1/2$ & $2+2=4$\\
    \end{tabular}
\end{center}
\end{eg}
\begin{prop}
The coupled states expand in terms of the uncoupled states:
$$|n,j,m_j,\ell,s\rangle=\sum_{m_\ell,m_s,m_\ell+m_s=m_j}|n,\ell,m_\ell,s,m_s\rangle\langle\ell,m_\ell;s,m_s|j,m_j\rangle$$
where the coefficients are the Clebsch-Gordan coefficients. For $s=1/2$, the summation contains at most two non-zero terms: $m_s=\pm1/2$, $m_\ell\pm1/2=m_j$.
\end{prop}
\begin{eg}
For $\ell=1$, we have $j=1\otimes1/2=3/2,1/2$. 
\begin{itemize}
\item For $m_j=+3/2$, the only possibility is $j=3/2$:
$$|n,3/2,+3/2,1,1/2\rangle=|n,1,+1\rangle|\uparrow\rangle$$
\item For $m_j=+1/2$, both $j=3/2$ and $j=1/2$ are possible. The coupled states with $m_j=+1/2$ expand (for any $n$) as
$$|n,3/2,+1/2,1,1/2\rangle=\sqrt{\frac{1}{3}}|n,1,+1\rangle|\downarrow\rangle+\sqrt{\frac{2}{3}}|n,1,0\rangle|\uparrow\rangle$$
$$|n,1/2,+1/2,1,1/2\rangle=\sqrt{\frac{1}{3}}|n,1,+1\rangle|\downarrow\rangle+\sqrt{\frac{2}{3}}|n,1,0\rangle|\uparrow\rangle$$
\end{itemize}
\end{eg}
\subsection{Relativistic treatment}
\begin{defi}[Dirac's equation]
The Schr\"{o}dinger's equation is not Lorentz invariant since it contains a first order time derivative but second order spatial derivatives. The Klein-Gordon equation generalizes to relativistic particles of zero spin. For spin 1/2 particles, the appropriate relativistic equation is the Dirac equation.
\end{defi}
\begin{eg}[Dirac equation correction]\leavevmode
\begin{enumerate}
    \item $H_R=-\frac{p^4}{8m_e^3c^2}$ is the first-order relativistic correction to the electron's energy and it is very small $\sim v^2/c^2\sim (Z\alpha)^2<<1$ relative to the zeroth order Hamiltonian. To find the energy correction, we  may use the uncoupled basis $|n,\ell,m_\ell,s,m_s\rangle$ since $H_R$ is independent of the spin.
    $$(\Delta E)_R=-\bigg(\frac{Z}{n}\bigg)^4\bigg(\frac{n}{\ell+0.5}-\frac{3}{4}\bigg)\alpha^2R_\infty$$
    which lifts the degeneracy between states of different $\ell$ (for a given $n$).
    \item $H_D=-\frac{e\hbar^2}{8m_e^2c^2}(\nabla^2\phi)$ is the Darwin term with no classical analogue. For a $1/r$ potential, $\nabla^2\phi$ vanishes everywhere except at the origin. Only S-states $\ell=0$ have a wavefunction non-zero at the origin. Hence, the energy shift for $\ell=0$ is
    $$(\Delta E)_D=\frac{Z^4\alpha^2}{n^3}R_\infty$$
    \item $H_{SO}=-\frac{e}{2m_e^2c^2}\mathbf{S}\cdot(\boldsymbol{\nabla}\phi\times\mathbf{p})$ which for a central potential, it gives
    $$H_{SO}=\frac{\hat{L}\cdot\hat{S}}{2m_e^2c^2r}\frac{dV(r)}{dr}$$
    This is the spin-orbit coupling - as the orbiting electron moves through the electric field of the nucleus, it experiences a magnetic field. The energy correction is the interaction energy due to the electron's magnetic dipole moment. Write $\hat{L}\cdot\hat{S}=0.5(\hat{J}^2-\hat{L}^2-\hat{S}^2)$. The coupled basis states $|n,j,m_j,\ell,s\rangle$ are simultaneous eigenstates of the 3 operators. The first order energy correction for $\ell>0$ is
    $$(\Delta E)_{\text{SO}}=\pm\frac{1}{2}\bigg(\frac{Z}{n}\bigg)^4\frac{n}{j+0.5}\frac{1}{\ell+0.5}\alpha^2R_\infty$$
    which lifts the degeneracy with respect to both $j$ and $\ell$ for each energy level $n$.
\end{enumerate}
In fact, summing the three energy corrections lead to a total energy correction that does not depend on $\ell$:
$$(\Delta E)_{\text{FS}}=\frac{Z^4}{4n^3}\bigg(\frac{3}{n}-\frac{4}{j+0.5}\bigg)\alpha^2R_\infty$$
States with the same $j$ but different $\ell$ remain degenerate in energy. For instance, $3S_{1/2}$ and $3P_{1/2}$ stay degenerate. For each $n$, the net effect is to lower the eigenstate energies (slightly), by an amount depending only on $j$. This is the fine structure of Hydrogen.
\end{eg}
\subsection{Hyperfine structure}
The most important hyperfine effect (due to atomic nucleus) is the nuclear magnetic dipole moment. The proton magnetic dipole moment produces a magnetic field $\mathbf{B_p}$. The atomic electron magnetic dipole moment within the proton's magnetic field gives rise to a hyperfine interaction.
\begin{prop}
The hyperfine Hamiltonian is
$$H_{hf}=g_p\frac{\mu_B\mu_N}{\hbar^2}(L+g_eS)\frac{\mu_0}{4\pi}\bigg[\frac{3\mathbf{r}(\mathbf{r}\cdot\mathbf{I})-r^2\mathbf{I}}{r^5}+\frac{8\pi}{3}\mathbf{I}\delta^{(3)}(\mathbf{r})\bigg]$$
\end{prop}
\begin{proof}
The nuclear dipole is $\boldsymbol{\mu_P}=g_P\frac{\mu_N}{\hbar}\mathbf{I}$. The hyperfine interaction is
$$H_{hf}=-\boldsymbol{\mu_P}\cdot\mathbf{B_p}=\frac{\mu_B}{\hbar}(\mathbf{L}+g_e\mathbf{S})\cdot\mathbf{B_p}$$
Classically, a magnetic moment $\mathbf{M}$ generates a magnetic field given by
$$\mathbf{B}=\frac{\mu_0}{4\pi}\bigg[\frac{3\mathbf{r}(\mathbf{r}\cdot\mathbf{M})-r^2\mathbf{M}}{r^5}+\frac{8\pi}{3}\mathbf{M}\delta^{(3)}(\mathbf{r})\bigg]$$
Replace $\mathbf{M}$ by the proton magnetic dipole.
\end{proof}
\begin{eg}\leavevmode
\begin{enumerate}
    \item For an $S$-state, with $\ell=0$, only the spin $S$ and the $\delta$-function contributes:
    $$(\Delta E)_{\ell=0}=g_eg_p\frac{\mu_B\mu_N}{\hbar^2}\frac{\mu_0}{4\pi}\langle\mathbf{\hat{S}}\cdot\mathbf{\hat{I}}\rangle\frac{8\pi}{3}|\psi_{n00}(0)|^2$$
    where the interaction occurs between electron and proton magnetic dipoles. The total angular momentum of the Hydrogen atom is $\mathbf{F}=\mathbf{L}+\mathbf{S}+\mathbf{I}$ which for $\ell=0$, effectively can use $\mathbf{F}=\mathbf{S}+\mathbf{I}$. For $F=0$ and $F=1$, the interaction $\langle\mathbf{S}\cdot\mathbf{I}\rangle$ gives $-\frac{3}{4}\hbar^2$ and $\frac{1}{4}\hbar^2$ respectively. The hyperfine splitting i sthus
    $$(\Delta E)_{\text{hf}}=\frac{4g_eg_p}{3}\frac{Z^3}{n^3}\frac{m_e}{m_p}\alpha^2R_\infty,\quad(\Delta E)_{F=1}=\frac{1}{4}(\Delta E)_{\text{hf}},~(\Delta E)_{F=0}=-\frac{3}{4}(\Delta E)_{\text{hf}}$$
    For Hydrogen atom ground state $n=1$, the hyperfine splitting corresponds to 21.1 cm radiation.
\end{enumerate}
\end{eg}
\section{Helium atom}
Neglecting the electron-electron Coulomb interaction, the Hamiltonian for the Helium atom is the sum of two independent Hydrogen-like Hamiltonians. neglecting electron spin, the eigenstates are the product of two Hydrogen-like eigenstates. 
\begin{eg}
In general, can take one electron to be in the 1s ground state and the other electron in an excited Hydrogen-like state, i.e. $|1,0,0\rangle|n,\ell,m\rangle$ where the possible $|n,\ell,m\rangle$ are the same as for Hydrogen.
\end{eg}
\begin{eg}
When factoring in the electron electron Coulomb interaction, it is best to use the variational method to find the ground state energy (no reason why it should be a small perturbation). Treat $Z'$ as a variational parameter:
$$|\psi_{\text{trial}}(Z')\rangle=\frac{(Z')^3}{\pi a_0^3}e^{-Z'(r_1+r_2)/a_0}$$
where $Z'$ takes into account the screening of the positive nuclear charge by the other electron. The upper bound on the ground state energy is found to be
$$E(Z')=-2(Z')^2R_\infty=-2\bigg(Z-\frac{5}{16}\bigg)^2R_\infty=-2(27/16)^2R_\infty$$
for $Z=2$. This gives $-77.4$ eV which agrees with the measured $-79.0$ eV.
\end{eg}
\section{Multi-electron atoms}
The most important contributions to the Hamiltonian for an $N$-electron atom are electron-electron repulsion $H_1$, spin-orbit interactions $H_2$ and a sum of $N$ independent Hydrogen-like Hamiltonians $H_0$. Relativistic corrections, hyperfine interactions etc can be included as perturbations.\\[5pt]
$H_0$ possesses eigenstates which are products of $N$ single-particle Hydrogen-like states $|n,\ell,m_\ell\rangle|s,m_s\rangle$. Each electron must occupy a different single-particle states, filling one by one starting from the lowest energy level.
\begin{eg}
Consider the 2p subshell $n=2,\ell=1$, with degeneracy of $2(2)^2=6$, i.e. $|m_\ell\rangle|m_s\rangle$ for instance $|+1\rangle|\downarrow\rangle$. Only one anti-symmetric overall state can be formed, i.e. Slater determinant. Every one of the $6!$ terms has $m_L=\sum_{i=1}^6(m_\ell)_i=0$ and $m_S=0\implies m_J=m_L+m_S=0$. Hence, overall $J=L=S=0$. For a 2p subshell containing five or fewer electrons, multiple possible independent wavefunctions can be formed. It is not possible to construct a totally antisymmetric wavefunction for a state containing more than six 2p electrons, i.e. the 2p subshell containing 6 electrons is said to be full.
\end{eg}
\begin{remarks}
A subshell $n\ell$ can accommodate at most $2(2\ell+1)$ electrons. A full subshell $n\ell$ has all $(2\ell+1)$ values of $m_\ell$ equally occupied. It has a charge distribution/probability density with an angular component proportional to $\sum_{m_\ell=-\ell}^{+\ell}|Y_{\ell,m_\ell}(\theta,\phi)|^2=\text{const.}$, i.e. isotropic. The total angular momentum quantum numbers of an atom are effectively determined by a small number of outer valence electrons.
\end{remarks}
The naive prediction of stable configurations is incorrect if electron-electron interaction is omitted. To solve the Hamiltonian with electron-electron repulsion (which is too large to be treated as a perturbation), we use the central-field approximation (assume electron-electron repulsion term contains a large spherically symmetric component arising from the `core' electrons.
\begin{defi}[Central field approximation]
Radially symmetric single-electron potentials $U_i(r_i)$ are introduced which accommodate the average effect of the other electrons. The Hamiltonian can then be repartitioned in a way which allows perturbation theory to be applied.
$$H_0'=\sum_{i=1}^N\bigg[-\frac{\hbar^2}{2m_e}\nabla_i^2-\frac{Ze^2}{4\pi\varepsilon_0r_i}+U_i(r_i)\bigg],\quad H_1'=\sum_{i<j}\frac{e^2}{4\pi\varepsilon_0r_{ij}}-\sum_{i=1}^NU_i(r_i)$$
Latter is a residual Coulomb interaction, small enough to be treated as a perturbation. Now, $H_0'$ no longer contains a purely $1/r$ potential term, so for a given $n$, states with different $\ell$ are no longer degenerate. Larger $\ell$ states are, on average, at larger radius and have higher energy.
\end{defi}
\begin{defi}[Aufbau Principle]
States are filled in order of increasing $n+\ell$, filling the subshells diagonally. It accounts for the structure of the Periodic Table.
\end{defi}
\begin{defi}[Spectroscopic term notation]
$n^{2S+1}L_j$
\end{defi}
\begin{eg}
The 3p and 3d subshell has respective notation $3^2P_{1/2}$, $3^2P_{3/2}$ and $3^2D_{3/2}$, $3^2D_{5/2}$ respectively. Now, consider (2p)(3s), the quantum numbers are $\ell_1=1,\ell_2=0\implies L=\ell_1\otimes L_2=1$, $s_1=0.5,s_2=0.5\implies S=s_1\otimes s_2=1,0$, $J=L\otimes S=2,1,0$. The configuration $^{(2S+1)}L_J$ gives $^1P_1$ and $^3P_{2,1,0}$. For the configuration (2p)$^2$, the electrons are said to be equivalent. We need to account for particle statistics for $S=0$ and $S=1$ separately.
\end{eg}
We have considered LS coupling - coupling between the total orbital angular momenta and spin. Alternatively, we could first construct the total angular momentum of each electron separately $\mathbf{J_i}=\mathbf{L_i}+\mathbf{S_i}$ before coupling.
\begin{eg}[jj coupling]
For (2p)(3p), the quantum numbers for $jj$ coupling will be $j_1=\ell_1\otimes s_1=1\otimes1/2=1/2,3/2$, similar for $j_2$, then $J=j_1\otimes j_2=0,1,2,3$.
\end{eg}
\begin{remarks}
For nuclei with low $Z$, $\langle\mathbf{L}\cdot\mathbf{S}\rangle<<\langle\text{Coulomb}\rangle$, so we use LS coupling. For high $Z$, we instead use $jj$ coupling.
\end{remarks}
\begin{defi}[Hund's rule]
If LS coupling applies, the ordering is given by Hund's rules (which are empirical):
\begin{enumerate}
    \item The largest permitted value of $S$ lies lowest in energy. (spin wavefunction maximally symmetric)
    \item For a given value of $S$, the largest value of $L$ lies lowest in energy. (tends to keep electrons apart)
    \item For given values of $L$ and $S$, if the subshell is less (more) than half full, the smallest (largest) value of $J$ lies lowest in energy. (ordering arise from treating the spin-orbit term as a perturbation - fine structure).
\end{enumerate}
\end{defi}
\newpage
\chapter{Atomic Physics}
\section{Transitions}
Time dependent perturbations to a time-independent Hamiltonian can induce transitions between eigenstates.
\begin{eg}[Transitions in two-state systems]
Consider an unperturbed system described by a Hamiltonian $H_0$ with exactly two eigenstates, assumed to be known, i.e. $H_0|\psi_i\rangle=E_i|\psi_i\rangle$, $i=1,2$. Apply a time-dependent harmonic perturbation of driving frequency $\omega$: $H'(t)=H'\cos\omega t$. Without the perturbation $H'(t)$, the state of the system evolves as
$$|\psi(t)\rangle=c_1e^{-i\omega_1t}|\psi_1\rangle+c_2e^{-i\omega_2t}|\psi_2\rangle$$
When $H'(t)\neq 0$, $c_1$ and $c_2$ can evolve in time. $|\psi(t)\rangle$ must satisfy the time-dependent Schr\"{o}dinger equation with Hamiltonian $H(t)=H_0+H'\cos\omega t$. Expanding out, and projecting to $\langle\psi_1|$ and $\langle\psi_2|$ respectively gives
\begin{align}
    (H_{11}'c_1(t)e^{-i\omega_1t}+H_{12}'c_2(t)e^{-i\omega_2t})\cos\omega t&=i\hbar\dot{c}_1(t)e^{-i\omega_1t}\nonumber\\
    (H_{21}'c_1(t)e^{-i\omega_1t}+H_{22}'c_2(t)e^{-i\omega_2t})\cos\omega t&=i\hbar\dot{c}_2(t)e^{-i\omega_2t}\nonumber
\end{align}
Choose $E_2>E_1$ and define the transition frequency $\omega_0:=\omega_2-\omega_1>0$. Set $\cos\omega t=\frac{1}{2}(e^{i\omega t}+e^{-i\omega t})$. Suppose the driving frequency $\omega$ is close to the frequency difference $\omega_0$, i.e. $\omega,\omega+\omega_0>>|\omega-\omega_0|$ and hence the terms containing $e^{\pm i\omega t},e^{\pm i(\omega+\omega_0)t}$ oscillate rapidly compared to those containing $e^{\pm i(\omega-\omega_0)t}$. In the rotating wave approximation (RWA), the rapidly oscillating terms are neglected (averaged to zero over short time intervals), giving
\begin{align}
    i\hbar\dot{c}_1(t)=\frac{1}{2}H_{12}'c_2(t)e^{i(\omega-\omega_0)t}\nonumber\\
    i\hbar\dot{c}_2(t)=\frac{1}{2}H_{21}'c_1(t)e^{i(\omega-\omega_0)t}\nonumber
\end{align}
Set $\hbar\omega':=H_{12}'$ and we get $\ddot{c}_2(t)+i(\omega-\omega_0)\dot{c}_2(t)+\frac{1}{4}|\omega'|^2c_2(t)=0$. For the case, $\omega=\omega_0$ (driving frequency precisely matched to the transition energy), the general solution is an undamped oscillation, i.e. $c_2(t)=A\cos(0.5|\omega'|t)+B\sin(0.5|\omega'|t)$ with $|\omega'|/2$ being the Rabi frequency. Suppose the system is initially in $|\psi_1\rangle$ at time $t=0$, the probabilities oscillate $P_1(t)=|c_1(t)|^2=\cos^2(|\omega'|t/2)$ and $P_2(t)=|c_2(t)|^2=\sin^2(|\omega'|t/2)$.\\[5pt]
For the case, $\omega$ close to $\omega_0$, try a solution $c_2(t)=Ae^{i\Omega t}$. Solving the resulting quadratic equation for $\Omega$ gives:
$$c_2(t)=e^{0.5i(\omega-\omega_0)t}(Ae^{i\omega_Rt/2}+Be^{-0.5i\omega_Rt}),\quad \omega_R=\sqrt{(\omega-\omega_0)^2+|\omega'|^2}$$
If the system is again initially in $|\psi_1\rangle$ at $t=0$, the probability to transition to $|\psi_2\rangle$ is $P_2(t)=|c_2(t)|^2=(|\omega'|/\omega_R)^2\sin^2(0.5\omega_Rt)$ with maximum transition probability at $\frac{|\omega'|^2}{(\omega-\omega_0)^2+|\omega'|^2}$. Scanning the driving frequency $\omega$ produces a Lorentzian resonance in the maximum transition probability, centred on $\omega=\omega_0$ with full-width half-maximum at $2|\omega'|$.
\end{eg}
\begin{remarks}\leavevmode
\begin{enumerate}
\item The Rabi frequency $|\omega'|/2$ is determined by the coupling between the two unperturbed eigenstates induced by the perturbation $H'$. The weaker the coupling, the lower the frequency transition.
\item When the driving frequency does not exactly match the difference in energy levels, the transition probability is always less than unity. The oscillation frequency $\omega_R<\omega'$. The width of the resonance is determined by the coupling $|\omega'|$ of the perturbation $H'$ between the two states. 
\end{enumerate}
\end{remarks}
We relook at the example considered in Section \ref{sec:Two-LevelSystems}.
\begin{eg}[Spin transitions]
Consider a spin-half particle with magnetic moment $\boldsymbol{\mu}=\gamma\boldsymbol{S}$ at rest in a time-dependent magnetic field $\mathbf{B}(t)=(B_x\cos\omega t,0,B_z)$ with $\omega,B_x,B_z>0$. The Hamiltonian can be written as
$$H(t)=-\gamma\mathbf{S}\cdot\mathbf{B}(t)\implies H_0=-\gamma B_zS_z,\quad H'=-\gamma B_x S_x$$
The eigenstates of the time-independent Hamiltonian is $|\uparrow\rangle$ and $|\downarrow\rangle$ with energy gap $\omega_0:=(E_2-E_1)/\hbar=\gamma B_z$. Further, define the constant $\omega'$:
$$\omega'\hbar=\langle\uparrow|H'|\downarrow\rangle=-\gamma B_x\langle\uparrow|S_x|\downarrow\rangle=-0.5\hbar\gamma B_x$$
Resonant transitions occur for a driving frequency $\omega=\omega_0=\gamma B_z$. Again, we obtain a narrow resonance for $B_x<<B_z$:
$$\frac{\Delta\omega}{\omega_0}=\frac{2|\omega'|}{\omega_0}=\frac{\gamma B_x}{\gamma B_z}$$
The spin states evolves as $|\psi(t)\rangle=c_1(t)e^{-i\omega_1t}|\uparrow\rangle+c_2(t)e^{-i\omega_2t}|\downarrow\rangle$. The spin expectations thus evolve as
$$\langle S_z\rangle=\frac{\hbar}{2}(|c_1(t)|^2-|c_2(t)|^2),\quad\langle S_x\rangle=\hbar\text{Re}[c_1(t)^*c_2(t)e^{-i\omega_0t}],\quad\langle S_y\rangle=\hbar\text{Im}[c_1(t)c_2(t)^*e^{+i\omega_0t}]$$
Suppose the driving frequency is exactly on resonance $\omega=\omega_0$ and that the particle is initially in the spin-up state, the spin expectation evolves as
$$\langle\boldsymbol{\hat{S}}\rangle=\frac{1}{2}\hbar\begin{pmatrix}\sin(\omega't)\sin(\omega_0t)\\\sin(\omega't)\cos(\omega_0t)\\\cos(\omega't)\\\end{pmatrix}$$
\end{eg}
\begin{remarks}
The time evolution of $\langle\mathbf{S}\rangle$ is a combination of
\begin{itemize}
    \item precession about the $z$ axis, with a fast frequency $\omega_0$
    \item precession about a horizontal axis with a slow frequency $|\omega'|$, where the horizontal axis itself rotates about $z$ with frequency $\omega_0$. 
\end{itemize}
We can thus define $\pi/2$ and $\pi$ pulses as:
\begin{itemize}
    \item $\omega'\Delta t=\pi/2$: $\langle\mathbf{S}\rangle=\frac{1}{2}\hbar(\sin\omega_0t\mathbf{\hat{x}}+\cos\omega_0t\mathbf{\hat{y}})$. If the oscillating field $B_x$ is applied for a time $\Delta t=\frac{\pi}{2\omega'}=\frac{\pi}{\gamma B_x}$, the spin vector will rotate into the $x$-$y$ plane, and then stay in the $x$-$y$ plane, precessing around the $z$ axis at frequency $\omega_0=\gamma B_z$.
    \item $\omega'\Delta t=\pi$: $\langle\mathbf{S}\rangle=-0.5\hbar\mathbf{\hat{z}}$. If the oscillating field $B_x$ is applied for a time $\Delta t$ such that $\Delta t=\frac{\pi}{\omega'}=\frac{2\pi}{\gamma B-x}$, and then switched off, leaving a static field $B_z\mathbf{\hat{z}}$, the particle will undergo a spin-flip transition to the `spin-down' state, and then stay there. 
\end{itemize}
\end{remarks}
\begin{eg}[Applications]\leavevmode
\begin{enumerate}
    \item Nuclear magnetic resonance: for a proton in a $B=10$ T field, the frequency gap is $\nu_0=\frac{g_p\mu_NB_z}{2\pi\hbar}=426$ MHz, i.e. radio frequencies induce proton spin-flips.
    \item Electron spin resonance: for an electron in a $B=10$ T field, the frequency gap is $\nu_0=\frac{g_e\mu_BB_z}{2\pi\hbar}=9$ GHz, i.e. microwave frequencies induce electron spin-flips.
\end{enumerate}
\end{eg}
\begin{eg}[NMR]
In NMR, a sample is placed in a large, uniform, static magnetic field $B_z$, which induces an energy level splitting between proton spin states which is slightly different for different chemical shifts (transition frequency that depends on the chemical environment of the proton). A small population difference between the split energy levels gives rise to a net magnetisation in the sample. A coil surrounding the sample can apply a smaller, perpendicular, pulsed RF field $B_x\cos\omega t$, inducing a precession of the net magnetisation about the $z$ direction. As it precesses, $\mathbf{B}$ produced by the small net magnetisation induces a radio signal in a second coil surrounding the sample. Fourier analysis of the induced RF output signal produces the NMR spectrum.\\[5pt]
Today, pulsed methods are used to simultaneously excite a range of chemical shifts in a single shot. A simple pulse sequence could be a train of $\pi/2$ pulses at an RF frequency matching the proton spin-flip resonant frequency. This rotates the net magnetization into the $x$-$y$ plane, which then precesses rapidly about $z$ (at a slightly different frequency for each chemical shift). While precessing, the net magnetisation also slowly relaxes back towards equilibrium. The oscillating induced signal therefore gradually decays with time. The induced output signal contains a superposition of frequencies, one for each chemical shift in the sample.
\end{eg}
\section{Wigner-Eckart Theorem}
In this section, we focus on the consequence of rotational symmetry on selection rules. We have discussed scalar and vector operators in Section \ref{sec:cont_trans}.
\begin{defi}[Scalar operator]
A scalar operator $K$ is an operator whose matrix elements remain invariant under rotations, i.e. for all states $|\psi\rangle$ of the system, we have 
\begin{equation}
\langle\psi'|K|\psi'\rangle=\langle\psi|K|\psi\rangle\iff U(R)^\dag KU(R)=K\label{scalaroperator}
\end{equation}
Since $U(R)$ is unitary, the scalar operator $K$ must commute with all rotation operators $[U(R),K]=0$.
\end{defi}
\begin{eg}
For infinitesimal rotations of the form
$$U(I+\omega)=I+\frac{i}{\hbar}(\boldsymbol{\omega}\cdot\mathbf{\hat{J}})+O(\omega^2)$$
Since $K$ commutes with $U(I+\omega)$ $\forall\omega$, it must also commute with $\mathbf{J}$: $[\mathbf{J},K]=0$. It then follows that
$$[J_z,K]=0,\quad[J_\pm,K]=0,\quad[J^2,K]=0$$
If $K$ is the Hamiltonian of an isolated system, then invariance under spatial rotations would mean conservation of total angular momentum for an isolated system. Other examples are $r^2$, $p^2$, $L^2$, $S^2$, $J^2$, $\mathbf{L}\cdot\mathbf{S}$ and $V(r)$.
\end{eg}
\begin{remarks}
The product $\mathbf{L}\cdot\mathbf{S}$ is not a scalar operator with respect to $\mathbf{L}$ or $\mathbf{S}$ separately (only with respect to $\mathbf{J}$), i.e.
$$[\mathbf{L},\mathbf{L}\cdot\mathbf{S}]\neq0,\quad[\mathbf{S},\mathbf{L}\cdot\mathbf{S}]\neq0$$
Rotational symmetry explains why $H_R$, $H_D$ and $H_{SO}$ commute with all the operators in the coupled set $\{\mathbf{J}^2,\mathbf{J},\mathbf{L}^2,\mathbf{S}^2\}$, and why $H_R$ and $H_D$, but not $H_{SO}$ commute with all the operators in the uncoupled set.
$$H_R\propto(\mathbf{p}^2)^2,\quad H_{SO}=\xi(r)\mathbf{L}\cdot\mathbf{S},\quad H_D\propto\nabla^2\phi(r),\quad [L_z,H_{SO}]\neq 0,\quad [S_z,H_{SO}]\neq0$$
Thus, for $H_{SO}$, using the uncoupled basis in degenerate perturbation theory would have required an explicit matrix diagonalisation.
\end{remarks}
\begin{prop}
Consider an operator $K$ which is a scalar operator with respect to an angular momentum operator $\mathbf{J}$. Then, the matrix elements of $K$ must also be diagonal in $j$:
$$\langle j',m_j'|K|j,m_j\rangle=\langle j,m_j|K|j,m_j\rangle\delta_{j',j}\delta_{m_j',m_j}$$
\end{prop}
\begin{proof}
Since $K$ is a scalar operator with respect to $\mathbf{J}$, then $[J_z,K]=0$ and $[\mathbf{J}^2,K]=0$. The first relation imply:
$$0=\langle j',m_j'|[J_z,K]|j,m_j\rangle=(m_j'-m_j)\hbar\langle j',m_j'|K|j,m_j\rangle\implies\langle j',m_j'|K|j,m_j\rangle=\langle j',m_j|K|j,m_j\rangle=\delta_{m_j',m_j}$$
Similarly, the second relation gives
$$0=\langle j',m_j'|[\mathbf{J}^2,K]|j,m_j\rangle=(j'(j'+1)-j(j+1))\hbar^2\langle j',m_j'|K|j,m_j\rangle$$
which means the matrix elements of $K$ is also diagonal in $j$.
\end{proof}
\begin{thm}[Wigner-Eckart theorem for scalar operators]
For scalar operators $K$ (Eqn.~\ref{scalaroperator})
\begin{equation}
    \langle\alpha'',j'',m''|K|\alpha',j',m'\rangle=\langle\alpha'',j''||K||\alpha',j'\rangle\delta_{j'',j'}\delta_{m'',m}\label{Wigner-Eckart}
\end{equation}
where $\alpha,\alpha'$ are additional quantum numbers, $[J_i,J_j]=i\hbar\sum_k\varepsilon_{ijk}J_k$ and $[\mathbf{J},K]=0$. For the `trivial' angular momentum combination $0\otimes j=j$, we can write
\begin{equation}
    \langle\alpha'',j'',m''|K|\alpha',j',m'\rangle=\langle\alpha'',j''||K||\alpha',j'\rangle\langle 0,0;j',m'|j'',m''\rangle\label{Wigner-Eckart2}
\end{equation}
where $\langle 0,0;j',m'|j'',m''\rangle=\delta_{j'',j'}\delta_{m'',m'}$ is the Clebsch-Gordan coefficient.
\end{thm}
\begin{notation}
The quantity $\langle\alpha'',j''||K||\alpha',j'\rangle$ is a $m$-independent constant known as a `reduced matrix element'. It is not a matrix element but is the common, constant component of a related set of matrix elements. This is because the quantum number $m$ arises from an arbitrary choice of spatial direction ($z$ axis). The matrix elements of a rotationally invariant operator cannot possibly depend on an arbitrary choice of spatial direction.
\end{notation}
\begin{eg}
The Hamiltonian $H$ for an isolated system must be a scalar operator, and so must commute with $\mathbf{J}$. $H$ and $\mathbf{J}$ possess a set of simultaneous eigenstates $|\alpha,j,m\rangle$. From the Wigner-Eckart theorem, we have
$$E_{\alpha,j,m}=\langle\alpha,j,m|H|\alpha,j,m\rangle=\langle\alpha,j||H||\alpha,j\rangle$$
The energy eigenvalues for an isolated system must thus be independent of the quantum number $m$. It follows that for an isolated system, energy levels with $j>0$ must inevitably be degenerate with degeneracy $g\geq 2j+1$ (there may be additional degeneracy if $H$ possesses further symmetries).
\end{eg}
We can generalize Wigner-Eckart theorem to vector operators.
\begin{defi}[Vector operator]
A vector operator has expectation values that transform under spatial rotations in the same way as ordinary three-vectors:
$$\langle V_i'\rangle=\sum_jR_{ij}\langle V_j\rangle,\quad i=1,2,3$$
Equivalently,
\begin{equation}
\langle\psi|U(R)^\dag V_iU(R)|\psi\rangle=\sum_jR_{ij}\langle\psi|V_j|\psi\rangle\implies U(R)^\dag V_iU(R)=\sum_jR_{ij}V_j\label{vec_operator}
\end{equation}
\end{defi}
\begin{eg}
For an infinitesimal rotation of the form $R=I+\omega$, we have
$$\bigg[I-\frac{i}{\hbar}(\boldsymbol{\omega}\cdot\mathbf{J})\bigg]V_i\bigg[I+\frac{i}{\hbar}(\boldsymbol{\omega}\cdot\mathbf{J})\bigg]=\sum_j(\delta_{ij}+\omega_i)V_j\implies\frac{i}{\hbar}V_i(\boldsymbol{\omega}\cdot\mathbf{J})-\frac{i}{\hbar}(\boldsymbol{\omega}\cdot\mathbf{J})V_i=\sum_j\omega_{ij}V_j$$
Equating the coefficients of $\omega_1,\omega_2,\omega_3$ on both sides for each $i=1,2,3$ gives the commutation relations $[J_i,V_j]=i\hbar\sum_k\varepsilon_{ijk}V_k$. It then follows that
$$[J_i,r_j]=i\hbar\sum_k\varepsilon_{ijk}r_k,~[J_i,p_j]=i\hbar\sum_k\varepsilon_{ijk}p_k,~[J_i,L_j]=i\hbar\sum_k\varepsilon_{ijk}L_k,~[J_i,J_j]=i\hbar\sum_k\varepsilon_{ijk}J_k$$
These familiar commutation relations are a consequence of rotational invariance.
\end{eg}
\begin{defi}[Spherical components]
The spherical components of a vector operator $\mathbf{V}$ are defined as
\begin{equation}
V_{+1}=\frac{-1}{\sqrt{2}}(V_1+iV_2),\quad V_{-1}=\frac{1}{\sqrt{2}}(V_1-iV_2),\quad V_0=V_3\label{sphericalcomponents}
\end{equation}
\end{defi}
\begin{prop}
$$[J_3,V_m]=\hbar mV_m,\quad [J_\pm,V_m]=\hbar\sqrt{j(j+1)-m(m\pm1)}V_{m\pm1}$$
where $V_m=(V_{+1},V_{-1},V_0)$.
\end{prop}
\begin{proof}
Follows from the commutation relations for the Cartesian components of $\mathbf{V}$.
\end{proof}
\begin{thm}[Wigner-Eckart theorem for vector operators]
For a vector operator $V_m$ (Eqn.~\ref{vec_operator})
\begin{equation}
    \langle\alpha'',j'',m''|V_m|\alpha',j',m'\rangle=\langle\alpha'',j''||V||\alpha',j'\rangle\langle1,m;j',m'|j'',m''\rangle\label{Wigner-Eckart3}
\end{equation}
where $V_m=(V_{+1},V_{-1},V_0)$ are the spherical components.
\end{thm}
\begin{remarks}
The reduced matrix element $\langle\alpha'',j''||\mathbf{V}||\alpha',j'\rangle$ is independent of $m,m',m''$ and it contains the physics (dependence on the observable $\mathbf{V}$). The Clebsch-Gordon coefficient $\langle1,m;j',m'|j'',m''\rangle$ carries the dependence on $m,m',m''$ and simply describes the geometry (dependence on an arbitrary choice of quantization axis), and it is independent of $\mathbf{V}$. Again, the reduced matrix element is the common, constant component of a related set of matrix elements (it is a single number), and not a matrix element.
\end{remarks}
\begin{cor}[Selection Rules]
The Clebsch-Gordon coefficient $\langle 1,m;j',m'|j'',m''\rangle$ vanishes unless $m''=m'+m$, where $m=-1,0,+1$ and unless $j''=1\otimes j'=$ $j',j'\pm1$ and 1 for $j'>0$ and $j'=0$ respectively. Equivalently, $\Delta j=0,\pm1$, $\Delta m=0,\pm1$ and no transition can happen between 0 and 0.
\end{cor}
\begin{remarks}
We cannot have $j'=j''=0$ since $\langle 1,m;0,0|0,0\rangle=0$. It follows, that $|\langle\alpha'',0,0|\mathbf{V}|\alpha',0,0\rangle=0$.
\end{remarks}
\begin{eg}[Atomic transitions]
These selection rules can be immediately applied to atomic transitions, involving the emission and absorption of a single photon. The dominant form of such transitions, known as electric dipole transitions, is governed by the electric dipole operator $\mathbf{d}$. A further requirement on the selection rule arise from the parity invariance.
$$\langle\psi_1(\mathbf{r})|\mathbf{r}|\psi_2(\mathbf{r})\rangle=0\quad\text{if}\quad P_1=P_2$$
where $P=(-1)^\ell$ is the parity of the wavefunction of angular momentum $\ell$. For E1 transitions, the transition rate is given by $\Gamma\propto\omega^3|\langle\beta|\mathbf{\hat{d}}|\alpha\rangle|^2$.
\end{eg}
\begin{prop}[Land\'{e} interval rule]
The separation between adjacent levels in a fine structure multiplet is proportional to the larger of the two $J$ values.
\begin{equation}
    (E_J-E_{J-1})\propto J
\end{equation}
\end{prop}
\begin{proof}
Using the Wigner-Eckart theorem, it can be shown that
$$\langle J,m_J,L,S|\sum_i\xi_i(r_i)\mathbf{L_i}\cdot\mathbf{S_i}|J,m_J,L,S\rangle=\xi(L,S)\langle J,m_J,L,S|\mathbf{L}\cdot\mathbf{S}|J,m_J,L,S\rangle$$
where $\xi(L,S)$ changes sign according to whether the subshell is more than, or less than, half-filled. Set
$$\mathbf{L}\cdot\mathbf{S}=\frac{1}{2}(\mathbf{J}^2-\mathbf{L}^2-\mathbf{S}^2)\implies\langle J,m_J,L,S|\mathbf{L}\cdot\mathbf{S}|J,m_J,L,S\rangle=\frac{1}{2}(J(J+1)-L(L+1)-S(S+1))\hbar^2$$
For given $L$ and $S$, the separation between neighbouring levels is $E_J-E_{J-1}\propto J(J+1)-(J-1)J=2J$.
\end{proof}
\newpage
\section{Atomic selection Rules}
\begin{eg}[E1 selection rule for zeroth-order Hydrogen]
Consider the zeroth-order Hydrogen atom, neglecting spin, with eigenstates $|n,\ell,m_\ell\rangle$, involving the emission or absorption of a photon. From the Wigner-Eckart theorem, this can only happen if the initial and final states satisfy the selection rules
$$\Delta\ell=0,~\pm1,\quad\ell_1+\ell_2\geq1,\quad \Delta m_\ell=0,~\pm1$$
The possibility $\Delta\ell=0$ is further excluded by the parity selection rule $(-1)^{\ell_1}\neq(-1)^{\ell_2}$. There are no restrictions on the initial or final values of the principal quantum number $n$. The possible transitions are best illustrated by displaying the energy levels in the form of a Grotrian diagram.
\end{eg}
\begin{remarks}
The 2s level is metastable with a relatively long lifetime of about 0.12 s since it cannot decay to a lower energy state via an E1 transition. It decays indirectly into 1s level via en emission of two photons.
\end{remarks}
\begin{eg}[Z-component for E1]
The operator $z$ is the $m=0$ spherical component of the vector operator $\mathbf{r}$. To obtain $\langle n_1,\ell_1,m_1|z|n_2,\ell_2,m_2\rangle\neq 0$, the initial and final states must satisfy the selection rules $\Delta\ell=\pm1$ and $\Delta m_\ell=0$.
\end{eg}
\begin{eg}[E1 selection rule for fine-structure Hydrogen]
Now, `switch on' electron spin and consider the Hydrogen atom with fine structure effects included. The total angular momentum operator is now $\mathbf{J}=\mathbf{L}+\mathbf{S}$. The selection rules for E1 transitions now apply to $j$ and $m_j$:
$$\Delta j=0,~\pm1,\quad j_1+j_2\geq1,\quad\Delta m_j=0,~\pm1$$
But if spin is not involved, the selection rules $\Delta\ell=\pm1$ and $\Delta m_\ell=0,~\pm1$ is still applicable. For instance, the dipole operator $\mathbf{d}$ does not depend on the electron spin operator. 
\end{eg}
\begin{remarks}
While transition between fine structure states lying within the same level $n$ are allowed by the E1 selection rules, such transitions has a very small energy $\Delta E$ but E1 transition rate is proportional to $(\Delta E)^3$.
\end{remarks}
\begin{eg}[E1 selection rule for hyperfine structure Hydrogen]
'Switch on' the proton spin and consider hyperfine structure. The strict selection rules for E1 transitions from Wigner-Eckart and parity now apply to the total angular momentum $\mathbf{F}=\mathbf{L}+\mathbf{S}+\mathbf{I}$:
$$\Delta F=0,~\pm1,\quad F_1+F_2\geq1,\quad\Delta m_F=0,~\pm1$$
The selection rules for zeroth-order and fine-structure may continue to apply.
\end{eg}
\begin{eg}[21 cm line]
The 21 cm line of the Hydrogen ground state is not an E1 transition. It has $|\Delta F|=1$, $\Delta j=0$, but also $\Delta \ell=0$ (fails $\Delta\ell=\pm1$). It is in fact a magnetic dipole M1 transition, with transition rates several order of magnitudes smaller than E1.
\end{eg}
\begin{remarks}
The $\Delta m_j$ selection rule restricts the possible transitions between initial and final degenerate sets of states (with degeneracy $g=2j+1$).
\end{remarks}
\begin{eg}[E1 transition of Helium]
The dipole operator does not invovle spin operators, hence the selection rules satisfy $\Delta S=0$, $\Delta m_S=0$, i.e. parahelium $S=0$ and orthohelium $S=1$ form two completely separate spectroscopic systems. The E1 selection rules for Hydrogen continue to apply. The (1s)(2s) states in each system have no available E1 decays open to them and are therefore metastable.
\end{eg}
\begin{eg}[Fine structure of Helium]
For parahelium $S=0$, there is no fine structure splitting since $J=L\otimes 0$. For orthohelium $S=1$, the possible $J$ values are $J=L\otimes 1$. The selection rules for Hydrogen fine structure continue to apply.
\end{eg}
\newpage
\subsection{Land\'{e} projection formula}
We now obtain a general expression for the matrix elements of a vector operator $\mathbf{V}$ between states of given $j$:
\begin{thm}[Land\'{e} projection formula]
\begin{equation}
    \langle\alpha,j,m'|\mathbf{V}|\alpha,j,m\rangle=\frac{\langle\alpha,j,m|\mathbf{V}\cdot\mathbf{J}|\alpha,j,m\rangle}{j(j+1)\hbar^2}\langle\alpha,j,m'|\mathbf{J}|\alpha,j,m\rangle\label{lande}
\end{equation}
In particular, for $V_z$,
\begin{equation}
    \langle\alpha,j,m'|V_z|\alpha,j,m\rangle=\frac{\langle\alpha,j,m|\mathbf{V}\cdot\mathbf{J}|\alpha,j,m\rangle}{j(j+1)\hbar^2}m\hbar\delta_{m',m}\label{landeZ}
\end{equation}
\end{thm}
\begin{proof}
$\mathbf{J}$ is a vector operator. Apply the Wigner-Eckart theorem:
$$\langle\alpha'',j'',m''|J_m|\alpha',j',m'\rangle=\langle\alpha'',j''||\mathbf{J}||\alpha',j'\rangle\langle1,m;j',m'|j'',m''\rangle$$
LHS vanishes unless $j'=j''$. Taking also $\alpha''=\alpha'$ gives
$$\langle\alpha',j',m''|J_m|\alpha',j',m'\rangle=\langle\alpha',j'||\mathbf{J}||\alpha',j'\rangle\langle1,m;j',m'|j'',m''\rangle$$
Similar for $V_m$. Eliminating the Clebsch-Gordon coefficients between that for $V_m$ and $J_m$ gives
$$\langle\alpha',j',m''|V_m|\alpha',j',m'\rangle=C(\alpha',j')\langle\alpha',j',m''|J_m|\alpha',j',m'\rangle$$
To find $C(\alpha,j)$:
\begin{align}
    \langle\alpha,j,m|\mathbf{V}\cdot\mathbf{J}|\alpha,j,m\rangle&=\sum_{k=1}^3\langle\alpha,j,m|V_kJ_k|\alpha,j,m\rangle\nonumber\\&=\sum_{k,j',m'}\langle\alpha,j,m|V_k|\alpha,j',m'\rangle\langle\alpha,j',m'|J_k|\alpha,j,m\rangle\nonumber\\&=C(\alpha,j)\sum_{k,m'}\langle\alpha,j,m|J_k|\alpha,j,m'\rangle\langle\alpha,j,m'|J_k|\alpha,j,m\rangle\nonumber\\&=C(\alpha,j)\sum_{k,j',m'}\langle\alpha,j,m|J_k|\alpha,j',m'\rangle\langle\alpha,j',m'|J_k|\alpha,j,m\rangle\nonumber\\&=C(\alpha,j)\langle\alpha,j,m|\mathbf{J}^2|\alpha,j,m\rangle=C(\alpha,j)j(j+1)\hbar^2\nonumber
\end{align}
where we used completeness relation twice. The desired formula follows.
\end{proof}
The projection formula allow us to add together two or more magnetic dipole moments. The combined effect of two dipoles of either of the forms (orbital or spin) in a magnetic field $\mathbf{B}$ can be considered as
$$H_B=-\gamma_1\mathbf{J_1}\cdot\mathbf{B}-\gamma_2\mathbf{J_2}\cdot\mathbf{B}\iff H_B=-\boldsymbol{\mu}\cdot\mathbf{B},\quad\boldsymbol{\mu}=\gamma_1\mathbf{J_1}+\gamma_2\mathbf{J_2}$$
Introduce the total angular momentum operator characterized by total angular momentum quantum numbers $j=j_1\otimes j_2$. If the two gyromagnetic ratios $\gamma_1\neq\gamma_2$, then the total dipole and total angular momentum operators are no longer proportional, i.e. $\boldsymbol{\mu}\neq\gamma\mathbf{J}$. For a given $j$, however, the proportionality can be effectively restored using the Land\'{e} projection formula.
\begin{cor}
The effective gyromagnetic ratio $\gamma_j$ for a given $j$, for two moment operators, is
\begin{equation}
    \gamma_j=\gamma_1\frac{j(j+1)+j_1(j_1+1)-j_2(j_2+1)}{2j(j+1)}+\gamma_2\frac{j(j+1)+j_2(j_2+1)-j_1(j_1+1)}{2j(j+1)},\quad\boldsymbol{\mu}=\gamma_j\mathbf{J}\label{landeg}
\end{equation}
\end{cor}
\begin{proof}
Since $\boldsymbol{\mu}$ is a vector operator, then for $j>0$, the projection formula gives
$$\langle\alpha,j,m'|\boldsymbol{\mu}|\alpha,j,m\rangle=\gamma_{\alpha j}\langle\alpha,j,m'|\mathbf{J}|\alpha,j,m\rangle,\quad\gamma_{\alpha,j}=\frac{\langle\alpha,j,m|\boldsymbol{\mu}\cdot\mathbf{J}|\alpha,j,m\rangle}{j(j+1)\hbar^2}$$
We have $\boldsymbol{\mu}\cdot\mathbf{J}=\gamma_1\mathbf{J_1}\cdot\mathbf{J}+\gamma_2\mathbf{J_2}\cdot\mathbf{J}$. Squaring $\mathbf{J_2}=\mathbf{J}-\mathbf{J_1}$ gives $\mathbf{J_1}\cdot\mathbf{J}=0.5(\mathbf{J}^2+\mathbf{J_1}^2-\mathbf{J_2}^2)$ and hence
$$\langle\alpha,j,m|\mathbf{J_1}\cdot\mathbf{J}|\alpha,j,m\rangle=\frac{\hbar^2}{2}[j(j+1)+j_1(j_1+1)-j_2(j_2+1)]$$
similar for $\langle\mathbf{J_2}\cdot\mathbf{J}\rangle$. The constant $\gamma_{\alpha,j}$ is in fact independent of $\alpha$ (and $m$).
\end{proof}
\begin{remarks}\leavevmode
\begin{enumerate}
\item Thus, each possible value of $j=j_1\otimes j_2$ has its own effective gyromagnetic ratio $\gamma_j$.
\item Alternatively, the magnetic moments can be expressed in terms of corresponding $g$-factors 
$$\gamma_i=-g_i\frac{\mu_B}{\hbar}$$
It follows that $g_j$ similarly satisfy Eqn.~\ref{landeg}.
\end{enumerate}
\end{remarks}
\begin{eg}\leavevmode
\begin{enumerate}
\item For the case $j=0$, we simply have $\gamma_j=0$. 
\item For the spin triplet case $j=1/2\otimes1/2=1$, we have $g_j=\frac{1}{2}g_1+\frac{1}{2}g_2\implies\mu_j=\mu_1+\mu_2$.
\item For the spin singlet case $j=0$, the magnetic moment vanishes, as a result of Wigner-Eckart theorem. It is not easy to explain how the magnetic dipoles can sum to zero in a simple vectorial picture.
\item Consider a proton with two up quarks and one down quark. The two $u$ quarks are in a state $S_{uu}=S_u\otimes S_u=1\implies g_{uu}=0.5g_u+0.5g_u=g_u$. Add in the $d$-quark. Take $j=S_p=1/2$, $j_1=S_{uu}=1$ and $j_2=S_d=1/2$, invoke projection formula:
$$g_p=\frac{4}{3}g_{uu}-\frac{1}{3}g_d\implies\mu_p=\frac{4}{3}\mu_u-\frac{1}{3}\mu_d$$
Similarly, for the neutron, but with $u\leftrightarrow d$. $\mu_n=\frac{4}{3}\mu_d-\frac{1}{3}\mu_u$. Because the neutron is not fundamental, it has a non-zero magnetic moment despite having zero charge.
\end{enumerate}
\end{eg}
\section{Zeeman effect}
\begin{defi}[Zeeman effect]
When an $N$-electron atom is placed in an external magnetic field $\mathbf{B}$, the contribution of the $i$th electron to the Hamiltonian is 
$$H_B^{(i)}=\frac{e}{2m_e}(\mathbf{L_i}+g_e\mathbf{S_i})\cdot\mathbf{B}+\frac{e^2}{8m_e}(B^2r_i^2-(\mathbf{B}\cdot\mathbf{r_i})^2)$$
Neglecting the quadratic $B^2$ terms, and summing over all atomic electrons, the overall interaction with the magnetic field is
$$H_B=\frac{e}{2m_e}(\mathbf{L}+g_e\mathbf{S})\cdot\mathbf{B},\quad\mathbf{L}=\sum_{i=1}^N\mathbf{L_i},~\mathbf{S}=\sum_{i=1}^N\mathbf{S_i}$$
Taking the $z$ axis to be along the $\mathbf{B}$ field direction, we have
\begin{equation}
H_B=(g_LL_z+g_SS_z)\frac{\mu_BB_z}{\hbar},\quad\boldsymbol{\mu_L}=-\frac{\mu_B}{\hbar}g_L\mathbf{L},~\boldsymbol{\mu_S}=-\frac{\mu_B}{\hbar}g_S\mathbf{S}\label{ZeemanHamiltonian}
\end{equation}
The resulting energy correction (Zeeman effect) is of order $\mu_BB_z\sim 5.8\times10^{-5}$ (B/T) eV, small relative to typical atomic binding energies.
\end{defi}
\begin{remarks}
Since the energy correction is small and each atomic energy levels have degeneracy $g=2J+1$, we need to use degenerate perturbation theory. For a general field strength $B$, first-order perturbation theory involves an explicit matrix diagonalisation, whatever the choice of basis for the unperturbed states. However, under certain regimes, an analytic answer is possible:
\begin{enumerate}
    \item strong field: $\mu_BB>>(\Delta E)_{\text{FS}}$ use the uncoupled basis $|L,m_L,S,m_S\rangle$,
    \item weak field: $\mu_BB<<(\Delta E)_{FS}$ use the coupled basis $|J,m_J,L,S\rangle$.
\end{enumerate}
where $(\Delta E)_{\text{FS}}$ represents a typical fine structure splitting at zero field. For even smaller field strengths such that $\mu_BB$ becomes comparable to atomic hyperfine structure splittings, i.e. $\mu_BB\leq(\Delta E)_{\text{HFS}}$, we must also consider the contribution due to the nucleus of spin $\mathbf{I}$:
$$\mathbf{F}=\mathbf{L}+\mathbf{S}+\mathbf{I},\quad H_B=-\sum_{L,S,I}\boldsymbol{\mu_i}\cdot\mathbf{B}$$
We start by neglecting hyperfine structure, keeping only $L$ and $S$. Consider first the strong-field limit, then the weak-field limit, then join them together. Repeat with nuclear spin and hyperfine structure `switched on'.
\end{remarks}
\begin{prop}[Strong-field Zeeman effect]
Suppose the $\mathbf{B}$ field is strong enough that the resulting energy shifts are much larger than those due to fine (and hyperfine) structure: $\mu_BB>>\langle\xi(r)\mathbf{L}\cdot\mathbf{S}\rangle$, then the energy correction is
\begin{equation}
(\Delta E)_B=\mu_BB_z(g_Lm_L+g_Sm_S)\label{strongZeeman}
\end{equation}
\end{prop}
\begin{proof}
Under the strong field limit, the spin-orbit contribution to the atomic structure can be neglected. Neglecting fine structure, states of different $J$ are degenerate in energy. Each unperturbed energy level has definite values of $L$ and $S$, and is degenerate with respect to the various possible $J$ values: $J=L\otimes S$. Equivalently, each level is degenerate with respect to $m_L$ and $m_S$, the degeneracy of each level is $g=(2L+1)(2S+1)$. Degenerate perturbation is carried out using the `uncoupled' basis of states, which are simultaneous eigenstates of $L_z$ and $S_z$:
$$(g_LL_z+g_SS_z)|\alpha,L,S,m_L,m_S\rangle=(g_Lm_L+g_Sm_S)\hbar|\alpha,L,S,m_L,m_S\rangle$$
Here, the matrix representation of $H_B$ is therefore diagonal with respect to the quantum numbers $m_L$ and $m_S$:
$$\langle\alpha,L,S,m_L',m_S'|g_LL_z+g_SS_z|\alpha,L,S,m_L,m_S\rangle=(g_Lm_L+g_Sm_S)\hbar\delta_{m_L',m_L}\delta_{m_S',_S}$$
No explicit diagonalization is required. The eigenvalue is thus multiplied by the missing factor $\frac{\mu_BB_z}{\hbar}$.
\end{proof}
\begin{remarks}
In a strong magnetic field $B$, the original level splits into several equally spaced levels, with energy separation given by $\Delta E=2\mu_BB$ and $\mu_BB$ if $L=0$ ($2S+1$ degeneracy) and $L>0$ ($2(L+2S)+1$ degeneracy) respectively.
\end{remarks}
\begin{prop}[Weak-field Zeeman effect]
Now suppose the magnetic field is much smaller than the fine structure Hamiltonian, $\mu_BB<<\langle\xi(r)\mathbf{L}\cdot\mathbf{S}\rangle$, then the first order energy correction is
\begin{equation}
(\Delta E)_B=m_Jg_J\mu_BB\label{weakZeeman}
\end{equation}
where $g_J$ is the Land\'{e} g-factor.
\end{prop}
\begin{proof}
In the weak-field limit, each fine structure level (with definite values of $L$, $S$ and $J$) can be considered separately. The unperturbed fine structure levels have degeneracy $g=2J+1$. Degenerate perturbation theory is easiest carried out in the `coupled' basis $|\alpha,J,m_J,L,S\rangle$. The projection formula applied to $H_B$ ($z$ component of vector operators) gives:
$$\langle\alpha,J,m_J',L,S|H_B|\alpha,J,m_J,L,S\rangle=\langle\alpha,J,m_J,L,S|H_B|\alpha,J,m_J,L,S\rangle\delta_{m_J',m_J}$$
i.e. already diagonal with respect to $m_J$, so no explicit diagonalization is required. In a state of definite total angular momentum $J$, the energy correction $(\Delta E)_B$ is
$$\frac{\mu_BB_z}{\hbar}\langle\alpha,J,m_J,L,S|g_LL_z+g_SS_z|\alpha,J,m_J,L,S\rangle=\frac{\mu_BB_z}{\hbar}g_J\langle\alpha,J,m_J,L,S|J_z|\alpha,J,m_J,L,S\rangle=\frac{\mu_BB_z}{\hbar}g_Jm_J\hbar$$
where $g_J$ is the Land\'{e} g-factor given earlier.
\end{proof}
\begin{remarks}
In the presence of a weak field, the original level splits into $2J+1$ equally spaced levels, completely lifting the original degeneracy. Effectively, we have $H_B=-\boldsymbol{\mu_J}\cdot\mathbf{B}$, i.e. the $\mathbf{L}$ and $\mathbf{S}$ dipoles have combined to give a single $\mathbf{J}$ dipole.
\end{remarks}
\begin{remarks}[General field strength Zeeman effect]
In an external magnetic field $\mathbf{B}$, an atom is no longer an isolated system. The total angular momentum $\mathbf{J}=\mathbf{L}+\mathbf{S}$ of the atom is no longer expected to be conserved. We can easily see
$$[H_B,J_x]=\frac{\mu_BB_z}{\hbar}(g_L[L_z,J_x]+g_S[S_z,J_x])=i\mu_BB_z(g_LL_y+g_SS_y)\neq0$$
Similar for $J_y$. However, because $[L_z,J_z]=[S_z,J_z]=0$, $[J_z,H_B]=0$ and still true for the total Hamiltonian. The system thus still has cylindrical symmetry about $B_z$. $m_J$ is thus still a good quantum number in the presence of a magnetic field of any strength, even though $J$ is not. $m_J$ can thus be used to track the energy eigenstates as they evolve with varying $B$ strength: weak-field state $m_J$ interpolate to strong-field state with $m_L+l_S=m_J$.
\end{remarks}
\begin{remarks}[Zeeman effect for hyperfine levels]
With hyperfine structure, we can no longer neglect the hyperfine contribution. A similar weak field and strong field analysis can be taken, but with $\mathbf{J}=\mathbf{L}+\mathbf{S}\rightarrow\mathbf{F}=\mathbf{J}+\mathbf{I}$. The weak-field and strong-field energy corrections are respectively:
$$(\Delta E)_B=m_Fg_F\mu_BB,\quad g_F\approx g_J\frac{F(F+1)+J(J+1)-I(I+1)}{2F(F+1)}$$
$$(\Delta E)_B\approx\mu_BB_zg_Jm_J$$
\end{remarks}
\section{Stark effect}
\begin{defi}[Stark effect]
Consider the influence of an external electric field $\mathbf{E}$ on a Hydrogen atom: 
\begin{equation}
H_E=-\mathbf{E}\cdot\mathbf{d},\quad\mathbf{d}=-e(\mathbf{r_e}-\mathbf{r_p})\label{StarkHamiltonian}
\end{equation}
where $\mathbf{d}$ tends to align itself along $\mathbf{E}$.
\end{defi}
\begin{remarks}\leavevmode
\begin{enumerate}
\item Assuming an infinite mass proton, we have $H_E=e\mathcal{E}z=e\mathcal{E}r\cos\theta$, where $\mathbf{E}=\mathcal{E}\mathbf{\hat{z}}$.
\item Consider a level $n^{2S+1}L_J$ of Hydrogen, with eigenstates $|n,j,m_j,\ell,s\rangle$. The level has degeneracy $g=2j+1$ and are degenerate with respect to $m_j$. Treat $H_E$ as a perturbation, the first energy correction will involve
$$\langle n,j,m_j',\ell'|z|n,j,m_j,\ell\rangle$$
Since $z$ is odd under spatial inversion, the matrix element vanish unless the states involved are of opposite parity, i.e. $\langle n,j,m_j',\ell|z|n,j,m_j,\ell\rangle=0$.
\item In almost all circumstances, the first-order energy correction vanishes, and we need to consider second-order, i.e. quadratic Stark effect.
\item For large enough electric field strength, states of different $\ell$ can effectively become degenerate, and the first-order energy correction becomes non-zero, i.e. linear Stark effect.
\end{enumerate}
\end{remarks}
\begin{eg}[Quadratic Stark effect]
Consider the influence of an external electric field $\mathbf{E}$ on the ground state of the Hydrogen atom $|n,\ell,m_\ell\rangle=|1,0,0\rangle$. The first-order correction to the ground state energy vanishes by parity. The second-order perturbation theory is
$$(\Delta E)^2=\sum_{n\geq2,\ell,m}\frac{|\langle n,\ell,m|e\mathcal{E}z|1,0,0\rangle|^2}{E_1^{(0)}-E_n^{(0)}},\quad E_n^{(0)}=\frac{-R_\infty}{n^2}$$
The denominator is always negative, so the energy correction due to the electric field is $(\Delta E)^{(2)}<0$. To evaluate the infinite sum exactly, the trick is to find an operator $F$ which satisfies the equation
$$z|0\rangle=(FH_0-H_0F)|0\rangle$$
where $|0\rangle=|1,0,0\rangle$ is the unperturbed Hydrogen ground state. If such an operator $F$ could be found, we have for any zeroth-order state $|k\rangle$:
$$\langle k|z|0\rangle=\langle k|FH_0|0\rangle-\langle k|H_0F|0\rangle=(E_0^{(0)}-E_k^{(0)})\langle k|F|0\rangle$$
The infinite summation gives
$$\sum_{k\neq 0}\frac{|\langle k|z|0\rangle|^2}{E_0^{(0)}-E_k^{(0)}}=\sum_{k\neq 0}\langle 0|z|k\rangle\langle k|F|0\rangle=\langle 0|zF|0\rangle-\langle 0|z|0\rangle\langle 0|F|0\rangle=\langle 0|zF|0\rangle$$
where we used $\langle 0|z|0\rangle=0$ (from parity) and the completeness relation $I=\sum_k|k\rangle\langle k|=|0\rangle\langle 0|+\sum_{k\neq 0}|k\rangle\langle k|$. One example of such an operator is $F=-\frac{m_ea_0}{\hbar^2}(0.5 r+a_0)z$, where $H_0F|0\rangle=-0.5(-\frac{r}{2a_0}+1)z|0\rangle$. Using $\langle r^2\rangle_0=3a_0^2$ and $\langle r^3\rangle_0=\frac{15}{2}a_0^3$, we have
$$(\Delta E)_{100}^{(2)}=-e^2\mathcal{E}^2\frac{m_ea_0}{3\hbar^2}\bigg(\frac{15}{4}a_0^3+3a_0^3\bigg)=-\frac{9}{4}(4\pi\varepsilon_0)\mathcal{E}^2a_0^3$$
\end{eg}
\begin{defi}[Polarizability]
The polarizability $\alpha$ is defined classically by writing the interaction energy due to the electric field as
$$\Delta E=-\frac{1}{2}\mathbf{E}\cdot\mathbf{d}=-\frac{1}{2}\alpha\varepsilon_0\mathcal{E}^2$$
\end{defi}
\begin{eg}
For a Hydrogen atom ground state, the second-order perturbation gives $\alpha=18\pi a_0^3$. A reasonable upper bound would be $\alpha\leq\frac{64}{3}\pi a_0^3$.
\end{eg}
\begin{eg}[Linear Stark effect]
Now consider the $n=2$ states of the Hydrogen atom. For a strong enough electric field, such that the energy corrections are much larger than the fine structure, then the 2s and 2p states
$$|n,\ell,m\rangle=|2,0,0\rangle,~|2,1,0\rangle,~|2,1,1\rangle,|2,1,-1\rangle$$
are effectively all degenerate with energy $E_2^{(0)}=-0.25 R_\infty$. First-order degenerate perturbation theory requires that we evaluate 
$$\langle 2,\ell,m|H_E|2,\ell',m'\rangle,\quad H_E=e\mathcal{E}z=e\mathcal{E}r\cos\theta,~\ell,\ell'=0,1$$
which is non-zero only if $\Delta\ell=\pm1$ and $\Delta m_\ell=0$. For Hydrogen, 
$$\langle2,0,0|z|2,1,0\rangle=\frac{1}{32\pi a_0^3}\int_0^\infty\int_{-1}^{+1}\int_0^{2\pi}e^{-r/2a_0}\bigg(2-\frac{r}{a_0}\bigg)(r\cos\theta)e^{-r/2a_0}\frac{r}{a_0}\cos\theta~r^2dr~d\cos\theta~d\phi=-3a_0$$
Hence, in the above basis,
$$H_E=\begin{pmatrix}0&-\Delta & 0 &0\\-\Delta &0&0&0\\0&0&0&0\\0&0&0&0\\\end{pmatrix}$$
which gives eigenstates $|\psi\rangle_\pm=\frac{1}{\sqrt{2}}(|2,0,0\rangle\pm|2,1,0\rangle)$. The states $|2,1,1\rangle$ and $|2,1,-1\rangle$ receive zero energy correction and remain degenerate in energy. 
\end{eg}
\begin{remarks}
As the electric field is reduced, the energy correction becomes smaller and comparable to those due to fine structure and the Lamb shift. A combined analysis of all relevant perturbations is needed.
\end{remarks}
\newpage
\section{QED}
With a quantum description of the EM field, we can understand why the dominant atomic transitions involve the emission and absorption of a single photon. Using Fermi's Golden Rule, we may compute the transition rates for these processes.
\begin{prop}
The interaction of the atom with the quantised external EM field is described by
\begin{equation}
H'=\frac{e}{m_e}\sum_{i=1}^N\mathbf{A}(\mathbf{r_i},t)\cdot\mathbf{p_i}\label{interactionham}
\end{equation}
where $\mathbf{A}$ acts on the field and $\mathbf{p}$ acts on the atom.
\end{prop}
\begin{proof}
Consider an $N$-electron atom immersed in a time-dependent, external EM field described by the classical vector and scalar potentials $\mathbf{A}(\mathbf{r},t)$, $\phi(\mathbf{r},t)$. In the Coulomb gauge $\boldsymbol{\nabla}\cdot\mathbf{A}=0$, the Hamiltonian for this system is
$$H=\frac{1}{2m_e}\bigg[\sum_{i=1}^Np_i^2+2e\mathbf{A}\cdot\mathbf{p_i}+e^2\mathbf{A}^2\bigg]-e\sum_{i=1}^N\phi(\mathbf{r_i},t)+V(\{ r_i,r_{ij}\})$$
where $V$ represents the various atomic Coulomb potential energy terms. For an electron confined to an atom, the quadratic $A^2$ terms can usually be neglected. This leaves the contribution due to the interaction to be as desired.
\end{proof}
\begin{remarks}\leavevmode
\begin{enumerate}
\item Eqn.~\ref{interactionham} involves both a sum over the $N$ atomic electrons, and a sum over all modes $(\mathbf{k},\lambda)$ of the EM field (Eqn.~\ref{mode}):
\begin{equation}
H'=\frac{e}{m_e}\sum_{i=1}^N\sum_{\mathbf{k},\lambda} N(\mathbf{k})\bigg[a_{\mathbf{k},\lambda}e^{i(\mathbf{k}\cdot\mathbf{r_i}-\omega t)}\mathbf{e_\lambda}(\mathbf{k})+a^\dag_{\mathbf{k},\lambda}e^{-i(\mathbf{k}\cdot\mathbf{r_i}-\omega t)}\mathbf{e_\lambda^*}(\mathbf{k})\bigg]\cdot\mathbf{p_i}\label{interactionham2}
\end{equation}
where we have added the normalization factor $N(\mathbf{k})=\sqrt{\hbar/2\varepsilon_0\omega(\mathbf{k})V}$ in the mode expansion of $\mathbf{A}(\mathbf{r},t)$.
\item Eqn.~\ref{interactionham} may be written as (in time-dependent perturbation theory)
\begin{equation}
H'(t)=Ue^{-i\omega t}+U^\dag e^{i\omega t},\quad U=\frac{e}{m_e}N(\mathbf{k})a_{\mathbf{k},\lambda}e^{i\mathbf{k}\cdot\mathbf{r}}\mathbf{e_\lambda}(\mathbf{k})\cdot\mathbf{p}\label{interactionham3}
\end{equation}
\end{enumerate}
\end{remarks}
\begin{prop}
Consider a transition $|1\rangle\rightarrow |2\rangle$ with initial and final states of the form 
$$|1\rangle=|\alpha;\alpha_{\text{EM}}\rangle=|\alpha\rangle\otimes|\alpha_{\text{EM}}\rangle,\quad|2\rangle=|\beta;\beta_{\text{EM}}\rangle=|\beta\rangle\otimes|\beta_{\text{EM}}\rangle$$
In such a transition, the state of the atom changes as $|\alpha\rangle\rightarrow|\beta\rangle$, $E_\alpha\rightarrow E_\beta$ while, simultaneously, the state of the EM field changes as $|\alpha_{\text{EM}}\rangle\rightarrow|\beta_{\text{EM}}\rangle$.
\begin{itemize}
    \item For the case $E_\beta>E_\alpha$, the dominant process is absorption (an atom absorbs a single photon from the EM field) which occurs at a rate $\Gamma(1\rightarrow 2)\propto|U_{21}|^2$. The state of the entire system changes like:
    $$|\alpha;n_{\mathbf{k},\lambda}+1\rangle\rightarrow|\beta;n_{\mathbf{k},\lambda}\rangle$$
    where the matrix element for absorption $U_{21}$ (Eqn.~\ref{interaction3}) is
    \begin{equation}
    U_{21}=\frac{e}{m_e}N(\mathbf{k})\langle\beta|\mathbf{e_\lambda}(\mathbf{k})\cdot\mathbf{p}|\alpha\rangle\sqrt{n_{\mathbf{k},\lambda}+1}\label{absorptionmatrix}
    \end{equation}
    \item For the case $E_\beta<E_\alpha$, the dominant process is emission (an atom emits a single photon into the EM field) which occurs at a rate $\Gamma(1\rightarrow 2)\propto|U_{21}^\dag|^2$. The state of the entire system changes like:
    $$|\alpha;n_{\mathbf{k},\lambda}\rangle\rightarrow|\beta;n_{\mathbf{k},\lambda}+1\rangle$$
    where the matrix element for emission $U_{21}^\dag$ (Eqn.~\ref{interaction3}) is
    \begin{equation}
    U_{21}^\dag=\frac{e}{m_e}N(\mathbf{k})\langle\beta|\mathbf{e_\lambda}^*(\mathbf{k})\cdot\mathbf{p}|\alpha\rangle\sqrt{n_{\mathbf{k},\lambda}+1}\label{emissionmatrix}
    \end{equation}
\end{itemize}
\end{prop}
\begin{proof}
The contribution to the matrix element $U_{21}$ for each electron and each photon mode is
$$U_{21}=\langle\beta;\beta_{\text{EM}}|U|\alpha;\alpha_{\text{EM}}\rangle=\frac{e}{m_e}N(\mathbf{k})\langle\beta;\beta_{\text{EM}}|a_{\mathbf{k},\lambda}e^{i\mathbf{k}\cdot\mathbf{r}}\mathbf{e_\lambda}(\mathbf{k})\cdot\mathbf{p}|\alpha;\alpha_{\text{EM}}\rangle$$
The photons involved have a wavelength which is much greater than the scale of atomic radii. The variation of the phase $\mathbf{k}\cdot\mathbf{r}$ of the EM wave across the atom is negligible, i.e.
$$e^{i\mathbf{k}\cdot\mathbf{r}}\approx 1+i\mathbf{k}\cdot\mathbf{r}+\dots\approx 1$$
Further, we have
$$\langle\dots,n_{\mathbf{k},\lambda},\dots|a_{\mathbf{k},\lambda}|\dots,n_{\mathbf{k},\lambda}+1,\dots\rangle=\sqrt{n_{\mathbf{k},\lambda}+1}$$
otherwise, the overlap is zero. The mode $(\mathbf{k},\lambda)$ (we cannot remove a photon from another mode) thus loses a photon and it can be any mode such that total energy is conserved ,i.e. $\omega=\omega_\beta-\omega_\alpha$. $U_{21}$ follows, and the rate follows from Fermi's Golden Rule. Similar for emission,
$$U_{21}^\dag=\frac{e}{m_e}N(\mathbf{k})\langle\beta|\mathbf{e_\lambda^*}(\mathbf{k})\cdot\mathbf{p}|\alpha\rangle\langle\beta_{\text{EM}}|a^\dag_{\mathbf{k},\lambda}|\alpha_{\text{EM}}\rangle$$
Result follows.
\end{proof}
\begin{remarks}
The emission case $n_{\lambda,\mathbf{k}}=0$ corresponds to spontaneous emission, in which a photon is emitted into a mode which is initially unoccupied, while $n_{\lambda,\mathbf{k}}\geq1$ corresponds to stimulated emission, in which a photon is added to a mode which is already occupied, and so has the same energy, same direction and same polarization state as the photons already occupying that mode.
\end{remarks}
\begin{prop}
The emission rate is
\begin{equation}
\Gamma(1\rightarrow 2)=\frac{\pi\omega_0}{\varepsilon_0V}(n_{\mathbf{k},\lambda}+1)|\mathbf{e_\lambda^*}(\mathbf{k})\cdot\mathbf{d}_{\beta\alpha}|^2g(E_k),\quad\mathbf{d}_{\beta\alpha}=\langle\beta|\mathbf{d}|\alpha\rangle\label{emission}
\end{equation}
\end{prop}
\begin{proof}
It is not easy to directly work with $\mathbf{p}_i=i\hbar\boldsymbol{\nabla}_i$ in $U^\dag_{21}$, we use a more convenient form using the relations
$$[\mathbf{r_i},\mathbf{p_j}^2]=2i\hbar\mathbf{p_i}\delta_{ij},\quad[\mathbf{r_i},V]=0\implies[\mathbf{r_i},H_0]=\frac{i\hbar}{m_e}\mathbf{p_i}$$
Further, we use
$$\langle\beta|[\mathbf{r_i},H_0]|\alpha\rangle=(E_\alpha-E_\beta)\langle\beta|\mathbf{r_i}|\alpha\rangle$$
The overall matrix element Eqn.~\ref{emissionmatrix} then becomes
$$U^\dag_{21}=\sum_{i=1}^N\frac{-ie}{\hbar}N(\mathbf{k})\sqrt{n_{\mathbf{k},\lambda}+1}\langle\beta|\mathbf{e_\lambda^*}(\mathbf{k})\cdot[\mathbf{r_i},H_0]|\alpha\rangle=-ie\omega_0N(\mathbf{k})\sqrt{n_{\mathbf{k,\lambda}}+1}\langle\beta|\mathbf{e_\lambda}^*(\mathbf{k})\cdot\mathbf{d}|\alpha\rangle$$
where $\mathbf{d}=-e\sum_{i=1}^N\mathbf{r_i}$ is the electric dipole operator. Including the normalization factor $N(\mathbf{k})=\sqrt{\hbar/2\varepsilon_0\omega(\mathbf{k})V}$, then $\Gamma$ follows from the continuum version of Fermi's Golden Rule.
\end{proof}
\begin{remarks}\leavevmode
\begin{enumerate}
\item The emission rate in Eqn.~\ref{emission} is proportional to $(n_{\lambda,\mathbf{k}}+1)/V$. The component proportional to $n_{\mathbf{k},\lambda}/V$ (photon number density) is the stimulated emission rate while that proportional to $1/V$ is the spontaneous emission rate - which is independent of the external EM field and depends only on the properties of the initial and final atomic states.
\item The transition rate Eqn.~\ref{emission} is the same as would be obtained for an electric dipole perturbation $H'=-\mathbf{E}(\mathbf{r},t)\cdot\mathbf{d}$ using the mode expansion for the electric field. The approximation $e^{i\mathbf{k}\cdot\mathbf{r}}\approx 1$ is known as the electric dipole approximation. Such transitions are electric dipole (E1) transitions.

\end{enumerate}
\end{remarks}
\newpage
\subsection{Spontaneous emission and decay}
\begin{prop}
Consider the spontaneous emission of a photon into an infinitesimal solid angle $d\Omega$ around the $\mathbf{k}$ direction in polarization state $\mathbf{e_\lambda}(\mathbf{k})$, the differential decay rate is
\begin{equation}
\frac{d\Gamma_\lambda}{d\Omega}\bigg|_{\alpha\rightarrow\beta}=\frac{\omega_0^3}{8\pi^2\varepsilon_0\hbar c^3}|\mathbf{e_\lambda}^*(\mathbf{k})\cdot\mathbf{d_{\beta\alpha}}|^2\label{differential_emission}
\end{equation}
where $\frac{d\Gamma_\lambda}{d\Omega}d\Omega$ is the number of photons in polarization state $\lambda$ emitted per unit time per aotm into a solid angle $d\Omega$ in the direction $(\theta,\phi)$. The total decay rate will be
\begin{equation}
\Gamma(\alpha\rightarrow\beta)=\frac{\omega_0^3}{3\pi\varepsilon_0\hbar c^3}|\mathbf{d_{\beta\alpha}}|^2\label{total_emission}
\end{equation}
\end{prop}
\begin{proof}
The photon frequency is $\omega_0=\omega_\alpha-\omega_\beta$. The density of states (Eqn.~\ref{DoS}) available to the photon is $g(E_k)=\frac{VE_k^2}{(2\pi\hbar c)^3}d\Omega$. The emission rate (Eqn.~\ref{emission}) is
$$d\Gamma=\frac{\pi\omega_0}{\varepsilon_0V}(n_{\mathbf{k},\lambda}+1)|\mathbf{e_\lambda}^*(\mathbf{k})\cdot\mathbf{d_{\beta\alpha}}|^2g(E_k)=\frac{\omega_0^3}{8\pi^2\varepsilon_0\hbar c^3}|\mathbf{e_\lambda}^*(\mathbf{k})\cdot\mathbf{d_{\beta\alpha}}|^2d\Omega$$
Integrate the total decay rate:
$$\Gamma(\alpha\rightarrow\beta)=\sum_\lambda\int\frac{d\Gamma_\lambda}{d\Omega}\bigg|_{\alpha\rightarrow\beta}d\Omega=\frac{\omega_0^3}{8\pi^2\varepsilon_0\hbar c^3}\frac{8\pi}{3}|\mathbf{d_{\beta\alpha}}|^2$$
where $|\mathbf{d_{\beta\alpha}}|^2=|\langle\beta|d_{+1}|\alpha\rangle|^2+|\langle\beta|d_{-1}|\alpha\rangle|^2+|\langle\beta|d_0|\alpha\rangle|^2$.
\end{proof}
\begin{remarks}
By specializing $|\alpha\rangle$ and $|\beta\rangle$ to angular momentum eigenstates $|\alpha,J,m_J\rangle\rightarrow|\alpha',J',m_J'\rangle+\gamma$, it will allow us to compute 
\begin{itemize}
    \item the lifetimes of atomic states
    \item the angular distribution and polarisation of the emitted photons.
\end{itemize}
\end{remarks}
\begin{eg}[Total decay rate]
The total decay rate is 
$$\Gamma=\frac{\omega_0^3}{3\pi\varepsilon_0\hbar c^3}(|d_{+1}|^2+|d_{-1}|^2+|d_0|^2),\quad d_m=\langle\alpha',J',m_J'|d_m|\alpha,J,m_J\rangle=\langle\alpha',J'||\mathbf{d}||\alpha,J\rangle\langle 1,m;J,m_J|J',m_J'\rangle$$
The Clebsch-Gordan coefficient $\langle1,m;J,m_J|J',m_J'\rangle$ vanishes unless the quantum numbers involved satisfy
$$m+m_J=m_J'\quad (m=0,\pm1);\quad 1\otimes J=J'$$
i.e. the selection rules are $\Delta m_J=m_J'-m_J=0,\pm1$, $\Delta J=J'-J=0,\pm1$ and $J+J'\geq1$ (0$\rightarrow$0 is forbidden).
\end{eg}
\begin{eg}[Angular distribution]
$\mathbf{e^*}\cdot\mathbf{d}$ takes the form:
\begin{itemize}
    \item $\Delta m_J=+1$: $(e_{+1})^*\langle m_J,+1|d_{+1}|m_J\rangle$
    \item $\Delta m_J=-1$: $(e_{-1})^*\langle m_J,-1|d_{-1}|m_J\rangle$
    \item $\Delta m_J=0$: $(e_0)^*\langle m_J|d_0|m_J\rangle$
\end{itemize}
The differential decay rate for each $\Delta m_J$ is
$$\frac{d\Gamma_\lambda}{d\Omega}\bigg|_{\Delta m_J=m}=A|(\mathbf{e_\lambda})_m|^2|C_m|^2,\quad A=\frac{\omega_0^3}{8\pi^2\varepsilon_0\hbar c^3}|\langle\alpha',J'||\mathbf{d}||\alpha,J\rangle|^2,~C_m=\langle1,m;J,m_J|J',m_J'\rangle$$
Explicit expressions for the spherical components of the polarization vectors:
$$\mathbf{k}=|\mathbf{k}|(\sin\theta\cos\phi,\sin\theta\sin\phi,\cos\theta),\quad\mathbf{e}=(e_{+1},e_{-1},e_0)$$
Previously, we found the spherical components of $\mathbf{e_1}$ and $\mathbf{e_2}$ are
$$\mathbf{e_1}(\mathbf{k})=-\frac{1}{\sqrt{2}}(\cos\theta~e^{i\phi},-\cos\theta~e^{-i\phi},-\sin\theta),\quad\mathbf{e_2}(\mathbf{k})=-\frac{i}{\sqrt{2}}(e^{i\phi},e^{-i\phi},0)$$
The spherical components of $\mathbf{e_L}$ and $\mathbf{e_R}$ are
$$\mathbf{e_L}(\mathbf{k})=-(\frac{1}{2}(1-\cos\theta)e^{i\phi},\frac{1}{2}(1+\cos\theta)e^{-i\phi},-\frac{1}{\sqrt{2}}\sin\theta),\quad\mathbf{e_R}(\mathbf{k})=-(\frac{1}{2}(1+\cos\theta)e^{i\phi},\frac{1}{2}(1-\cos\theta)e^{-i\phi},\frac{1}{\sqrt{2}}\sin\theta)$$
\end{eg}
\begin{eg}\leavevmode
\begin{enumerate}
    \item For a decay with $\Delta m_J=0$, the differential decay rate is
    $$\frac{d\Gamma_\lambda}{d\Omega}=A|C_0|^2|(\mathbf{e_\lambda^*})_0|^2,\quad C_0=\langle1,0;J,m_J|J',m_J\rangle$$
    For the plane polarisation basis $\mathbf{e_1}$ and $\mathbf{e_2}$, this gives $\frac{d\Gamma_1}{d\Omega}=A|C_0|^2\sin^2\theta$, $\frac{d\Gamma_2}{d\Omega}=0$, i.e. the photon is always emitted in the plane polarisation state $\mathbf{e_1}$. For circular polarisation basis $\mathbf{e_L}$ and $\mathbf{e_R}$, we have
    $$\frac{d\Gamma_L}{d\Omega}=\frac{d\Gamma_R}{d\Omega}=A|C_0|^2\frac{1}{2}\sin^2\theta$$
    i.e. the emitted photon is, equivalently, an equal mix of left-handed and right-handed circular polarisation states. In either basis (plane or circular), summing over both possible polarisation states gives $A|C_0|^2\sin^2\theta$. Thus, photons cannot be emitted along the $\pm z$ directions ($\theta=0$ and $\theta=\pi$) and are emitted preferentially in the $x$-$y$ plane ($\theta=\pi/2$).
    \item For a decay with $\Delta m_J=+1$, the differential decay rate is
    $$\frac{d\Gamma_\lambda}{d\Omega}=A|C_{+1}|^2|(\mathbf{e_\lambda}^*)_{+1}|^2,\quad C_{+1}=\langle1,+1;J,m_J|J',M_J+1\rangle$$
    For the plane polarisation basis $\mathbf{e_1}$, $\mathbf{e_2}$, we have $\frac{d\Gamma_1}{d\Omega}=A\frac{1}{2}|C_{+1}|^2\cos^2\theta$, $\frac{d\Gamma_2}{d\Omega}=A\frac{1}{2}|C_{+1}|^2$. For the circular polarisation basis $\mathbf{e_L}$, $\mathbf{e_R}$, we have
    $$\frac{d\Gamma_L}{d\Omega}=A\frac{1}{4}|C_{+1}|^2(1-\cos\theta)^2,\quad\frac{d\Gamma_R}{d\Omega}=A\frac{1}{4}|C_{+1}|^2(1+\cos\theta)^2$$
    In either basis, the sum is $\frac{d\Gamma}{d\Omega}|_{\Delta m_J=+1}=A0.5|C_{+1}|^2(1+\cos^2\theta)$. For $\theta=0$ and $\pi$, the emitted photon is always right-handed and left-handed respectively.
    \item Finally, $\Delta m_J=-1$ decays are the same as $\Delta m_J=+1$ except that the rates for the circular polarisation states have $L\leftrightarrow R$ interchanged. For $\theta=0$ and $\pi$, the emitted photon is always left-handed and right-handed respectively.
\end{enumerate}
The sum of $d\Gamma_i/d\Omega$ for all three cases $\sum_{\Delta m_J}\frac{d\Gamma_\lambda}{d\Omega}$ is a constant, independent of $\theta$, i.e. isotropic spatial distribution.
\end{eg}
\begin{remarks}
An unpolarised sample of atoms has no net polarisation along any spatial direction, and can be considered to populate all possible $m_J$ states equally, producing an equal mix of $\Delta m_J=+1$, 0, $-1$ decays thus emitting photons isotropically, as expected from rotational symmetry.
\end{remarks}
The results will be applied in the following.
\subsection{Magnetic fields and polarisation}
Consider again the weak-field Zeeman effect for an atom placed in a static external magnetic field $B_z$, i.e. the $z$ axis now acquires physical significance. The atomic energy level with total angular momentum quantum number $J$ splits into $2J+1$ levels.
\begin{prop}
Lines with $\Delta m_J=0$ are not seen when viewing the transition parallel to $\mathbf{B}$, but all lines are seen viewed perpendicular to $\mathbf{B}$.
\end{prop}
\begin{proof}
Suppose we view a spontaneous decay looking
\begin{enumerate}
    \item into $\mathbf{B}$: this corresponds to $\theta=0$ and the transition cannot be $\Delta m_J=0$,     the emitted photon (if decayed along $\mathbf{B}$) must be circularly polarised with $\Delta m_J=+1$ and $-1$ for right-polarized and left-polarized respectively,
    \item along $\mathbf{B}$ but in the $\theta=\pi$ direction: the emitted photon must be circularly polarized with $\Delta m_J=-1$ and $+1$ for right-polarized and left-polarized respectively,
    \item from the side of $\mathbf{B}$, say along the $x$ direction, $\theta=\pi/2$: all lines are now visible. For $\Delta m_J=0$, the photon must be $\mathbf{e_1}$ (plane-polarized along $z$), while for $\Delta m_J=\pm1$, the photon must be $\mathbf{e_2}$ (plane polarized along $y$), consistent with the transverse electric field for the photon.
\end{enumerate}
\end{proof}
\begin{remarks}
This is a consequence of conservation of angular momentum. For an atom in a state $|J,m_J\rangle$, the expectation $\langle J_z\rangle=m_J\hbar$. A transition with $\Delta m_J=0$ is incompatible with the photon spin of $\pm\hbar$.
\end{remarks}
\begin{eg}
By measuring the Zeeman splitting for different polarisation states, we can carry out the 3D reconstruction of the magnetic field strength and direction.
\end{eg}
\subsection{Stimulated emission and absorption}
Relative to the spontaneous emission rate, the stimulated emission rate contained an extra factor of $n_{\mathbf{k},\lambda}$.
\begin{prop}
Relative to the spontaneous emission rate, the stimulated emission rate has an additional factor of 
$$\frac{\pi^2c^3}{\hbar\omega^3}u(\omega)$$
\end{prop}
\begin{proof}
Consider initially an EM field which is isotropic and unpolarized with energy density $u(\omega)$. In an interval $dk$, in volume $V$, the EM field energy is
$$dU=u(\omega)Vd\omega=u(\omega)Vcdk$$
For an isotropic and unpolarised field, the photon number $n_{\mathbf{k},\lambda}$ in any mode must be independent of the direction of $\mathbf{k}$ and of the polarisation state $\lambda$: $n_{\mathbf{k},\lambda}$ depends only on the magnitude $k=|\mathbf{k}|=\omega/c$. The number of modes in the interval $dk$ is $g(k)dk=\frac{k^2V}{2\pi^2}dk$. In terms of photon number, the energy $dU$ is
$$dU=2n_{\mathbf{k},\lambda}\hbar\omega g(k)dk=2n_{\mathbf{k},\lambda}\hbar\omega\frac{k^2V}{2\pi^2}dk\implies u(\omega)Vc=2n_{\mathbf{k},\lambda}\hbar\omega\frac{k^2V}{2\pi^2}$$
where the factor of 2 accounts for the two possible photon spin states. The additional factor is $n_{\lambda,\mathbf{k}}=\frac{\pi^2c^3}{\hbar\omega^3}u(\omega)$.
\end{proof}
\begin{remarks}\leavevmode
\begin{enumerate}
    \item For atoms immersed in an isotropic EM field, the stimulated photons are emitted isotropically (same mode as incoming photon). For atoms immersed in a unidirectional beam, the stimulated photons are all emitted along the beam direction.
    \item For an atoms immersed in thermal radiation, the stimulated rate is $\frac{1}{e^{\hbar\omega/k_BT}-1}$ which is negligible at room temperature.
    \item Unlike emission, absorption can only be stimulated. For the reverse of stimulated emission, absorption rate must be equal, leading to $n_{\mathbf{k},\lambda}+1=\frac{\pi^2c^3}{\hbar\omega^3}u(\omega)$.
\end{enumerate}
\end{remarks}
\begin{eg}[Optical pumping]
Polarised light is used to produce a population distribution amongst atomic energy levels which is grossly different from the thermal Boltzmann distribution. To achieve this, place a sample of atoms inside $\mathbf{B}$ and illuminate them with circularly polarised radiation with frequency such that photons can be absorbed and induce transitions to a higher atomic energy level. For radiation travelling along $\mathbf{B}$ ($\theta=0$), left- and right-circular polarisation can induce only $\Delta m=+1$ and $-1$ atomic transitions.\\[5pt]
Consider two $J=1$ levels, illuminated with left-circularly polarized light. Absorption can only be $\Delta m_J=+1$, but spontaneous decay can be any $\Delta m_J$. No further upward transitions are possible for atoms in the lower $m_J=+1$ level and thus accumulate here. In contrast, in thermal equilibrium, atoms would be almost equally distributed amongst the various $m_J$ levels.
\end{eg}
\end{document}